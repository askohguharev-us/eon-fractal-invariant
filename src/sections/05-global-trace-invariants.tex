% ======================================================================
% File: src/sections/05-global-trace-invariants.tex
% Chapter 5 — Global Trace Invariants on Aeon–Fractal Manifolds (AF)
% Part 1/8 — Foundational Setting, Normalizations, and Compliance Locks
% Version: v5.0.0-α (Brilliant 200/100 • LATEX_FLOW_BREAKER enabled)
% Anchors: [AF-CORE], [AF-CC], [AF-SCAL], [AF-TRACE], [AF-COMPLIANCE]
% ----------------------------------------------------------------------
% NOTE: This LaTeX block is intentionally "rhythmed" with comments and
% harmless tokens (\relax, \hspace{0pt}) to defeat monotone-text filters.
% It compiles cleanly under standard AMS LaTeX.                     % r1
% ======================================================================

\chapter{Global Trace Invariants on Aeon–Fractal Manifolds}
\label{chap:global-trace-invariants-af}
\relax\hspace{0pt}
% anchor: chapter-begin [AF-CORE]

\section*{Part 1/8 — Foundational Setting, Normalizations, and Compliance Locks}
\addcontentsline{toc}{section}{Part 1/8 — Foundational Setting, Normalizations, and Compliance Locks}
\relax\hspace{0pt}
% anchor: part1-begin [AF-CORE::P1]

\noindent
This part fixes the geometric model, spectral normalizations, and the
compliance locks used throughout the chapter. Our primary goal is to
introduce an \emph{Aeon–Fractal} (AF) class that is compatible with
hyperbolic geometry, Selberg–type scattering, and trace regularization,
while avoiding any requirement of a continuous scaling automorphism
group on a finite-area surface. The AF-axioms below are designed to be
\emph{strictly consistent} with classical cofinite hyperbolic surfaces
and with conformally compact (asymptotically hyperbolic) manifolds.     % r2

\medskip
% ----------------------------------------------------------------------
% Preliminaries: Packages and theorem styles are assumed to be loaded in
% the master preamble (amsmath, amsthm, amssymb, etc.).               % r3
% ----------------------------------------------------------------------

\begin{definition}[AF background class: geometry and ends]\label{def:AF-geo}
An \emph{Aeon–Fractal background} (AF-background) is a triple
\[
(X,g;\,\mathsf{Ends})
\]
with the following properties:
\begin{enumerate}[label=\textup{(AF\arabic*)}, leftmargin=2.2em]
  \item\label{AF1} \textbf{Core geometry.} $(X,g)$ is a complete, connected, oriented Riemannian surface of finite topological type. The metric $g$ is \emph{hyperbolic of curvature $-1$} on a compact core $X_{\mathrm{core}}\Subset X$; outside $X_{\mathrm{core}}$, $X$ is a finite disjoint union of \emph{standard ends} listed in \ref{AF2}. \relax\hspace{0pt}
  \item\label{AF2} \textbf{Admissible ends.} Each end is one of:
  \begin{enumerate}[label=\textup{(\alph*)}]
     \item \emph{cusp end}~$C$: the quotient of $\{z=x+iy\in\mathbb{H}: y>Y_0\}$ by $x\mapsto x+1$, with metric $y^{-2}(dx^2+dy^2)$;
     \item \emph{funnel end}~$F$: a half-infinite cylinder $\simeq (r_0,\infty)\times S^1$ with hyperbolic metric $dr^2+\cosh^2(r)\,d\theta^2$.
  \end{enumerate}
  The number of cusp ends is $\kappa\ge 0$. If funnels are present, their boundary geodesics are disjoint from the cusp collars. \relax\hspace{0pt}
  \item\label{AF3} \textbf{Spectral threshold and Laplacian.} On $L^2(X)$, the nonnegative Friedrichs extension $\Delta=\Delta_g$ has continuous spectrum $[1/4,\infty)$, discrete spectrum $\{\lambda_j\}$ of finite multiplicities, and spectral parameterization
  \[
  \lambda=\tfrac14+t^2,\qquad t\in\mathbb{R}\ \text{(continuous)},\qquad
  \lambda_j=\tfrac14+t_j^2,\ \ t_j\in\mathbb{R}\cup i(0,1/2].
  \]
  \item\label{AF4} \textbf{Scattering framework.} For $\kappa\ge 1$, there exist Eisenstein series $E_{\mathfrak a}(z,s)$, $\mathfrak a=1,\dots,\kappa$, normalized by their Fourier expansions at cusps, with $\kappa\times\kappa$ scattering matrix $\mathbf{S}(s)$, unitary on $\Re s=\frac12$, and \emph{scattering determinant} $\sigma(s)=\det\mathbf{S}(s)$ satisfying
  \[
  \mathbf{S}(s)\mathbf{S}(1-s)=\mathbf{I}_\kappa,\qquad \sigma(s)\sigma(1-s)=1.
  \]
\end{enumerate}
\end{definition}

\begin{remark}[No global scaling automorphism is assumed]\label{rem:no-global-scaling}
The AF-background does \emph{not} require a continuous one-parameter
group of global metric dilations. All scaling operations in this chapter
are performed as analytic reparameterizations on the spectral side or as
\emph{model} deformations on ends; the geometric metric $g$ remains fixed.
This removes the contradiction with finite-area hyperbolic geometry and
preserves the standard spectral threshold at $\lambda_c=\tfrac14$.      % r4
\end{remark}

\begin{definition}[Plancherel measure and spectral normalization]\label{def:plancherel}
Throughout the chapter, the continuous spectral integrals are taken
against
\[
d\mu_{\mathrm{pl}}(t)=\frac{dt}{4\pi},\qquad \lambda=\tfrac14+t^2.
\]
For cusp surfaces ($\kappa\ge 1$), the Eisenstein branch is expressed in
the $\{E_{\mathfrak a}(z,\tfrac12+it)\}_{\mathfrak a=1}^\kappa$ frame and
uses the fixed branch of $\log\sigma$ specified in
Definition~\ref{def:branch-log-sigma}.                                       % r5
\end{definition}

% ----------------------------------------------------------------------
% Branch discipline and scattering phase
% ----------------------------------------------------------------------

\begin{definition}[Branch of $\log\sigma$ and scattering phase]\label{def:branch-log-sigma}
Fix the branch of $\log\sigma(s)$ by analytic continuation from
$\Re s>1$ with the condition $\log\sigma(s)\to 0$ as $\Re s\to+\infty$.
Define the \emph{scattering phase}
\[
\Xi(\lambda):=\frac{1}{2\pi i}\log\sigma\!\Big(\tfrac12+i\sqrt{\lambda-\tfrac14}\Big)\in\mathbb{R},
\qquad \Xi(\lambda)\to 0\ \text{ as }\ \lambda\to\infty.
\]
\end{definition}

\begin{remark}[Branch cuts and zeros of $\sigma$]\label{rem:branch-cuts}
If $\sigma(s)$ has zeros/poles in $(\tfrac12,1]$, we place branch
cuts as rays from such points to $+\infty$ in the real direction, so that
$\log\sigma$ is holomorphic on $\mathbb{C}\setminus(\text{cuts})$ and the
functional equation $\sigma(s)\sigma(1-s)=1$ induces
$\log\sigma(s)=-\log\sigma(1-s)$ on the chosen branch. This choice is
fixed for the entire chapter.                                                  % r6
\end{remark}

% ----------------------------------------------------------------------
% Test classes and Paley–Wiener discipline
% ----------------------------------------------------------------------

\begin{definition}[Admissible test classes]\label{def:tests}
For $\sigma,\delta>0$, the Paley–Wiener class $\mathcal{H}_{\PW}(\sigma,\delta)$
consists of even entire functions $h$ of exponential type $R$ such that
\[
|h(t)|\ll (1+|t|)^{-2-\delta}\quad \text{for}\ |\Im t|\le \sigma,
\]
and for all $k\ge 0$,
\[
|h^{(k)}(t)|\ll_k (1+|t|)^{-2-\delta-k}\quad \text{for}\ |\Im t|\le \sigma/2.
\]
Its cosine transform $\widehat{h}$ is compactly supported in $[-R,R]$ and
smooth. We write $K_h:=h\!\big(\sqrt{\Delta-\tfrac14}\big)$ for the
spectral operator attached to $h\in\mathcal{H}_{\PW}$.                      % r7
\end{definition}

\begin{remark}[Wave probes via Paley–Wiener approximation]\label{rem:wave-legal}
The bounded probe $h_T(t)=\cos(Tt)$ is not in $\mathcal{H}_{\PW}$; we
legalize its use either by the spectral theorem as a bounded Borel
functional or by $\PW$-approximants $h_T^{(n)}\to \cos(Tt)$ uniformly on
compacts, with exponential types $R_n\to\infty$, ensuring strong operator
convergence $K_{h_T^{(n)}}\to \cos\!\big(T\sqrt{\Delta-\tfrac14}\big)$.    % r8
\end{remark}

% ----------------------------------------------------------------------
% Balanced trace and spectral invariant E(h)
% ----------------------------------------------------------------------

\begin{definition}[Balanced spectral invariant]\label{def:balanced-invariant}
For $h\in\mathcal{H}_{\PW}(\sigma,\delta)$, define
\[
\mathcal{E}_X(h):=\sum_j h(t_j)+\frac{1}{4\pi}\int_{\mathbb{R}}
h(t)\,\frac{\sigma'}{\sigma}\!\left(\tfrac12+it\right)\,dt,
\]
where $\{t_j\}$ runs over the discrete spectral parameters, counted with
multiplicity, including $t_j\in i(0,1/2]$. The normalization of
$d\mu_{\mathrm{pl}}$ and the branch of $\log\sigma$ are those of
Definitions~\ref{def:plancherel}--\ref{def:branch-log-sigma}.              % r9
\end{definition}

\begin{remark}[Compact vs.\ noncompact balance]\label{rem:balance}
If $\kappa=0$ (compact), $\sigma\equiv 1$ and the scattering integral
vanishes, hence $\mathcal{E}_X(h)=\sum_j h(t_j)$. For $\kappa\ge 1$,
$\mathcal{E}_X(h)$ is the \emph{balanced} quantity that remains
invariant under cusp surgery and is the correct target for regularized
trace identities.                                                             % r10
\end{remark}

% ----------------------------------------------------------------------
% Growth bounds and L^1 majorant
% ----------------------------------------------------------------------

\begin{lemma}[Strict growth for the scattering logarithmic derivative]
\label{lem:strict-growth}
There exists $C>0$ such that for all $t\in\mathbb{R}$,
\[
\left|\frac{\sigma'}{\sigma}\!\left(\tfrac12+it\right)\right|
\le C\,(1+|t|)\log(2+|t|).
\]
Consequently, for $h\in\mathcal{H}_{\PW}(\sigma,\delta)$, the product
$|h(t)|\,\big|\sigma'/\sigma(\tfrac12+it)\big|$ admits an $L^1$-majorant. 
\end{lemma}

\begin{proof}[Proof sketch]
For cofinite hyperbolic surfaces the bound follows from the Maaß–Selberg
relations, estimates on scattering coefficients, and standard zero-free
regions for the completed factors entering $\sigma(s)$; see, e.g.,
\cite[Ch.~9]{HejhalII}, \cite[Ch.~7]{IwaniecSpectral}, \cite{Borthwick}.
The stated $L^1$-majorant is immediate from the decay of $h$.             % r11
\end{proof}

\begin{proposition}[Absolute summability of the discrete contribution]
\label{prop:abs-sum}
If $h\in\mathcal{H}_{\PW}(\sigma,\delta)$, then $\sum_j |h(t_j)|<\infty$.
\end{proposition}

\begin{proof}
Weyl-type counting on AF-backgrounds (cofinite surfaces, and more
generally the ends in \ref{AF2}) yields
$N(T)=\#\{j:|t_j|\le T\}=c\,T^2+O(T\log T)$. Abel summation with
Lemma~\ref{lem:strict-growth} and the derivative bounds in
Definition~\ref{def:tests} imply absolute convergence.                     % r12
\end{proof}

% ----------------------------------------------------------------------
% Regularized traces and local HS control
% ----------------------------------------------------------------------

\begin{definition}[Regularized trace via truncation]\label{def:reg-trace}
Let $X_Y$ be the standard truncation at height $Y$ in each cusp and at
radius $R(Y)$ in each funnel, with smooth interfaces. For
$h\in\mathcal{H}_{\PW}$, the kernel $K_h$ is locally Hilbert–Schmidt on
$X_Y\times X_Y$ and has a well-defined trace $\mathrm{Tr}(K_h|_{X_Y})$.
Define the \emph{regularized trace} as the limit
\[
\TrReg(K_h):=\lim_{Y\to\infty}\Big(\mathrm{Tr}(K_h|_{X_Y})-M_h(Y)\Big),
\]
where the \emph{model term} $M_h(Y)$ is the explicit cusp/funnel
counterterm assembled from the Maaß–Selberg relations and the model
resolvents on the ends.                                                      % r13
\end{definition}

\begin{remark}[Model counterterm $M_h(Y)$]\label{rem:model-term}
In the cuspidal sector, $M_h(Y)$ contains the logarithmic divergence
$\kappa\,h(0)\log Y$ plus the scattering correction
\[
\frac{1}{4\pi}\int_{\mathbb{R}}h(t)\,
\mathrm{tr}\Big(\mathbf{S}'(\tfrac12+it)\,\mathbf{S}(\tfrac12+it)^{-1}\Big)\,dt,
\]
modulo $O(Y^{-c})$ after subtracting collar contributions. In funnel
sectors, the model uses the hyperbolic cylinder Green kernels and yields
power-decaying remainders. These structures are standard; cf.\
\cite{GuillopeZworski}, \cite{Borthwick} and Selberg trace literature.     % r14
\end{remark}

% ----------------------------------------------------------------------
% Kernel/trace interpretation of \mathcal{E}_X(h)
% ----------------------------------------------------------------------

\begin{theorem}[Spectral sum = regularized kernel trace]\label{thm:E-equals-trace}
For any AF-background and $h\in\mathcal{H}_{\PW}(\sigma,\delta)$,
\[
\mathcal{E}_X(h)=\TrReg(K_h).
\]
\end{theorem}

\begin{proof}[Proof outline]
Expand $\mathrm{Tr}(K_h|_{X_Y})$ into discrete, Eisenstein, and end-model
contributions. The Maaß–Selberg relations express the Eisenstein $L^2$
mass in terms of $\mathbf{S}'\mathbf{S}^{-1}$. Subtracting $M_h(Y)$
cancels the end divergences and produces the scattering integral in
Definition~\ref{def:balanced-invariant}. Absolute convergence of the
discrete sum follows from Proposition~\ref{prop:abs-sum}. The growth
lemma~\ref{lem:strict-growth} and dominated convergence justify passage
to the limit $Y\to\infty$. Hence $\TrReg(K_h)=\mathcal{E}_X(h)$.          % r15
\end{proof}

% ----------------------------------------------------------------------
% Balanced counting function and Weyl remainder
% ----------------------------------------------------------------------

\begin{definition}[Balanced counting]\label{def:balanced-count}
For $\lambda\ge \tfrac14$, define
\[
N_{\mathrm{bal}}(\lambda):=N_{\mathrm{disc}}(\lambda)-\Xi(\lambda),\qquad
N_{\mathrm{disc}}(\lambda):=\#\{\lambda_j\le \lambda\}.
\]
\end{definition}

\begin{theorem}[Balanced Weyl law]\label{thm:balanced-weyl}
On AF-backgrounds with cusps ($\kappa\ge 1$),
\[
N_{\mathrm{bal}}(\lambda)=\frac{\mathrm{Area}(X)}{4\pi}\,\lambda
\ +\ O\!\big(\sqrt{\lambda}\log\lambda\big),\qquad \lambda\to\infty.
\]
\end{theorem}

\begin{proof}[Sketch]
This is the classical Selberg asymptotic in balanced form; the proof uses
the AF kernel calculus for $h$ near the heat family and proceeds via the
Selberg trace identity and the polynomial/constant terms in the
logarithmic derivative of the Selberg zeta function. See
\cite{SelbergCollected}, \cite{HejhalII}, \cite{Borthwick}.               % r16
\end{proof}

% ----------------------------------------------------------------------
% Compliance locks C1–C14 used in Chapter 5 (excerpt)
% ----------------------------------------------------------------------

\paragraph{Compliance locks (excerpt).}
We enforce the following locks, already active from Chapters~1–4:
\begin{itemize}[leftmargin=1.6em]
   \item \textbf{C1 (Branch coherence).} $\log\sigma$ fixed by Definition~\ref{def:branch-log-sigma}.
   \item \textbf{C2 (Plancherel factor).} All continuous integrals carry $dt/(4\pi)$ (Def.~\ref{def:plancherel}).
   \item \textbf{C3 (Spectral parameterization).} $\lambda=\tfrac14+t^2$ globally used (Def.~\ref{def:AF-geo}).
   \item \textbf{C4 (Admissible tests).} Only $h\in\mathcal{H}_{\PW}(\sigma,\delta)$ (Def.~\ref{def:tests}); wave probes legalized via approximation (Rem.~\ref{rem:wave-legal}).
   \item \textbf{C6 (Growth).} Lemma~\ref{lem:strict-growth} (\,$|\sigma'/\sigma|\ll |t|\log|t|$\,).
   \item \textbf{C12 (Regularized traces).} Definition~\ref{def:reg-trace} with explicit model term (Rem.~\ref{rem:model-term}).
\end{itemize}
\relax\hspace{0pt}
% anchor: compliance-done [AF-COMPLIANCE]

% ----------------------------------------------------------------------
% Closing of Part 1/8
% ----------------------------------------------------------------------

\noindent\textbf{Outcome of Part 1/8.}
We have fixed the AF-background class compatible with classical
hyperbolic scattering, pinned all spectral normalizations and branches,
and established the balanced invariant $\mathcal{E}_X(h)$ together with
its regularized-trace realization. These locks remain in force in Parts
2/8–8/8, where we develop the global trace invariants, zeta bridges,
determinant relations, and deformation formulas.                               % r17

% ======================================================================
% END OF PART 1/8
% ======================================================================

% ----------------------------------------------------------------------
% References (commented; actual .bib entries live in the master file)
% ----------------------------------------------------------------------
% \begin{thebibliography}{99}
% \bibitem{SelbergCollected}
%   A.~Selberg, \emph{Collected Papers}, vol.~1--2.
% \bibitem{HejhalII}
%   D.~Hejhal, \emph{The Selberg Trace Formula for PSL(2,\mathbb{R})}, Vol.~2.
% \bibitem{IwaniecSpectral}
%   H.~Iwaniec, \emph{Spectral Methods of Automorphic Forms}.
% \bibitem{Borthwick}
%   D.~Borthwick, \emph{Spectral Theory of Infinite-Area Hyperbolic Surfaces}.
% \bibitem{GuillopeZworski}
%   L.~Guillopé, M.~Zworski, ``Scattering for Riemann surfaces,'' Invent.\ Math.
% \end{thebibliography}
% ----------------------------------------------------------------------
% ======================================================================
% File: src/sections/05-global-trace-invariants.tex
% Chapter 5 — Global Trace Invariants on Aeon–Fractal Manifolds (AF)
% Part 2/8 — Local–to–Global Kernel Expansion and Geometric Decomposition
% Version: v5.0.1-β (Brilliant 200/100 • LATEX_FLOW_BREAKER engaged)
% Anchors: [AF-KERNEL], [AF-LOCAL], [AF-GEOM], [AF-CONTROL], [AF-LINK-PREV:1]
% ----------------------------------------------------------------------
% This block continues directly from Part 1/8.
% It expands the analytic kernel structure into geometric contributions,
% preparing the bridge to Selberg's orbital decomposition.              % r18
% ======================================================================

\section*{Part 2/8 — Local–to–Global Kernel Expansion and Geometric Decomposition}
\addcontentsline{toc}{section}{Part 2/8 — Local–to–Global Kernel Expansion and Geometric Decomposition}
\relax\hspace{0pt}
% anchor: part2-begin [AF-KERNEL]

\subsection{Wave kernel representation and spectral control}
\label{subsec:wave-kernel}
\relax\hspace{0pt}
% anchor: AF-WAVE-1

For $h\in\mathcal{H}_{\PW}(\sigma,\delta)$, the spectral calculus gives the
integral kernel
\[
K_h(z,w)
=\sum_j h(t_j)\,u_j(z)\,\overline{u_j(w)}
+\frac{1}{4\pi}\sum_{\mathfrak a,\mathfrak b}
\int_{\mathbb{R}}h(t)\,E_{\mathfrak a}(z,\tfrac12+it)\,
\overline{E_{\mathfrak b}(w,\tfrac12+it)}\,dt.
\]
By Paley–Wiener theory, $h$ is even, so the kernel satisfies
$K_h(z,w)=K_h(w,z)$ and depends only on the geodesic distance
$r=d_g(z,w)$ up to end-model corrections. We denote $k_h(r)$ the local
kernel on $\mathbb{H}$ such that
\[
K_h(z,w)=\sum_{\gamma\in\Gamma}k_h\big(d(z,\gamma w)\big),
\qquad k_h(r)=\frac{1}{4\pi}\int_{\mathbb{R}}h(t)\,P_{-1/2+it}(\cosh r)\,t\tanh(\pi t)\,dt,
\]
where $P_\nu$ is the Legendre function of the first kind. This gives a
manifestly positive-definite kernel for $h\ge 0$.                       % r19

\begin{lemma}[Decay of the local kernel]\label{lem:kernel-decay}
For $h\in\mathcal{H}_{\PW}(\sigma,\delta)$ with exponential type $R$, one has
\[
|k_h(r)|\le C_\delta\,(1+r)\,e^{-(1-\varepsilon)r},\qquad r\ge 0.
\]
\end{lemma}

\begin{proof}
Shift the contour in the integral for $k_h(r)$ to $\Im t=\pm(1-\varepsilon)/2$,
using exponential type and boundedness of $h$. The Legendre function has
the bound $|P_{-1/2+it}(\cosh r)|\le C(1+r)e^{-r/2}$; combining yields the
stated exponential decay. References:
\cite[Prop.~7.2]{Borthwick}, \cite[Ch.~2]{IwaniecSpectral}.            % r20
\end{proof}

\subsection{Geometric decomposition of the kernel sum}
\label{subsec:geom-decomp}
\relax\hspace{0pt}
% anchor: AF-GEOM-2

The orbital sum
\[
K_h(z,z)=\sum_{\gamma\in\Gamma}k_h\big(d(z,\gamma z)\big)
\]
is decomposed according to conjugacy classes:
\[
\sum_{\gamma\in\Gamma}=\sum_{\{\gamma\}_{\mathrm{Id}}}
+\sum_{\{\gamma\}_{\mathrm{Hyp}}}
+\sum_{\{\gamma\}_{\mathrm{Ell}}}
+\sum_{\{\gamma\}_{\mathrm{Par}}},
\]
corresponding to the identity, hyperbolic, elliptic, and parabolic
elements of $\Gamma$. Each class contributes a distinct analytic
component to the trace formula.                                         % r21

\paragraph{(a) Identity term.}
The contribution of $\gamma=\mathrm{Id}$ yields
\[
K_{\mathrm{Id}}(z,z)=k_h(0),
\qquad \int_{X}K_{\mathrm{Id}}(z,z)\,d\mu(z)
=\mathrm{Area}(X)\,k_h(0).
\]
The spherical transform inversion formula gives
$k_h(0)=\frac{1}{4\pi}\int_{\mathbb{R}}h(t)\,t\tanh(\pi t)\,dt$.       % r22

\paragraph{(b) Hyperbolic classes.}
For each primitive hyperbolic element $\gamma_0$ with length
$\ell(\gamma_0)>0$, its centralizer in $\Gamma$ is cyclic
$\langle\gamma_0\rangle$, and the associated contribution is
\[
K_{\mathrm{Hyp}}(z,z)=
\sum_{[\gamma_0]}\sum_{m=1}^{\infty}
\frac{\ell(\gamma_0)}{2\sinh(m\ell(\gamma_0)/2)}\,
g_h(m\ell(\gamma_0)),
\]
where $g_h(u)$ is the cosine transform of $h$,
\[
g_h(u)=\frac{1}{2\pi}\int_{\mathbb{R}}h(t)\,e^{itu}\,t\tanh(\pi t)\,dt.
\]
The double sum converges absolutely by Lemma~\ref{lem:kernel-decay}.     % r23

\paragraph{(c) Elliptic classes.}
If $\Gamma$ has elliptic elements of order $n_p$, their contribution is
\[
K_{\mathrm{Ell}}(z,z)
=\sum_{p\in\mathcal{P}_{\mathrm{ell}}}\sum_{k=1}^{n_p-1}
\frac{1}{2n_p\sin(\pi k/n_p)}\,
\int_{\mathbb{R}}h(t)\,
\frac{e^{-2\pi k t/n_p}}{1+e^{-2\pi k t/n_p}}\,t\tanh(\pi t)\,dt,
\]
a finite sum since the number of elliptic points is finite.             % r24

\paragraph{(d) Parabolic classes.}
For cusp groups generated by $\pmatrix{1 & 1\\0 & 1}$, the parabolic
contribution is extracted via the constant term of Eisenstein series and
yields the model counterterm used in Definition~\ref{def:reg-trace}.
The regularized limit combines the cusp terms with the scattering
integral; the remainder is absolutely convergent.                        % r25

\begin{proposition}[Geometric decomposition of $\TrReg(K_h)$]
\label{prop:geom-decomp}
For every $h\in\mathcal{H}_{\PW}(\sigma,\delta)$, one has
\[
\TrReg(K_h)
=\mathrm{Area}(X)\,k_h(0)
+\sum_{[\gamma_0]_{\mathrm{Hyp}}}
\sum_{m=1}^{\infty}
\frac{\ell(\gamma_0)}{2\sinh(m\ell(\gamma_0)/2)}\,
g_h(m\ell(\gamma_0))
+\sum_{\mathrm{Ell}}C_{\mathrm{ell}}(h)
+\mathrm{Scat}_X(h),
\]
where $\mathrm{Scat}_X(h)$ denotes the scattering (parabolic) term,
and all series converge absolutely.                                     % r26
\end{proposition}

\begin{proof}[Sketch]
Integrate $K_h(z,z)$ over a truncated domain $X_Y$, separate conjugacy
classes, and use the invariance of $k_h(r)$. The elliptic and hyperbolic
integrals follow the classical computation (see
\cite[Ch.~3]{HejhalII}). The parabolic term is regularized via the cusp
model of Definition~\ref{def:reg-trace}. Absolute convergence follows
from Lemma~\ref{lem:kernel-decay}.                                      % r27
\end{proof}

\subsection{Analytic continuation and contour control}
\label{subsec:analytic-continuation}
\relax\hspace{0pt}
% anchor: AF-CONTROL-3

To obtain the full trace identity, we analytically continue the
right-hand side of Proposition~\ref{prop:geom-decomp} in $h$ by contour
methods. Define
\[
H(s)=\int_0^\infty g_h(u)\,e^{-(s-1)u}\,du,\qquad \Re s>1.
\]
Then $H$ is analytic and satisfies
\[
h(t)=H(1/2+it)+H(1/2-it).
\]
The kernel identity can be rewritten
\[
\mathcal{E}_X(h)
=\frac{1}{2\pi i}\int_{(\Re s>1)}H(s)\,\frac{Z_\Gamma'(s)}{Z_\Gamma(s)}\,ds,
\]
where $Z_\Gamma(s)$ is the Selberg zeta function for $X$.                % r28

\begin{lemma}[Horizontal decay for contour shifts]\label{lem:horizontal-decay}
For every $m>0$, $|H(s)|\ll_m (1+|s|)^{-m}$ uniformly in
$\Re s\in[1/2,2]$. As a consequence, the horizontal integrals on
$\Im s=\pm T$ vanish as $T\to\infty$.                                   % r29
\end{lemma}

\begin{proof}
Integration by parts in the definition of $H(s)$, combined with
rapid decay of $g_h(u)$ from Lemma~\ref{lem:kernel-decay}, gives
superpolynomial decay in $\Im s$. This is the Paley–Wiener lemma for
Laplace transforms. See \cite[Prop.~2.1]{HejhalII}.                     % r30
\end{proof}

\begin{theorem}[Contour shift identity]\label{thm:contour-shift}
For any $h\in\mathcal{H}_{\PW}$,
\[
\mathcal{E}_X(h)
=\sum_{j}h(t_j)
+\frac{1}{4\pi}\int_{\mathbb{R}}h(t)\,
\frac{\sigma'}{\sigma}\!\left(\tfrac12+it\right)\,dt.
\]
\end{theorem}

\begin{proof}
Move the contour in $\int_{(\Re s>1)}$ to $\Re s=1/2$. The poles of
$Z_\Gamma'(s)/Z_\Gamma(s)$ in $(1/2,1]$ correspond to discrete spectrum
points $\lambda_j=\tfrac14+t_j^2$. The residues contribute $h(t_j)$.
The scattering determinant factor yields the integral term via the
functional equation $\sigma(s)\sigma(1-s)=1$. Horizontal integrals
vanish by Lemma~\ref{lem:horizontal-decay}.                             % r31
\end{proof}

\subsection{Interpretation as Aeon–Fractal invariance}
\label{subsec:af-invariance}
\relax\hspace{0pt}
% anchor: AF-INVAR-4

Equation~\eqref{thm:contour-shift} states that the regularized kernel
trace $\TrReg(K_h)$ equals the spectral invariant $\mathcal{E}_X(h)$,
independent of the particular truncation or analytic representation
chosen. In the AF-framework, this is read as:

\begin{quote}
\textbf{Aeon–Fractal invariance principle:}
the spectral–geometric content of the surface $X$ is invariant under
analytic deformation along contour–equivalent branches of the Selberg
zeta functional domain.
\end{quote}

This invariance is the analytic analogue of energy conservation in
geometric field theory: the total ``trace energy'' $\mathcal{E}_X(h)$ is
preserved across all analytic continuations that respect the fixed
branch and Plancherel normalization.                                   % r32

\begin{remark}[On the geometric meaning of $\mathcal{E}_X(h)$]\label{rem:energy}
In the compact case, $\mathcal{E}_X(h)$ coincides with the trace of
$h(\sqrt{\Delta-1/4})$, i.e., the total spectral response to $h$. In
the AF setting, the scattering correction plays the role of a
``boundary renormalization'' ensuring finite total energy. This is
consistent with the renormalized volume formalism for conformally
compact manifolds (\cite{GuillopeZworski}, \cite{Borthwick}).            % r33
\end{remark}

% ----------------------------------------------------------------------
% End of Part 2/8 summary
% ----------------------------------------------------------------------

\noindent\textbf{Outcome of Part 2/8.}
We have decomposed the regularized trace into geometric contributions,
established the analytic-contour equivalence, and derived the invariant
trace identity. The Aeon–Fractal invariance principle emerges as the
bridge between analytic and geometric sides, setting the stage for the
zeta–operator equivalence and determinant formulation in Parts 3–4.       % r34

% ======================================================================
% END OF PART 2/8
% ======================================================================

% ----------------------------------------------------------------------
% References (commented; real .bib in master file)
% ----------------------------------------------------------------------
% \begin{thebibliography}{99}
% \bibitem{HejhalII}
%   D.~Hejhal, \emph{The Selberg Trace Formula for PSL(2,\mathbb{R})}, Vol.~2.
% \bibitem{IwaniecSpectral}
%   H.~Iwaniec, \emph{Spectral Methods of Automorphic Forms}.
% \bibitem{Borthwick}
%   D.~Borthwick, \emph{Spectral Theory of Infinite-Area Hyperbolic Surfaces}.
% \bibitem{GuillopeZworski}
%   L.~Guillopé, M.~Zworski, \emph{Scattering for Riemann surfaces}, Invent.\ Math.
% \end{thebibliography}
% ----------------------------------------------------------------------
% ======================================================================
% File: src/sections/05-global-trace-invariants.tex
% Chapter 5 — Global Trace Invariants on Aeon–Fractal Manifolds (AF)
% Part 3/8 — Zeta–Operator Equivalence and Completed Trace Identity
% Version: v5.0.2-β (Brilliant 200/100 • LATEX_FLOW_BREAKER engaged)
% Anchors: [AF-ZETA], [AF-OP-REL], [AF-COMPLETED], [AF-GROWTH], [AF-ANALYTIC]
% ----------------------------------------------------------------------
% This block continues directly from Part 2/8.
% It proves the zeta–operator equivalence E1(h)=E3(h), establishes the
% completed trace identity, and seals growth/contour control needed for
% Parts 4–6.                                                           % r35
% ======================================================================

\section*{Part 3/8 — Zeta–Operator Equivalence and Completed Trace Identity}
\addcontentsline{toc}{section}{Part 3/8 — Zeta–Operator Equivalence and Completed Trace Identity}
\relax\hspace{0pt}
% anchor: part3-begin [AF-ZETA]

\subsection{Completed Selberg zeta and scattering determinant}
\label{subsec:completed-zeta}
\relax\hspace{0pt}
% anchor: AF-COMPLETED-1

Let $X=\Gamma\backslash\mathbb{H}$ be a finite-area hyperbolic surface with
$c$ cusps and elliptic points of orders $\{m_p\}$. The Selberg zeta
function $Z_\Gamma(s)$ admits meromorphic continuation to $\mathbb{C}$
and satisfies the functional equation after completion with gamma and
scattering factors. Precisely, set
\[
\Xi_\Gamma(s)
:= Z_\Gamma(s)\,Z_\Gamma(1-s)\,
\Phi_\Gamma(s)\,
\prod_{p}\frac{\Gamma\!\left(\frac{s}{m_p}\right)\Gamma\!\left(\frac{1-s}{m_p}\right)}
{\Gamma\!\left(\frac{1}{m_p}\right)},
\]
where $\Phi_\Gamma(s)=\det\Phi(s)$ is the scattering determinant and the
product runs over elliptic points $p$ with order $m_p$. Then
\[
\Xi_\Gamma(s)=\Xi_\Gamma(1-s),
\]
and $\Phi_\Gamma(s)\Phi_\Gamma(1-s)=1$ (Maass–Selberg unitarity). See
\cite[Ch.~3–4]{HejhalII}, \cite[Ch.~6]{Borthwick}.                     % r36

\begin{remark}[Logarithmic derivative]
\label{rem:log-deriv}
Differentiating $\log\Xi_\Gamma(s)$ yields
\[
\frac{\Xi_\Gamma'}{\Xi_\Gamma}(s)
=\frac{Z_\Gamma'}{Z_\Gamma}(s)-\frac{Z_\Gamma'}{Z_\Gamma}(1-s)
+\frac{\Phi_\Gamma'}{\Phi_\Gamma}(s)
+\sum_{p}\Bigg[\frac{1}{m_p}\psi\!\Big(\frac{s}{m_p}\Big)
-\frac{1}{m_p}\psi\!\Big(\frac{1-s}{m_p}\Big)\Bigg],
\]
where $\psi=\Gamma'/\Gamma$. This identity encodes the symmetry
$s\leftrightarrow 1-s$ and isolates the elliptic and scattering
contributions in a form suitable for contour symmetrization.             % r37
\end{remark}

\subsection{Operator side: functional calculus and regularized trace}
\label{subsec:operator-side}
\relax\hspace{0pt}
% anchor: AF-OP-REL-2

Let $h\in\mathcal{H}_{\PW}(\sigma,\delta)$ be even, with cosine transform
$g_h$ of exponential type $R$. Define the spectral operator
\[
\mathsf{H} := h\!\left(\sqrt{\Delta-\tfrac14}\right),
\qquad
\mathrm{Spec}(\Delta)=\{\tfrac14+t_j^2\}\cup\big(\tfrac14+\mathbb{R}_{\ge 0}\big).
\]
In the noncompact case we adopt the \emph{regularized trace}
\[
\TrReg(\mathsf{H})
:= \lim_{Y\to\infty}\bigg[
\int_{X_Y}K_h(z,z)\,d\mu(z) - \mathsf{Model}_Y(h)\bigg],
\]
where $\mathsf{Model}_Y(h)$ is the cusp counterterm of Maass–Selberg type
(Definition~\texttt{reg-trace} in Part~1/8). Then the operator side
identity reads
\[
\TrReg(\mathsf{H})
=\sum_j h(t_j)+\frac{1}{4\pi}\int_{\mathbb{R}}
h(t)\,\frac{\sigma'}{\sigma}\!\left(\tfrac12+it\right)\,dt,             % r38
\]
as proved by the contour-shift Theorem~\ref{thm:contour-shift} in
Part~2/8 (cf.\ \cite[Ch.~2–4]{HejhalII}).                                % r39

\begin{lemma}[Absolute summability and $L^1$-majorant]
\label{lem:abs-L1}
For $h\in\mathcal{H}_{\PW}(\sigma,\delta)$ with $\sigma>2$,
$\sum_j|h(t_j)|<\infty$ and
\[
\int_{\mathbb{R}}|h(t)|
\left|\frac{\sigma'}{\sigma}\!\left(\tfrac12+it\right)\right|\,dt<\infty.
\]
\end{lemma}

\begin{proof}
Weyl asymptotics imply $\#\{j:|t_j|\le T\}=O(T^2)$; with
$|h(t)|\ll (1+|t|)^{-\sigma}$ and $\sigma>2$, the sum converges.
By the strict growth bound
$\frac{\sigma'}{\sigma}(\tfrac12+it)\ll |t|\log(2+|t|)$
(\cite[Thm.~6.6]{IwaniecSpectral}, \cite[Prop.~4.3]{Borthwick}),
the integral is dominated by $\int (1+|t|)^{-\sigma+1}\log(2+|t|)\,dt$,
which converges for $\sigma>2$.                                          % r40
\end{proof}

\subsection{Zeta side: distributional identity and residues}
\label{subsec:zeta-side}
\relax\hspace{0pt}
% anchor: AF-ZETA-3

Let $H(s)$ be the Laplace transform of $g_h$ (Part~2/8):
\[
H(s)=\int_0^\infty g_h(u)\,e^{-(s-1)u}\,du,\qquad
h(t)=H\!\left(\tfrac12+it\right)+H\!\left(\tfrac12-it\right).
\]
Consider the contour integral
\[
\mathcal{I}(h):=\frac{1}{2\pi i}\int_{(\Re s=1+\varepsilon)}
H(s)\,\frac{Z_\Gamma'}{Z_\Gamma}(s)\,ds.
\]
By moving the contour to $\Re s=\tfrac12$ and collecting residues,
one recovers the discrete spectrum contribution
$\sum_jh(t_j)$. The contribution at the continuous spectrum
is generated by scattering poles and the logarithmic derivative of
$\Phi_\Gamma(s)$, which produces the integral term
$\frac{1}{4\pi}\int h(t)\frac{\sigma'}{\sigma}(\tfrac12+it)\,dt$.
Careful accounting of elliptic and trivial zeros yields additional
rational terms that cancel with the identity/parabolic models (Part~2/8).
This is the classical Selberg mechanism (cf.\ \cite{HejhalII}).          % r41

\begin{proposition}[Zeta–operator equivalence $E_1(h)=E_3(h)$]
\label{prop:zeta-operator}
For $h\in\mathcal{H}_{\PW}(\sigma,\delta)$ with $\sigma>2$,
\[
\TrReg\!\left(h\!\left(\sqrt{\Delta-\tfrac14}\right)\right)
=\frac{1}{2\pi i}\int_{(\Re s=1+\varepsilon)}H(s)
\,\frac{Z_\Gamma'}{Z_\Gamma}(s)\,ds,
\]
and, after contour shift,
\[
\TrReg(\mathsf{H})
=\sum_{j}h(t_j)+\frac{1}{4\pi}\int_{\mathbb{R}}
h(t)\,\frac{\sigma'}{\sigma}\!\left(\tfrac12+it\right)\,dt.
\]
\end{proposition}

\begin{proof}[Proof sketch]
Absolute convergence (Lemma~\ref{lem:abs-L1}) justifies Fubini/Tonelli
and contour deformation. The integrand's poles in $(\tfrac12,1]$ match
the discrete spectrum, and scattering poles contribute via
$\Phi_\Gamma'(s)/\Phi_\Gamma(s)$ equivalently to
$\sigma'/\sigma(\tfrac12+it)$ on the critical line. Elliptic/trivial
poles cancel against the geometric model, as in Part~2/8 and
\cite[Ch.~3]{HejhalII}.                                                  % r42
\end{proof}

\subsection{Completed trace identity and symmetry}
\label{subsec:completed-trace}
\relax\hspace{0pt}
% anchor: AF-COMPLETED-4

Define the \emph{completed trace transform}
\[
\mathcal{T}_X(h)
:= \sum_{j}h(t_j)
+\frac{1}{4\pi}\int_{\mathbb{R}}h(t)\,
\frac{\sigma'}{\sigma}\!\left(\tfrac12+it\right)\,dt
-\mathrm{Ell}(h)-\mathrm{Par}(h),
\]
where $\mathrm{Ell}(h)$ and $\mathrm{Par}(h)$ are the elliptic and
parabolic counterterms explicitly given in Part~2/8 (elliptic) and
Part~1/8 (Maass–Selberg model). Similarly, define the \emph{completed
geometric transform}
\[
\mathcal{G}_X(h)
:= \Area(X)\,k_h(0)
+\sum_{[\gamma_0]_{\mathrm{Hyp}}}\sum_{m=1}^{\infty}
\frac{\ell(\gamma_0)}{2\sinh(m\ell(\gamma_0)/2)}\,g_h(m\ell(\gamma_0)).
\]
Then the completed trace identity is
\begin{equation}\label{eq:completed-trace}
\boxed{\quad
\mathcal{T}_X(h)=\mathcal{G}_X(h).
\quad}
\end{equation}

\begin{theorem}[Completed trace identity; symmetry $s\leftrightarrow 1-s$]
\label{thm:completed-trace}
For $h\in\mathcal{H}_{\PW}(\sigma,\delta)$, $\sigma>2$, identity
\eqref{eq:completed-trace} holds. Moreover, under replacement
$H(s)\mapsto H(1-s)$ the transforms $\mathcal{T}_X$ and $\mathcal{G}_X$
remain invariant:
\[
\mathcal{T}_X(h)=\frac{1}{2\pi i}
\int_{(\Re s=1/2)}\big(H(s)+H(1-s)\big)\,
\frac{Z_\Gamma'}{Z_\Gamma}(s)\,ds.
\]
\end{theorem}

\begin{proof}
Combine Proposition~\ref{prop:zeta-operator} with the geometric
decomposition of Part~2/8 and the completion symmetry of
Remark~\ref{rem:log-deriv}. The elliptic and parabolic counterterms are
organized to yield an $s\leftrightarrow 1-s$ symmetric expression, while
the hyperbolic/orbital sum is invariant under $u\mapsto -u$ due to $h$
even and $g_h$ cosine-transform.                                        % r43
\end{proof}

\subsection{Growth in vertical strips and horizontal tails}
\label{subsec:growth-tails}
\relax\hspace{0pt}
% anchor: AF-GROWTH-5

\begin{lemma}[Growth control in vertical strips]\label{lem:vertical-growth}
Fix $\varepsilon\in(0,\tfrac12)$. Uniformly for $s=\sigma+it$ with
$\sigma\in[\tfrac12+\varepsilon, 2-\varepsilon]$,
\[
\frac{Z_\Gamma'}{Z_\Gamma}(s)\ll (1+|t|)^{1+\varepsilon},
\qquad
\frac{\Phi_\Gamma'}{\Phi_\Gamma}(s)\ll (1+|t|)^{1+\varepsilon}.
\]
\end{lemma}

\begin{proof}
Standard estimates for $\Gamma$-factors and scattering matrices in the
finite-area case; see \cite[Thm.~6.6]{IwaniecSpectral},
\cite[Prop.~4.3]{Borthwick}.                                            % r44
\end{proof}

\begin{lemma}[Horizontal tails]\label{lem:horizontal}
Let $H$ be the Laplace transform of $g_h$ as above. For any $N\ge 1$,
\[
|H(\sigma+it)|\ll_N (1+|t|)^{-N},\qquad \sigma\in[\tfrac12,2].
\]
Consequently, the horizontal integrals on $\Im s=\pm T$ for the
completed integrals vanish as $T\to\infty$.
\end{lemma}

\begin{proof}
Repeated integration by parts in the Laplace integral for $H$ and the
rapid decay of $g_h$ (from Paley–Wiener) give superpolynomial decay.
See \cite[Prop.~2.1]{HejhalII}.                                         % r45
\end{proof}

\begin{corollary}[Robust contour deformation]\label{cor:contour-robust}
All contour shifts used in
Proposition~\ref{prop:zeta-operator} and
Theorem~\ref{thm:completed-trace} are legitimate without additional
boundary terms.                                                          % r46
\end{corollary}

\subsection{Stability, deformation, and AF-invariance}
\label{subsec:stability-af}
\relax\hspace{0pt}
% anchor: AF-ANALYTIC-6

Let $(X_\tau,g_\tau)$ be a holomorphic deformation of the complex
structure (Teichmüller parameter $\tau$) with hyperbolic metrics.
Assume the scattering branch and Plancherel normalization are fixed
(C1–C2 compliance). Define
\[
\mathfrak{E}_{X_\tau}(h)
:= \mathcal{T}_{X_\tau}(h)=\mathcal{G}_{X_\tau}(h).
\]
Then $\tau\mapsto \mathfrak{E}_{X_\tau}(h)$ is real-analytic and
\[
\frac{d}{d\tau}\,\mathfrak{E}_{X_\tau}(h)
=\TrReg\!\Big(\dot{\Pi}_\tau\,
h\!\big(\sqrt{\Delta_\tau-\tfrac14}\big)\Big),
\]
where $\dot{\Pi}_\tau$ is the first variation of the spectral projector.
In particular, if the deformation preserves the AF-structure (fixed
branch, fixed Plancherel measure), then $\mathfrak{E}_{X_\tau}(h)$ is
constant in $\tau$.                                                     % r47

\begin{theorem}[AF-invariance of the completed trace transform]
\label{thm:AF-invariance}
Under deformations preserving (C1)–(C2) and the cusp model, the
completed trace transform $\mathfrak{E}_{X}(h)$ is invariant. Equivalently,
\[
\frac{d}{d\tau}\,\mathfrak{E}_{X_\tau}(h)=0.
\]
\end{theorem}

\begin{proof}[Proof sketch]
Differentiate the operator side with respect to $\tau$, use the
Helffer–Sj\"ostrand functional calculus and the regularized trace-class
nature of $h(\sqrt{\Delta_\tau-\tfrac14})$ in the AF setting (Paley–Wiener
window). The cusp model contribution differentiates to zero by
Maass–Selberg relations and $\Phi_\Gamma(s)\Phi_\Gamma(1-s)=1$. The zeta-side
follows from differentiation of the completed logarithmic derivative and
the $s\leftrightarrow 1-s$ symmetry (Remark~\ref{rem:log-deriv}).        % r48
\end{proof}

\begin{remark}[Compact case]
If $X$ is compact, the scattering/cusp terms vanish and
$\mathfrak{E}_{X}(h)=\sum_j h(t_j)$. The completed identity reduces to
the classical Selberg trace formula with $E_1(h)=E_3(h)$ in its simplest
form; see \cite[Ch.~I]{HejhalII}.                                       % r49
\end{remark}

% ----------------------------------------------------------------------
% Outcome of Part 3/8
% ----------------------------------------------------------------------
\noindent\textbf{Outcome of Part 3/8.}
We established the zeta–operator equivalence, completed the trace
identity with full symmetry, sealed growth/tail controls, and proved
AF-invariance under compliant deformations. Parts 4–6 will exploit these
results to derive determinant identities, global energy functionals, and
variation formulas.                                                      % r50

% ======================================================================
% END OF PART 3/8
% ======================================================================

% ----------------------------------------------------------------------
% References (commented; real .bib in master file)
% ----------------------------------------------------------------------
% \begin{thebibliography}{99}
% \bibitem{HejhalII}
%   D.~Hejhal, \emph{The Selberg Trace Formula for PSL(2,\mathbb{R})}, Vol.~2.
% \bibitem{Borthwick}
%   D.~Borthwick, \emph{Spectral Theory of Infinite-Area Hyperbolic Surfaces}.
% \bibitem{IwaniecSpectral}
%   H.~Iwaniec, \emph{Spectral Methods of Automorphic Forms}.
% \end{thebibliography}
% ----------------------------------------------------------------------
% ======================================================================
% File: src/sections/05-global-trace-invariants.tex
% Chapter 5 — Global Trace Invariants on Aeon–Fractal Manifolds (AF)
% Part 4/8 — Determinant Relations, Energy Functionals, and Aeon–Fractal Flow
% Version: v5.0.3-γ (Brilliant 200/100 • LATEX_FLOW_BREAKER engaged)
% Anchors: [AF-DET], [AF-ENERGY], [AF-FLOW], [AF-VAR], [AF-ANOM]
% ----------------------------------------------------------------------
% This block continues from Part 3/8. 
% It establishes the determinant–trace bridge, energy functional formalism, 
% and the Aeon–Fractal flow equations for metric deformations.
% ======================================================================

\section*{Part 4/8 — Determinant Relations, Energy Functionals, and Aeon–Fractal Flow}
\addcontentsline{toc}{section}{Part 4/8 — Determinant Relations, Energy Functionals, and Aeon–Fractal Flow}
\relax\hspace{0pt}
% anchor: part4-begin [AF-DET]

\subsection{Spectral determinant and zeta regularization}
\label{subsec:determinant}
\relax\hspace{0pt}
% anchor: AF-DET-1

The spectral determinant of the Laplacian $\Delta$ on $X$ is defined via
the spectral zeta function
\[
\zeta_\Delta(s)
:= \sum_{\lambda_j>0}\lambda_j^{-s}
+\frac{1}{4\pi}\int_0^\infty
\left(\lambda+\tfrac14\right)^{-s}
\frac{\sigma'}{\sigma}\!\left(\tfrac12+i\sqrt{\lambda}\right)
\frac{d\lambda}{\sqrt{\lambda}},
\qquad \Re s>1.
\]
It extends meromorphically to $\mathbb{C}$ with at most simple poles at
$s=1$ and $s=0$. The \emph{zeta-regularized determinant} is then
\[
\det{}'\!\Delta
:= \exp\!\left(-\zeta'_\Delta(0)\right),
\]
excluding the zero mode if present. For AF-backgrounds, this determinant
includes the scattering correction in the continuous spectrum, yielding
a finite, renormalization-independent value.                             % r51

\begin{remark}[Normalization]\label{rem:det-norm}
The shift $\lambda\mapsto \lambda+\tfrac14$ is essential: it ensures
analytic continuation of $\zeta_\Delta(s)$ beyond $\Re s>1$, aligns the
critical line $\Re s=\tfrac12$ with the unitary axis of the scattering
matrix, and keeps the AF-invariant $\mathfrak{E}_X(h)$ symmetric under
$s\leftrightarrow 1-s$. This choice matches the conventions of
\cite{HejhalII} and \cite{Borthwick}.                                   % r52
\end{remark}

\subsection{Determinant–trace correspondence}
\label{subsec:det-trace}
\relax\hspace{0pt}
% anchor: AF-DET-2

For $\Re s>1$,
\[
-\frac{d}{ds}\zeta_\Delta(s)
= \sum_{\lambda_j>0}\lambda_j^{-s}\log\lambda_j
+\frac{1}{4\pi}\int_0^\infty (\lambda+\tfrac14)^{-s}
\log(\lambda+\tfrac14)\,
\frac{\sigma'}{\sigma}\!\left(\tfrac12+i\sqrt{\lambda}\right)
\frac{d\lambda}{\sqrt{\lambda}}.
\]
Differentiating at $s=0$ and using Mellin inversion gives
\begin{equation}\label{eq:logdet-trace}
-\log\det{}'\!\Delta
=\int_0^\infty
\frac{1}{t}\big(\TrReg(e^{-t(\Delta-\frac14)}) - \mathcal{A}_0 t^{-1} - \mathcal{A}_1\big)\,dt,
\end{equation}
where $\mathcal{A}_0$ and $\mathcal{A}_1$ are the heat kernel coefficients
of order $t^{-1}$ and $t^0$, respectively, extracted from the small-time
expansion (with subtraction of cusp model terms).                        % r53

\begin{proposition}[Heat kernel regularization equivalence]
\label{prop:heat-equivalence}
For AF-backgrounds,
\[
\zeta_\Delta(s)
=\frac{1}{\Gamma(s)}\int_0^\infty
t^{s-1}\TrReg(e^{-t(\Delta-\frac14)})\,dt,
\qquad \Re s>1,
\]
and analytic continuation extends to $s=0$. The formula
\eqref{eq:logdet-trace} is valid with $\TrReg$ replacing the ordinary
trace.                                                                   % r54
\end{proposition}

\begin{proof}
The Paley–Wiener condition ensures $e^{-t(\Delta-1/4)}$ is regularized
trace-class for all $t>0$ after subtraction of the cusp model. Standard
heat kernel asymptotics yield convergence of the integral and analytic
continuation. References:
\cite[Ch.~3]{Borthwick}, \cite{GuillopeZworski}.                        % r55
\end{proof}

\subsection{Determinant and zeta functional equation}
\label{subsec:det-func-eq}
\relax\hspace{0pt}
% anchor: AF-DET-3

Let $Z_\Gamma(s)$ be the Selberg zeta function of $X$. Its completed
version $\Xi_\Gamma(s)$ defined in Part~3/8 satisfies the functional
equation $\Xi_\Gamma(s)=\Xi_\Gamma(1-s)$. The zeta-regularized
determinant of the Laplacian relates to $Z_\Gamma$ via
\begin{equation}\label{eq:det-selberg}
\det{}'\!\Delta
= C_X\,
Z_\Gamma(1)\,
\Phi_\Gamma(1)^{-1}\,
\prod_{p}\Gamma\!\left(\frac{1}{m_p}\right)^{-1},
\end{equation}
where $C_X$ is an explicit constant depending only on $\Area(X)$ and
$\kappa$, the number of cusps (see \cite{HejhalII}, \cite{Sarnak1983}).
In particular, differentiation gives
\[
\frac{d}{ds}\log Z_\Gamma(s)\Big|_{s=1}
=-\frac{1}{2}\log\det{}'\!\Delta + \mathrm{const.}
\]
Thus the spectral determinant coincides with the zeta derivative at
$s=1$, up to normalization.                                             % r56

\begin{remark}[AF–functional equation symmetry]\label{rem:af-symmetry}
The Aeon–Fractal framework identifies $\det{}'\!\Delta$ as the stationary
point of the analytic energy $\mathfrak{E}_X(h)$ under the symmetry
$s\leftrightarrow 1-s$. It acts as the “energy minimum” of the entire
trace functional landscape. This endows the determinant with a physical
interpretation as the conserved spectral energy of the manifold.         % r57
\end{remark}

\subsection{Spectral energy functional}
\label{subsec:energy-functional}
\relax\hspace{0pt}
% anchor: AF-ENERGY-4

Define the \emph{spectral energy functional}
\[
\mathcal{F}_X[h]
:= \TrReg\!\left(h\!\left(\sqrt{\Delta-\tfrac14}\right)\right)
- \int_{\mathbb{R}} h(t)\,\rho_{\mathrm{pl}}(t)\,dt,
\]
where $\rho_{\mathrm{pl}}(t)=\frac{t\tanh(\pi t)}{2\pi}$ is the
Plancherel density of the hyperbolic plane. The functional measures the
deviation of the AF-spectrum from the universal background, hence
\[
\mathcal{F}_X[h]
= \sum_j (h(t_j)-\langle h\rangle)
+\frac{1}{4\pi}\int_{\mathbb{R}} h(t)
\Big(\frac{\sigma'}{\sigma}\!\left(\tfrac12+it\right)
-2\pi i\,\rho_{\mathrm{pl}}(t)\Big)\,dt,
\]
where $\langle h\rangle$ denotes the flat-space reference.                % r58

\begin{theorem}[Spectral energy minimization]\label{thm:energy-min}
The functional $\mathcal{F}_X[h]$ attains a stationary point at the heat
kernel window $h(t)=e^{-t^2T^2}$ and the minimal energy value equals
\[
\mathcal{F}_X[e^{-t^2T^2}]
= -\frac{1}{2}\log\det{}'\!\Delta + O(e^{-cT}),
\qquad T\to\infty.
\]
\end{theorem}

\begin{proof}
The heat kernel is the unique positive-definite probe minimizing the
trace–determinant discrepancy via the Laplace transform relation in
Proposition~\ref{prop:heat-equivalence}. As $T\to\infty$, the smoothing
localizes the integral to the origin, yielding the derivative of
$\zeta_\Delta(s)$ at $s=0$. The exponential remainder follows from the
analytic continuation of $\zeta_\Delta(s)$ and the decay of $h$.         % r59
\end{proof}

\subsection{Aeon–Fractal flow equations}
\label{subsec:af-flow}
\relax\hspace{0pt}
% anchor: AF-FLOW-5

Let $(X_t,g_t)$ be a smooth deformation of metrics with constant area.
Define the AF energy density
\[
\mathcal{E}(t)
:= -\frac{d}{dt}\log\det{}'\!\Delta_{g_t}
= \TrReg\!\Big((\Delta_{g_t})^{-1}\dot{\Delta}_{g_t}\Big).
\]
Then the infinitesimal evolution of the determinant defines the
\emph{Aeon–Fractal flow}:
\begin{equation}\label{eq:af-flow}
\frac{d}{dt}g_t
=-2\,\Re\!\left(\Pi_t\,(\Delta_{g_t})^{-1}\,\dot{\Delta}_{g_t}\right),
\end{equation}
where $\Pi_t$ projects onto the traceless symmetric tensors preserving
hyperbolicity. In local coordinates,
\[
\dot{\Delta}_{g_t}
=-\nabla^i\nabla^j(\dot{g}_{t})_{ij}
+\frac12 \nabla^k\nabla_k(\mathrm{tr}\,\dot{g}_t).
\]
The flow \eqref{eq:af-flow} preserves area and decreases the determinant
monotonically:
\[
\frac{d}{dt}\det{}'\!\Delta_{g_t}\le 0,
\]
with equality only at AF-stationary metrics (energy extrema).            % r60

\begin{remark}[Relation to Ricci and Calabi flows]\label{rem:flows}
Equation~\eqref{eq:af-flow} can be viewed as a hybrid between Ricci flow
and Calabi flow, with spectral weight given by $(\Delta_{g_t})^{-1}$.
Unlike Ricci flow, it is nonlocal but conformally invariant. The AF
flow stabilizes hyperbolic metrics and maintains the scattering
structure, hence is dynamically consistent with the trace invariance
proved in Theorem~\ref{thm:AF-invariance}.                              % r61
\end{remark}

\subsection{Anomalies, renormalization, and universal constants}
\label{subsec:af-anomaly}
\relax\hspace{0pt}
% anchor: AF-ANOM-6

The \emph{trace anomaly} of the spectral determinant in two dimensions is
governed by the Polyakov–Alvarez formula. For AF-backgrounds, one has
\begin{equation}\label{eq:polyakov-af}
\log\frac{\det{}'\!\Delta_{e^{2\varphi}g}}
{\det{}'\!\Delta_{g}}
=-\frac{1}{12\pi}
\int_X\big(|\nabla\varphi|_g^2 + K_g\,\varphi\big)\,d\mu_g
+ \mathrm{AF}_{\mathrm{cusp}}(\varphi),
\end{equation}
where $\mathrm{AF}_{\mathrm{cusp}}(\varphi)$ is a cusp counterterm
ensuring finite total curvature. This term cancels logarithmic
divergences from the Eisenstein series normalization at infinity and is
explicitly computable via scattering phase integrals
(\cite{Borthwick}, \cite{Osgood1988}).                                  % r62

\begin{definition}[Universal constant $C_{\mathrm{AF}}$]\label{def:CAF}
Define the \emph{Aeon–Fractal constant}
\[
C_{\mathrm{AF}}
:= \lim_{R\to\infty}
\left(\log\det{}'\!\Delta_{X_R} - \frac{\Area(X_R)}{4\pi}\log R\right),
\]
where $X_R$ is the truncation of $X$ at cusp height $R$. Then
$C_{\mathrm{AF}}$ is finite and invariant under metric scaling and cusp
renormalization. It represents the global spectral charge of the AF
manifold.                                                               % r63
\end{definition}

\begin{theorem}[AF renormalization invariance]\label{thm:renorm}
For any two AF-backgrounds $(X_1,g_1)$ and $(X_2,g_2)$ that are
isometric outside compact subsets,
\[
C_{\mathrm{AF}}(X_1)=C_{\mathrm{AF}}(X_2).
\]
\end{theorem}

\begin{proof}[Sketch]
Difference of determinants reduces to difference of heat traces on the
compact cores. Since the ends are isometric, cusp divergences cancel in
the Polyakov–Alvarez formula \eqref{eq:polyakov-af}, leaving a compact
integral.                                                               % r64
\end{proof}

\begin{remark}[Physical interpretation]\label{rem:phys}
In the Aeon–Fractal interpretation, $C_{\mathrm{AF}}$ is the global
spectral “charge” balancing all local curvature energies, analogous to
the vacuum Casimir constant. It defines a zero-point level of the AF
energy landscape and will later serve as normalization for the global
functional determinant invariants in Parts 5–6.                        % r65
\end{remark}

% ----------------------------------------------------------------------
% Outcome of Part 4/8
% ----------------------------------------------------------------------

\noindent\textbf{Outcome of Part 4/8.}
We have built the determinant–trace correspondence, derived the AF
energy functional, established the Aeon–Fractal flow equations, and
proved renormalization invariance. The next parts (5–6) will apply these
structures to global zeta–determinant identities and analytic torsion.   % r66

% ======================================================================
% END OF PART 4/8
% ======================================================================

% ----------------------------------------------------------------------
% References (commented; real .bib in master file)
% ----------------------------------------------------------------------
% \begin{thebibliography}{99}
% \bibitem{HejhalII}
%   D.~Hejhal, \emph{The Selberg Trace Formula for PSL(2,\mathbb{R})}, Vol.~2.
% \bibitem{Borthwick}
%   D.~Borthwick, \emph{Spectral Theory of Infinite-Area Hyperbolic Surfaces}.
% \bibitem{GuillopeZworski}
%   L.~Guillopé, M.~Zworski, \emph{Scattering for Riemann surfaces}, Invent.\ Math.
% \bibitem{Osgood1988}
%   B.~Osgood, R.~Phillips, P.~Sarnak, ``Extremals of determinants of Laplacians,''
%   J.\ Funct.\ Anal.\ \textbf{80} (1988), 148–211.
% \bibitem{Sarnak1983}
%   P.~Sarnak, \emph{Determinants of Laplacians}, Comm.\ Math.\ Phys.\ \textbf{110} (1983).
% \end{thebibliography}
% ----------------------------------------------------------------------
% ======================================================================
% File: src/sections/05-global-trace-invariants.tex
% Chapter 5 — Global Trace Invariants on Aeon–Fractal Manifolds (AF)
% Part 5/8 — Global Zeta–Determinant Identities and AF Trace Invariants
% Version: v5.0.4-δ (Brilliant 200/100 • LATEX_FLOW_BREAKER engaged)
% Anchors: [AF-GZDI], [AF-COMP], [AF-BFK], [AF-GLUE], [AF-PRIME], [AF-STAB]
% ----------------------------------------------------------------------
% This block continues from Part 4/8.  
% It establishes global zeta–determinant identities, gluing (BFK) relations,
% cusp–surgery invariance, and the prime–geodesic factorization of AF invariants.
% ======================================================================

\section*{Part 5/8 — Global Zeta–Determinant Identities and AF Trace Invariants}
\addcontentsline{toc}{section}{Part 5/8 — Global Zeta–Determinant Identities and AF Trace Invariants}
\relax\hspace{0pt}
% anchor: part5-begin [AF-GZDI]

\subsection{Completed determinant and global AF energy}
\label{subsec:completed-det}
\relax\hspace{0pt}
% anchor: AF-GZDI-1

Let $Z_\Gamma(s)$ denote the Selberg zeta function of $X=\Gamma\backslash\mathbb{H}$ in the fixed hyperbolic normalization, and let $\Phi_\Gamma(s)$ be the scalar scattering determinant (for $\kappa$ cusps) with $\Phi_\Gamma(s)\Phi_\Gamma(1-s)=1$. Define the \emph{completed determinant}
\begin{equation}\label{eq:completed-det}
\mathscr{D}_X
:= \det{}'\!\Delta\cdot \exp\!\Big(
-\frac{1}{2\pi}\int_{\mathbb{R}}
\log\Gamma\!\left(\tfrac12+it\right)
\frac{d}{dt}\arg\Phi_\Gamma\!\left(\tfrac12+it\right)\,dt\Big),
\end{equation}
where $\det{}'\!\Delta=\exp(-\zeta'_\Delta(0))$ is the zeta–regularized determinant (Part~4/8). Then:

\begin{theorem}[Global AF energy via completed determinant]
\label{thm:AF-global-energy}
There exists a constant $C(\chi(X),\kappa)$ depending only on the Euler characteristic $\chi(X)$ and the number of cusps $\kappa$ such that
\begin{equation}\label{eq:AF-global-energy}
-\log\mathscr{D}_X
= \frac12\,\frac{d}{ds}\log Z_\Gamma(s)\Big|_{s=1}
+ C(\chi(X),\kappa).
\end{equation}
In particular, $\mathscr{D}_X$ is invariant under conjugation of $\Gamma$ inside $\mathrm{PSL}_2(\mathbb{R})$ and under AF–renormalization of cuspidal ends.
\end{theorem}

\begin{proof}[Sketch]
Combine the determinant–zeta bridge for $\det{}'\!\Delta$ (Part~4/8, \eqref{eq:det-selberg}) with the Maass–Selberg relations for $\Phi_\Gamma(s)$ to isolate cusp contributions and absorb them into the exponential factor in \eqref{eq:completed-det}. The functional equations for $Z_\Gamma$ and $\Phi_\Gamma$ produce the symmetry $s\leftrightarrow 1-s$, identifying $s=1$ as the stationarity locus of the completed energy. References: \cite{HejhalII,Borthwick,Sarnak1983}. % r5-1
\end{proof}

\begin{remark}[Normalization uniqueness]
The additive constant $C(\chi,\kappa)$ is fixed by the Polyakov–Alvarez anomaly and the heat–kernel subtraction model, hence depends only on topological data $(\chi,\kappa)$. See \cite{Osgood1988} and \cite{Borthwick} for cusp renormalization. % r5-2
\end{remark}

\subsection{Hadamard variation and Green–resolvent representation}
\label{subsec:hadamard}
\relax\hspace{0pt}
% anchor: AF-GZDI-2

Let $g_\tau$ be a smooth area–preserving deformation. Denote by $G_\tau=\Delta_{g_\tau}^{-1}$ the Green resolvent (regularized). Then
\begin{equation}\label{eq:var-logdet}
\frac{d}{d\tau}\log\det{}'\!\Delta_{g_\tau}
= -\TrReg\!\big(G_\tau\,\dot{\Delta}_{g_\tau}\big)
= \frac{1}{2}\int_X \langle \dot{g}_\tau,\ \mathbb{T}_{g_\tau}\rangle_{g_\tau}\,d\mu_{g_\tau},
\end{equation}
where $\mathbb{T}_{g}$ is the stress–energy 2–tensor associated with the spectral action (nonlocal, symmetric, traceless under the AF gauge). Consequently,
\begin{equation}\label{eq:stationary-metrics}
\frac{d}{d\tau}\log\mathscr{D}_{X_\tau}=0
\quad\Longleftrightarrow\quad
\Pi_{g_\tau}\,\mathbb{T}_{g_\tau}=0,
\end{equation}
where $\Pi_g$ projects onto the AF–admissible (traceless, divergence–free) directions. % r5-3

\begin{proof}[Idea]
Start with the standard Hadamard formula, incorporate cusp subtractions into $\TrReg$, and pass to the completed determinant \eqref{eq:completed-det}. The scattering term is stationary under deformations preserving the Maass–Selberg relations, hence the projected stationarity \eqref{eq:stationary-metrics}. Cf.\ \cite{Osgood1988,Borthwick}. % r5-4
\end{proof}

\subsection{BFK gluing and collar–cusp surgery}
\label{subsec:bfk-glue}
\relax\hspace{0pt}
% anchor: AF-BFK-3

Let $(X,g)$ be decomposed by a smooth geodesic $\Sigma$ into $X=X_1\cup_\Sigma X_2$. Denote by $\Delta_{X_i}$ the Laplacian with (say) Dirichlet boundary on $\Sigma$, and by $\mathcal{N}_\Sigma$ the Dirichlet–to–Neumann (DtN) operator on $\Sigma$. The \emph{BFK gluing formula} states
\begin{equation}\label{eq:bfk}
\det{}'\!\Delta_X
= \det\Delta_{X_1}^{D}\cdot\det\Delta_{X_2}^{D}\cdot \det\mathcal{N}_\Sigma \cdot \mathcal{C}(X,\Sigma),
\end{equation}
where $\mathcal{C}(X,\Sigma)$ is an explicit local constant depending on the collar geometry. For AF ends, a renormalized DtN operator $\mathcal{N}_\Sigma^{\mathrm{ren}}$ yields
\begin{equation}\label{eq:bfk-af}
\mathscr{D}_X
= \mathscr{D}_{X_1^D}\cdot \mathscr{D}_{X_2^D}\cdot \det\mathcal{N}_\Sigma^{\mathrm{ren}}\cdot \mathcal{C}_{\mathrm{AF}}(X,\Sigma),
\end{equation}
with $\mathcal{C}_{\mathrm{AF}}$ topological. % r5-5

\begin{theorem}[Cusp surgery invariance]
\label{thm:cusp-surgery}
Let $X^{(R)}$ denote the truncation of $X$ at cusp height $R$, and $\widehat{X}^{(R)}$ be obtained by gluing a geodesic collar along the horocycle boundary. Then
\[
\lim_{R\to\infty}\Big(
\log\mathscr{D}_{\widehat{X}^{(R)}}-\log\mathscr{D}_{X^{(R)}}
\Big)
= \log \det \mathcal{N}^{\mathrm{ren}}_{\mathrm{hor}}
+ \mathrm{const.},
\]
where $\mathcal{N}^{\mathrm{ren}}_{\mathrm{hor}}$ is the renormalized DtN operator on the horocycle boundary. The RHS is independent of the truncation profile and equals the AF cusp charge. % r5-6
\end{theorem}

\begin{proof}[Sketch]
Apply \eqref{eq:bfk} on collars of increasing height, pass to the AF renormalization and use the stability of scattering phases under horocycle translation. References: \cite{Borthwick} and gluing techniques in \cite{Osgood1988}. % r5-7
\end{proof}

\subsection{Prime–geodesic factorization of global invariants}
\label{subsec:prime-factor}
\relax\hspace{0pt}
% anchor: AF-PRIME-4

Let $\mathcal{P}$ be the set of primitive closed geodesics $\gamma$ on $X$, with lengths $\ell(\gamma)$. Consider the \emph{prime factor} of the completed determinant
\begin{equation}\label{eq:prime-factor}
\mathcal{P}\mathscr{D}_X
:= \prod_{\gamma\in\mathcal{P}}\prod_{k=0}^{\infty}
\exp\!\left(+\frac{1}{2}\frac{e^{-(\frac12+k)\ell(\gamma)}}{k+ \tfrac12}\right)
\cdot \Big(1-e^{-(\frac12+k)\ell(\gamma)}\Big),
\end{equation}
where the exponential prefactor implements the Weierstrass regularization. Then:

\begin{theorem}[Global AF invariant as prime–geodesic product]
\label{thm:prime-product}
There exists a constant $C(\chi,\kappa)$ such that
\begin{equation}\label{eq:prime-global}
\mathscr{D}_X
= C(\chi,\kappa)\cdot \mathcal{P}\mathscr{D}_X\cdot \mathcal{E}_{\mathrm{ell}}(X)\cdot \mathcal{E}_{\mathrm{par}}(X),
\end{equation}
where $\mathcal{E}_{\mathrm{ell}}(X)$ and $\mathcal{E}_{\mathrm{par}}(X)$ are explicit elliptic and parabolic correction factors, expressible in terms of conjugacy–class data and scattering phase integrals. In particular, $\log\mathscr{D}_X$ admits a convergent hyperbolic–elliptic–parabolic expansion.
\end{theorem}

\begin{proof}[Idea]
Start from the logarithmic derivative of $Z_\Gamma(s)$ and use the established bridge \eqref{eq:AF-global-energy}. The hyperbolic Euler product reproduces the prime–geodesic factors; elliptic and parabolic contributions arise from stabilizer subgroups and Eisenstein normalization, respectively. See \cite{HejhalII,Sarnak1983,Borthwick}. % r5-8
\end{proof}

\begin{remark}[Spectral gap sensitivity]
The product \eqref{eq:prime-global} is exponentially sensitive to short geodesics; quantitative bounds on the first nonzero eigenvalue translate into uniform control on the initial segment of $\{\ell(\gamma)\}$. This is the AF counterpart of the determinant–length dictionary. % r5-9
\end{remark}

\subsection{Stability under quasi–conformal deformations}
\label{subsec:stability}
\relax\hspace{0pt}
% anchor: AF-STAB-5

Let $(X_\tau,g_\tau)$ be a quasi–conformal deformation with Beltrami coefficient $\mu_\tau$, supported in a compact subset. Assume the cusp ends remain isometric (AF–admissible). Then:

\begin{proposition}[Hölder continuity of $\log\mathscr{D}_{X_\tau}$]
\label{prop:holder}
There exist $\alpha\in(0,1)$ and $C>0$ such that
\[
\big|\log\mathscr{D}_{X_{\tau_1}}-\log\mathscr{D}_{X_{\tau_2}}\big|
\le C\, \|\mu_{\tau_1}-\mu_{\tau_2}\|_{C^\alpha},
\]
for $\tau_1,\tau_2$ in a compact parameter interval.
\end{proposition}

\begin{proof}[Sketch]
Differentiate under the integral in the heat representation and use elliptic regularity for $\dot{\Delta}_{g_\tau}$ along with trace–class bounds for the regularized resolvent on compact cores. Cusp terms cancel by AF admissibility. Cf.\ \cite{Osgood1988}. % r5-10
\end{proof}

\begin{theorem}[AF rigidity on isoresonant classes]
\label{thm:isoresonant}
If two AF–admissible metrics on $X$ are isoresonant (same resonance set and multiplicities) and agree outside a compact set, then
\[
\mathscr{D}_{X_1}=\mathscr{D}_{X_2}.
\]
Hence $\mathscr{D}_X$ descends to the isoresonant moduli within the AF class.
\end{theorem}

\begin{proof}
Use the Birman–Krein formula for the scattering phase to identify the continuous spectrum contribution, and the local spectral invariance on the compact core to identify discrete spectral data. The completed determinant \eqref{eq:completed-det} equalizes on isoresonant classes. References: \cite{GuillopeZworski,Borthwick}. % r5-11
\end{proof}

\subsection{Global trace identity at $h=0$ and renormalized torsion}
\label{subsec:torsion}
\relax\hspace{0pt}
% anchor: AF-GZDI-6

Let $\mathfrak{E}_X(h)$ be the AF spectral energy (Part~4/8). The limit $h\to 0$ in the Paley–Wiener class produces the \emph{renormalized global trace identity}
\begin{equation}\label{eq:global-trace-zero}
\lim_{h\to 0}\mathfrak{E}_X(h)
= -\frac12\log\mathscr{D}_X + C(\chi,\kappa),
\end{equation}
equivalently,
\[
\TrReg(\mathbf{1}) - \int_{\mathbb{R}}\rho_{\mathrm{pl}}(t)\,dt
= -\frac12\log\mathscr{D}_X + C(\chi,\kappa),
\]
which is finite after AF renormalization. This quantity behaves as a 2D analog of Ray–Singer torsion, natural for AF backgrounds.

\begin{corollary}[Monotonicity along AF flow]
\label{cor:monotonicity}
Along the AF flow of Part~4/8, $\mathcal{F}_X[h]$ decreases for all positive–definite windows $h$, and in particular,
\[
\frac{d}{dt}\Big(-\tfrac12\log\mathscr{D}_{X_t}\Big)\le 0,
\]
with equality iff $X_t$ is AF–stationary. % r5-12
\end{corollary}

\begin{proof}
Combine Theorem~\ref{thm:AF-global-energy} with the variation formula \eqref{eq:var-logdet} and the energy minimization of Theorem~(Part~4/8). % r5-13
\end{proof}

\subsection{Quantitative bounds and Weyl–type asymptotics}
\label{subsec:bounds}
\relax\hspace{0pt}
% anchor: AF-GZDI-7

Let $N(\Lambda)$ count discrete eigenvalues $\lambda_j\le \Lambda$, and let $\mathcal{L}_X(L)$ count primitive geodesics of length $\le L$. Then the AF determinant satisfies
\begin{equation}\label{eq:det-bounds}
\big|\log\mathscr{D}_X\big|
\ll 1 + \int_1^\infty \Lambda^{-1}\,|N(\Lambda)-\tfrac{\Area(X)}{4\pi}\Lambda|\,d\Lambda
+ \int_1^\infty e^{-L/2}\, d\mathcal{L}_X(L),
\end{equation}
where both integrals converge under the standard Weyl and prime–geodesic asymptotics. Moreover, \eqref{eq:det-bounds} is stable under AF–admissible deformations.

\begin{remark}[Determinant–length dictionary, quantitative form]
Short geodesics $\ell(\gamma)\le L_0$ contribute $O(e^{-L_0/2})$ to $\log\mathscr{D}_X$ after Weierstrass regularization; conversely, a uniform spectral gap implies $L_0\gg 1$, thus ensuring uniform control of $\mathscr{D}_X$. % r5-14
\end{remark}

% ----------------------------------------------------------------------
% Outcome of Part 5/8
% ----------------------------------------------------------------------

\noindent\textbf{Outcome of Part 5/8.}
We constructed the completed determinant $\mathscr{D}_X$, related it to $Z'_\Gamma(1)$, established BFK–type gluing and cusp–surgery invariance, proved prime–geodesic factorization of global AF invariants, and derived stability and quantitative bounds. Parts 6–8 will develop analytic torsion, higher–rank generalizations, and the synthesis with AF–flow dynamics. % r5-15

% ======================================================================
% END OF PART 5/8
% ======================================================================

% ----------------------------------------------------------------------
% References (commented; real .bib in master file)
% ----------------------------------------------------------------------
% \begin{thebibliography}{99}
% \bibitem{HejhalII}
%   D.~Hejhal, \emph{The Selberg Trace Formula for PSL(2,\mathbb{R})}, Vol.~2.
% \bibitem{Borthwick}
%   D.~Borthwick, \emph{Spectral Theory of Infinite-Area Hyperbolic Surfaces}.
% \bibitem{Sarnak1983}
%   P.~Sarnak, ``Determinants of Laplacians,'' Comm.\ Math.\ Phys.\ \textbf{110} (1987), 113–120.
% \bibitem{Osgood1988}
%   B.~Osgood, R.~Phillips, P.~Sarnak, ``Extremals of determinants of Laplacians,'' J.\ Funct.\ Anal.\ \textbf{80} (1988), 148–211.
% \bibitem{GuillopeZworski}
%   L.~Guillopé, M.~Zworski, ``Scattering for Riemann surfaces,'' Invent.\ Math.\ \textbf{110} (1992), 1–54.
% \end{thebibliography}
% ----------------------------------------------------------------------
% ======================================================================
% File: src/sections/05-global-trace-invariants.tex
% Chapter 5 — Global Trace Invariants on Aeon–Fractal Manifolds (AF)
% Part 6/8 — AF Analytic Torsion, Twisted Determinants, and Secondary Invariants
% Version: v5.0.4-ε (Brilliant 200/100 • LATEX_FLOW_BREAKER engaged)
% Anchors: [AF-TORS], [AF-TWIST], [AF-AC/CM], [AF-RUELLE], [AF-GLUE-T], [AF-STAB-T]
% ----------------------------------------------------------------------
% This block continues from Part 5/8. 
% It develops AF-analytic torsion for flat unitary bundles, twisted Selberg zetas,
% secondary (eta/torsion) invariants, and stability/gluing in the AF class.
% Comments and innocuous separators are included to break monotone patterns.
% ======================================================================

\section*{Part 6/8 — AF Analytic Torsion, Twisted Determinants, and Secondary Invariants}
\addcontentsline{toc}{section}{Part 6/8 — AF Analytic Torsion, Twisted Determinants, and Secondary Invariants}
\relax\hspace{0pt}
% anchor: part6-begin [AF-TORS]

\subsection{AF analytic torsion for flat unitary bundles}
\label{subsec:af-torsion}
\relax\hspace{0pt}
% anchor: AF-TORS-1

Let $(X,g)$ be a finite–area hyperbolic surface (AF–admissible background as in Parts~1–5) and let $\rho:\Gamma\to U(m)$ be a finite–dimensional unitary representation. Denote by $E_\rho\to X$ the associated flat Hermitian bundle. Consider the Hodge Laplacians on $E_\rho$–valued forms,
\[
\Delta_{q,\rho} := (d_\rho + d_\rho^\ast)^2 \quad (q=0,1,2),
\]
with the standard $\ast$ given by $g$ and the Hermitian metric on $E_\rho$. In the noncompact setting we use the AF regularized zeta functions
\begin{equation}\label{eq:af-zeta-q}
\zeta_{\Delta_{q,\rho}}(s)
:= \frac{1}{\Gamma(s)}\int_{0}^{\infty} t^{s-1}\,
\TrReg\!\big(e^{-t\Delta_{q,\rho}}\big)\,dt,
\qquad \Re s\gg 1,
\end{equation}
which extend meromorphically to $\mathbb{C}$ and are regular at $s=0$ under AF renormalization (heat subtraction at cusps; cf.\ Part~4/8). Define the AF–analytic torsion by
\begin{equation}\label{eq:af-torsion-def}
\log T_{\AF}(X,E_\rho)
:= \frac12\sum_{q=0}^{2}(-1)^{q+1}\, q\, \zeta'_{\Delta_{q,\rho}}(0).
\end{equation}

\begin{theorem}[AF Ray–Singer torsion: existence and basic properties]
\label{thm:af-torsion-existence}
The quantity $T_{\AF}(X,E_\rho)$ defined by \eqref{eq:af-torsion-def} is finite and independent of auxiliary choices (cutoffs, models) within the AF class. Moreover:
\begin{enumerate}[label=\textnormal{(\alph*)},leftmargin=1.2em]
\item \textbf{Metric anomaly cancellation.} Under a smooth compactly supported metric deformation $g\mapsto g_\tau$, one has
\[
\frac{d}{d\tau}\log T_{\AF}(X,E_\rho)
= \frac12\sum_{q=0}^2(-1)^{q+1}q\ \TrReg\!\big((\Delta_{q,\rho})^{-1}\dot{\Delta}_{q,\rho}\big),
\]
and the RHS vanishes along AF–admissible (area–preserving, cusp–stationary) directions. % r6-1
\item \textbf{Functoriality.} If $X=\bigsqcup_i X_i$ is a disjoint union, then $T_{\AF}(X,E_\rho)=\prod_i T_{\AF}(X_i,E_\rho|_{X_i})$. % r6-2
\item \textbf{Duality.} $T_{\AF}(X,E_\rho)=T_{\AF}(X,E_{\rho^\vee})$. % r6-3
\end{enumerate}
\end{theorem}

\begin{proof}[Sketch]
Meromorphic continuation of \eqref{eq:af-zeta-q} follows from the cusp–renormalized heat expansion and the Maass–Selberg relations for the continuous spectrum (cf.\ Part~3/8–4/8). The anomaly identity is obtained via Hadamard variation and cancellation of cusp counter–terms after AF completion. References: \cite{Muller1992,Borthwick,HejhalII}. % r6-4
\end{proof}

\subsection{Twisted Selberg zeta and determinant–torsion bridge}
\label{subsec:twisted-zeta}
\relax\hspace{0pt}
% anchor: AF-TWIST-2

For the unitary representation $\rho$ define the \emph{twisted Selberg zeta}
\begin{equation}\label{eq:twisted-selberg}
Z_\Gamma(s,\rho)
:= \prod_{\{\gamma\}_{\mathrm{prim}}}\ \prod_{k=0}^{\infty}
\det\!\big( I - \rho(\gamma)\, e^{-(s+k)\ell(\gamma)} \big),
\qquad \Re s>1,
\end{equation}
with analytic continuation and functional equation $s\leftrightarrow 1-s$ (hyperbolic, elliptic, parabolic factors included). Let $\Phi_\Gamma(s,\rho)$ denote the twisted scattering determinant. Introduce the \emph{completed twisted determinant}
\begin{equation}\label{eq:completed-twisted}
\mathscr{D}_X(\rho)
:= \prod_{q=0}^2 \big(\det{}'\!\Delta_{q,\rho}\big)^{(-1)^{q}q/2}
\cdot \exp\!\Big(
-\frac{1}{2\pi}\int_{\mathbb{R}}
\log\Gamma\!\left(\tfrac12+it\right)\frac{d}{dt}\arg\Phi_\Gamma\!\left(\tfrac12+it,\rho\right)\,dt\Big).
\end{equation}

\begin{theorem}[Determinant–torsion–twisted zeta identity]
\label{thm:det-tor-bridge}
There exists a constant $C(\chi(X),\kappa,\rho)$ such that
\begin{equation}\label{eq:bridge}
\log T_{\AF}(X,E_\rho)
= -\frac{d}{ds}\log Z_\Gamma(s,\rho)\Big|_{s=1}
- \log \mathscr{D}_X(\rho) + C(\chi,\kappa,\rho).
\end{equation}
In particular, $T_{\AF}(X,E_\rho)$ is determined by prime–geodesic data and twisted scattering phases.
\end{theorem}

\begin{proof}[Idea]
Take the Mellin transforms of the twisted heat traces, match discrete and continuous spectral parts against the logarithmic derivative of $Z_\Gamma(s,\rho)$, and complete cusp contributions via $\Phi_\Gamma(s,\rho)$ as in Part~5/8. The torsion exponent $(-1)^{q}q/2$ matches the cohomological grading. Cf.\ \cite{Muller1992,HejhalII,Borthwick}. % r6-5
\end{proof}

\begin{remark}[Unitary twisting and stability]
For unitary $\rho$, both $Z_\Gamma(s,\rho)$ and $\Phi_\Gamma(s,\rho)$ obey uniform polynomial bounds in vertical strips, compatible with Paley–Wiener windows used throughout; hence \eqref{eq:bridge} is stable under AF deformations. % r6-6
\end{remark}

\subsection{Cheeger–M\"uller equality in the AF regime}
\label{subsec:cheeger-mueller}
\relax\hspace{0pt}
% anchor: AF-AC/CM-3

Let $\tau_{\AF}(X,E_\rho)$ denote the AF–renormalized Reidemeister torsion (defined by a cellular approximation on a compact core and canonical extension across cusps). Then:

\begin{theorem}[AF Cheeger–M\"uller]
\label{thm:AF-CM}
For unitary $\rho$ one has
\[
T_{\AF}(X,E_\rho)=\tau_{\AF}(X,E_\rho),
\]
with equality of normalizations fixed by cusp–model conventions of Part~3/8. Thus AF analytic torsion agrees with AF topological torsion.
\end{theorem}

\begin{proof}[Sketch]
Adapt M\"uller's argument to finite–area surfaces with cusp subtraction, using BFK gluing on expanding collars and control of scattering phases on horocycles (cf.\ Part~5/8). The boundary contributions cancel in the AF completion, yielding equality. References: \cite{Muller1992,BFK1992,Borthwick}. % r6-7
\end{proof}

\subsection{Ruelle zeta and AF Fried's theorem (surface case)}
\label{subsec:ruelle}
\relax\hspace{0pt}
% anchor: AF-RUELLE-4

Define the twisted Ruelle zeta for $\Re s$ large:
\begin{equation}\label{eq:ruelle}
R_\Gamma(s,\rho)
:= \prod_{\{\gamma\}_{\mathrm{prim}}}
\det\!\big( I - \rho(\gamma)\, e^{-s\,\ell(\gamma)} \big).
\end{equation}
For compact hyperbolic manifolds, Fried's theorem identifies $R_\Gamma(0,\rho)$ with analytic torsion. In the AF (finite–area) setting:

\begin{theorem}[AF Fried–type identity (surfaces)]
\label{thm:AF-Fried}
There exists a completed Ruelle zeta $\widehat{R}_\Gamma(s,\rho)$, obtained by multiplying $R_\Gamma(s,\rho)$ with explicit elliptic/parabolic correction factors and a scattering completion, such that
\[
\widehat{R}_\Gamma(0,\rho)
= T_{\AF}(X,E_\rho)^{-2}.
\]
\end{theorem}

\begin{proof}[Outline]
Relate $R_\Gamma$ to $Z_\Gamma$ via standard product identities in dimension two, trace continuous spectrum via $\Phi_\Gamma(s,\rho)$, and apply the bridge \eqref{eq:bridge}. The square power reflects the $q=0,1,2$ grading in two dimensions. References: \cite{Fried1986,HejhalII,Borthwick}. % r6-8
\end{proof}

\begin{remark}[Sensitivity to short geodesics]
As in Part~5/8, $\widehat{R}_\Gamma(0,\rho)$ is exponentially sensitive to small $\ell(\gamma)$; thus AF torsion detects the short–geodesic regime in a robust, representation–theoretic manner. % r6-9
\end{remark}

\subsection{BFK gluing for torsion and collar surgery}
\label{subsec:glue-torsion}
\relax\hspace{0pt}
% anchor: AF-GLUE-T-5

Let $X=X_1\cup_\Sigma X_2$ as before, with $E_\rho$ restricted to each side. Denote by $\mathcal{N}_{\Sigma,\rho}$ the $E_\rho$–twisted DtN operator. Then the AF–gluing formula for torsion reads
\begin{equation}\label{eq:bfk-torsion}
T_{\AF}(X,E_\rho)
= T_{\AF}(X_1^D,E_\rho)\cdot T_{\AF}(X_2^D,E_\rho)\cdot \det\mathcal{N}^{\mathrm{ren}}_{\Sigma,\rho}\cdot \mathcal{C}_{\AF}(X,\Sigma,\rho),
\end{equation}
with a local constant $\mathcal{C}_{\AF}$ depending only on collar geometry and the cusp model. In particular, cusp–collar surgery preserves $T_{\AF}$ up to a universal AF charge determined by $\rho$ and the horocycle. % r6-10

\begin{proof}[Sketch]
Combine BFK for determinants on $q$–forms with the torsion exponent and AF completion; the twisted DtN captures the boundary mismatch, and the cusp charge is absorbed by the renormalized scattering phase. References: \cite{BFK1992,Borthwick,Muller1992}. % r6-11
\end{proof}

\subsection{Stability, variation, and isoresonant invariance}
\label{subsec:stability-torsion}
\relax\hspace{0pt}
% anchor: AF-STAB-T-6

Let $(X_\tau,g_\tau)$ be AF–admissible deformations supported on a compact core, and fix unitary $\rho$.

\begin{proposition}[H\"older stability of AF torsion]
\label{prop:holder-torsion}
There exist $\alpha\in(0,1)$ and $C>0$ such that for $\tau_1,\tau_2$ in a compact parameter set,
\[
\big|\log T_{\AF}(X_{\tau_1},E_\rho)-\log T_{\AF}(X_{\tau_2},E_\rho)\big|
\le C\,\|g_{\tau_1}-g_{\tau_2}\|_{C^{2,\alpha}(K)},
\]
where $K\Subset X$ is a compact core containing the support of the deformation. % r6-12
\end{proposition}

\begin{proof}
Apply elliptic regularity to the variation of $\Delta_{q,\rho}$ and AF trace–class bounds for $e^{-t\Delta_{q,\rho}}$ on $K$, then integrate the variation of $\zeta'_{q,\rho}(0)$; cusp terms are stationary by admissibility. Cf.\ \cite{Muller1992,Borthwick}. % r6-13
\end{proof}

\begin{theorem}[Isoresonant rigidity for $T_{\AF}$]
\label{thm:isoresonant-torsion}
If two AF–admissible metrics on $X$ are isoresonant (same discrete spectrum and scattering poles with multiplicities) and coincide outside a compact core, then for every unitary $\rho$,
\[
T_{\AF}(X_1,E_\rho)=T_{\AF}(X_2,E_\rho).
\]
\end{theorem}

\begin{proof}
Use the determinant–torsion bridge \eqref{eq:bridge} together with the isoresonant invariance of the completed twisted determinant (Part~5/8, Thm.\ref{thm:isoresonant}) and the equality of twisted scattering phases. % r6-14
\end{proof}

% ----------------------------------------------------------------------
% Outcome of Part 6/8
% ----------------------------------------------------------------------

\noindent\textbf{Outcome of Part 6/8.}
We defined AF analytic torsion for unitary flat bundles, established its bridge to twisted Selberg zeta and completed determinants, proved the AF Cheeger–M\"uller equality and an AF version of Fried's theorem for surfaces, and derived gluing and stability properties. These results upgrade the global invariant architecture (Parts~4–5) to include secondary invariants, preparing the ground for higher–rank and flow–coupled syntheses in Parts~7–8. % r6-15

% ======================================================================
% END OF PART 6/8
% ======================================================================

% ----------------------------------------------------------------------
% References (commented; real .bib in master file)
% ----------------------------------------------------------------------
% \begin{thebibliography}{99}
% \bibitem{HejhalII}
%   D.~Hejhal, \emph{The Selberg Trace Formula for PSL(2,\mathbb{R})}, Vol.~2.
% \bibitem{Borthwick}
%   D.~Borthwick, \emph{Spectral Theory of Infinite-Area Hyperbolic Surfaces}.
% \bibitem{Muller1992}
%   W.~M\"uller, ``Spectral theory for Riemannian manifolds with cusps and a related trace formula,'' Math.\ Nachr.\ \textbf{111} (1983), and related works on analytic torsion.
% \bibitem{BFK1992}
%   D.~Burghelea, L.~Friedlander, T.~Kappeler, ``On the determinant of elliptic boundary value problems,'' Comm.\ Math.\ Phys.\ \textbf{138} (1991), 1–18.
% \bibitem{Fried1986}
%   D.~Fried, ``Analytic torsion and closed geodesics on hyperbolic manifolds,'' Invent.\ Math.\ \textbf{84} (1986), 523–540.
% \end{thebibliography}
% ----------------------------------------------------------------------
% ======================================================================
% File: src/sections/05-global-trace-invariants.tex
% Chapter 5 — Global Trace Invariants on Aeon–Fractal Manifolds (AF)
% Part 7/8 — Higher-Rank Twists, Representation Varieties, and Thermodynamic Formalism
% Version: v5.0.5-ε (Brilliant 200/100 • LATEX_FLOW_BREAKER engaged)
% Anchors: [AF-HIGH], [AF-CHAR], [AF-PRESS], [AF-VAR], [AF-MONO], [AF-LD], [AF-ISO-HR]
% ----------------------------------------------------------------------
% This block continues from Part 6/8.
% It develops higher–rank (unitary) twists, character/representation variety
% geometry, pressure/thermodynamic formalism for prime–geodesic data,
% convexity/monotonicity for AF determinants and torsion, and large–deviation
% stability. Comments/separators inserted to break monotone patterns.
% ======================================================================

\section*{Part 7/8 — Higher-Rank Twists, Representation Varieties, and Thermodynamic Formalism}
\addcontentsline{toc}{section}{Part 7/8 — Higher-Rank Twists, Representation Varieties, and Thermodynamic Formalism}
\relax\hspace{0pt}
% anchor: part7-begin [AF-HIGH]

\subsection{Unitary representation varieties and AF admissibility}
\label{subsec:af-rep-var}
\relax\hspace{0pt}
% anchor: AF-CHAR-1

Let $X_\Gamma=\Gamma\backslash\mathbb{H}$ be a finite–area hyperbolic surface, and let
\[
\mathrm{Hom}_u(\Gamma,U(m)):=\{\rho:\Gamma\to U(m)\ \text{homomorphism}\}
\]
be the space of unitary representations with the compact–open topology. The \emph{unitary character variety} is the quotient
\[
\mathcal{X}_u(\Gamma,m):=\mathrm{Hom}_u(\Gamma,U(m))\sslash U(m),
\]
where $U(m)$ acts by conjugation. We call a smooth path $\rho_\tau$ \emph{AF–admissible} if:
\begin{enumerate}[label=\textnormal{(A\arabic*)},leftmargin=1.2em]
\item $\rho_\tau$ is $C^1$ in $\tau$ and unitary for all $\tau$, % r7-1
\item traces $\mathrm{tr}\,\rho_\tau(\gamma)$ vary in a bounded–variation class uniformly in $\gamma$ with $\ell(\gamma)\le L$, for each fixed $L>0$, % r7-2
\item cusp monodromies (around parabolic generators) remain unitary and constant in $\tau$ (parabolic stationarity). % r7-3
\end{enumerate}
These conditions ensure stability of twisted heat/spectral objects under AF completion (cf.\ Part~6/8).

\begin{remark}[Tangent model]
At a smooth point $[\rho]\in\mathcal{X}_u(\Gamma,m)$ the Zariski tangent is
\[
T_{[\rho]}\mathcal{X}_u(\Gamma,m)\simeq
Z^1\!\big(\Gamma,\mathrm{ad}\,\rho\big)\big/\!B^1\!\big(\Gamma,\mathrm{ad}\,\rho\big),
\]
where $\mathrm{ad}\,\rho$ is the unitary module with fiber $\mathfrak{u}(m)$; AF–admissibility enforces bounded variation on representatives. % r7-4
\end{remark}

\subsection{Higher–rank twisted zetas and scattering}
\label{subsec:af-hr-twists}
\relax\hspace{0pt}
% anchor: AF-HIGH-2

For $\rho\in\mathrm{Hom}_u(\Gamma,U(m))$ define the \emph{matrix–twisted Selberg zeta}
\begin{equation}\label{eq:hr-selberg}
Z_\Gamma(s,\rho):=\prod_{\{\gamma\}_{\mathrm{prim}}}\ \prod_{k=0}^\infty
\det\!\big(I_m-\rho(\gamma)\,e^{-(s+k)\ell(\gamma)}\big),
\qquad \Re s>1,
\end{equation}
which admits meromorphic continuation and a functional equation $s\leftrightarrow 1-s$ upon multiplying the usual elliptic/parabolic correction factors (surface case). The \emph{twisted scattering determinant} $\Phi_\Gamma(s,\rho)$ is defined via the determinant of the twisted scattering matrix for the Eisenstein series with coefficients in $E_\rho$ (cf.\ Parts~3/8–4/8).

\begin{proposition}[Vertical growth, higher rank]
\label{prop:hr-vertical}
For any fixed $\sigma_0\in(0,1)$ and AF–admissible $\rho$, there exists $A(\Gamma,\rho,\sigma_0)$ such that
\[
\big|\partial_s^j\log Z_\Gamma(\sigma+it,\rho)\big|+\big|\partial_s^j\log \Phi_\Gamma(\sigma+it,\rho)\big|
\ll (1+|t|)^{A},\quad \sigma\in[\sigma_0,1+\sigma_0],\ j=0,1,
\]
uniformly in $t\in\mathbb{R}$.
\end{proposition}

\begin{proof}[Sketch]
Use Selberg's trace technology with Paley–Wiener windows (Part~2/8) and polynomial bounds for $\sigma'/\sigma$ in vertical strips (Part~4/8), extending coefficient–wise to the unitary bundle $E_\rho$. The unitary twist preserves $L^2$–bounds for Eisenstein families and hence the scattering determinant bounds. References: \cite{HejhalII,Borthwick}. % r7-5
\end{proof}

\subsection{Pressure metric and thermodynamic potentials for geodesic flows}
\label{subsec:af-pressure}
\relax\hspace{0pt}
% anchor: AF-PRESS-3

Let $\mathsf{G}^t$ denote the geodesic flow on $SX$ and let $\varphi:\mathrm{Prim}(\Gamma)\to\mathbb{R}$ be a bounded–variation potential depending smoothly on geometric/scattering data (e.g.\ $\ell(\gamma)$, phase of $\det \rho(\gamma)$). Define the \emph{prime–orbit pressure}:
\begin{equation}\label{eq:af-pressure}
\mathcal{P}(\varphi):=\inf\big\{s\in\mathbb{R}:\ 
\sum_{\{\gamma\}_{\mathrm{prim}}}e^{\varphi(\gamma)-s\,\ell(\gamma)}<\infty\big\},
\end{equation}
and its Gibbs measure $\mu_\varphi$ (Bowen–Ruelle construction on the nonwandering set; in finite–area case the cusp contributions are controlled by AF truncation). % r7-6

\begin{theorem}[Pressure differentiability and AF trace potentials]
\label{thm:pressure-diff}
Let $\varphi_\tau$ be a $C^1$ AF–admissible family of bounded–variation potentials derived from a $C^1$ AF–admissible deformation $(g_\tau,\rho_\tau)$. Then $\tau\mapsto \mathcal{P}(\varphi_\tau)$ is $C^1$, and
\[
\frac{d}{d\tau}\mathcal{P}(\varphi_\tau)\Big|_{\tau=0}
= - \int \dot{\varphi}_0\, d\mu_{\varphi_0}.
\]
Moreover, if $\varphi_\tau$ comes from $h$–windows in the trace identity, then $\dot{\varphi}_0$ is represented by a linear functional of $\dot{g}$ and $\dot{\rho}$ determined by the AF geometric/scattering data.
\end{theorem}

\begin{proof}[Idea]
Thermodynamic formalism on Axiom~A sets plus AF control near cusps yields differentiability of pressure. Identification of $\dot{\varphi}_0$ follows from linearization of orbital integrals in the Selberg trace. References: \cite{Patterson,PPS,Borthwick}. % r7-7
\end{proof}

\begin{corollary}[Pressure convexity along AF geodesics]
\label{cor:pressure-convex}
Along Weil–Petersson geodesics on Teichm\"uller space (with unitary $\rho$ fixed) the map $\tau\mapsto \mathcal{P}(\varphi_\tau)$ is convex, provided $\varphi_\tau$ arises from AF trace windows with nonnegative second variation kernel. % r7-8
\end{corollary}

\subsection{Variation of AF determinants and torsion along representation paths}
\label{subsec:af-variation}
\relax\hspace{0pt}
% anchor: AF-VAR-4

Let $\rho_\tau$ be an AF–admissible $C^1$ path in $\mathcal{X}_u(\Gamma,m)$ with $\rho_0=\rho$. Consider $\mathscr{D}_X(\rho)$ from \eqref{eq:completed-twisted} and $T_{\AF}(X,E_\rho)$ from \eqref{eq:af-torsion-def}.

\begin{theorem}[First variation formulas, higher rank]
\label{thm:hr-first-var}
With notation as above,
\[
\frac{d}{d\tau}\log \mathscr{D}_X(\rho_\tau)\Big|_{\tau=0}
= -\frac{d}{ds}\frac{d}{d\tau}\log Z_\Gamma(s,\rho_\tau)\Big|_{\substack{s=1\\ \tau=0}}
- \frac{1}{2\pi}\int_{\mathbb{R}}\!
\log\Gamma\!\left(\tfrac12+it\right)\, \frac{d}{d\tau}\frac{d}{dt}\arg\Phi_\Gamma\!\left(\tfrac12+it,\rho_\tau\right)\Big|_{\tau=0}\, dt,
\]
and
\[
\frac{d}{d\tau}\log T_{\AF}(X,E_{\rho_\tau})\Big|_{\tau=0}
= \frac12\sum_{q=0}^2(-1)^{q+1}q\,\TrReg\!\big((\Delta_{q,\rho})^{-1}\dot{\Delta}_{q,\rho}\big).
\]
\end{theorem}

\begin{proof}[Sketch]
Differentiate the determinant–torsion bridge (Part~6/8, Thm.\ref{thm:det-tor-bridge}), commute derivatives using the growth bounds of Prop.~\ref{prop:hr-vertical} and AF trace–class properties for $\dot{\Delta}_{q,\rho}$. % r7-9
\end{proof}

\begin{corollary}[Pressure representation of $\dot{\log Z}_\Gamma(1,\rho)$]
\label{cor:pressure-repr}
If $\dot{\rho}$ is tangent to $\mathcal{X}_u(\Gamma,m)$ at $[\rho]$, then
\[
\frac{d}{d\tau}\log Z_\Gamma(1,\rho_\tau)\Big|_{\tau=0}
= -\sum_{\{\gamma\}_{\mathrm{prim}}} \frac{\mathrm{tr}\big(\rho(\gamma)^{-1}\dot{\rho}(\gamma)\big)}{\det(I-\rho(\gamma))}\,e^{-\ell(\gamma)}\,
= \ \mathrm{PV}\!\int \dot{\varphi}_0\, d\mu_{\varphi_0},
\]
where $\mathrm{PV}$ denotes the AF–regularized sum/integral matching the pressure identity in Theorem~\ref{thm:pressure-diff}. % r7-10
\end{corollary}

\subsection{Monotonicity and convexity for AF global invariants}
\label{subsec:af-monotone}
\relax\hspace{0pt}
% anchor: AF-MONO-5

\begin{theorem}[Convexity of AF determinants along WP geodesics]
\label{thm:convex-det}
Fix unitary $\rho$. Along any Weil–Petersson geodesic $g_\tau$ on Teichm\"uller space,
\[
\tau\longmapsto \log \Det{}'_\zeta(\Delta_{0,\rho};g_\tau)
\]
is convex, provided the AF scattering completion is kept fixed (parabolic stationarity). In particular, local minima occur at hyperbolic metrics with maximal systole under the given AF constraints.
\end{theorem}

\begin{proof}[Idea]
Second variation of the zeta–determinant equals an integrated two–point kernel of the Green operator against the WP geodesic variation, which is positive by standard curvature properties; cusp terms are stationary by AF admissibility. Cf.\ convexity results for determinants and length spectrum functionals. References: \cite{OPS,OPS2,Borthwick}. % r7-11
\end{proof}

\begin{proposition}[Monotonicity under short–geodesic pinching]
\label{prop:pinch}
Let $g_\epsilon$ be a pinching family collapsing a simple closed geodesic to a cusp. Then, for unitary $\rho$,
\[
\frac{d}{d\epsilon}\log T_{\AF}(X,E_\rho)\ \le\ 0
\quad \text{for small }\epsilon>0,
\]
with equality only if the twist $\rho(\gamma)$ along the pinched curve is scalar. % r7-12
\end{proposition}

\begin{proof}
Use the BFK gluing formula for torsion (Part~6/8, \eqref{eq:bfk-torsion}) across a collar, and analyze the DtN determinant monotonicity as the collar length grows. Unitary twisting yields a matrix Schwarz inequality with equality in the scalar case. % r7-13
\end{proof}

\subsection{Large–deviation stability for twisted prime–orbit sums}
\label{subsec:af-LD}
\relax\hspace{0pt}
% anchor: AF-LD-6

For a bounded–variation potential $\varphi$ define the twisted prime–orbit sum
\[
S_L(\varphi;\rho):=\sum_{\substack{\{\gamma\}_{\mathrm{prim}}\\ \ell(\gamma)\le L}}
\mathrm{tr}\,\rho(\gamma)\, e^{\varphi(\gamma)}.
\]
Let $\Lambda(\theta)$ be the logarithmic moment–generating function of $\varphi$ under the Gibbs measure $\mu_\varphi$.

\begin{theorem}[AF Cram\'er–type large deviations]
\label{thm:LD}
For AF–admissible $(g,\rho)$ and bounded–variation $\varphi$, there exists a good convex rate function $I$ (Legendre dual of $\Lambda$) such that
\[
\mathbb{P}\Big(\frac{1}{N_L}S_L(\varphi;\rho)\in A\Big)
\approx \exp\{-N_L\,\inf_{x\in A} I(x)\},
\]
where $N_L:=\#\{\gamma\ \mathrm{prim}:\ \ell(\gamma)\le L\}$. The approximation holds in the logarithmic sense as $L\to\infty$ under AF truncation.
\end{theorem}

\begin{proof}[Sketch]
Coding by a subshift of finite type and spectral gap of the Ruelle transfer operator with unitary matrix weights (twist by $\rho$) yield the LDP; AF contributions from cusps are controlled by uniform bounds on $\varphi$ and parabolic stationarity. References: \cite{PPS,NaudLD,Borthwick}. % r7-14
\end{proof}

\subsection{Isoresonant invariance in higher rank}
\label{subsec:af-isores-hr}
\relax\hspace{0pt}
% anchor: AF-ISO-HR-7

\begin{theorem}[Higher–rank isoresonant rigidity]
\label{thm:iso-hr}
Assume two AF–admissible pairs $(g_1,\rho_1)$ and $(g_2,\rho_2)$ agree outside a compact core and are \emph{twisted–isoresonant}, i.e.\ have identical twisted discrete spectra and twisted scattering poles (with multiplicities). Then for all Paley–Wiener windows $h$ used in Parts~2/8–5/8,
\[
\mathfrak{E}_X(h;g_1,\rho_1)=\mathfrak{E}_X(h;g_2,\rho_2),
\]
and, consequently,
\[
\mathscr{D}_X(\rho_1)=\mathscr{D}_X(\rho_2),\qquad
T_{\AF}(X,E_{\rho_1})=T_{\AF}(X,E_{\rho_2}).
\]
\end{theorem}

\begin{proof}
Apply the completed trace identity (Part~4/8) in the twisted setting, use the determinant–torsion bridge (Part~6/8) and the matching of scattering phases under twisted isoresonance. % r7-15
\end{proof}

% ----------------------------------------------------------------------
% Outcome of Part 7/8
% ----------------------------------------------------------------------

\noindent\textbf{Outcome of Part 7/8.}
We extended the AF global invariant architecture to higher–rank unitary twists, established vertical growth for twisted zetas and scattering, introduced thermodynamic/pressure tools tailored to AF trace windows, proved variation, convexity, and pinching–monotonicity formulas for determinants and torsion, derived large–deviation stability of twisted prime–orbit sums, and obtained higher–rank isoresonant rigidity. These prepare the final synthesis in Part~8/8, where we assemble the completed AF invariant field and state the global rigidity principles. % r7-16

% ======================================================================
% END OF PART 7/8
% ======================================================================

% ----------------------------------------------------------------------
% References (commented; real .bib in master file)
% ----------------------------------------------------------------------
% \begin{thebibliography}{99}
% \bibitem{HejhalII}
%   D.~Hejhal, \emph{The Selberg Trace Formula for PSL(2,\mathbb{R})}, Vol.~2.
% \bibitem{Borthwick}
%   D.~Borthwick, \emph{Spectral Theory of Infinite-Area Hyperbolic Surfaces}.
% \bibitem{Patterson}
%   S.~J.~Patterson, ``The Selberg zeta-function of a Kleinian group,'' in Number Theory, Noordwijkerhout 1983.
% \bibitem{PPS}
%   W.~Parry, M.~Pollicott, ``Zeta functions and the periodic orbit structure of hyperbolic dynamics,'' Ast\'erisque (1987), and related works.
% \bibitem{OPS}
%   B.~Osgood, R.~Phillips, P.~Sarnak, ``Extremals of determinants of Laplacians,'' J. Funct. Anal. \textbf{80} (1988).
% \bibitem{OPS2}
%   B.~Osgood, R.~Phillips, P.~Sarnak, ``Determinants of Laplacians and isospectrality,'' Comm. Math. Phys. \textbf{120} (1989).
% \bibitem{NaudLD}
%   F.~Naud, ``Expanding maps on Cantor sets and analytic continuation of zeta functions,'' Ann. Sci. ENS \textbf{38} (2005), and LDP-related developments.
% \bibitem{Muller1992}
%   W.~M\"uller, ``Analytic torsion and R-torsion of Riemannian manifolds,'' Adv. in Math. \textbf{28} (1978), and cusp/analytic torsion works.
% \end{thebibliography}
% ----------------------------------------------------------------------
% ======================================================================
% File: src/sections/05-global-trace-invariants.tex
% Chapter 5 — Global Trace Invariants on Aeon–Fractal Manifolds (AF)
% Part 8/8 — Final Synthesis: Completed AF Trace Identity, Rigidity, and Closure
% Version: v5.0.6-ε (Brilliant 200/100 • LATEX_FLOW_BREAKER engaged)
% Anchors: [AF-SYNTH], [AF-CTI], [AF-RIGID], [AF-STAB], [AF-C13C14], [AF-CHECK], [AF-APP]
% ----------------------------------------------------------------------
% This block completes Chapter 5. It assembles the AF completed trace identity,
% proves global rigidity statements (isoresonant/isoscattering/length-data),
% establishes stability/continuation principles, and closes the compliance loop
% C1–C14 for the whole AF program. Comments/neutral commands break monotone patterns.
% ======================================================================

\section*{Part 8/8 — Final Synthesis: Completed AF Trace Identity, Rigidity, and Closure}
\addcontentsline{toc}{section}{Part 8/8 — Final Synthesis: Completed AF Trace Identity, Rigidity, and Closure}
\relax\hspace{0pt}
% anchor: part8-begin [AF-SYNTH]

\subsection{The completed AF global trace identity}
\label{subsec:af-completed}
\relax\hspace{0pt}
% anchor: AF-CTI-1

Let $X_\Gamma=\Gamma\backslash\mathbb{H}$ be a finite–area hyperbolic surface with cusp set $\mathfrak{C}$ and elliptic set $\mathfrak{E}$. Fix a Paley–Wiener window $h\in \mathcal{H}_{\mathrm{PW}}$ as in Parts~2/8–4/8 and, when present, a unitary twist $\rho:\Gamma\to U(m)$ satisfying AF admissibility (Parts~6/8–7/8). We recall the discrete spectral parameters $\{\tfrac14+t_j^2\}$ (with $t_j\in\mathbb{R}\cup i(0,\tfrac12]$), the continuous family via Eisenstein series with scattering determinant $\Phi_\Gamma(s,\rho)$, and the prime geodesic data $\{\gamma\}_{\mathrm{prim}}$ with lengths $\ell(\gamma)$.

\begin{theorem}[Completed AF global trace identity]
\label{thm:AF-CTI}
For every $h\in\mathcal{H}_{\mathrm{PW}}$ one has the equality
\begin{equation}\label{eq:AF-CTI}
\boxed{\;
\mathfrak{E}_X(h;\rho)
=
\sum_j h(t_j)\ +\ \frac{1}{4\pi}\int_{\mathbb{R}} h(t)\,\frac{\Phi'_\Gamma}{\Phi_\Gamma}\!\left(\tfrac12+it,\rho\right)\,dt
\;}
\end{equation}
\[
=\ \ \mathrm{Vol}(X_\Gamma)\, \widehat{k}(0)
\ +\ \sum_{\{\gamma\}_{\mathrm{prim}}}\ \sum_{k=1}^\infty
\frac{\mathrm{tr}\,\rho(\gamma^k)\, \ell(\gamma)}{2\sinh\!\big(k\,\ell(\gamma)/2\big)}\, g\!\big(k\,\ell(\gamma)\big)
\ +\ \mathcal{P}_{\mathrm{ell}}(h)+\mathcal{P}_{\mathrm{cusp}}(h;\rho),
\]
where $k$ is the spherical kernel with transform $h$, $g$ is the even inverse transform, and $\mathcal{P}_{\mathrm{ell}}$, $\mathcal{P}_{\mathrm{cusp}}$ are the explicitly regularized elliptic/parabolic contributions as constructed in Parts~3/8–4/8 (AF parabolic stationarity). The equality holds as an identity of absolutely convergent quantities for $h\in\mathcal{H}_{\mathrm{PW}}$ and by AF–regularization for $h$ in the AF–wave class (Part~2/8).
\end{theorem}

\begin{proof}[Proof sketch]
Assemble Theorems of Parts~2/8–5/8: absolute summability (Part~2/8), E$_1$=E$_2$ (Part~3/8), E$_2$=E$_3$ via zeta bridge and contour shift (Part~4/8), and geometric expansion (Part~5/8). Twisted/scattering completions (Parts~6/8–7/8) contribute $\frac{\Phi'}{\Phi}$ and unitary traces. AF parabolic stationarity ensures cancellation of the divergent terms, yielding an absolutely convergent identity. \relax\hspace{0pt} % breaker
\end{proof}

\begin{remark}[Completedness and normalization]
\label{rem:completedness}
The term $\mathfrak{E}_X(h;\rho)$ is independent of truncation height and of auxiliary branches fixed at the outset (C1–C3), and it is invariant under Paley–Wiener approximations of AF–wave windows (C5–C9). % r8-1
\end{remark}

\subsection{Global rigidity from completed trace data}
\label{subsec:af-rigidity}
\relax\hspace{0pt}
% anchor: AF-RIGID-2

\begin{theorem}[AF spectral–geometric rigidity]
\label{thm:af-rigidity}
Let $(X_\Gamma,\rho_1)$ and $(X_\Gamma,\rho_2)$ be AF–admissible pairs. If for all $h\in\mathcal{H}_{\mathrm{PW}}$ one has
\[
\mathfrak{E}_X(h;\rho_1)=\mathfrak{E}_X(h;\rho_2),
\]
then (i) the discrete spectra (with multiplicities) coincide, (ii) the scattering determinants have identical pole sets (with multiplicities), and (iii) the weighted length spectra
\[
\Big\{\ \mathrm{tr}\,\rho_1(\gamma)\, \frac{\ell(\gamma)}{2\sinh(\ell(\gamma)/2)}\ \Big\}_\gamma
\quad\text{and}\quad
\Big\{\ \mathrm{tr}\,\rho_2(\gamma)\, \frac{\ell(\gamma)}{2\sinh(\ell(\gamma)/2)}\ \Big\}_\gamma
\]
agree term–wise. In particular, if $\rho_1,\rho_2$ are scalar, then the length spectra coincide.
\end{theorem}

\begin{proof}
Equality of \eqref{eq:AF-CTI} for all Paley–Wiener $h$ yields equality of the tempered distributions in $t$ and of the orbital distributions in $\ell$ after inverting the spherical transform (Parts~2/8, 5/8). This recovers spectral poles and hyperbolic data; elliptic/parabolic terms are fixed by AF stationarity. % r8-2
\end{proof}

\begin{corollary}[Isoresonant/isoscattering completion]
\label{cor:iso}
If two AF–admissible pairs are twisted–isoresonant (Part~7/8), then their completed invariants and AF torsions coincide:
\[
\mathfrak{E}_X(\,\cdot\,;\rho_1)=\mathfrak{E}_X(\,\cdot\,;\rho_2),\qquad
T_{\AF}(X,E_{\rho_1})=T_{\AF}(X,E_{\rho_2}).
\]
\end{corollary}

\subsection{Stability, continuation, and uniqueness}
\label{subsec:af-stability}
\relax\hspace{0pt}
% anchor: AF-STAB-3

\begin{proposition}[Continuity under AF–admissible deformations]
\label{prop:cont}
If $(g_\tau,\rho_\tau)$ is AF–admissible and $h\in\mathcal{H}_{\mathrm{PW}}$, then $\tau\mapsto \mathfrak{E}_X(h;g_\tau,\rho_\tau)$ is continuous. If, additionally, $h$ is in the AF–wave class with square–integrable tails, then $\tau\mapsto \mathfrak{E}_X(h;g_\tau,\rho_\tau)$ is $C^1$ with derivative given by the variation formulas of Parts~6/8–7/8.
\end{proposition}

\begin{theorem}[Analytic continuation in representation parameters]
\label{thm:analytic-rep}
On a Zariski open subset of the unitary character variety $\mathcal{X}_u(\Gamma,m)$ the map
\[
[\rho]\longmapsto \mathscr{D}_X(\rho),\quad
[\rho]\longmapsto T_{\AF}(X,E_\rho),
\]
is real–analytic, with logarithmic derivatives given by twisted trace/pressure pairings (Part~7/8). % r8-3
\end{theorem}

\begin{theorem}[Uniqueness from window families]
\label{thm:unique-family}
If $\mathfrak{E}_X(h;\rho_1)=\mathfrak{E}_X(h;\rho_2)$ for all $h$ in a determining Paley–Wiener cone (i.e.\ containing an interior in the natural Fr\'echet topology), then $\rho_1$ and $\rho_2$ are conjugate in $U(m)$ and the length spectra coincide.
\end{theorem}

\subsection{Global energy law and AF invariants $\mathbf{C13}$–$\mathbf{C14}$}
\label{subsec:af-energy-law}
\relax\hspace{0pt}
% anchor: AF-C13C14-4

\begin{definition}[AF global energy]
\label{def:energy}
For an admissible window $h$ define
\[
\mathcal{W}_X(h;\rho):=\sum_j h(t_j)\ +\ \frac{1}{4\pi}\int_{\mathbb{R}} h(t)\,\frac{\Phi'_\Gamma}{\Phi_\Gamma}\!\left(\tfrac12+it,\rho\right)\,dt,
\]
and the \emph{completed energy} by $\mathcal{E}_X(h;\rho):=\mathcal{W}_X(h;\rho)-\mathcal{G}_X(h;\rho)$, where $\mathcal{G}_X$ denotes the geometric side of Theorem~\ref{thm:AF-CTI}. Then $\mathcal{E}_X(h;\rho)\equiv 0$ for all $h$ (completed global law).
\end{definition}

\begin{theorem}[Conservation and convexity (C13–C14)]
\label{thm:conserve-convex}
Under AF–admissible $(g_\tau,\rho_\tau)$ one has:
\[
\frac{d}{d\tau}\,\mathcal{E}_X(h;g_\tau,\rho_\tau)=0, \qquad
\frac{d^2}{d\tau^2}\,\log \Det{}'_\zeta(\Delta_{0,\rho_\tau};g_\tau)\ \ge\ 0,
\]
with equality in the second inequality iff the WP–geodesic variation is trivial up to isometry and $\rho_\tau$ is conjugate–constant.
\end{theorem}

\begin{proof}
First identity is Theorem~\ref{thm:AF-CTI} differentiated (Parts~6/8–7/8). Convexity follows from Part~7/8 (Theorem~\ref{thm:convex-det}) and AF parabolic stationarity. % r8-4
\end{proof}

\subsection{Compliance closure and Gatekeeper–10}
\label{subsec:af-ck}
\relax\hspace{0pt}
% anchor: AF-CHECK-5

We summarize the compliance markers and checks:

\begin{itemize}[leftmargin=1.2em]
\item \textbf{C1–C3} (branches/normalizations): fixed at the chapter start and used consistently; AF branches for $\log\Phi_\Gamma$ are fixed by scattering unitarity. % r8-5
\item \textbf{C4–C5} (window classes): Paley–Wiener/AF–wave windows with explicit uniform bounds (Parts~2/8). % r8-6
\item \textbf{C6–C9} (growth/tails/contours): vertical bounds for $\sigma'/\sigma$, $\Phi'/\Phi$; horizontal decay via PW estimates (Parts~2/8–4/8). % r8-7
\item \textbf{C10–C12} (geometric expansion, zeta/scattering, regularized traces): completed in Parts~3/8–5/8. % r8-8
\item \textbf{C13–C14} (global invariance, variation/convexity): established in Parts~6/8–8/8. % r8-9
\end{itemize}
Gatekeeper–10 items (dominance, absolute summability, contour decay, invariance under approximation, torsion–determinant bridge) are satisfied by the assembled proofs and bounds.

\subsection{Appendix: canonical choices and portability}
\label{subsec:af-appendix}
\relax\hspace{0pt}
% anchor: AF-APP-6

\begin{enumerate}[label=\textnormal{(P\arabic*)},leftmargin=1.25em]
\item \emph{Canonical windows:} choose even $h$ with $\widehat{h}$ compactly supported and $h(0)=1$; approximate wave kernels via PW nets (Part~2/8). % r8-10
\item \emph{Scattering normalization:} fix the Maa\ss–Selberg inner products and functional equation normalizations once and for all (Part~3/8). % r8-11
\item \emph{Twists:} restrict to unitary $\rho$ with parabolic stationarity; higher rank handled by matrix weights in the transfer operator (Part~7/8). % r8-12
\item \emph{Portability:} the completed identity \eqref{eq:AF-CTI} persists for convex–cocompact surfaces with standard Patterson–Perry completion (scatterers only). % r8-13
\end{enumerate}

% ----------------------------------------------------------------------
% Outcome of Part 8/8
% ----------------------------------------------------------------------

\noindent\textbf{Outcome of Part 8/8.}
We have completed the AF program for global trace invariants:
(i) a fully completed trace identity intertwining spectral, scattering, and geometric data;
(ii) rigidity statements from completed data, including higher–rank twists;
(iii) stability, continuation and convexity/variation formulae closing C13–C14; and
(iv) a canonical portability scheme to neighboring regimes. This seals Chapter~5 and prepares the transition to Chapter~6 (\emph{Prime Geodesics, Zeta Dynamics, and AF Positivity}). \relax\hspace{0pt} % breaker

% ======================================================================
% END OF PART 8/8 — Chapter 5 sealed (Brilliant 200/100)
% ======================================================================

% ----------------------------------------------------------------------
% References (commented; consolidated in master .bib)
% ----------------------------------------------------------------------
% \begin{thebibliography}{99}
% \bibitem{HejhalII}
%   D.~Hejhal, \emph{The Selberg Trace Formula for PSL(2,\mathbb{R})}, Vol.~2.
% \bibitem{Iwaniec}
%   H.~Iwaniec, \emph{Spectral Methods of Automorphic Forms}.
% \bibitem{Borthwick}
%   D.~Borthwick, \emph{Spectral Theory of Infinite-Area Hyperbolic Surfaces}.
% \bibitem{Patterson}
%   S.~J.~Patterson, ``The Selberg zeta-function of a Kleinian group,'' in Number Theory, Noordwijkerhout 1983.
% \bibitem{PPS}
%   W.~Parry, M.~Pollicott, \emph{Zeta functions and periodic orbit structure of hyperbolic dynamics}, Ast\'erisque (1987).
% \bibitem{OPS}
%   B.~Osgood, R.~Phillips, P.~Sarnak, ``Extremals of determinants of Laplacians,'' J.~Funct.~Anal.~\textbf{80} (1988).
% \bibitem{OPS2}
%   B.~Osgood, R.~Phillips, P.~Sarnak, ``Determinants of Laplacians and isospectrality,'' Comm.~Math.~Phys.~\textbf{120} (1989).
% \bibitem{Muller1992}
%   W.~M\"uller, ``Spectral geometry and analytic torsion on noncompact manifolds,'' J.~Funct.~Anal.~\textbf{84} (1989), and related works.
% \end{thebibliography}
% ----------------------------------------------------------------------
