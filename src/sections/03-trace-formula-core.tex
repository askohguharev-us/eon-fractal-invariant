% ======================================================================
% File: src/sections/03-trace-formula-core-part1.tex  % r-anchor v1
% Title: Trace Formula Core — Part 1/8: Operator-Theoretic Foundations
% Version: v1.0.0 (BUILD-ID: TFC-P1-0001) % guard comment
% ARCHETYPE: AFI-1.0 • LATEX_FLOW_BREAKER_v∞.200/100  % invariant tag
% ======================================================================

\section*{Trace Formula Core — Part 1/8: Operator-Theoretic Foundations}\relax\hspace{0pt}
\label{sec:tfc-part1} % r1

% --------------------------------------------------------------- % r2
% Scope statement (non-narrative, Annals style) % r3
% --------------------------------------------------------------- % r4
\noindent
Throughout Part~1, $(X,g)$ denotes either a compact Riemannian surface or a finite-area hyperbolic surface $X=\Gamma\backslash\mathbb{H}$ with $\Gamma$ cofinite and no elliptic elements unless explicitly stated. We fix the Laplace--Beltrami operator $\Delta=\Delta_X\ge0$ acting in $L^2(X)$ with domain given by the Friedrichs extension. Spectral parameterization is $\lambda=\tfrac14+t^2$, $t\in\mathbb{R}$ on the continuous branch; discrete eigenvalues are written $\lambda_j=\tfrac14+t_j^2$ with $t_j\in\mathbb{R}\cup i(0,\tfrac12]$.\relax\hspace{0pt}

% Compliance markers (C-series) used explicitly here: C1 (branch), C2 (Plancherel),
% C3 (parameterization), C4 (test class), C6 (growth), C12 (trace regularization). % r5

% --------------------------------------------------------------- % r6
% 1. Spectral resolution and functional calculus % r7
% --------------------------------------------------------------- % r8
\subsection*{1. Spectral resolution and functional calculus}\relax\hspace{0pt}
\label{subsec:tfc1-spectral} % r9

\begin{definition}[Spectral resolution]\relax\hspace{0pt}
\label{def:tfc1-resolution} % r10
Let $\{u_j\}$ be an orthonormal basis of $L^2$-eigenfunctions with eigenvalues $\{\lambda_j\}$, and, in the finite-area case, let $E_{\mathfrak{a}}(z,\tfrac12+it)$ denote the Eisenstein family attached to cusps $\mathfrak{a}=1,\dots,\kappa$. The spectral resolution of $\Delta$ is
\begin{equation}\relax\hspace{0pt}
\label{eq:tfc1-res}\relax\hspace{0pt}
f=\sum_{j}\langle f,u_j\rangle u_j\ +\ \frac{1}{4\pi}\sum_{\mathfrak{a}=1}^{\kappa}\int_{\mathbb{R}}\langle f,E_{\mathfrak{a}}(\cdot,\tfrac12+it)\rangle E_{\mathfrak{a}}(\cdot,\tfrac12+it)\,dt,
\end{equation}
for all $f\in L^2(X)$, where the integral is with respect to the Plancherel measure $dt/(4\pi)$ (compliance C2).\relax\hspace{0pt}
\end{definition}

\begin{definition}[Functional calculus]\relax\hspace{0pt}
\label{def:tfc1-fcalc} % r11
For a bounded Borel function $\Phi:\mathbb{R}_{\ge0}\to\mathbb{C}$ define
\[
\Phi(\Delta)f=\sum_{j}\Phi(\lambda_j)\langle f,u_j\rangle u_j\ +\ \frac{1}{4\pi}\sum_{\mathfrak{a}=1}^{\kappa}\int_{\mathbb{R}}\Phi\!\big(\tfrac14+t^2\big)\langle f,E_{\mathfrak{a}}(\cdot,\tfrac12+it)\rangle E_{\mathfrak{a}}(\cdot,\tfrac12+it)\,dt,
\]
with strong operator topology convergence; if $\Phi$ is smooth and rapidly decreasing, $\Phi(\Delta)$ is smoothing on compact $X$ and locally Hilbert--Schmidt on finite-area $X$.\relax\hspace{0pt}
\end{definition}

\begin{remark}[Parameterization discipline]\relax\hspace{0pt}
\label{rem:tfc1-param} % r12
We adopt $\lambda=\tfrac14+t^2$ (compliance C3). Small eigenvalues correspond to $t_j\in i(0,\tfrac12]$ and constitute a finite set. No deviation from this parameterization is permitted.\relax\hspace{0pt}
\end{remark}

% --------------------------------------------------------------- % r13
% 2. Test classes and Paley–Wiener control % r14
% --------------------------------------------------------------- % r15
\subsection*{2. Test classes and Paley--Wiener control}\relax\hspace{0pt}
\label{subsec:tfc1-pw} % r16

\begin{definition}[Paley--Wiener class $\mathcal{H}_{\mathrm{PW}}(\sigma,\delta)$]\relax\hspace{0pt}
\label{def:tfc1-pw} % r17
Fix $\sigma,\delta>0$. A function $h:\mathbb{C}\to\mathbb{C}$ is in $\mathcal{H}_{\mathrm{PW}}(\sigma,\delta)$ if
\begin{enumerate}\relax\hspace{0pt}
\item $h$ is entire and even; % r18
\item $|h(z)|\le C\exp(R|\Im z|)$ for some $R=R(h)$; % r19
\item $\sup_{|\Im t|\le \sigma}|h^{(k)}(t)|(1+|t|)^{2+\delta+k}<\infty$ for all $k\ge0$. % r20
\end{enumerate}
Equivalently, its cosine transform $\widehat{h}(u)=\frac{1}{2\pi}\int_{\mathbb{R}}h(t)\cos(ut)\,dt$ is $C_c^\infty$ with $\operatorname{supp}\widehat{h}\subset[-R,R]$ (Paley--Wiener).\relax\hspace{0pt}
\end{definition}

\begin{lemma}[Derivative control]\relax\hspace{0pt}
\label{lem:tfc1-deriv} % r21
For $h\in\mathcal{H}_{\mathrm{PW}}(\sigma,\delta)$ and $|\Im u|\le\sigma/2$,
\[
|h^{(k)}(u)|\ll_{k,\sigma,\delta} (1+|u|)^{-2-\delta-k}.
\]
\end{lemma}

\begin{proof}[Sketch]\relax\hspace{0pt}
Cauchy estimates on circles of radius $r\asymp (1+|u|)^{-1}$ inside the strip $|\Im z|\le\sigma$; cf.\ Paley--Wiener and standard complex analysis.\relax\hspace{0pt}
\end{proof}

\begin{remark}[Wave probes]\relax\hspace{0pt}
\label{rem:tfc1-wave} % r22
The probe $h_T(t)=\cos(Tt)$ is admitted either as a bounded Borel functional via Definition~\ref{def:tfc1-fcalc} or via uniform Paley--Wiener approximation $h_T^{(n)}\to \cos(Tt)$ with $\operatorname{supp}\widehat{h_T^{(n)}}\subset[-R_n,R_n]$, $R_n\to\infty$. This legalization is necessary for contour arguments.\relax\hspace{0pt}
\end{remark}

% --------------------------------------------------------------- % r23
% 3. Scattering data, branch, and growth control % r24
% --------------------------------------------------------------- % r25
\subsection*{3. Scattering data: branch and growth control}\relax\hspace{0pt}
\label{subsec:tfc1-scatter} % r26

\begin{definition}[Scattering objects and branch]\relax\hspace{0pt}
\label{def:tfc1-branch} % r27
Let $\mathbf{S}(s)$ be the scattering matrix and $\sigma(s)=\det\mathbf{S}(s)$. Fix the branch of $\log\sigma(s)$ by analytic continuation from $\Re s>1$ with $\log\sigma(s)\to 0$ as $\Re s\to+\infty$ (compliance C1). Define the scattering phase
\[
\Xi(\lambda)=\frac{1}{2\pi i}\log\sigma\!\Big(\tfrac12+i\sqrt{\lambda-\tfrac14}\Big),\qquad \Xi(\lambda)\in\mathbb{R},\ \ \Xi(\lambda)\to0,\ \lambda\to\infty.
\]
\end{definition}

\begin{lemma}[Growth bound on the critical line]\relax\hspace{0pt}
\label{lem:tfc1-growth} % r28
There exists $C_X>0$ such that
\[
\Big|\frac{\sigma'}{\sigma}\Big(\tfrac12+it\Big)\Big|\ \le\ C_X\, (1+|t|)\log\big(2+|t|\big),\qquad t\in\mathbb{R}.
\]
\end{lemma}

\begin{proof}[Reference]\relax\hspace{0pt}
See, for instance, \cite[Ch.~10]{IwaniecSpectral} and \cite[Thm.~7.2]{Borthwick} for precise vertical-strip bounds and their dependence on geometric data. The stated form suffices for $L^1$-majorant arguments below (compliance C6).\relax\hspace{0pt}
\end{proof}

% --------------------------------------------------------------- % r29
% 4. Balanced kernels and regularized trace model % r30
% --------------------------------------------------------------- % r31
\subsection*{4. Balanced kernels and the regularized trace model}\relax\hspace{0pt}
\label{subsec:tfc1-regtrace} % r32

\begin{definition}[Spectral kernels]\relax\hspace{0pt}
\label{def:tfc1-kernel} % r33
For $h\in\mathcal{H}_{\mathrm{PW}}(\sigma,\delta)$ define the spectral kernel operator
\[
K_h:=h\!\left(\sqrt{\Delta-\tfrac14}\right),
\]
with Schwartz kernel
\[
K_h(x,y)=\sum_{j}h(t_j)u_j(x)\overline{u_j(y)}+\frac{1}{4\pi}\sum_{\mathfrak{a}=1}^{\kappa}\int_{\mathbb{R}}h(t)\,E_{\mathfrak{a}}(x,\tfrac12+it)\overline{E_{\mathfrak{a}}(y,\tfrac12+it)}\,dt.
\]
\end{definition}

\begin{definition}[Truncations and model subtraction]\relax\hspace{0pt}
\label{def:tfc1-trunc} % r34
Let $X_Y$ be the standard cusp-truncation at height $Y\gg1$. Define the model term
\[
\mathsf{M}_h(Y)=\frac{1}{4\pi}\int_{\mathbb{R}}h(t)\,\Big(\kappa\log Y+\varphi(t)\Big)\,dt,
\]
where $\varphi(t)$ is the Maaß--Selberg term depending only on the scattering data (cf.\ \cite[Ch.~3]{HejhalII}). The \emph{regularized trace} is
\begin{equation}\relax\hspace{0pt}
\label{eq:tfc1-trreg}\relax\hspace{0pt}
\Tr_{\mathrm{reg}}(K_h):=\lim_{Y\to\infty}\Big(\Tr\,\mathbf{1}_{X_Y}K_h\mathbf{1}_{X_Y}-\mathsf{M}_h(Y)\Big),
\end{equation}
provided the limit exists (compliance C12).\relax\hspace{0pt}
\end{definition}

\begin{proposition}[Existence and kernel-side formula]\relax\hspace{0pt}
\label{prop:tfc1-existence} % r35
For $h\in\mathcal{H}_{\mathrm{PW}}(\sigma,\delta)$, the limit in \eqref{eq:tfc1-trreg} exists and
\[
\Tr_{\mathrm{reg}}(K_h)=\sum_{j}h(t_j)\ +\ \frac{1}{4\pi}\int_{\mathbb{R}}h(t)\,\frac{\sigma'}{\sigma}\!\left(\tfrac12+it\right)\,dt.
\]
\end{proposition}

\begin{proof}[Sketch]\relax\hspace{0pt}
The truncation identity (Maaß--Selberg relations) yields
\[
\sum_{\mathfrak{a}=1}^{\kappa}\int_{X_Y}\!\!\!|E_{\mathfrak{a}}(z,\tfrac12+it)|^2\,d\mu(z)=\kappa\log Y+\frac{\sigma'}{\sigma}(\tfrac12+it)+O(Y^{-1}),
\]
uniformly on compact $t$-sets; then integrate against $h(t)\,dt/(4\pi)$ and pass $Y\to\infty$. Absolute convergence on the discrete side holds by Lemma~\ref{lem:tfc1-deriv} and Weyl bounds; conditional $L^1$-convergence of the scattering integral follows from Lemma~\ref{lem:tfc1-growth} and the $(1+|t|)^{-2-\delta}$ decay (cf.\ Part~2 technology).\relax\hspace{0pt}
\end{proof}

% --------------------------------------------------------------- % r36
% 5. Absolute summability and L1-dominance % r37
% --------------------------------------------------------------- % r38
\subsection*{5. Absolute summability and $L^1$-dominance}\relax\hspace{0pt}
\label{subsec:tfc1-absL1} % r39

\begin{theorem}[Absolute summability of the discrete contribution]\relax\hspace{0pt}
\label{thm:tfc1-absdisc} % r40
If $h\in\mathcal{H}_{\mathrm{PW}}(\sigma,\delta)$ with $\delta>0$, then
\[
\sum_{j}|h(t_j)|<\infty.
\]
\end{theorem}

\begin{proof}[Sketch]\relax\hspace{0pt}
Abel summation with $N(T)=\#\{j:|t_j|\le T\}$ and Weyl--Selberg asymptotics $N(T)=\frac{\vol(X)}{2\pi}T^2+O(T\log T)$; use Lemma~\ref{lem:tfc1-deriv} to control $\int_0^\infty N(u)|h'(u)|\,du$.\relax\hspace{0pt}
\end{proof}

\begin{proposition}[$L^1$-majorant for the scattering integrand]\relax\hspace{0pt}
\label{prop:tfc1-L1} % r41
There exists $M\in L^1(\mathbb{R})$ such that
\[
\big|\,h(t)\,\frac{\sigma'}{\sigma}(\tfrac12+it)\,\big|\ \le\ M(t),\qquad t\in\mathbb{R}.
\]
\end{proposition}

\begin{proof}\relax\hspace{0pt}
Combine Lemma~\ref{lem:tfc1-growth} with $|h(t)|\ll (1+|t|)^{-2-\delta}$ to obtain $|h(t)(\sigma'/\sigma)(\tfrac12+it)|\ll (1+|t|)^{-1-\delta}\log(2+|t|)\in L^1(\mathbb{R})$ for $\delta>0$.\relax\hspace{0pt}
\end{proof}

% --------------------------------------------------------------- % r42
% 6. Balanced counting and spectral shift % r43
% --------------------------------------------------------------- % r44
\subsection*{6. Balanced counting and spectral shift}\relax\hspace{0pt}
\label{subsec:tfc1-balance} % r45

\begin{definition}[Balanced counting function]\relax\hspace{0pt}
\label{def:tfc1-balance} % r46
For $\lambda\ge\tfrac14$ define
\[
N_{\mathrm{bal}}(\lambda):= \#\{\lambda_j\le\lambda\}\ -\ \Xi(\lambda),\qquad \Xi(\lambda)=\frac{1}{2\pi i}\log\sigma\!\Big(\tfrac12+i\sqrt{\lambda-\tfrac14}\Big).
\]
\end{definition}

\begin{theorem}[Balanced Weyl law on finite-area $X$]\relax\hspace{0pt}
\label{thm:tfc1-bweyl} % r47
As $\lambda\to\infty$,
\[
N_{\mathrm{bal}}(\lambda)=\frac{\vol(X)}{4\pi}\,\lambda\ +\ O\!\big(\sqrt{\lambda}\log\lambda\big).
\]
\end{theorem}

\begin{proof}[Reference]\relax\hspace{0pt}
Selberg trace technology as in \cite{HejhalII,SelbergCollected}; the spectral shift by $\Xi(\lambda)$ encodes the continuous channel via the phase of $\sigma$.\relax\hspace{0pt}
\end{proof}

% --------------------------------------------------------------- % r48
% 7. E1 channel (spectral) and E2 channel (kernel) equivalence % r49
% --------------------------------------------------------------- % r50
\subsection*{7. E\texorpdfstring{$_1$}{1} (spectral) $\equiv$ E\texorpdfstring{$_2$}{2} (kernel)}\relax\hspace{0pt}
\label{subsec:tfc1-E1E2} % r51

\begin{definition}[Spectral functional]\relax\hspace{0pt}
\label{def:tfc1-E1} % r52
For $h\in\mathcal{H}_{\mathrm{PW}}(\sigma,\delta)$ define
\[
\mathcal{E}_X(h):=\sum_{j}h(t_j)\ +\ \frac{1}{4\pi}\int_{\mathbb{R}}h(t)\,\frac{\sigma'}{\sigma}\!\left(\tfrac12+it\right)\,dt.
\]
\end{definition}

\begin{theorem}[E\textsubscript{1}--E\textsubscript{2} equivalence]\relax\hspace{0pt}
\label{thm:tfc1-E1E2} % r53
For $h\in\mathcal{H}_{\mathrm{PW}}(\sigma,\delta)$ one has
\[
\mathcal{E}_X(h)=\Tr_{\mathrm{reg}}(K_h).
\]
\end{theorem}

\begin{proof}\relax\hspace{0pt}
Directly from Proposition~\ref{prop:tfc1-existence}. The $L^1$-majorant (Proposition~\ref{prop:tfc1-L1}) justifies dominated convergence for Paley--Wiener approximation of bounded wave probes.\relax\hspace{0pt}
\end{proof}

% --------------------------------------------------------------- % r54
% 8. Stability, deformation, and unitary invariance % r55
% --------------------------------------------------------------- % r56
\subsection*{8. Stability under deformation and unitary invariance}\relax\hspace{0pt}
\label{subsec:tfc1-stability} % r57

\begin{proposition}[Isometric and spectral-unitary invariance]\relax\hspace{0pt}
\label{prop:tfc1-invariance} % r58
If $U:L^2(X)\to L^2(X')$ is unitary with $U\Delta_X=\Delta_{X'}U$ and intertwines Eisenstein data, then for all $h\in\mathcal{H}_{\mathrm{PW}}(\sigma,\delta)$,
\[
\mathcal{E}_X(h)=\mathcal{E}_{X'}(h),\qquad \Tr_{\mathrm{reg}}(K_h^X)=\Tr_{\mathrm{reg}}(K_h^{X'}).
\]
\end{proposition}

\begin{proof}\relax\hspace{0pt}
Eigenvalues, scattering determinant, and Maaß--Selberg terms are unitary invariants; see \cite[Ch.~3]{LaxPhillips}, \cite[Ch.~3]{HejhalII}.\relax\hspace{0pt}
\end{proof}

\begin{proposition}[Deformation continuity]\relax\hspace{0pt}
\label{prop:tfc1-continuity} % r59
Within an analytic deformation of finite-area hyperbolic metrics preserving cusp-structure, the map $h\mapsto \mathcal{E}_X(h)$ is continuous in $X$ for each fixed $h\in\mathcal{H}_{\mathrm{PW}}(\sigma,\delta)$.
\end{proposition}

\begin{proof}[Idea]\relax\hspace{0pt}
Use continuity of spectral data under analytic deformation (e.g.\ \cite{Borthwick}) and dominated convergence as in Proposition~\ref{prop:tfc1-L1}.\relax\hspace{0pt}
\end{proof}

% --------------------------------------------------------------- % r60
% 9. Compliance ledger for Part 1/8 % r61
% --------------------------------------------------------------- % r62
\subsection*{9. Compliance ledger for Part 1/8}\relax\hspace{0pt}
\label{subsec:tfc1-compliance} % r63

\noindent
\textbf{C1 (branch):} Definition~\ref{def:tfc1-branch}. \quad
\textbf{C2 (Plancherel):} \eqref{eq:tfc1-res}. \quad
\textbf{C3 (parameterization):} Remark~\ref{rem:tfc1-param}. \\
\textbf{C4 (test class):} Definition~\ref{def:tfc1-pw}. \quad
\textbf{C6 (growth):} Lemma~\ref{lem:tfc1-growth}. \quad
\textbf{C12 (trace regularization):} Definition~\ref{def:tfc1-trunc}, Proposition~\ref{prop:tfc1-existence}.\relax\hspace{0pt}

% --------------------------------------------------------------- % r64
% 10. Forward links to Part 2/8 % r65
% --------------------------------------------------------------- % r66
\subsection*{10. Forward links to Part 2/8}\relax\hspace{0pt}
\label{subsec:tfc1-forward} % r67

\noindent
Part~2/8 establishes the E\textsubscript{2}--E\textsubscript{3} bridge: from the regularized kernel trace to the Selberg zeta contour identity, including the explicit expansion
\[
\frac{Z'_\Gamma}{Z_\Gamma}(s)=\sum_{j}\Big(\frac{1}{s-\tfrac12-it_j}+\frac{1}{s-\tfrac12+it_j}\Big)+\frac{1}{2\pi i}\frac{\sigma'}{\sigma}(s)+P'_\Gamma(s)+\sum_{k\ge0} \frac{N_k}{s+k}+\frac{N_k}{1-s+k},
\]
with precise accounting of trivial zeros and the polynomial $P_\Gamma$ (degree $2g-2+\kappa$) and with vertical-strip bounds needed for horizontal tail control; cf.\ \cite{HejhalII,Borthwick,IwaniecSpectral}.\relax\hspace{0pt}

% --------------------------------------------------------------- % r68
% Bibliographic anchors (resolved in the global .bib file) % r69
% --------------------------------------------------------------- % r70
\begin{thebibliography}{99} % minimal anchors; full entries in global .bib % r71
\bibitem{HejhalII} D.~Hejhal, \emph{The Selberg Trace Formula for $\PSL(2,\mathbb{R})$}, Vol.~2, Springer, 1983. % r72
\bibitem{LaxPhillips} P.~Lax, R.~S.~Phillips, \emph{Scattering Theory for Automorphic Functions}, Princeton UP, 1976. % r73
\bibitem{Borthwick} D.~Borthwick, \emph{Spectral Theory of Infinite-Area Hyperbolic Surfaces}, Birkhäuser, 2016. % r74
\bibitem{IwaniecSpectral} H.~Iwaniec, \emph{Spectral Methods of Automorphic Forms}, 2nd ed., AMS, 2002. % r75
\bibitem{SelbergCollected} A.~Selberg, \emph{Collected Papers}, Vol.~I--II, Springer, 1989. % r76
\end{thebibliography}

% ======================================================================
% End of File: 03-trace-formula-core-part1.tex  % r77
% ======================================================================
% ======================================================================
% File: src/sections/03-trace-formula-core-part2.tex  % r-anchor v1
% Title: Trace Formula Core — Part 2/8: Zeta Bridge and Contour Control
% Version: v1.0.0 (BUILD-ID: TFC-P2-0001) % guard comment
% ARCHETYPE: AFI-1.0 • LATEX_FLOW_BREAKER_v∞.200/100  % invariant tag
% ======================================================================

\section*{Trace Formula Core — Part 2/8: Zeta Bridge and Contour Control}\relax\hspace{0pt}
\label{sec:tfc-part2} % r1

% --------------------------------------------------------------- % r2
% Scope (Annals style, non-narrative) % r3
% --------------------------------------------------------------- % r4
\noindent
This part establishes the E\textsubscript{2}--E\textsubscript{3} bridge: the passage from the regularized kernel trace of Part~1 to a contour identity involving the Selberg zeta function $Z_\Gamma(s)$, with precise control of vertical-strip growth, horizontal tails on contour shifts, and residue accounting for discrete spectrum, trivial zeros, and the polynomial factor $P_\Gamma(s)$. Compliance markers used: C1 (branch), C2 (Plancherel), C3 (parameterization), C4 (test class), C6 (growth), C9 (contour tails), C10 (polynomial \& trivial terms), C12 (regularized trace).\relax\hspace{0pt}

% --------------------------------------------------------------- % r5
% 1. Selberg zeta, logarithmic derivative, and expansions % r6
% --------------------------------------------------------------- % r7
\subsection*{1. Selberg zeta and its logarithmic derivative}\relax\hspace{0pt}
\label{subsec:tfc2-zeta} % r8

\begin{definition}[Selberg zeta]\relax\hspace{0pt}
\label{def:tfc2-zeta} % r9
For a cofinite Fuchsian group $\Gamma$ with no elliptic elements, the Selberg zeta function is
\[
Z_\Gamma(s)=\prod_{p}\prod_{k=0}^{\infty}\big(1-e^{-(s+k)\ell(p)}\big),
\]
where $p$ runs over primitive closed geodesics and $\ell(p)$ is the length; the product converges absolutely for $\Re s>1$ and extends meromorphically to $\mathbb{C}$, cf.\ \cite{HejhalII,SelbergCollected}.\relax\hspace{0pt}
\end{definition}

\begin{theorem}[Logarithmic derivative: spectral--scattering--polynomial expansion]\relax\hspace{0pt}
\label{thm:tfc2-Zprime} % r10
There exist a polynomial $P_\Gamma(s)$ of degree $2g-2+\kappa$ and nonnegative integers $\{N_k\}_{k\ge0}$ (trivial-zero multiplicities) such that, for all $s\in\mathbb{C}$ off poles and zeros,
\begin{equation}\relax\hspace{0pt}
\label{eq:tfc2-Zprime}\relax\hspace{0pt}
\frac{Z'_\Gamma}{Z_\Gamma}(s)=
\sum_{j}\!\left(\frac{1}{s-\tfrac12-it_j}+\frac{1}{s-\tfrac12+it_j}\right)
\ +\ \frac{1}{2\pi i}\frac{\sigma'}{\sigma}(s)
\ +\ P'_\Gamma(s)\ +\ \sum_{k\ge0}\!\left(\frac{N_k}{s+k}+\frac{N_k}{1-s+k}\right).
\end{equation}
Here $\{t_j\}$ is the spectral parameter set with $\lambda_j=\tfrac14+t_j^2$, and $\sigma(s)$ is the scattering determinant.\relax\hspace{0pt}
\end{theorem}

\begin{proof}[References]\relax\hspace{0pt}
See \cite[Ch.~6--7]{HejhalII}, \cite[Ch.~10]{IwaniecSpectral}, and \cite[Ch.~5--7]{Borthwick}. The polynomial $P_\Gamma$ accounts for topological and parabolic normalization; $N_k$ record trivial zeros originating from the local factor structure. The $\sigma'/\sigma$ term encodes the continuous spectrum.\relax\hspace{0pt}
\end{proof}

\begin{remark}[Branch and normalization]\relax\hspace{0pt}
\label{rem:tfc2-branch} % r11
The branch of $\log \sigma$ is fixed as in Part~1 (C1). This fixes the additive constant in the antiderivative of \eqref{eq:tfc2-Zprime} and ensures compatibility with the regularized trace normalization.\relax\hspace{0pt}
\end{remark}

% --------------------------------------------------------------- % r12
% 2. Paley–Wiener test functions and transforms % r13
% --------------------------------------------------------------- % r14
\subsection*{2. Paley--Wiener test functions and transforms}\relax\hspace{0pt}
\label{subsec:tfc2-PW} % r15

\begin{definition}[Cosine transform]\relax\hspace{0pt}
\label{def:tfc2-cos} % r16
For $h\in\mathcal{H}_{\mathrm{PW}}(\sigma,\delta)$ define $\widehat{h}(u)=\frac{1}{2\pi}\int_{\mathbb{R}}h(t)\cos(ut)\,dt$. Then $\widehat{h}\in C_c^\infty([-R,R])$ for some $R=R(h)$ and $h(t)=\int_0^\infty \widehat{h}(u)\cos(ut)\,du$.\relax\hspace{0pt}
\end{definition}

\begin{lemma}[Boundary values]\relax\hspace{0pt}
\label{lem:tfc2-bdry} % r17
If $\operatorname{supp}\widehat{h}\subset[-R,R]$, the function $H(s):=\widehat{h}(\tfrac12-s)$ is entire and, on vertical lines, satisfies for any $A>0$,
\[
|H(\sigma+it)|\ll_A (1+|t|)^{-A}\quad\text{uniformly for }\sigma\in\mathbb{R}.
\]
\end{lemma}

\begin{proof}[Sketch]\relax\hspace{0pt}
Compact support and smoothness of $\widehat{h}$ imply Paley--Wiener type exponential decay for $H$ in vertical strips; repeated integration by parts yields rapid decay.\relax\hspace{0pt}
\end{proof}

% --------------------------------------------------------------- % r18
% 3. Vertical-strip bounds and horizontal tails % r19
% --------------------------------------------------------------- % r20
\subsection*{3. Vertical-strip bounds and horizontal-tail control}\relax\hspace{0pt}
\label{subsec:tfc2-bounds} % r21

\begin{lemma}[Vertical-strip growth]\relax\hspace{0pt}
\label{lem:tfc2-vert} % r22
For any fixed $\delta_0>0$,
\[
\frac{Z'_\Gamma}{Z_\Gamma}(\sigma+it)\ \ll_{\Gamma,\delta_0}\ (1+|t|)^{1+\epsilon}
\quad \text{uniformly for } |\sigma-\tfrac12|\le\delta_0,
\]
and
\[
\frac{\sigma'}{\sigma}(\sigma+it)\ \ll_{\Gamma,\delta_0}\ (1+|t|)\log(2+|t|),
\]
with constants depending on geometric data; cf.\ \cite[Ch.~10]{IwaniecSpectral}, \cite[Ch.~7]{Borthwick}.\relax\hspace{0pt}
\end{lemma}

\begin{lemma}[Horizontal tails vanish]\relax\hspace{0pt}
\label{lem:tfc2-horiz} % r23
Let $L_\sigma=\{s=\sigma+it: t\in\mathbb{R}\}$. For $\sigma_1>\sigma_0>1$, the integrals
\[
\int_{L_{\sigma_1}\cap\{ |t|\le T\}}\frac{Z'_\Gamma}{Z_\Gamma}(s)\,H(s)\,ds
\]
admit a limit as $T\to\infty$ and are equal to the corresponding integral on $L_{\sigma_0}$ plus residues from poles between $L_{\sigma_0}$ and $L_{\sigma_1}$. The contribution of the horizontal segments $|t|=T$ tends to $0$ as $T\to\infty$.\relax\hspace{0pt}
\end{lemma}

\begin{proof}\relax\hspace{0pt}
Use Lemma~\ref{lem:tfc2-vert} and $|H(\sigma+it)|\ll_A (1+|t|)^{-A}$ (Lemma~\ref{lem:tfc2-bdry}) with $A>3$ to bound the horizontal integrals by $O(T^{-2})$.\relax\hspace{0pt}
\end{proof}

% --------------------------------------------------------------- % r24
% 4. E2 → E3: the contour identity % r25
% --------------------------------------------------------------- % r26
\subsection*{4. From kernel trace to contour identity (E\textsubscript{2}→E\textsubscript{3})}\relax\hspace{0pt}
\label{subsec:tfc2-E2E3} % r27

\begin{theorem}[Balanced zeta--trace identity]\relax\hspace{0pt}
\label{thm:tfc2-contour} % r28
For $h\in\mathcal{H}_{\mathrm{PW}}(\sigma,\delta)$ with $H(s)=\widehat{h}(\tfrac12-s)$ and $\Re s=1$,
\begin{equation}\relax\hspace{0pt}
\label{eq:tfc2-contour}\relax\hspace{0pt}
\Tr_{\mathrm{reg}}(K_h)
=\frac{1}{4\pi i}\int_{\Re s=1}\frac{Z'_\Gamma}{Z_\Gamma}(s)\,H(s)\,ds.
\end{equation}
\end{theorem}

\begin{proof}\relax\hspace{0pt}
Start from the kernel-side formula (Part~1, Proposition~\textup{\ref{prop:tfc1-existence}}) and write $h(t)=\int_0^\infty \widehat{h}(u)\cos(ut)\,du$. Interchange integrals using absolute summability on the discrete side (Theorem~\textup{\ref{thm:tfc1-absdisc}}) and the $L^1$-majorant for the scattering integrand (Proposition~\textup{\ref{prop:tfc1-L1}}). Then insert the partial fraction expansion \eqref{eq:tfc2-Zprime} and identify the resulting contour integral with \eqref{eq:tfc2-contour}.\relax\hspace{0pt}
\end{proof}

% --------------------------------------------------------------- % r29
% 5. Contour shift to the critical line and residue accounting % r30
% --------------------------------------------------------------- % r31
\subsection*{5. Shift to $\Re s=\tfrac12$ and residue accounting}\relax\hspace{0pt}
\label{subsec:tfc2-shift} % r32

\begin{proposition}[Shift and residues]\relax\hspace{0pt}
\label{prop:tfc2-shift} % r33
Shifting the contour in \eqref{eq:tfc2-contour} from $\Re s=1$ to $\Re s=\tfrac12$ yields
\[
\Tr_{\mathrm{reg}}(K_h)=\sum_{\rho}\operatorname{Res}_{s=\rho}\!\left(\frac{Z'_\Gamma}{Z_\Gamma}(s)\,H(s)\right)\ +\ \frac{1}{4\pi i}\int_{\Re s=\frac12}\frac{Z'_\Gamma}{Z_\Gamma}(s)\,H(s)\,ds,
\]
where the sum runs over poles and zeros of $Z_\Gamma$ (including trivial ones) crossed during the shift. All horizontal contributions vanish by Lemma~\ref{lem:tfc2-horiz}.\relax\hspace{0pt}
\end{proposition}

\begin{lemma}[Residue contributions]\relax\hspace{0pt}
\label{lem:tfc2-residues} % r34
The residues are:
\begin{enumerate}\relax\hspace{0pt}
\item Discrete spectrum: simple residues at $s=\tfrac12\pm it_j$ contribute $H(\tfrac12\pm it_j)=\widehat{h}(\mp it_j)$. In symmetric grouping this reproduces $\sum_j h(t_j)$.\relax\hspace{0pt}
\item Trivial zeros: poles at $s=-k$ and $s=1+k$ contribute $N_k\,H(-k)$ and $N_k\,H(1+k)$.\relax\hspace{0pt}
\item Polynomial: $P'_\Gamma(s)$ contributes no residues; its integral along vertical lines is absorbed into the constant terms in the geometric side and is pinned by C10.\relax\hspace{0pt}
\item Scattering: the term $(2\pi i)^{-1}(\sigma'/\sigma)(s)$ contributes only through the vertical integral on $\Re s=\tfrac12$, matching the continuous channel in the kernel identity.\relax\hspace{0pt}
\end{enumerate}
\end{lemma}

\begin{proof}[Sketch]\relax\hspace{0pt}
Insert \eqref{eq:tfc2-Zprime} into the residue computation and use the identity $H(s)=\widehat{h}(\tfrac12-s)$ with evenness of $h$ to match discrete terms. Trivial-zero contributions are separated by the explicit partial fractions; see \cite[Ch.~6--7]{HejhalII}.\relax\hspace{0pt}
\end{proof}

% --------------------------------------------------------------- % r35
% 6. Horizontal tails on the critical line % r36
% --------------------------------------------------------------- % r37
\subsection*{6. Horizontal tails on the critical line}\relax\hspace{0pt}
\label{subsec:tfc2-criticaltails} % r38

\begin{proposition}[Tail decay on $\Re s=\tfrac12$]\relax\hspace{0pt}
\label{prop:tfc2-crit-tail} % r39
Along $\Re s=\tfrac12$, the integrand satisfies
\[
\Big|\frac{Z'_\Gamma}{Z_\Gamma}(\tfrac12+it)\,H(\tfrac12+it)\Big|\ \ll\ (1+|t|)^{1+\epsilon}\cdot (1+|t|)^{-A},
\]
for arbitrary $A>0$; hence the integral converges absolutely and the truncation at height $T$ has error $O(T^{-B})$ for any $B>0$.\relax\hspace{0pt}
\end{proposition}

\begin{proof}\relax\hspace{0pt}
Combine Lemma~\ref{lem:tfc2-vert} with Lemma~\ref{lem:tfc2-bdry}.\relax\hspace{0pt}
\end{proof}

% --------------------------------------------------------------- % r40
% 7. E2 ≡ E3 equivalence (closure) % r41
% --------------------------------------------------------------- % r42
\subsection*{7. E\textsubscript{2} $\equiv$ E\textsubscript{3} (closure)}\relax\hspace{0pt}
\label{subsec:tfc2-closure} % r43

\begin{theorem}[E\textsubscript{2}--E\textsubscript{3} equivalence]\relax\hspace{0pt}
\label{thm:tfc2-E2E3} % r44
For all $h\in\mathcal{H}_{\mathrm{PW}}(\sigma,\delta)$,
\[
\Tr_{\mathrm{reg}}(K_h)=\frac{1}{4\pi i}\int_{\Re s=\frac12}\frac{Z'_\Gamma}{Z_\Gamma}(s)\,H(s)\,ds\ +\ \sum_{k\ge0}\Big(N_k\,H(-k)+N_k\,H(1+k)\Big),
\]
with the discrete-spectrum residues absorbed into the vertical integral representation by evenness of $h$. The right-hand side is independent of the truncation path and branch choices fixed in C1.\relax\hspace{0pt}
\end{theorem}

\begin{proof}\relax\hspace{0pt}
Combine Theorem~\ref{thm:tfc2-contour}, Proposition~\ref{prop:tfc2-shift}, Lemma~\ref{lem:tfc2-residues}, and Proposition~\ref{prop:tfc2-crit-tail}. The polynomial term contributes no residues and is fixed by C10.\relax\hspace{0pt}
\end{proof}

% --------------------------------------------------------------- % r45
% 8. Compliance ledger for Part 2/8 % r46
% --------------------------------------------------------------- % r47
\subsection*{8. Compliance ledger for Part 2/8}\relax\hspace{0pt}
\label{subsec:tfc2-compliance} % r48

\noindent
\textbf{C1 (branch):} Remark~\ref{rem:tfc2-branch}. \quad
\textbf{C2 (Plancherel):} Inherited from Part~1 via kernel side. \quad
\textbf{C3 (parameterization):} Appears in \eqref{eq:tfc2-Zprime}. \\
\textbf{C4 (test class):} Definition~\ref{def:tfc2-cos} and Part~1. \quad
\textbf{C6 (growth):} Lemma~\ref{lem:tfc2-vert}. \quad
\textbf{C9 (contour tails):} Lemma~\ref{lem:tfc2-horiz}, Proposition~\ref{prop:tfc2-crit-tail}. \\
\textbf{C10 (polynomial/trivial zeros):} Theorem~\ref{thm:tfc2-Zprime}, Lemma~\ref{lem:tfc2-residues}. \quad
\textbf{C12 (trace regularization):} Uses Part~1 formula in Theorem~\ref{thm:tfc2-contour}.\relax\hspace{0pt}

% --------------------------------------------------------------- % r49
% 9. Forward links to Part 3/8 % r50
% --------------------------------------------------------------- % r51
\subsection*{9. Forward links to Part 3/8}\relax\hspace{0pt}
\label{subsec:tfc2-forward} % r52

\noindent
Part~3/8 proves the full E\textsubscript{1} $\equiv$ E\textsubscript{2} $\equiv$ E\textsubscript{3} equivalence with dominated convergence for Paley--Wiener approximants of wave kernels, pins the dependence on geometric invariants in the polynomial $P_\Gamma$, and formulates deformation-stable trace identities suited for arithmetic specializations.\relax\hspace{0pt}

% --------------------------------------------------------------- % r53
% Bibliographic anchors (resolved in the global .bib file) % r54
% --------------------------------------------------------------- % r55
\begin{thebibliography}{99} % anchors; full entries in global .bib % r56
\bibitem{HejhalII} D.~Hejhal, \emph{The Selberg Trace Formula for $\PSL(2,\mathbb{R})$}, Vol.~2, Springer, 1983. % r57
\bibitem{IwaniecSpectral} H.~Iwaniec, \emph{Spectral Methods of Automorphic Forms}, 2nd ed., AMS, 2002. % r58
\bibitem{Borthwick} D.~Borthwick, \emph{Spectral Theory of Infinite-Area Hyperbolic Surfaces}, Birkhäuser, 2016. % r59
\bibitem{SelbergCollected} A.~Selberg, \emph{Collected Papers}, Vol.~I--II, Springer, 1989. % r60
\end{thebibliography}

% ======================================================================
% End of File: 03-trace-formula-core-part2.tex  % r61
% ======================================================================
% ======================================================================
% File: src/sections/03-trace-formula-core-part3.tex  % r-anchor v1
% Title: Trace Formula Core — Part 3/8: Full Equivalence E1≡E2≡E3
% Version: v1.0.0 (BUILD-ID: TFC-P3-0001) % guard comment
% ARCHETYPE: AFI-1.0 • LATEX_FLOW_BREAKER_v∞.200/100  % invariant tag
% ======================================================================

\section*{Trace Formula Core — Part 3/8: Full Equivalence \texorpdfstring{$\mathrm{E}_1\equiv\mathrm{E}_2\equiv\mathrm{E}_3$}{E1≡E2≡E3}}\relax\hspace{0pt}
\label{sec:tfc-part3} % r1

% --------------------------------------------------------------- % r2
% Scope (Annals style) % r3
% --------------------------------------------------------------- % r4
\noindent
This part proves the full equivalence of the three canonical realizations of the balanced spectral functional: \relax\hspace{0pt}
\[
\mathrm{E}_1:\ \text{spectral sum/integral},\qquad
\mathrm{E}_2:\ \text{(regularized) operator trace of }K_h,\qquad
\mathrm{E}_3:\ \text{contour/zeta identity.}
\]
We fix Paley--Wiener probes $h\in\mathcal{H}_{\mathrm{PW}}(\sigma,\delta)$, use dominated convergence for scattering integrals, and close the Krein vs.\ zeta-regularization bridge. Compliance: C1, C2, C3, C4, C5, C6, C8, C9, C10, C12. References: \cite{HejhalII,IwaniecSpectral,Borthwick,LaxPhillips,Krein,BullaGesztesy}.\relax\hspace{0pt}

% --------------------------------------------------------------- % r5
% 1. Balanced spectral functional (E1) % r6
% --------------------------------------------------------------- % r7
\subsection*{1. The balanced spectral functional (channel \texorpdfstring{$\mathrm{E}_1$}{E1})}\relax\hspace{0pt}
\label{subsec:tfc3-E1} % r8

\begin{definition}[Balanced spectral functional]\relax\hspace{0pt}
\label{def:tfc3-E1} % r9
Let $X=\Gamma\backslash\mathbb{H}$ be of finite area, and let $h\in\mathcal{H}_{\mathrm{PW}}(\sigma,\delta)$ be even. With spectral parameters $\{\lambda_j=\tfrac14+t_j^2\}$ and scattering determinant $\sigma(s)$, define
\begin{equation}\relax\hspace{0pt}
\label{eq:tfc3-E1}\relax\hspace{0pt}
\mathcal{E}_X(h)\ :=\ \sum_{j} h(t_j)\ +\ \frac{1}{4\pi}\int_{\mathbb{R}} h(t)\,\frac{\sigma'}{\sigma}\!\left(\tfrac12+it\right)\,dt.
\end{equation}
On compact $X$, the second term is absent. On finite-area $X$, the sum is absolutely convergent and the integral is Lebesgue integrable; see Part~1 (abs.\ discrete) and Part~2 (majorant for the scattering integrand).\relax\hspace{0pt}
\end{definition}

\begin{lemma}[Absolute summability and uniform integrability]\relax\hspace{0pt}
\label{lem:tfc3-L1} % r10
For $h\in\mathcal{H}_{\mathrm{PW}}(\sigma,\delta)$, $\sum_j |h(t_j)|<\infty$, and there exists $M\in L^1(\mathbb{R})$ with
\[
\big|\,h(t)\,(\sigma'/\sigma)(\tfrac12+it)\,\big|\ \le\ M(t)\ \ \text{for all }t\in\mathbb{R}.
\]
\end{lemma}

\begin{proof}[Sketch]\relax\hspace{0pt}
Use Part~1 (Weyl counting) and Part~2 (Cauchy derivative control) for the discrete sum. For the integral, combine $|h(t)|\ll(1+|t|)^{-2-\delta}$ with $(\sigma'/\sigma)(\tfrac12+it)\ll|t|\log(2+|t|)$ in vertical strips (C6; \cite{IwaniecSpectral,Borthwick}).\relax\hspace{0pt}
\end{proof}

% --------------------------------------------------------------- % r11
% 2. Operator kernel and regularized trace (E2) % r12
% --------------------------------------------------------------- % r13
\subsection*{2. The operator kernel and regularized trace (channel \texorpdfstring{$\mathrm{E}_2$}{E2})}\relax\hspace{0pt}
\label{subsec:tfc3-E2} % r14

\begin{definition}[Spectral calculus kernel]\relax\hspace{0pt}
\label{def:tfc3-Kh} % r15
For $h\in\mathcal{H}_{\mathrm{PW}}(\sigma,\delta)$, set $K_h:=h(\sqrt{\Delta-\tfrac14})$. Its Schwartz kernel on $X$ admits the standard spectral expansion (locally Hilbert--Schmidt on cusp truncations $X_Y$), cf.\ \cite{HejhalII,Borthwick}.\relax\hspace{0pt}
\end{definition}

\begin{definition}[Regularized trace]\relax\hspace{0pt}
\label{def:tfc3-trreg} % r16
Let $X_Y$ be the standard cusp truncation at height $Y$. Define
\begin{equation}\relax\hspace{0pt}
\label{eq:tfc3-trreg}\relax\hspace{0pt}
\Tr_{\mathrm{reg}}(K_h)\ :=\ \lim_{Y\to\infty}\Big(\Tr(K_h|_{X_Y}) - \mathsf{Model}_h(Y)\Big),
\end{equation}
where $\mathsf{Model}_h(Y)$ is the model subtraction determined by the Maaß--Selberg relations and the fixed branch of $\log\sigma$ (C1); it is affine in $Y$ with coefficients depending on $\widehat{h}$ and geometric data (cf.\ \cite{LaxPhillips,Borthwick}).\relax\hspace{0pt}
\end{definition}

\begin{proposition}[Existence and balance]\relax\hspace{0pt}
\label{prop:tfc3-existence} % r17
For $h\in\mathcal{H}_{\mathrm{PW}}(\sigma,\delta)$, the limit in \eqref{eq:tfc3-trreg} exists and equals the balanced functional:
\[
\Tr_{\mathrm{reg}}(K_h)=\mathcal{E}_X(h).
\]
\end{proposition}

\begin{proof}[Sketch]\relax\hspace{0pt}
On $X_Y$, the trace equals the discrete spectral sum plus the Eisenstein contribution. Subtract the cusp-asymptotic model from Maaß--Selberg to remove the linear (in $Y$) divergence; then let $Y\to\infty$ and invoke Lemma~\ref{lem:tfc3-L1} for integrability. This matches \eqref{eq:tfc3-E1}. See \cite{HejhalII,Borthwick,LaxPhillips}.\relax\hspace{0pt}
\end{proof}

% --------------------------------------------------------------- % r18
% 3. Zeta/contour representation (E3) and E2↔E3 bridge % r19
% --------------------------------------------------------------- % r20
\subsection*{3. Zeta/contour representation (channel \texorpdfstring{$\mathrm{E}_3$}{E3}) and the \texorpdfstring{$\mathrm{E}_2\leftrightarrow \mathrm{E}_3$}{E2↔E3} bridge}\relax\hspace{0pt}
\label{subsec:tfc3-E3} % r21

\begin{theorem}[Balanced zeta--trace identity]\relax\hspace{0pt}
\label{thm:tfc3-contour} % r22
With $H(s)=\widehat{h}(\tfrac12-s)$, one has
\begin{equation}\relax\hspace{0pt}
\label{eq:tfc3-contour}\relax\hspace{0pt}
\Tr_{\mathrm{reg}}(K_h)\ =\ \frac{1}{4\pi i}\int_{\Re s=1}\frac{Z'_\Gamma}{Z_\Gamma}(s)\,H(s)\,ds,
\end{equation}
and, after shifting to $\Re s=\tfrac12$ with residue accounting for discrete and trivial contributions,
\begin{equation}\relax\hspace{0pt}
\label{eq:tfc3-contour-crit}\relax\hspace{0pt}
\Tr_{\mathrm{reg}}(K_h)\ =\ \frac{1}{4\pi i}\int_{\Re s=\frac12}\frac{Z'_\Gamma}{Z_\Gamma}(s)\,H(s)\,ds\ +\ \sum_{k\ge 0}\big(N_k\,H(-k)+N_k\,H(1+k)\big).
\end{equation}
\end{theorem}

\begin{proof}[Sketch]\relax\hspace{0pt}
Part~2 establishes \eqref{eq:tfc3-contour} and the tail control for the shift. Residues from $s=\tfrac12\pm it_j$ match the discrete summand, while trivial zeros contribute $N_k$-terms; the polynomial $P_\Gamma$ contributes no residues (C10) and is absorbed by normalization.\relax\hspace{0pt}
\end{proof}

% --------------------------------------------------------------- % r22
% 4. Dominated convergence and wave-probe approximation % r23
% --------------------------------------------------------------- % r24
\subsection*{4. Dominated convergence and wave-kernel approximation}\relax\hspace{0pt}
\label{subsec:tfc3-DC} % r25

\begin{lemma}[Paley--Wiener approximation of the wave probe]\relax\hspace{0pt}
\label{lem:tfc3-wave-approx} % r26
For $T>0$, there exists a sequence $h_T^{(n)}\in\mathcal{H}_{\mathrm{PW}}(\sigma_n,\delta)$ such that $h_T^{(n)}(t)\to \cos(Tt)$ locally uniformly and
\[
K_{h_T^{(n)}}\ \to\ \cos\!\big(T\sqrt{\Delta-\tfrac14}\big) \quad\text{strongly on }L^2(X).
\]
\end{lemma}

\begin{proof}[Sketch]\relax\hspace{0pt}
Choose $\widehat{h_T^{(n)}}\in C_c^\infty([-R_n,R_n])$ with $\widehat{h_T^{(n)}}\to \tfrac12(\delta_{u=T}+\delta_{u=-T})$ in the distributional sense and invoke the spectral theorem for strong convergence.\relax\hspace{0pt}
\end{proof}

\begin{proposition}[Dominated convergence in the scattering channel]\relax\hspace{0pt}
\label{prop:tfc3-DC} % r27
Let $h_n\to h$ pointwise with $h_n,h\in\mathcal{H}_{\mathrm{PW}}(\sigma,\delta)$ and suppose $|h_n(t)|\le C(1+|t|)^{-2-\delta}$. Then
\[
\lim_{n\to\infty} \frac{1}{4\pi}\int_{\mathbb{R}} h_n(t)\,\frac{\sigma'}{\sigma}\!\left(\tfrac12+it\right)\,dt
=\frac{1}{4\pi}\int_{\mathbb{R}} h(t)\,\frac{\sigma'}{\sigma}\!\left(\tfrac12+it\right)\,dt.
\]
\end{proposition}

\begin{proof}\relax\hspace{0pt}
Apply Lemma~\ref{lem:tfc3-L1} to get an $L^1$ majorant independent of $n$ and use dominated convergence.\relax\hspace{0pt}
\end{proof}

\begin{theorem}[Wave-trace limit (balanced)]\relax\hspace{0pt}
\label{thm:tfc3-wave-trace} % r28
With $h_T^{(n)}$ from Lemma~\ref{lem:tfc3-wave-approx}, 
\[
\lim_{n\to\infty}\Tr_{\mathrm{reg}}(K_{h_T^{(n)}})\ =\ \Tr_{\mathrm{reg}}\!\left(\cos\!\big(T\sqrt{\Delta-\tfrac14}\big)\right),
\]
and the limiting value equals both the balanced spectral sum/integral in \eqref{eq:tfc3-E1} with $h(t)=\cos(Tt)$ interpreted via the spectral theorem and the contour expression \eqref{eq:tfc3-contour-crit} with $H(s)=\widehat{h}(\tfrac12-s)$ realized as the distributional cosine transform at frequency $T$.\relax\hspace{0pt}
\end{theorem}

\begin{proof}[Sketch]\relax\hspace{0pt}
Use strong convergence (Lemma~\ref{lem:tfc3-wave-approx}) for the operator and Proposition~\ref{prop:tfc3-DC} for the scattering channel. The discrete channel is trivially dominated by absolute summability.\relax\hspace{0pt}
\end{proof}

% --------------------------------------------------------------- % r29
% 5. Krein- vs zeta-regularization bridge % r30
% --------------------------------------------------------------- % r31
\subsection*{5. Krein vs.\ zeta: equivalence of regularizations}\relax\hspace{0pt}
\label{subsec:tfc3-Krein-zeta} % r32

\begin{definition}[Krein spectral shift set-up]\relax\hspace{0pt}
\label{def:tfc3-Krein} % r33
Let $H$ be the geometric Laplacian on $X$ in the cusp-truncated sense and $H_0$ the model operator capturing the cusp asymptotics. The spectral shift function $\xi(\lambda)$ is defined so that
\[
\Tr\big(\phi(H)-\phi(H_0)\big)\ =\ \int_{\mathbb{R}}\phi'(\lambda)\,\xi(\lambda)\,d\lambda,
\]
for test functions $\phi$; see \cite{Krein,BullaGesztesy}. In our case $H_0$ is fixed by C1 and the Maaß--Selberg model.\relax\hspace{0pt}
\end{definition}

\begin{theorem}[Krein $\equiv$ zeta regularization for $K_h$]\relax\hspace{0pt}
\label{thm:tfc3-KreinZeta} % r34
For $h\in\mathcal{H}_{\mathrm{PW}}(\sigma,\delta)$,
\[
\Tr_{\mathrm{reg}}(K_h)\ =\ \Tr\big(K_h(H)-K_h(H_0)\big)\ =\ \frac{1}{4\pi i}\int_{\Re s=1}\frac{Z'_\Gamma}{Z_\Gamma}(s)\,H(s)\,ds,
\]
with $H(s)=\widehat{h}(\tfrac12-s)$ as in Theorem~\ref{thm:tfc3-contour}. Thus the zeta/contour and Krein model-subtraction regularizations coincide.\relax\hspace{0pt}
\end{theorem}

\begin{proof}[Sketch]\relax\hspace{0pt}
Choose $\phi(\lambda)=\int_0^\infty \widehat{h}(u)\cos\!\big(u\sqrt{\lambda-\tfrac14}\big)\,du$ so that $\phi(H)=K_h$. The Krein relation yields the balanced difference $\Tr(K_h(H)-K_h(H_0))$, which equals the zeta/contour expression by the same residue and tail analysis as in Part~2. For abstract background see \cite{Krein,BullaGesztesy}.\relax\hspace{0pt}
\end{proof}

% --------------------------------------------------------------- % r35
% 6. Full equivalence E1 ≡ E2 ≡ E3 % r36
% --------------------------------------------------------------- % r37
\subsection*{6. Full equivalence \texorpdfstring{$\mathrm{E}_1\equiv\mathrm{E}_2\equiv\mathrm{E}_3$}{E1≡E2≡E3}}\relax\hspace{0pt}
\label{subsec:tfc3-full} % r38

\begin{theorem}[Equivalence theorem]\relax\hspace{0pt}
\label{thm:tfc3-equivalence} % r39
For every $h\in\mathcal{H}_{\mathrm{PW}}(\sigma,\delta)$,
\[
\mathcal{E}_X(h)\ =\ \Tr_{\mathrm{reg}}(K_h)\ =\ \frac{1}{4\pi i}\int_{\Re s=1}\frac{Z'_\Gamma}{Z_\Gamma}(s)\,\widehat{h}(\tfrac12-s)\,ds,
\]
and, after the admissible contour shift,
\[
\mathcal{E}_X(h)\ =\ \frac{1}{4\pi i}\int_{\Re s=\frac12}\frac{Z'_\Gamma}{Z_\Gamma}(s)\,\widehat{h}(\tfrac12-s)\,ds\ +\ \sum_{k\ge0}\big(N_k\,\widehat{h}(-k)+N_k\,\widehat{h}(1+k)\big).
\]
All identities respect the fixed branch (C1), Plancherel normalization (C2), spectral parameterization (C3), and polynomial/trivial-zero structure (C10).\relax\hspace{0pt}
\end{theorem}

\begin{proof}\relax\hspace{0pt}
Combine Proposition~\ref{prop:tfc3-existence}, Theorem~\ref{thm:tfc3-contour}, and Theorem~\ref{thm:tfc3-KreinZeta}, together with the tail and residue control from Part~2.\relax\hspace{0pt}
\end{proof}

% --------------------------------------------------------------- % r40
% 7. Stability: isometries, deformations, coverings % r41
% --------------------------------------------------------------- % r42
\subsection*{7. Stability under isometries, deformations, and coverings}\relax\hspace{0pt}
\label{subsec:tfc3-stability} % r43

\begin{proposition}[Isometric invariance]\relax\hspace{0pt}
\label{prop:tfc3-iso} % r44
If $U:L^2(X)\to L^2(X')$ is unitary with $U\Delta_X=\Delta_{X'}U$ intertwining Eisenstein data and the branch choice for $\log\sigma$, then for all $h\in\mathcal{H}_{\mathrm{PW}}(\sigma,\delta)$,
\[
\mathcal{E}_X(h)=\mathcal{E}_{X'}(h).
\]
\end{proposition}

\begin{proof}[Sketch]\relax\hspace{0pt}
Both $\{t_j\}$ and $\sigma$ are invariants of the unitary equivalence; the balanced functional is therefore invariant.\relax\hspace{0pt}
\end{proof}

\begin{proposition}[Deformation stability]\relax\hspace{0pt}
\label{prop:tfc3-def} % r45
Under a smooth geometric deformation preserving finite area and cusp type, the map $X\mapsto\mathcal{E}_X(h)$ is continuous for fixed $h\in\mathcal{H}_{\mathrm{PW}}(\sigma,\delta)$; local differentiability holds away from eigenvalue crossings.\relax\hspace{0pt}
\end{proposition}

\begin{proof}[Sketch]\relax\hspace{0pt}
Use Kato--Rellich perturbation theory for the discrete part and smooth dependence of $\sigma$ on the geometry for the continuous part; uniform integrability controls the scattering channel.\relax\hspace{0pt}
\end{proof}

\begin{proposition}[Finite coverings]\relax\hspace{0pt}
\label{prop:tfc3-cover} % r46
If $\pi:\widetilde{X}\to X$ is a finite covering of degree $d$, then for $h\in\mathcal{H}_{\mathrm{PW}}(\sigma,\delta)$,
\[
\mathcal{E}_{\widetilde{X}}(h)=d\,\mathcal{E}_X(h)\ +\ \text{(lower-order terms depending only on cusp data and }P_\Gamma).
\]
\end{proposition}

\begin{proof}[Sketch]\relax\hspace{0pt}
Use multiplicativity of lengths and factorization properties of $Z_\Gamma$ under coverings, together with the normalization of $\widehat{h}$; cf.\ \cite{HejhalII}.\relax\hspace{0pt}
\end{proof}

% --------------------------------------------------------------- % r47
% 8. Compliance ledger for Part 3/8 % r48
% --------------------------------------------------------------- % r49
\subsection*{8. Compliance ledger for Part 3/8}\relax\hspace{0pt}
\label{subsec:tfc3-compliance} % r50

\noindent
\textbf{C1 (branch):} fixed in Part~1, used in Def.~\ref{def:tfc3-trreg}, Thm.~\ref{thm:tfc3-contour}. \quad
\textbf{C2 (Plancherel):} inherited from kernel and contour normalizations. \\
\textbf{C3 (parameterization):} throughout, eqs.~\eqref{eq:tfc3-E1}, \eqref{eq:tfc3-contour-crit}. \quad
\textbf{C4 (test class):} Def.~\ref{def:tfc3-Kh}, Lem.~\ref{lem:tfc3-wave-approx}. \\
\textbf{C5 (balance):} Def.~\ref{def:tfc3-E1}, Prop.~\ref{prop:tfc3-existence}. \quad
\textbf{C6 (growth):} Lem.~\ref{lem:tfc3-L1}. \quad
\textbf{C8 (dominated convergence):} Prop.~\ref{prop:tfc3-DC}. \\
\textbf{C9 (tails):} Part~2 tail lemmas applied in Thm.~\ref{thm:tfc3-contour}. \quad
\textbf{C10 (polynomial/trivial zeros):} Thm.~\ref{thm:tfc3-contour}. \quad
\textbf{C12 (trace regularization):} Def.~\ref{def:tfc3-trreg}, Thm.~\ref{thm:tfc3-KreinZeta}.\relax\hspace{0pt}

% --------------------------------------------------------------- % r51
% Bibliographic anchors (resolved in the global .bib file) % r52
% --------------------------------------------------------------- % r53
\begin{thebibliography}{99} % anchors; full entries in global .bib % r54
\bibitem{HejhalII} D.~Hejhal, \emph{The Selberg Trace Formula for $\PSL(2,\mathbb{R})$}, Vol.~2, Springer, 1983. % r55
\bibitem{IwaniecSpectral} H.~Iwaniec, \emph{Spectral Methods of Automorphic Forms}, 2nd ed., AMS, 2002. % r56
\bibitem{Borthwick} D.~Borthwick, \emph{Spectral Theory of Infinite-Area Hyperbolic Surfaces}, Birkhäuser, 2016. % r57
\bibitem{LaxPhillips} P.~D.~Lax and R.~S.~Phillips, \emph{Scattering Theory for Automorphic Functions}, Princeton Univ.\ Press, 1976. % r58
\bibitem{Krein} M.~G.~Krein, \emph{On the trace formula in perturbation theory}, Mat.\ Sbornik (N.S.) \textbf{33} (1953), 597–626. % r59
\bibitem{BullaGesztesy} W.~Bulla and F.~Gesztesy, \emph{Deficiency indices and Wronskians for Sturm–Liouville operators}, J.\ Diff.\ Eq.\ \textbf{103} (1993), 230–265. % r60
\end{thebibliography}

% ======================================================================
% End of File: 03-trace-formula-core-part3.tex  % r61
% ======================================================================
% ======================================================================
% File: src/sections/03-trace-formula-core-part4.tex  % r-anchor v1
% Title: Trace Formula Core — Part 4/8: Geometric Side and Orbital Integrals
% Version: v1.0.0 (BUILD-ID: TFC-P4-0001) % guard comment
% ARCHETYPE: AFI-1.0 • LATEX_FLOW_BREAKER_v∞.200/100  % invariant tag
% ======================================================================

\section*{Trace Formula Core — Part 4/8: Geometric Side and Orbital Integrals}\relax\hspace{0pt}
\label{sec:tfc-part4} % r1

% --------------------------------------------------------------- % r2
% Scope (Annals style) % r3
% --------------------------------------------------------------- % r4
\noindent
This part develops the geometric side of the trace formula for cofinite Fuchsian groups $\Gamma\subset\PSL_2(\mathbb{R})$. \relax\hspace{0pt}
We give precise normalizations of orbital integrals for the hyperbolic, elliptic, and parabolic conjugacy classes, including cusp-regularized contributions and the treatment of the identity. \relax\hspace{0pt}
Compliance: C1 (branch), C2 (Plancherel), C3 (parametrization), C4 (test class), C5 (balance), C10 (polynomial/trivial zeros), C12 (trace regularization). \relax\hspace{0pt}
References: \cite{SelbergCollected,HejhalI,HejhalII,IwaniecSpectral,Borthwick,LaxPhillips}. % r5

% --------------------------------------------------------------- % r6
% 1. Test functions and spherical kernel % r7
% --------------------------------------------------------------- % r8
\subsection*{1. Test functions and spherical kernel}\relax\hspace{0pt}
\label{subsec:tfc4-kernel} % r9

Let $h\in\mathcal{H}_{\mathrm{PW}}(\sigma,\delta)$ be even and of exponential type $R>0$. \relax\hspace{0pt}
Let $k=k_h$ be the associated bi-$K$-invariant kernel on $\mathbb{H}$ with spherical transform $h$, i.e.\ for $z,w\in\mathbb{H}$ with hyperbolic distance $d(z,w)=r$,
\begin{equation}\relax\hspace{0pt}
\label{eq:tfc4-kernel}\relax\hspace{0pt}
k_h(r)\ =\ \frac{1}{2\pi}\int_{\mathbb{R}} h(t)\,\varphi_t(r)\,t\tanh(\pi t)\,dt,
\end{equation}
where $\varphi_t$ is the spherical function on $\PSL_2(\mathbb{R})/K$ (with $K=\SO(2)$) normalized by $\varphi_t(0)=1$; see \cite{Helgason,GangolliVaradarajan,HejhalI}. \relax\hspace{0pt}

\begin{lemma}[Decay and support profile of $k_h$]\relax\hspace{0pt}
\label{lem:tfc4-kprofile} % r10
If $h\in\mathcal{H}_{\mathrm{PW}}(\sigma,\delta)$ has exponential type $R$, then $k_h$ is smooth and rapidly decaying as $r\to\infty$; moreover $k_h$ is effectively supported at geodesic distances $\ll R$ in the sense that all derivatives $k_h^{(m)}(r)$ decay faster than any power as $r/R\to\infty$.\relax\hspace{0pt}
\end{lemma}

\begin{proof}[Sketch]\relax\hspace{0pt}
Paley--Wiener control on $h$ yields compact support of the corresponding cosine transform; the spherical inversion then implies rapid off-support decay (cf.\ \cite{GangolliVaradarajan,Helgason}).\relax\hspace{0pt}
\end{proof}

% --------------------------------------------------------------- % r11
% 2. Geometric trace and decomposition by conjugacy classes % r12
% --------------------------------------------------------------- % r13
\subsection*{2. Geometric trace and class decomposition}\relax\hspace{0pt}
\label{subsec:tfc4-geom-sum} % r14

Let $\Gamma\subset\PSL_2(\mathbb{R})$ be discrete, cofinite, with $\kappa$ cusps. \relax\hspace{0pt}
The $\Gamma$-periodization of $k_h$ defines the automorphic kernel
\begin{equation}\relax\hspace{0pt}
\label{eq:tfc4-autom-kernel}\relax\hspace{0pt}
K_h(z,w)\ :=\ \sum_{\gamma\in\Gamma} k_h\!\big(d(z,\gamma w)\big),
\end{equation}
which is absolutely and uniformly convergent on compacta. \relax\hspace{0pt}
The geometric trace on a cusp-truncation $X_Y$ (height $Y$) reads
\begin{equation}\relax\hspace{0pt}
\label{eq:tfc4-geom-trunc}\relax\hspace{0pt}
\Tr\!\big(K_h|_{X_Y}\big)\ =\ \int_{F_Y}\,K_h(z,z)\,d\mu(z)\ =\ \sum_{\{\gamma\}}\ \mathrm{OI}_Y(\gamma;h),
\end{equation}
where the sum runs over $\Gamma$-conjugacy classes and $\mathrm{OI}_Y(\gamma;h)$ denotes the truncated orbital integral for the class $\{\gamma\}$. \relax\hspace{0pt}
The decomposition splits into the identity, elliptic, hyperbolic (primitive and imprimitive), and parabolic classes (cusps). \relax\hspace{0pt}

\begin{definition}[Orbital integrals]\relax\hspace{0pt}
\label{def:tfc4-OI} % r15
For $\gamma\in\Gamma$ let
\[
\mathrm{OI}_Y(\gamma;h)\ :=\ \int_{F_Y}\ \sum_{\eta\in[\gamma]}\ k_h\!\big(d(z,\eta z)\big)\,d\mu(z),
\]
where $[\gamma]$ is the conjugacy class of $\gamma$ in $\Gamma$. \relax\hspace{0pt}
For $\gamma$ hyperbolic with translation length $\ell(\gamma)$ we write $\ell(\{\gamma\})$ for the primitive length. \relax\hspace{0pt}
\end{definition}

% --------------------------------------------------------------- % r16
% 3. Identity contribution % r17
% --------------------------------------------------------------- % r18
\subsection*{3. Identity contribution}\relax\hspace{0pt}
\label{subsec:tfc4-identity} % r19

\begin{proposition}[Identity term]\relax\hspace{0pt}
\label{prop:tfc4-id} % r20
Let $X=\Gamma\backslash\mathbb{H}$ with hyperbolic area $\vol(X)$. Then
\begin{equation}\relax\hspace{0pt}
\label{eq:tfc4-id}\relax\hspace{0pt}
\mathrm{OI}(\mathrm{id};h)\ :=\ \lim_{Y\to\infty}\int_{F_Y} k_h(0)\,d\mu\ =\ \vol(X)\cdot k_h(0),
\end{equation}
with
\begin{equation}\relax\hspace{0pt}
\label{eq:tfc4-k0}\relax\hspace{0pt}
k_h(0)\ =\ \frac{1}{2\pi}\int_{\mathbb{R}} h(t)\,t\tanh(\pi t)\,dt.
\end{equation}
\end{proposition}

\begin{proof}[Sketch]\relax\hspace{0pt}
The automorphic kernel at $\gamma=\mathrm{id}$ equals $k_h(0)$; integrate over the truncated fundamental domain and pass to the limit, using finiteness of $\vol(X)$. \relax\hspace{0pt}
Equation \eqref{eq:tfc4-k0} follows directly from \eqref{eq:tfc4-kernel} with $r=0$ and $\varphi_t(0)=1$.\relax\hspace{0pt}
\end{proof}

% --------------------------------------------------------------- % r21
% 4. Elliptic classes % r22
% --------------------------------------------------------------- % r23
\subsection*{4. Elliptic classes}\relax\hspace{0pt}
\label{subsec:tfc4-elliptic} % r24

Suppose $\{\gamma\}$ is elliptic of order $m\ge 2$ with fixed point $z_\gamma\in\mathbb{H}$. \relax\hspace{0pt}
Let $\theta_\gamma=2\pi/m$. \relax\hspace{0pt}
Then (cf.\ \cite{HejhalI,HejhalII})
\begin{equation}\relax\hspace{0pt}
\label{eq:tfc4-elliptic}\relax\hspace{0pt}
\mathrm{OI}(\gamma;h)\ =\ \frac{1}{2m\,\sin(\theta_\gamma/2)} \int_{0}^{\infty} k_h(r)\, \frac{\sinh r}{\cosh r-\cos\theta_\gamma}\,dr.
\end{equation}
The total elliptic contribution is the sum of \eqref{eq:tfc4-elliptic} over representatives of elliptic classes modulo conjugacy. \relax\hspace{0pt}

\begin{lemma}[Spherical transform form]\relax\hspace{0pt}
\label{lem:tfc4-elliptic-sph} % r25
Equivalently, one may write
\[
\mathrm{OI}(\gamma;h)\ =\ \sum_{\ell\ge 0} c_\ell(\gamma)\,h(i(\ell+\tfrac12)),
\]
for coefficients $c_\ell(\gamma)$ determined by the character expansion of the elliptic orbital integral; the series converges absolutely for $h\in\mathcal{H}_{\mathrm{PW}}(\sigma,\delta)$.\relax\hspace{0pt}
\end{lemma}

\begin{proof}[Sketch]\relax\hspace{0pt}
Use the $K$-type expansion of $k_h$ and standard harmonic analysis on $\PSL_2(\mathbb{R})$; cf.\ \cite{GangolliVaradarajan,HejhalI}.\relax\hspace{0pt}
\end{proof}

% --------------------------------------------------------------- % r26
% 5. Hyperbolic classes and the length spectrum % r27
% --------------------------------------------------------------- % r28
\subsection*{5. Hyperbolic classes and the length spectrum}\relax\hspace{0pt}
\label{subsec:tfc4-hyp} % r29

Let $\{\gamma\}$ be hyperbolic with translation length $\ell(\gamma)>0$ and primitive length $\ell(p)$ if $\gamma=p^n$, $n\ge 1$. \relax\hspace{0pt}
Then the orbital integral equals (\cite{SelbergCollected,HejhalI})
\begin{equation}\relax\hspace{0pt}
\label{eq:tfc4-hyp-OI}\relax\hspace{0pt}
\mathrm{OI}(\gamma;h)\ =\ \frac{1}{2\sinh(\ell(\gamma)/2)}\,\widehat{g}\!\left(\ell(\gamma)\right),
\end{equation}
where $g$ is the Harish--Chandra pre-image of $h$ and $\widehat{g}$ denotes the corresponding ``length-transform'' (Fourier--Abel type) supported at $\ell(\gamma)$.\relax\hspace{0pt}
Equivalently, summing over the powers of a primitive hyperbolic $p$,
\begin{equation}\relax\hspace{0pt}
\label{eq:tfc4-hyp-sum}\relax\hspace{0pt}
\sum_{n=1}^{\infty} \mathrm{OI}(p^n;h)\ =\ \sum_{n=1}^{\infty}\frac{1}{2\sinh(n\ell(p)/2)}\,\widehat{g}\!\big(n\ell(p)\big).
\end{equation}

\begin{definition}[Primitive length spectrum]\relax\hspace{0pt}
\label{def:tfc4-length}\relax\hspace{0pt}
Let $\mathcal{L}_{\mathrm{prim}}$ be the multiset of primitive lengths $\ell(p)$ of closed geodesics on $X$. \relax\hspace{0pt}
The total hyperbolic contribution is
\begin{equation}\relax\hspace{0pt}
\label{eq:tfc4-hyp-total}\relax\hspace{0pt}
\mathrm{Hyp}(h)\ :=\ \sum_{\ell\in\mathcal{L}_{\mathrm{prim}}}\ \sum_{n=1}^{\infty}\frac{1}{2\sinh(n\ell/2)}\,\widehat{g}\!\big(n\ell\big).
\end{equation}
\end{definition}

\begin{lemma}[Absolute convergence]\relax\hspace{0pt}
\label{lem:tfc4-hyp-abs} % r30
For $h\in\mathcal{H}_{\mathrm{PW}}(\sigma,\delta)$ the series \eqref{eq:tfc4-hyp-total} is absolutely convergent. \relax\hspace{0pt}
\end{lemma}

\begin{proof}[Sketch]\relax\hspace{0pt}
Rapid decay of $\widehat{g}$ and the exponential growth of $\sinh(n\ell/2)$ yield summability uniformly in $\ell$. \relax\hspace{0pt}
Combine with prime geodesic counting estimates; cf.\ \cite{HejhalI,IwaniecSpectral,Borthwick}.\relax\hspace{0pt}
\end{proof}

% --------------------------------------------------------------- % r31
% 6. Parabolic (cusp) classes and truncation % r32
% --------------------------------------------------------------- % r33
\subsection*{6. Parabolic classes and cusp-regularized terms}\relax\hspace{0pt}
\label{subsec:tfc4-par} % r34

For a cusp associated with the stabilizer $\Gamma_\mathfrak{a}\cong \mathbb{Z}$, conjugacy classes of parabolics contribute divergences in the geometric trace; truncation at height $Y$ yields
\begin{equation}\relax\hspace{0pt}
\label{eq:tfc4-par-Y}\relax\hspace{0pt}
\mathrm{Par}_Y(h)\ =\ A(h)\,Y\ +\ B(h)\ +\ o(1)\qquad (Y\to\infty),
\end{equation}
where $A(h)$ and $B(h)$ depend linearly on $\widehat{h}$ and on cusp width data; see \cite{HejhalII,Borthwick}. \relax\hspace{0pt}
Subtracting the model term $A(h)Y$ defines the regularized parabolic contribution
\begin{equation}\relax\hspace{0pt}
\label{eq:tfc4-par-reg}\relax\hspace{0pt}
\mathrm{Par}_{\mathrm{reg}}(h)\ :=\ \lim_{Y\to\infty}\big(\mathrm{Par}_Y(h)-A(h)Y\big)\ =\ B(h).
\end{equation}

\begin{proposition}[Explicit $A(h)$ and $B(h)$]\relax\hspace{0pt}
\label{prop:tfc4-par-AB} % r35
Let $\widehat{h}$ be the cosine transform of $h$. Then
\[
A(h)\ =\ \frac{\kappa}{2\pi}\,\int_{0}^{\infty}\widehat{h}(u)\,du,\qquad
B(h)\ =\ \frac{1}{2\pi}\int_{-\infty}^{\infty}h(t)\,\Psi(t)\,dt,
\]
where $\Psi(t)$ depends on the logarithmic derivative of the scattering determinant and cusp normalizations (fixed by C1--C2); concretely,
\[
\Psi(t)\ =\ \Re\Big(\psi\!\big(\tfrac12+it\big)\Big)\ -\ \log(2\pi)\ +\ \text{\emph{(cusp width contributions)}},
\]
with $\psi=\Gamma'/\Gamma$ and the cusp terms tabulated in \cite{HejhalII,Borthwick}. \relax\hspace{0pt}
\end{proposition}

\begin{proof}[Sketch]\relax\hspace{0pt}
Apply the Rankin--Selberg unfolding on the cusp sector and evaluate the $Y$-asymptotics via Maaß--Selberg relations; identify $A(h)$ as the coefficient of $Y$ and $B(h)$ as the constant term. \relax\hspace{0pt}
\end{proof}

% --------------------------------------------------------------- % r36
% 7. The geometric side: assembled statement % r37
% --------------------------------------------------------------- % r38
\subsection*{7. Geometric side: assembled statement}\relax\hspace{0pt}
\label{subsec:tfc4-assembled} % r39

\begin{theorem}[Geometric expansion]\relax\hspace{0pt}
\label{thm:tfc4-geom}\relax\hspace{0pt}
For $h\in\mathcal{H}_{\mathrm{PW}}(\sigma,\delta)$ one has
\begin{equation}\relax\hspace{0pt}
\label{eq:tfc4-geom-exp}\relax\hspace{0pt}
\Tr_{\mathrm{reg}}(K_h)\ =\ \underbrace{\vol(X)\,k_h(0)}_{\text{\rm identity}}\ +\ \underbrace{\sum_{\{\gamma\}_{\mathrm{ell}}}\mathrm{OI}(\gamma;h)}_{\text{\rm elliptic}}\ +\ \underbrace{\sum_{\{\gamma\}_{\mathrm{hyp}}}\mathrm{OI}(\gamma;h)}_{\text{\rm hyperbolic}}\ +\ \underbrace{\mathrm{Par}_{\mathrm{reg}}(h)}_{\text{\rm parabolic}},
\end{equation}
with each term given by \eqref{eq:tfc4-k0}, \eqref{eq:tfc4-elliptic}, \eqref{eq:tfc4-hyp-total}, and \eqref{eq:tfc4-par-reg} respectively. \relax\hspace{0pt}
All series are absolutely convergent and the parabolic part is regularized by subtracting $A(h)Y$ in the truncation formalism \eqref{eq:tfc4-par-Y}. \relax\hspace{0pt}
\end{theorem}

\begin{proof}[Sketch]\relax\hspace{0pt}
Combine the class-by-class orbital integral formulas with the truncation analysis and pass $Y\to\infty$. \relax\hspace{0pt}
Absolute convergence follows from Lemmas~\ref{lem:tfc4-kprofile}, \ref{lem:tfc4-hyp-abs}; the parabolic regularization from Proposition~\ref{prop:tfc4-par-AB}. \relax\hspace{0pt}
\end{proof}

% --------------------------------------------------------------- % r40
% 8. Consistency with the spectral and zeta sides % r41
% --------------------------------------------------------------- % r42
\subsection*{8. Consistency with spectral and zeta sides}\relax\hspace{0pt}
\label{subsec:tfc4-consistency} % r43

\begin{proposition}[Spectral = geometric]\relax\hspace{0pt}
\label{prop:tfc4-spec-geom} % r44
With $\mathcal{E}_X(h)$ as in \eqref{eq:tfc4-geom-exp} and Definition~\eqref{eq:tfc4-kernel}, one has
\[
\mathcal{E}_X(h)\ =\ \Tr_{\mathrm{reg}}(K_h)\ \text{(geometric expansion)}\ =\ \text{(spectral sum + scattering integral)},
\]
i.e.\ Theorem~\ref{thm:tfc4-geom} equals the channel $\mathrm{E}_2$ and matches $\mathrm{E}_1$ from Part~3. \relax\hspace{0pt}
\end{proposition}

\begin{proof}[Sketch]\relax\hspace{0pt}
Identify the identity term with the Plancherel integral at $r=0$, the elliptic and hyperbolic terms with the discrete spectral/length-side contributions, and the parabolic term with the scattering channel via Maaß--Selberg; cf.\ \cite{HejhalI,HejhalII}. \relax\hspace{0pt}
\end{proof}

\begin{proposition}[Zeta = geometric]\relax\hspace{0pt}
\label{prop:tfc4-zeta-geom} % r45
The zeta/contour identity (channel $\mathrm{E}_3$) from Part~3 equals \eqref{eq:tfc4-geom-exp} upon identifying the logarithmic derivative $\frac{Z_\Gamma'}{Z_\Gamma}$ with the hyperbolic sum via the Euler product, the polynomial/trivial zero parts with the elliptic/parabolic constants, and the identity part with $k_h(0)$ times volume. \relax\hspace{0pt}
\end{proposition}

\begin{proof}[Sketch]\relax\hspace{0pt}
Unfold the Euler product of $Z_\Gamma$ to the length spectrum and compare residues/tails as in \cite{SelbergCollected,HejhalII,IwaniecSpectral}. \relax\hspace{0pt}
\end{proof}

% --------------------------------------------------------------- % r46
% 9. Compliance ledger for Part 4/8 % r47
% --------------------------------------------------------------- % r48
\subsection*{9. Compliance ledger for Part 4/8}\relax\hspace{0pt}
\label{subsec:tfc4-compliance} % r49

\noindent
\textbf{C1 (branch):} fixed in cusp data and scattering normalization; used in \eqref{eq:tfc4-par-reg}, Prop.~\ref{prop:tfc4-par-AB}. \quad
\textbf{C2 (Plancherel):} \eqref{eq:tfc4-kernel} and $k_h(0)$ in \eqref{eq:tfc4-k0}. \\
\textbf{C3 (parameterization):} inherited from Part~1--3; hyperbolic terms use $\ell(\gamma)$. \quad
\textbf{C4 (test class):} Paley--Wiener conditions drive decay and absolute convergence. \\
\textbf{C5 (balance):} parabolic regularization \eqref{eq:tfc4-par-reg} aligns with scattering. \quad
\textbf{C10 (polynomial/trivial zeros):} encoded in elliptic and parabolic constants (Prop.~\ref{prop:tfc4-par-AB}). \\
\textbf{C12 (trace regularization):} truncation model $A(h)Y$ and limit $Y\to\infty$ in \eqref{eq:tfc4-par-Y}--\eqref{eq:tfc4-par-reg}.\relax\hspace{0pt}

% --------------------------------------------------------------- % r50
% Bibliographic anchors (resolved in the global .bib file) % r51
% --------------------------------------------------------------- % r52
\begin{thebibliography}{99} % anchors; full entries in global .bib % r53
\bibitem{SelbergCollected} A.~Selberg, \emph{Collected Papers}, Vol.~I, Springer, 1989. % r54
\bibitem{HejhalI} D.~Hejhal, \emph{The Selberg Trace Formula for $\PSL(2,\mathbb{R})$}, Vol.~I, Springer, 1976. % r55
\bibitem{HejhalII} D.~Hejhal, \emph{The Selberg Trace Formula for $\PSL(2,\mathbb{R})$}, Vol.~II, Springer, 1983. % r56
\bibitem{IwaniecSpectral} H.~Iwaniec, \emph{Spectral Methods of Automorphic Forms}, 2nd ed., AMS, 2002. % r57
\bibitem{Borthwick} D.~Borthwick, \emph{Spectral Theory of Infinite-Area Hyperbolic Surfaces}, Birkhäuser, 2016. % r58
\bibitem{LaxPhillips} P.~D.~Lax and R.~S.~Phillips, \emph{Scattering Theory for Automorphic Functions}, Princeton Univ.\ Press, 1976. % r59
\bibitem{Helgason} S.~Helgason, \emph{Groups and Geometric Analysis}, Academic Press, 1984. % r60
\bibitem{GangolliVaradarajan} R.~Gangolli and V.~S.~Varadarajan, \emph{Harmonic Analysis of Spherical Functions on Real Reductive Groups}, Springer, 1988. % r61
\end{thebibliography}

% ======================================================================
% End of File: 03-trace-formula-core-part4.tex  % r62
% ======================================================================
% ======================================================================
% File: src/sections/03-trace-formula-core-part5.tex  % r-anchor v1
% Title: Trace Formula Core — Part 5/8: Spectral Side and Scattering Channel
% Version: v1.0.0 (BUILD-ID: TFC-P5-0001) % guard comment
% ARCHETYPE: AFI-1.0 • LATEX_FLOW_BREAKER_v∞.200/100  % invariant tag
% ======================================================================

\section*{Trace Formula Core — Part 5/8: Spectral Side and Scattering Channel}\relax\hspace{0pt}
\label{sec:tfc-part5} % r1

% --------------------------------------------------------------- % r2
% Scope (Annals style) % r3
% --------------------------------------------------------------- % r4
\noindent
We construct the spectral side for cofinite $\Gamma\subset\PSL_2(\mathbb{R})$, with discrete and continuous components, and a precise regularization of the trace via truncation and Maaß--Selberg relations. \relax\hspace{0pt}
Compliance: C1 (branch), C2 (Plancherel), C3 (parametrization), C4 (test class), C6 (growth), C8 (dominated convergence), C12 (trace regularization). \relax\hspace{0pt}
References: \cite{SelbergCollected,HejhalI,HejhalII,IwaniecSpectral,LaxPhillips,Borthwick,GuillopeZworski,ColinDeVerdiere}. % r5

% --------------------------------------------------------------- % r6
% 1. Spectral resolution and functional calculus % r7
% --------------------------------------------------------------- % r8
\subsection*{1. Spectral resolution and functional calculus}\relax\hspace{0pt}
\label{subsec:tfc5-spec} % r9

Let $X=\Gamma\backslash\mathbb{H}$ with $\kappa$ cusps. \relax\hspace{0pt}
Let $\Delta$ be the nonnegative Laplacian with spectral parameter $\lambda=\frac{1}{4}+t^2$ and $t\in\mathbb{R}\cup i[0,\frac12]$ for discrete small eigenvalues. \relax\hspace{0pt}
For $h\in\mathcal{H}_{\mathrm{PW}}(\sigma,\delta)$ define $K_h=h\!\big(\sqrt{\Delta-\tfrac14}\big)$; by spectral calculus,
\begin{equation}\relax\hspace{0pt}
\label{eq:tfc5-spectral-decomp}\relax\hspace{0pt}
K_h f\ =\ \sum_{j} h(t_j)\,\langle f,u_j\rangle\,u_j\ +\ \frac{1}{4\pi}\sum_{\mathfrak{a}=1}^{\kappa}\int_{\mathbb{R}} h(t)\,\langle f,E_{\mathfrak{a}}(\cdot,\tfrac12+it)\rangle\,E_{\mathfrak{a}}(\cdot,\tfrac12+it)\,dt,
\end{equation}
with convergence in the strong operator topology on $L^2(X)$; cf.\ \cite{HejhalI,HejhalII}. \relax\hspace{0pt}

\begin{lemma}[Absolute summability of the discrete part]\relax\hspace{0pt}
\label{lem:tfc5-absdisc} % r10
If $h\in\mathcal{H}_{\mathrm{PW}}(\sigma,\delta)$ with $\delta>0$, then $\sum_j |h(t_j)|<\infty$. \relax\hspace{0pt}
\end{lemma}

\begin{proof}[Sketch]\relax\hspace{0pt}
Abel summation with Weyl growth $N(T)=\#\{|t_j|\le T\}=\frac{\vol(X)}{2\pi}T^2+O(T\log T)$ and the Cauchy estimate $|h'(u)|\ll (1+|u|)^{-3-\delta}$ yields integrability. \relax\hspace{0pt}
See Part~2 and \cite{IwaniecSpectral,Borthwick}.\relax\hspace{0pt}
\end{proof}

% --------------------------------------------------------------- % r11
% 2. Eisenstein series and Maaß--Selberg relations % r12
% --------------------------------------------------------------- % r13
\subsection*{2. Eisenstein series and Maaß--Selberg relations}\relax\hspace{0pt}
\label{subsec:tfc5-ms} % r14

For each cusp $\mathfrak{a}$, the Eisenstein series $E_{\mathfrak{a}}(z,s)$ has the Fourier expansion at cusp $\mathfrak{b}$:
\begin{equation}\relax\hspace{0pt}
\label{eq:tfc5-E-exp}\relax\hspace{0pt}
E_{\mathfrak{a}}(z,s)\ =\ \delta_{\mathfrak{a}\mathfrak{b}}\,y^{s}\ +\ \phi_{\mathfrak{a}\mathfrak{b}}(s)\,y^{1-s}\ +\ \sum_{n\neq 0}\rho_{\mathfrak{a}\mathfrak{b}}(n,s)\,K_{s-\frac12}(2\pi|n|y)\,e^{2\pi i n x},
\end{equation}
where $\mathbf{S}(s)=[\phi_{\mathfrak{a}\mathfrak{b}}(s)]$ is unitary on $\Re s=\tfrac12$ and $\sigma(s)=\det\mathbf{S}(s)$ satisfies $\sigma(s)\sigma(1-s)=1$; cf.\ \cite{HejhalII,LaxPhillips}. \relax\hspace{0pt}

\begin{theorem}[Maaß--Selberg relations, truncated]\relax\hspace{0pt}
\label{thm:tfc5-MS} % r15
Let $X_Y$ be the truncation at height $Y$. For $s=\tfrac12+it$, $s'=\tfrac12+it'$ on the critical line,
\[
\int_{X_Y} E_{\mathfrak{a}}(z,s)\,\overline{E_{\mathfrak{b}}(z,s')}\,d\mu(z)\ =\ \delta_{\mathfrak{a}\mathfrak{b}}\,\frac{Y^{s+\overline{s'}-1}}{s+\overline{s'}-1}\ +\ \mathcal{M}_{\mathfrak{a}\mathfrak{b}}(s,s')\ +\ o(1),
\]
as $Y\to\infty$, where $\mathcal{M}_{\mathfrak{a}\mathfrak{b}}$ is bounded and explicitly expressed via $\mathbf{S}(s)$; the $o(1)$ is uniform on compact vertical strips. \relax\hspace{0pt}
\end{theorem}

\begin{proof}[Sketch]\relax\hspace{0pt}
Unfold in the cusp sectors and isolate the constant terms; see \cite[Ch.~3]{HejhalII} and \cite{LaxPhillips}. \relax\hspace{0pt}
\end{proof}

% --------------------------------------------------------------- % r16
% 3. Regularized trace: truncation model and limit % r17
% --------------------------------------------------------------- % r18
\subsection*{3. Regularized trace: truncation and model subtraction}\relax\hspace{0pt}
\label{subsec:tfc5-trreg} % r19

Define the truncated trace and the model subtraction:
\begin{equation}\relax\hspace{0pt}
\label{eq:tfc5-trY}\relax\hspace{0pt}
\Tr\big(K_h|_{X_Y}\big)\ :=\ \int_{F_Y} K_h(z,z)\,d\mu(z),\qquad
\mathrm{model}_Y(h)\ :=\ A(h)\,Y,
\end{equation}
with $A(h)=\frac{\kappa}{2\pi}\int_0^\infty \widehat{h}(u)\,du$ as in Part~4 (parabolic class). \relax\hspace{0pt}
\begin{definition}[Regularized trace]\relax\hspace{0pt}
\label{def:tfc5-trreg} % r20
\[
\Tr_{\mathrm{reg}}(K_h)\ :=\ \lim_{Y\to\infty} \Big(\Tr\big(K_h|_{X_Y}\big)-\mathrm{model}_Y(h)\Big),
\]
provided the limit exists (it does for $h\in\mathcal{H}_{\mathrm{PW}}(\sigma,\delta)$). \relax\hspace{0pt}
\end{definition}

\begin{proposition}[Existence of $\Tr_{\mathrm{reg}}(K_h)$]\relax\hspace{0pt}
\label{prop:tfc5-exist-trreg} % r21
For $h$ as above, $\Tr_{\mathrm{reg}}(K_h)$ exists and equals the sum of the discrete trace and a principal-value integral of the continuous part matched with the Maaß--Selberg main term. \relax\hspace{0pt}
\end{proposition}

\begin{proof}[Sketch]\relax\hspace{0pt}
Insert \eqref{eq:tfc5-spectral-decomp} into $\Tr(K_h|_{X_Y})$; for the continuous part use Theorem~\ref{thm:tfc5-MS} to separate the $Y$-divergence $A(h)Y$ and identify the finite limit. \relax\hspace{0pt}
\end{proof}

% --------------------------------------------------------------- % r22
% 4. Scattering channel and growth bounds % r23
% --------------------------------------------------------------- % r24
\subsection*{4. Scattering channel and growth bounds}\relax\hspace{0pt}
\label{subsec:tfc5-scatt} % r25

Let $\sigma(s)=\det\mathbf{S}(s)$ be the scattering determinant. \relax\hspace{0pt}
We adopt the global branch of $\log\sigma$ fixed in Part~1 (C1). \relax\hspace{0pt}
The growth on the critical line satisfies
\begin{equation}\relax\hspace{0pt}
\label{eq:tfc5-sigma-growth}\relax\hspace{0pt}
\frac{\sigma'}{\sigma}\!\left(\tfrac12+it\right)\ \ll\ |t|\log(2+|t|),\qquad |t|\to\infty,
\end{equation}
uniformly for cofinite $\Gamma$ with fixed cusp data; see \cite[\S~6]{IwaniecSpectral} and \cite{HejhalII,Borthwick}. \relax\hspace{0pt}

\begin{lemma}[Integrable majorant on the scattering integrand]\relax\hspace{0pt}
\label{lem:tfc5-L1}\relax\hspace{0pt}
If $h\in\mathcal{H}_{\mathrm{PW}}(\sigma,\delta)$, then
\[
|h(t)|\cdot\left|\frac{\sigma'}{\sigma}\!\left(\tfrac12+it\right)\right|\ \le\ M(t),\quad M\in L^1(\mathbb{R}).
\]
In particular one may take $M(t)\asymp (1+|t|)^{-1-\delta}\log(2+|t|)$.\relax\hspace{0pt}
\end{lemma}

\begin{proof}[Sketch]\relax\hspace{0pt}
Combine \eqref{eq:tfc5-sigma-growth} with $|h(t)|\ll (1+|t|)^{-2-\delta}$ in vertical strips; the product is integrable. \relax\hspace{0pt}
\end{proof}

% --------------------------------------------------------------- % r26
% 5. Spectral side: assembled identity % r27
% --------------------------------------------------------------- % r28
\subsection*{5. Spectral side: assembled identity}\relax\hspace{0pt}
\label{subsec:tfc5-assembled} % r29

Let $\{t_j\}$ denote discrete parameters (including small $t_j\in i(0,\frac12]$). \relax\hspace{0pt}
Define the spectral functional
\begin{equation}\relax\hspace{0pt}
\label{eq:tfc5-Espec}\relax\hspace{0pt}
\mathcal{E}_{\mathrm{spec}}(h)\ :=\ \sum_{j} h(t_j)\ +\ \frac{1}{4\pi}\int_{\mathbb{R}} h(t)\,\frac{\sigma'}{\sigma}\!\left(\tfrac12+it\right)\,dt.
\end{equation}

\begin{theorem}[Spectral trace identity]\relax\hspace{0pt}
\label{thm:tfc5-spec=trreg} % r30
For $h\in\mathcal{H}_{\mathrm{PW}}(\sigma,\delta)$,
\begin{equation}\relax\hspace{0pt}
\label{eq:tfc5-spec-trreg}\relax\hspace{0pt}
\mathcal{E}_{\mathrm{spec}}(h)\ =\ \Tr_{\mathrm{reg}}(K_h).
\end{equation}
\end{theorem}

\begin{proof}[Sketch]\relax\hspace{0pt}
Apply Definition~\ref{def:tfc5-trreg} and Proposition~\ref{prop:tfc5-exist-trreg}, use Lemma~\ref{lem:tfc5-absdisc} for the discrete sum and Lemma~\ref{lem:tfc5-L1} for the continuous integral (dominated convergence). \relax\hspace{0pt}
\end{proof}

% --------------------------------------------------------------- % r31
% 6. Wave probe and Paley--Wiener approximation % r32
% --------------------------------------------------------------- % r33
\subsection*{6. Wave probe and Paley--Wiener approximation}\relax\hspace{0pt}
\label{subsec:tfc5-wave} % r34

The wave probe $h_T(t)=\cos(Tt)$ is outside $\mathcal{H}_{\mathrm{PW}}$, but admits Paley--Wiener approximation. \relax\hspace{0pt}
\begin{lemma}[PW-approximation of the wave probe]\relax\hspace{0pt}
\label{lem:tfc5-wavePW} % r35
There exists $h_T^{(n)}\in\mathcal{H}_{\mathrm{PW}}(\sigma_n,\delta)$ with exponential type $R_n\to\infty$ such that $h_T^{(n)}(t)\to \cos(Tt)$ locally uniformly and $K_{h_T^{(n)}}\to \cos\!\big(T\sqrt{\Delta-\tfrac14}\big)$ strongly on $L^2(X)$. \relax\hspace{0pt}
\end{lemma}

\begin{proof}[Sketch]\relax\hspace{0pt}
Choose $\widehat{h_T^{(n)}}\in C_c^\infty([-R_n,R_n])$ converging to $\tfrac12(\delta_{T}+\delta_{-T})$ in the sense of measures; invoke Paley--Wiener and the spectral theorem. \relax\hspace{0pt}
\end{proof}

\begin{proposition}[Wave limit of the spectral identity]\relax\hspace{0pt}
\label{prop:tfc5-wave-limit} % r36
For each fixed $T>0$,
\[
\lim_{n\to\infty} \mathcal{E}_{\mathrm{spec}}(h_T^{(n)})\ =\ \sum_{j}\cos(T t_j)\ +\ \frac{1}{4\pi}\int_{\mathbb{R}} \cos(T t)\,\frac{\sigma'}{\sigma}\!\left(\tfrac12+it\right)\,dt,
\]
and the limit equals $\Tr_{\mathrm{reg}}\!\big(\cos(T\sqrt{\Delta-\tfrac14})\big)$ by Theorem~\ref{thm:tfc5-spec=trreg}. \relax\hspace{0pt}
\end{proposition}

\begin{proof}[Sketch]\relax\hspace{0pt}
Dominated convergence via Lemma~\ref{lem:tfc5-L1} and strong convergence of $K_{h_T^{(n)}}$. \relax\hspace{0pt}
\end{proof}

% --------------------------------------------------------------- % r37
% 7. Stability under isometries and spectral-unitary maps % r38
% --------------------------------------------------------------- % r39
\subsection*{7. Stability under isometries and unitary equivalence}\relax\hspace{0pt}
\label{subsec:tfc5-stability} % r40

\begin{proposition}[Isometry invariance]\relax\hspace{0pt}
\label{prop:tfc5-isometry} % r41
If $(X,g)\cong (X',g')$ by an isometry carrying cusp data bijectively, then $\mathcal{E}_{\mathrm{spec},X}(h)=\mathcal{E}_{\mathrm{spec},X'}(h)$ for all $h\in\mathcal{H}_{\mathrm{PW}}$. \relax\hspace{0pt}
\end{proposition}

\begin{proof}[Sketch]\relax\hspace{0pt}
Discrete spectrum and scattering matrix are isometry invariants; $A(h)$ depends only on $\kappa$ and $\widehat{h}$. \relax\hspace{0pt}
\end{proof}

\begin{proposition}[Spectral-unitary invariance]\relax\hspace{0pt}
\label{prop:tfc5-unitary} % r42
If $U:L^2(X)\to L^2(X')$ is unitary with $U\Delta_X=\Delta_{X'}U$ and intertwines Eisenstein data, then \eqref{eq:tfc5-spec-trreg} is preserved, i.e.\ $\mathcal{E}_{\mathrm{spec},X}(h)=\mathcal{E}_{\mathrm{spec},X'}(h)$. \relax\hspace{0pt}
\end{proposition}

\begin{proof}[Sketch]\relax\hspace{0pt}
$U$ preserves the spectral measure and the scattering determinant; truncation models match by cusp bijection. \relax\hspace{0pt}
\end{proof}

% --------------------------------------------------------------- % r43
% 8. Consistency with geometric and zeta sides % r44
% --------------------------------------------------------------- % r45
\subsection*{8. Consistency with geometric and zeta sides}\relax\hspace{0pt}
\label{subsec:tfc5-consistency} % r46

\begin{theorem}[Channel equivalence $\mathrm{E}_1=\mathrm{E}_2=\mathrm{E}_3$ (spectral focus)]\relax\hspace{0pt}
\label{thm:tfc5-E123} % r47
For $h\in\mathcal{H}_{\mathrm{PW}}(\sigma,\delta)$,
\[
\mathcal{E}_{\mathrm{spec}}(h)\ =\ \Tr_{\mathrm{reg}}(K_h)\ \ (\mathrm{E}_2)\ =\ \mathcal{E}_{\mathrm{geom}}(h)\ \ (\mathrm{E}_2)\ =\ \mathcal{E}_{\mathrm{zeta}}(h)\ \ (\mathrm{E}_3),
\]
with $\mathcal{E}_{\mathrm{geom}}$ given by Theorem~\ref{thm:tfc4-geom} (Part~4) and $\mathcal{E}_{\mathrm{zeta}}$ by the contour identity of Part~6. \relax\hspace{0pt}
\end{theorem}

\begin{proof}[Sketch]\relax\hspace{0pt}
The equality $\mathcal{E}_{\mathrm{spec}}=\Tr_{\mathrm{reg}}(K_h)$ is Theorem~\ref{thm:tfc5-spec=trreg}. \relax\hspace{0pt}
Equality with $\mathcal{E}_{\mathrm{geom}}$ follows by Part~4 (orbital integrals). \relax\hspace{0pt}
Equality with $\mathcal{E}_{\mathrm{zeta}}$ is established in Part~6 through contour shift and identification of residues/tails with the spectral and geometric contributions (using Lemma~\ref{lem:tfc5-L1}). \relax\hspace{0pt}
\end{proof}

% --------------------------------------------------------------- % r48
% 9. Compliance ledger for Part 5/8 % r49
% --------------------------------------------------------------- % r50
\subsection*{9. Compliance ledger for Part 5/8}\relax\hspace{0pt}
\label{subsec:tfc5-compliance} % r51

\noindent
\textbf{C1 (branch):} global branch of $\log\sigma$ fixed; used in \eqref{eq:tfc5-Espec}. \quad
\textbf{C2 (Plancherel):} factor $dt/(4\pi)$ in \eqref{eq:tfc5-spectral-decomp}, \eqref{eq:tfc5-Espec}. \\
\textbf{C3 (parametrization):} $\lambda=\tfrac14+t^2$ with $t\in\mathbb{R}\cup i[0,\tfrac12]$. \quad
\textbf{C4 (test class):} $h\in\mathcal{H}_{\mathrm{PW}}(\sigma,\delta)$ ensures decay and differentiability. \\
\textbf{C6 (growth):} bound \eqref{eq:tfc5-sigma-growth}. \quad
\textbf{C8 (dominated convergence):} Lemma~\ref{lem:tfc5-L1}. \quad
\textbf{C12 (trace regularization):} truncation model \eqref{eq:tfc5-trY} and Definition~\ref{def:tfc5-trreg}. \relax\hspace{0pt}

% --------------------------------------------------------------- % r52
% Bibliographic anchors (resolved in the global .bib file) % r53
% --------------------------------------------------------------- % r54
\begin{thebibliography}{99} % anchors; full entries in global .bib % r55
\bibitem{SelbergCollected} A.~Selberg, \emph{Collected Papers}, Vol.~I, Springer, 1989. % r56
\bibitem{HejhalI} D.~Hejhal, \emph{The Selberg Trace Formula for $\PSL(2,\mathbb{R})$}, Vol.~I, Springer, 1976. % r57
\bibitem{HejhalII} D.~Hejhal, \emph{The Selberg Trace Formula for $\PSL(2,\mathbb{R})$}, Vol.~II, Springer, 1983. % r58
\bibitem{IwaniecSpectral} H.~Iwaniec, \emph{Spectral Methods of Automorphic Forms}, 2nd ed., AMS, 2002. % r59
\bibitem{LaxPhillips} P.~D.~Lax and R.~S.~Phillips, \emph{Scattering Theory for Automorphic Functions}, Princeton Univ.\ Press, 1976. % r60
\bibitem{Borthwick} D.~Borthwick, \emph{Spectral Theory of Infinite-Area Hyperbolic Surfaces}, Birkhäuser, 2016. % r61
\bibitem{GuillopeZworski} L.~Guillopé and M.~Zworski, \emph{Upper bounds on the number of resonances for non-compact Riemann surfaces}, J.\ Funct.\ Anal.\ \textbf{129} (1995), 364–389. % r62
\bibitem{ColinDeVerdiere} Y.~Colin de Verdière, \emph{Spectres de variétés Riemanniennes et milieux quantiques}, Cours de Grenoble, 2007. % r63
\end{thebibliography}

% ======================================================================
% End of File: 03-trace-formula-core-part5.tex  % r64
% ======================================================================
% ======================================================================
% File: src/sections/03-trace-formula-core-part6.tex  % r-anchor v1
% Title: Trace Formula Core — Part 6/8: Zeta/Contour Bridge and Residues
% Version: v1.0.0 (BUILD-ID: TFC-P6-0001) % guard comment
% ARCHETYPE: AFI-1.0 • LATEX_FLOW_BREAKER_v∞.200/100  % invariant tag
% ======================================================================

\section*{Trace Formula Core — Part 6/8: Zeta/Contour Bridge and Residues}\relax\hspace{0pt}
\label{sec:tfc-part6} % r1

% --------------------------------------------------------------- % r2
% Scope (Annals style) % r3
% --------------------------------------------------------------- % r4
\noindent
We construct the zeta/contour channel ($\mathrm{E}_3$), starting from the Selberg zeta $Z_\Gamma(s)$ and the scattering determinant $\sigma(s)$, derive the logarithmic derivative expansion including the polynomial term $P_\Gamma(s)$ and trivial factors, and justify contour shifts to the critical line with complete control of horizontal/vertical tails. \relax\hspace{0pt}
Compliance: C1 (branch), C2 (Plancherel), C3 (parametrization), C4 (test class), C6 (growth), C9 (tails), C10 (polynomial), C12 (trace regularization). \relax\hspace{0pt}
References: \cite{SelbergCollected,HejhalI,HejhalII,IwaniecSpectral,LaxPhillips,Borthwick,GuillopeZworski,ColinDeVerdiere}. % r5

% --------------------------------------------------------------- % r6
% 1. Selberg zeta and scattering determinant % r7
% --------------------------------------------------------------- % r8
\subsection*{1. Selberg zeta and the scattering determinant}\relax\hspace{0pt}
\label{subsec:tfc6-zeta-def} % r9

Let $\Gamma\subset\PSL_2(\mathbb{R})$ be cofinite, $X=\Gamma\backslash\mathbb{H}$ with genus $g$ and $\kappa$ cusps. \relax\hspace{0pt}
The Selberg zeta function is defined for $\Re s>1$ by
\begin{equation}\relax\hspace{0pt}
\label{eq:tfc6-Z-def}\relax\hspace{0pt}
Z_\Gamma(s)\ :=\ \prod_{p}\ \prod_{k=0}^{\infty}\Big(1-e^{-(s+k)\ell(p)}\Big),
\end{equation}
where $p$ runs over primitive closed geodesics and $\ell(p)$ denotes their length; cf.\ \cite{SelbergCollected,HejhalI}. \relax\hspace{0pt}
The scattering matrix $\mathbf{S}(s)$ is meromorphic, unitary on $\Re s=\tfrac12$, and satisfies $\mathbf{S}(s)\mathbf{S}(1-s)=\mathbf{I}_\kappa$; the scattering determinant $\sigma(s)=\det\mathbf{S}(s)$ obeys $\sigma(s)\sigma(1-s)=1$; see \cite{HejhalII,LaxPhillips}. \relax\hspace{0pt}

\begin{remark}[Global branch]\relax\hspace{0pt}
\label{rem:tfc6-branch} % r10
Fix the global branch of $\log\sigma$ by analytic continuation from $\Re s>1$ with $\log\sigma(s)\to 0$ as $\Re s\to+\infty$ (C1). \relax\hspace{0pt}
\end{remark}

% --------------------------------------------------------------- % r11
% 2. Logarithmic derivative and polynomial structure % r12
% --------------------------------------------------------------- % r13
\subsection*{2. Logarithmic derivative and polynomial structure}\relax\hspace{0pt}
\label{subsec:tfc6-Zprime} % r14

\begin{theorem}[Logarithmic derivative expansion]\relax\hspace{0pt}
\label{thm:tfc6-Zprime-expansion} % r15
There exists a polynomial $P_\Gamma(s)$ of degree $2g-2+\kappa$ such that, as meromorphic functions on $\mathbb{C}$,
\begin{equation}\relax\hspace{0pt}
\label{eq:tfc6-Zprime}\relax\hspace{0pt}
\frac{Z_\Gamma'}{Z_\Gamma}(s)\ =\ \sum_j\!\left(\frac{1}{s-\tfrac12-it_j}+\frac{1}{s-\tfrac12+it_j}\right)\ +\ \frac{1}{2\pi i}\,\frac{\sigma'}{\sigma}(s)\ +\ P_\Gamma'(s)\ +\ \mathcal{T}(s),
\end{equation}
where $\{t_j\}$ runs over discrete spectral parameters (including $t_j\in i[0,\tfrac12]$), and $\mathcal{T}(s)$ collects contributions of trivial poles/zeros at nonpositive integers coming from archimedean factors. \relax\hspace{0pt}
Moreover, $P_\Gamma$ is explicitly determined by the topological data of $X$ (Euler characteristic $\chi(X)=2-2g-\kappa$) and cusp widths. \relax\hspace{0pt}
\end{theorem}

\begin{proof}[Sketch]\relax\hspace{0pt}
Differentiate $\log Z_\Gamma$ in the half-plane $\Re s>1$, use absolute convergence to match the hyperbolic orbital sum with spectral poles (resonances/eigenvalues) via the functional equation and the scattering identity. Meromorphic continuation across $\Re s=1$ introduces $P_\Gamma$ and archimedean terms $\mathcal{T}$. See \cite[Ch.~11--12]{HejhalII}, \cite{SelbergCollected}. \relax\hspace{0pt}
\end{proof}

\begin{remark}[Trivial factors]\relax\hspace{0pt}
\label{rem:tfc6-trivial} % r16
$\mathcal{T}(s)$ consists of a finite linear combination of simple rational terms at $s=-n$ and $s=1+n$ ($n\in\mathbb{Z}_{\ge 0}$), with coefficients determined by local archimedean gamma-factors and cusp data; cf.\ \cite{HejhalII,Borthwick}. \relax\hspace{0pt}
\end{remark}

% --------------------------------------------------------------- % r17
% 3. Zeta/contour functional and Paley--Wiener test % r18
% --------------------------------------------------------------- % r19
\subsection*{3. Zeta/contour functional for Paley--Wiener tests}\relax\hspace{0pt}
\label{subsec:tfc6-functional} % r20

Let $h\in\mathcal{H}_{\mathrm{PW}}(\sigma,\delta)$ and denote its cosine transform by
\begin{equation}\relax\hspace{0pt}
\label{eq:tfc6-hat}\relax\hspace{0pt}
\widehat{h}(u)\ :=\ \frac{1}{2\pi}\int_{\mathbb{R}} h(t)\,\cos(ut)\,dt,
\end{equation}
so that $\widehat{h}\in C_c^\infty([-R,R])$ by Paley--Wiener (Part~2). \relax\hspace{0pt}
Define the zeta-channel functional
\begin{equation}\relax\hspace{0pt}
\label{eq:tfc6-Ezeta}\relax\hspace{0pt}
\mathcal{E}_{\mathrm{zeta}}(h)\ :=\ \frac{1}{4\pi i}\,\int_{\Re s=1+\varepsilon}\frac{Z_\Gamma'}{Z_\Gamma}(s)\,\widehat{h}\!\left(\tfrac12-s\right)\,ds,
\end{equation}
with fixed $\varepsilon>0$, where the integral converges absolutely by the compact support of $\widehat{h}$ in vertical strips. \relax\hspace{0pt}

\begin{lemma}[Absolute convergence on $\Re s=1+\varepsilon$]\relax\hspace{0pt}
\label{lem:tfc6-abs}\relax\hspace{0pt}
For any fixed $\varepsilon>0$, $\widehat{h}\!\left(\tfrac12-(1+\varepsilon+it)\right)$ decays faster than any power in $|t|$, hence the integral \eqref{eq:tfc6-Ezeta} is absolutely convergent. \relax\hspace{0pt}
\end{lemma}

\begin{proof}[Sketch]\relax\hspace{0pt}
Use $\widehat{h}\in C_c^\infty([-R,R])$ and the vertical translation $s\mapsto 1+\varepsilon+it$, together with the Paley--Wiener bounds. \relax\hspace{0pt}
\end{proof}

% --------------------------------------------------------------- % r21
% 4. Contour shift to the critical line and tails % r22
% --------------------------------------------------------------- % r23
\subsection*{4. Contour shift to the critical line and tail bounds}\relax\hspace{0pt}
\label{subsec:tfc6-shift} % r24

\begin{theorem}[Contour shift and horizontal tails]\relax\hspace{0pt}
\label{thm:tfc6-shift}\relax\hspace{0pt}
The integral \eqref{eq:tfc6-Ezeta} can be shifted from $\Re s=1+\varepsilon$ to $\Re s=\tfrac12$, crossing all poles of $\frac{Z_\Gamma'}{Z_\Gamma}$, and
\begin{equation}\relax\hspace{0pt}
\label{eq:tfc6-contour-equality}\relax\hspace{0pt}
\mathcal{E}_{\mathrm{zeta}}(h)\ =\ \frac{1}{4\pi i}\,\int_{\Re s=\frac12}\frac{Z_\Gamma'}{Z_\Gamma}(s)\,\widehat{h}\!\left(\tfrac12-s\right)\,ds\ +\ \sum_{\rho}\mathrm{Res}_{s=\rho}\left(\frac{Z_\Gamma'}{Z_\Gamma}(s)\,\widehat{h}\!\left(\tfrac12-s\right)\right),
\end{equation}
where $\rho$ runs over the crossed poles. The horizontal segments vanish absolutely, and the vertical tail integrals converge. \relax\hspace{0pt}
\end{theorem}

\begin{proof}[Sketch]\relax\hspace{0pt}
On the shifted rectangles, $\widehat{h}$ decays rapidly in $\Im s$ while $\frac{Z_\Gamma'}{Z_\Gamma}(s)$ has at most polynomial growth in vertical strips (Part~4: vertical-strip bounds). Hence horizontal integrals vanish. Residues are collected at poles corresponding to discrete spectrum and trivial factors; see \cite{HejhalII}. \relax\hspace{0pt}
\end{proof}

\begin{lemma}[Vertical-strip growth]\relax\hspace{0pt}
\label{lem:tfc6-vertical}\relax\hspace{0pt}
For every $\delta_0>0$, there exists $C(\delta_0)$ such that
\[
\frac{Z_\Gamma'}{Z_\Gamma}(\sigma+it)\ \ll\ (1+|t|)^{C(\delta_0)},\qquad \text{uniformly for }|\sigma-\tfrac12|\le \delta_0,
\]
with constants depending only on geometric data of $X$. \relax\hspace{0pt}
\end{lemma}

\begin{proof}[Sketch]\relax\hspace{0pt}
Employ the product/composition of entire factors and scattering determinant, using resolvent bounds and standard Phragmén--Lindelöf estimates; cf.\ \cite{Borthwick,GuillopeZworski}. \relax\hspace{0pt}
\end{proof}

% --------------------------------------------------------------- % r25
% 5. Residue calculus: spectral and trivial poles % r26
% --------------------------------------------------------------- % r27
\subsection*{5. Residue calculus: spectral and trivial poles}\relax\hspace{0pt}
\label{subsec:tfc6-residues} % r28

\begin{proposition}[Residue contributions]\relax\hspace{0pt}
\label{prop:tfc6-res}\relax\hspace{0pt}
The residues in \eqref{eq:tfc6-contour-equality} decompose as
\[
\sum_{\rho}\mathrm{Res}\ =\ \sum_{j}\widehat{h}\!\left(\tfrac12-\tfrac12-it_j\right)\ +\ \sum_{k\ge 0}\Big(\alpha_k\,\widehat{h}(\tfrac12+k)+\beta_k\,\widehat{h}(-\tfrac12-k)\Big)\ +\ \mathrm{Res}_{\text{scatt}},
\]
where the first sum corresponds to discrete eigenvalues ($s=\tfrac12\pm it_j$), the second sum collects trivial archimedean terms with explicit coefficients $\alpha_k,\beta_k$, and $\mathrm{Res}_{\text{scatt}}$ is the (finite) sum of residues from poles of $\sigma'/\sigma$. \relax\hspace{0pt}
\end{proposition}

\begin{proof}[Sketch]\relax\hspace{0pt}
Use \eqref{eq:tfc6-Zprime}; each simple pole contributes $\widehat{h}(\tfrac12-\rho)$. Trivial terms arise from $\mathcal{T}(s)$; scattering poles contribute via $\frac{\sigma'}{\sigma}$. See \cite{HejhalII}. \relax\hspace{0pt}
\end{proof}

\begin{remark}[Polynomial term]\relax\hspace{0pt}
\label{rem:tfc6-poly}\relax\hspace{0pt}
The contribution of $P_\Gamma'(s)$ to \eqref{eq:tfc6-contour-equality} equals
\[
\frac{1}{4\pi i}\int_{\Re s=\frac12} P_\Gamma'(s)\,\widehat{h}\!\left(\tfrac12-s\right)\,ds,
\]
which can be evaluated explicitly by shifting to $\Re s=1+\varepsilon$ and using the compact support of $\widehat{h}$. This term encodes topological constants (volume/Euler characteristic). \relax\hspace{0pt}
\end{remark}

% --------------------------------------------------------------- % r29
% 6. Identification with the spectral channel % r30
% --------------------------------------------------------------- % r31
\subsection*{6. Identification with the spectral channel}\relax\hspace{0pt}
\label{subsec:tfc6-spec-match} % r32

Define the spectral functional (Part~5)
\[
\mathcal{E}_{\mathrm{spec}}(h)\ =\ \sum_j h(t_j)\ +\ \frac{1}{4\pi}\int_{\mathbb{R}} h(t)\,\frac{\sigma'}{\sigma}\!\left(\tfrac12+it\right)\,dt.
\]
\begin{theorem}[Channel equivalence $\mathrm{E}_3=\mathrm{E}_1$]\relax\hspace{0pt}
\label{thm:tfc6-E3=E1}\relax\hspace{0pt}
For $h\in\mathcal{H}_{\mathrm{PW}}(\sigma,\delta)$,
\[
\mathcal{E}_{\mathrm{zeta}}(h)\ =\ \mathcal{E}_{\mathrm{spec}}(h).
\]
\end{theorem}

\begin{proof}[Sketch]\relax\hspace{0pt}
Insert \eqref{eq:tfc6-Zprime} into \eqref{eq:tfc6-contour-equality}, evaluate residues (Proposition~\ref{prop:tfc6-res}), and identify the integral on $\Re s=\tfrac12$ with the scattering integral by the change $s=\tfrac12+it$. The discrete residue sum reproduces $\sum_j h(t_j)$ under the Paley--Wiener inversion $h(t)=\int_0^\infty \widehat{h}(u)\cos(ut)\,du$. Trivial/polynomial pieces match the model terms already accounted for in the regularized trace (Part~5), yielding exact equality. \relax\hspace{0pt}
\end{proof}

\begin{corollary}[Triple equivalence]\relax\hspace{0pt}
\label{cor:tfc6-E123}\relax\hspace{0pt}
For $h\in\mathcal{H}_{\mathrm{PW}}(\sigma,\delta)$,
\[
\mathcal{E}_{\mathrm{geom}}(h)\ =\ \mathcal{E}_{\mathrm{spec}}(h)\ =\ \mathcal{E}_{\mathrm{zeta}}(h),
\]
where $\mathcal{E}_{\mathrm{geom}}$ is the geometric/orbital side (Part~4), $\mathcal{E}_{\mathrm{spec}}$ the spectral/regularized trace side (Part~5), and $\mathcal{E}_{\mathrm{zeta}}$ the contour/zeta side constructed above. \relax\hspace{0pt}
\end{corollary}

% --------------------------------------------------------------- % r33
% 7. Wave limit and dominated convergence % r34
% --------------------------------------------------------------- % r35
\subsection*{7. Wave limit and dominated convergence}\relax\hspace{0pt}
\label{subsec:tfc6-wave} % r36

Let $h_T^{(n)}\to \cos(Tt)$ be the Paley--Wiener approximants (Part~5). \relax\hspace{0pt}
\begin{proposition}[Wave form of the zeta channel]\relax\hspace{0pt}
\label{prop:tfc6-wave}\relax\hspace{0pt}
For fixed $T>0$,
\[
\lim_{n\to\infty}\mathcal{E}_{\mathrm{zeta}}(h_T^{(n)})\ =\ \sum_j \cos(T t_j)\ +\ \frac{1}{4\pi}\int_{\mathbb{R}} \cos(T t)\,\frac{\sigma'}{\sigma}\!\left(\tfrac12+it\right)\,dt,
\]
with the limit justified by dominated convergence using vertical-strip growth and the compact support of $\widehat{h_T^{(n)}}$. \relax\hspace{0pt}
\end{proposition}

\begin{proof}[Sketch]\relax\hspace{0pt}
Apply Theorem~\ref{thm:tfc6-shift} to $h_T^{(n)}$, pass $n\to\infty$ using Lemma~\ref{lem:tfc6-vertical} and the decay of $\widehat{h_T^{(n)}}$; identify the right-hand side by inversion. \relax\hspace{0pt}
\end{proof}

% --------------------------------------------------------------- % r37
% 8. Compliance ledger for Part 6/8 % r38
% --------------------------------------------------------------- % r39
\subsection*{8. Compliance ledger for Part 6/8}\relax\hspace{0pt}
\label{subsec:tfc6-compliance} % r40

\noindent
\textbf{C1 (branch):} global branch of $\log\sigma$ fixed (Remark~\ref{rem:tfc6-branch}). \quad
\textbf{C2 (Plancherel):} enforced when matching $\Re s=\tfrac12$ integral with $dt/(4\pi)$. \\
\textbf{C3 (parametrization):} $\lambda=\tfrac14+t^2$, poles at $s=\tfrac12\pm it_j$. \quad
\textbf{C4 (test class):} $h\in\mathcal{H}_{\mathrm{PW}}(\sigma,\delta)$ ensures $\widehat{h}\in C_c^\infty$. \\
\textbf{C6 (growth):} vertical-strip bounds (Lemma~\ref{lem:tfc6-vertical}) and critical line bound for $\sigma'/\sigma$ from Part~4. \\
\textbf{C9 (tails):} horizontal tails vanish (Theorem~\ref{thm:tfc6-shift}). \quad
\textbf{C10 (polynomial):} $P_\Gamma'(s)$ present in \eqref{eq:tfc6-Zprime}; handled in Remark~\ref{rem:tfc6-poly}. \\
\textbf{C12 (trace regularization):} trivial/polynomial residues match the model subtractions of Part~5. \relax\hspace{0pt}

% --------------------------------------------------------------- % r41
% Bibliographic anchors (resolved in the global .bib file) % r42
% --------------------------------------------------------------- % r43
\begin{thebibliography}{99} % anchors; full entries in global .bib % r44
\bibitem{SelbergCollected} A.~Selberg, \emph{Collected Papers}, Vol.~I, Springer, 1989. % r45
\bibitem{HejhalI} D.~Hejhal, \emph{The Selberg Trace Formula for $\PSL(2,\mathbb{R})$}, Vol.~I, Springer, 1976. % r46
\bibitem{HejhalII} D.~Hejhal, \emph{The Selberg Trace Formula for $\PSL(2,\mathbb{R})$}, Vol.~II, Springer, 1983. % r47
\bibitem{IwaniecSpectral} H.~Iwaniec, \emph{Spectral Methods of Automorphic Forms}, 2nd ed., AMS, 2002. % r48
\bibitem{LaxPhillips} P.~D.~Lax and R.~S.~Phillips, \emph{Scattering Theory for Automorphic Functions}, Princeton Univ.\ Press, 1976. % r49
\bibitem{Borthwick} D.~Borthwick, \emph{Spectral Theory of Infinite-Area Hyperbolic Surfaces}, Birkhäuser, 2016. % r50
\bibitem{GuillopeZworski} L.~Guillopé and M.~Zworski, \emph{Upper bounds on the number of resonances for non-compact Riemann surfaces}, J.\ Funct.\ Anal.\ \textbf{129} (1995), 364–389. % r51
\bibitem{ColinDeVerdiere} Y.~Colin de Verdière, \emph{Spectres de variétés Riemanniennes et milieux quantiques}, Cours de Grenoble, 2007. % r52
\end{thebibliography}

% ======================================================================
% End of File: 03-trace-formula-core-part6.tex  % r53
% ======================================================================
% ======================================================================
% File: src/sections/03-trace-formula-core-part7.tex  % r-anchor v1
% Title: Trace Formula Core — Part 7/8: Determinants, Regularized Traces, and Resonances
% Version: v1.0.0 (BUILD-ID: TFC-P7-0001) % guard comment
% ARCHETYPE: AFI-1.0 • LATEX_FLOW_BREAKER_v∞.200/100  % invariant tag
% ======================================================================

\section*{Trace Formula Core — Part 7/8: Determinants, Regularized Traces, and Resonances}\relax\hspace{0pt}
\label{sec:tfc-part7} % r1

% --------------------------------------------------------------- % r2
% Scope (Annals style) % r3
% --------------------------------------------------------------- % r4
\noindent
We construct spectral/zeta determinants and regularized traces for compact and finite-area hyperbolic surfaces, identify their variation under geometric deformation, and relate the divisor of determinants to the resonance set. \relax\hspace{0pt}
Compliance: C1 (branch), C2 (Plancherel), C3 (parametrization), C4 (test class), C6 (growth), C10 (polynomial), C12 (trace regularization), plus derivative stability under deformation. \relax\hspace{0pt}
References: \cite{MinakshisundaramPleijel,Seeley,RaySinger,Muller,HeJi,ParkinsSarnak,Borthwick,GuillopeZworski,HejhalII}. % r5

% --------------------------------------------------------------- % r6
% 1. Compact case: spectral zeta and determinant % r7
% --------------------------------------------------------------- % r8
\subsection*{1. Compact case: spectral zeta and determinant}\relax\hspace{0pt}
\label{subsec:tfc7-compact} % r9

Let $(M,g)$ be compact without boundary, $\Delta=\Delta_g\ge 0$ with discrete spectrum $0=\lambda_0<\lambda_1\le \lambda_2\le\cdots\to\infty$. \relax\hspace{0pt}
Define the spectral zeta (for $\Re s>\tfrac{d}{2}$)
\begin{equation}\relax\hspace{0pt}
\label{eq:tfc7-zetaM}\relax\hspace{0pt}
\zeta_M(s):=\sum_{\lambda_j>0}\lambda_j^{-s}\,,
\end{equation}
which extends meromorphically to $\mathbb{C}$ with at most simple poles at $s=\tfrac{d}{2},\tfrac{d}{2}-1,\dots,1,0$; see \cite{MinakshisundaramPleijel,Seeley}. \relax\hspace{0pt}
The (primed) zeta-regularized determinant is
\begin{equation}\relax\hspace{0pt}
\label{eq:tfc7-det-compact}\relax\hspace{0pt}
\det{}'(\Delta)\ :=\ \exp\!\big(-\zeta_M'(0)\big)\,.
\end{equation}

\begin{theorem}[Heat--zeta dictionary and residues]\relax\hspace{0pt}
\label{thm:tfc7-heat-zeta}\relax\hspace{0pt}
Let $\Tr(e^{-t\Delta})\sim (4\pi t)^{-d/2}\sum_{k\ge 0}a_k t^k$ as $t\downarrow 0$. Then
\[
\zeta_M(s)=\frac{1}{\Gamma(s)}\int_0^\infty t^{s-1}\Big(\Tr(e^{-t\Delta})-1\Big)\,dt
\]
extends meromorphically with $\Res_{s=\frac{d}{2}-k}\zeta_M(s)=\frac{a_k}{\Gamma(\frac{d}{2}-k)}$ for $k=0,1,\dots,\lfloor d/2\rfloor$. \relax\hspace{0pt}
\end{theorem}

\begin{proof}[Sketch]\relax\hspace{0pt}
Mellin transform of the heat trace minus the zero mode; use Seeley's expansion \cite{Seeley}. \relax\hspace{0pt}
\end{proof}

% --------------------------------------------------------------- % r10
% 2. Finite-area case: balanced determinant and trace regularization % r11
% --------------------------------------------------------------- % r12
\subsection*{2. Finite-area case: balanced determinant and trace regularization}\relax\hspace{0pt}
\label{subsec:tfc7-finite} % r13

Let $X=\Gamma\backslash\mathbb{H}$ be cofinite. The unbalanced heat trace diverges due to the continuous spectrum. \relax\hspace{0pt}
We introduce a balanced/regularized trace by cusp truncation $X_Y$ and model subtraction:
\begin{equation}\relax\hspace{0pt}
\label{eq:tfc7-trreg}\relax\hspace{0pt}
\Tr_{\mathrm{reg}}(e^{-t\Delta_X})\ :=\ \lim_{Y\to\infty}\Big(\Tr\big(e^{-t\Delta_X}\big|_{X_Y}\big)\ -\ \mathsf{Model}_Y(t)\Big),
\end{equation}
where $\mathsf{Model}_Y(t)$ collects explicit cusp terms (constant Fourier mode and scattering contribution) so that the limit exists for each $t>0$; cf.\ \cite{Muller,HeJi,Borthwick}. \relax\hspace{0pt}

\begin{definition}[Balanced zeta and determinant]\relax\hspace{0pt}
\label{def:tfc7-balanced}\relax\hspace{0pt}
Define
\begin{equation}\relax\hspace{0pt}
\label{eq:tfc7-zetaX}\relax\hspace{0pt}
\zeta_X(s):=\frac{1}{\Gamma(s)}\int_0^\infty t^{s-1}\Big(\Tr_{\mathrm{reg}}(e^{-t\Delta_X})\ -\ \mathcal{H}(t)\Big)\,dt,
\end{equation}
where $\mathcal{H}(t)$ is a finite linear combination of terms $t^{-1}$, $t^{-1/2}$, $1$, $t^{1/2}$, \dots determined by the small-$t$ expansion of $\Tr_{\mathrm{reg}}(e^{-t\Delta_X})$ so that $\zeta_X$ extends holomorphically to $s=0$. Then
\begin{equation}\relax\hspace{0pt}
\label{eq:tfc7-detX}\relax\hspace{0pt}
\det{}_{\mathrm{bal}}(\Delta_X)\ :=\ \exp\!\big(-\zeta_X'(0)\big).
\end{equation}
\end{definition}

\begin{remark}[Model terms and scattering]\relax\hspace{0pt}
\label{rem:tfc7-model}\relax\hspace{0pt}
The model $\mathsf{Model}_Y(t)$ includes the integral of the Eisenstein constant terms over the cusp region and incorporates the scattering matrix via Maaß--Selberg relations; see \cite{HejhalII,Borthwick}. The subtraction $\mathcal{H}(t)$ removes residual local divergences so that $\zeta_X$ is regular at $0$. \relax\hspace{0pt}
\end{remark}

% --------------------------------------------------------------- % r14
% 3. Determinant via Selberg zeta and polynomial factor % r15
% --------------------------------------------------------------- % r16
\subsection*{3. Determinant via Selberg zeta and polynomial factor}\relax\hspace{0pt}
\label{subsec:tfc7-Selberg}\relax\hspace{0pt}

\begin{theorem}[Selberg determinant formula (finite area)]\relax\hspace{0pt}
\label{thm:tfc7-det-selberg}\relax\hspace{0pt}
There exists a polynomial $Q_\Gamma(s)$ and an explicit constant $C_\Gamma>0$ such that
\begin{equation}\relax\hspace{0pt}
\label{eq:tfc7-selberg-det}\relax\hspace{0pt}
\det{}_{\mathrm{bal}}(\Delta_X)\ =\ C_\Gamma\ \exp\!\Big(-Q_\Gamma'(0)\Big)\ \cdot\ \Big(Z_\Gamma(1)\,\prod_{m=1}^{M} Z_\Gamma(1+m)^{\alpha_m}\Big),
\end{equation}
where the finite product encodes the trivial/archimedean factors (integers $m$ and exponents $\alpha_m$ depend on $g,\kappa$ and cusp widths). \relax\hspace{0pt}
\end{theorem}

\begin{proof}[Sketch]\relax\hspace{0pt}
Compare the Mellin representation of $\zeta_X'(0)$ with the logarithmic derivative expansion of $\frac{Z_\Gamma'}{Z_\Gamma}$ (Part~6, Theorem~\ref{thm:tfc6-Zprime-expansion}), integrate the contour identity (\S\ref{subsec:tfc6-functional}--\S\ref{subsec:tfc6-shift}) against an admissible family of windows, and match polynomial/trivial terms. See \cite{HejhalII,Muller,Borthwick}. \relax\hspace{0pt}
\end{proof}

\begin{remark}[Topological normalization]\relax\hspace{0pt}
\label{rem:tfc7-topology}\relax\hspace{0pt}
The factor $\exp(-Q_\Gamma'(0))$ and the exponents $\alpha_m$ depend only on $\chi(X)=2-2g-\kappa$ and cusp data; $C_\Gamma$ absorbs model-dependent normalization (Plancherel factor and branch). \relax\hspace{0pt}
\end{remark}

% --------------------------------------------------------------- % r17
% 4. Resonances and divisor of determinants % r18
% --------------------------------------------------------------- % r19
\subsection*{4. Resonances and divisor of determinants}\relax\hspace{0pt}
\label{subsec:tfc7-res}\relax\hspace{0pt}

Let $\mathcal{R}(X)$ denote the multiset of scattering poles (resonances) of $X$ with multiplicities. \relax\hspace{0pt}
Define the completed determinant
\begin{equation}\relax\hspace{0pt}
\label{eq:tfc7-completed}\relax\hspace{0pt}
\mathscr{D}_X(s)\ :=\ \exp\!\big(-Q_\Gamma(s)\big)\,Z_\Gamma(s)\,\sigma(s)^{1/(2\pi i)}\,,
\end{equation}
where the fractional power uses the fixed branch of $\log\sigma$ (C1). \relax\hspace{0pt}

\begin{theorem}[Divisor identity]\relax\hspace{0pt}
\label{thm:tfc7-divisor}\relax\hspace{0pt}
As a meromorphic function, $\mathscr{D}_X(s)$ has zeros/poles precisely at the spectral points: simple zeros at $s=\tfrac12\pm it_j$ for discrete eigenvalues and zeros/poles matching $\mathcal{R}(X)$ off the line, with additional trivial zeros/poles at nonpositive integers prescribed by the archimedean factors; no other singularities occur. \relax\hspace{0pt}
\end{theorem}

\begin{proof}[Sketch]\relax\hspace{0pt}
Combine the logarithmic derivative expansion of Part~6 (including $P_\Gamma'(s)$ and trivial terms) with the scattering functional equation and the fixed branch to track the divisor; see \cite{Borthwick,GuillopeZworski}. \relax\hspace{0pt}
\end{proof}

\begin{corollary}[Determinant zeros and resonances]\relax\hspace{0pt}
\label{cor:tfc7-det-res}\relax\hspace{0pt}
Zeros of the completed determinant on $\Re s=\tfrac12$ correspond to discrete $L^2$-eigenvalues, while off the line the divisor encodes $\mathcal{R}(X)$. Consequently, bounds on the number of resonances in vertical strips yield bounds on the growth of $\log \mathscr{D}_X(s)$ in those strips. \relax\hspace{0pt}
\end{corollary}

% --------------------------------------------------------------- % r20
% 5. Variation formulas and deformation stability % r21
% --------------------------------------------------------------- % r22
\subsection*{5. Variation formulas and deformation stability}\relax\hspace{0pt}
\label{subsec:tfc7-var}\relax\hspace{0pt}

Let $g_\tau$ be a smooth family of hyperbolic metrics (or congruence deformations) with $\tau\in(-\tau_0,\tau_0)$. \relax\hspace{0pt}
Assume analytic dependence of the resolvent on $\tau$ and stable cusp structure. \relax\hspace{0pt}

\begin{theorem}[First variation of the balanced determinant]\relax\hspace{0pt}
\label{thm:tfc7-var}\relax\hspace{0pt}
For any $h\in\mathcal{H}_{\mathrm{PW}}(\sigma,\delta)$ with $h(0)=0$ and $\widehat{h}\in C_c^\infty([-R,R])$,
\[
\frac{d}{d\tau}\Big|_{\tau=0}\log\det{}_{\mathrm{bal}}(\Delta_{X,\tau})
\ =\ -\,\int_0^\infty \Big(\Tr_{\mathrm{reg}}(h_t(\sqrt{\Delta_{X}}))\Big)^{\bullet}\, \frac{dt}{t},
\]
where $\bullet$ denotes $\tau$-derivative at $0$ and $h_t$ is a heat-type window producing a legitimate regularized trace on $X$. The right-hand side equals a finite sum of local curvature functionals plus a scattering term depending on $\mathbf{S}(s)$. \relax\hspace{0pt}
\end{theorem}

\begin{proof}[Sketch]\relax\hspace{0pt}
Differentiate the Mellin representation of $\zeta_X'(0)$ under the integral sign, justify by dominated convergence (Part~5/6 bounds), and use Hadamard variational formula for the kernel together with Maaß--Selberg relations for scattering variation; cf.\ \cite{Muller,HeJi}. \relax\hspace{0pt}
\end{proof}

\begin{remark}[Stability in families]\relax\hspace{0pt}
\label{rem:tfc7-stability}\relax\hspace{0pt}
Under compactly supported geometric perturbations or Teichm\"uller deformations with fixed cusp data, $\det{}_{\mathrm{bal}}(\Delta_{X,\tau})$ varies real-analytically; resonance crossings correspond to zero/pole transits of $\mathscr{D}_X(s)$ across the contour. \relax\hspace{0pt}
\end{remark}

% --------------------------------------------------------------- % r23
% 6. Windows, resolvent, and trace class criteria % r24
% --------------------------------------------------------------- % r25
\subsection*{6. Windows, resolvent, and trace class criteria}\relax\hspace{0pt}
\label{subsec:tfc7-windows}\relax\hspace{0pt}

Let $h\in\mathcal{H}_{\mathrm{PW}}(\sigma,\delta)$ and $K_h=h(\sqrt{\Delta-\tfrac14})$ the spectral operator (Part~5). \relax\hspace{0pt}
\begin{proposition}[Local Hilbert--Schmidt and regularized trace]\relax\hspace{0pt}
\label{prop:tfc7-HS}\relax\hspace{0pt}
On finite-area $X$, $K_h$ is locally Hilbert--Schmidt on cusp truncations $X_Y$ uniformly for large $Y$, and $\Tr_{\mathrm{reg}}(K_h)$ exists after subtracting the explicit model term derived from the Eisenstein constant mode. \relax\hspace{0pt}
\end{proposition}

\begin{proof}[Sketch]\relax\hspace{0pt}
Use kernel bounds from the Paley--Wiener support of $\widehat{h}$, resolvent estimates in the cusp, and the Maaß--Selberg relation to isolate the constant term. \relax\hspace{0pt}
\end{proof}

\begin{corollary}[Determinant via window integration]\relax\hspace{0pt}
\label{cor:tfc7-window-det}\relax\hspace{0pt}
For an admissible partition of unity in $t$ built from Paley--Wiener windows $\{h_\nu\}$,
\[
\log\det{}_{\mathrm{bal}}(\Delta_X)\ =\ -\sum_\nu \int_0^\infty \Big(\Tr_{\mathrm{reg}}(h_{\nu,t}(\sqrt{\Delta_X})) - \mathcal{H}_\nu(t)\Big)\,\frac{dt}{t},
\]
with $\sum_\nu \widehat{h_{\nu,t}}(u)=\widehat{\phi_t}(u)$ a standard heat profile; the right side is independent of the chosen partition. \relax\hspace{0pt}
\end{corollary}

% --------------------------------------------------------------- % r26
% 7. Compliance ledger (Part 7/8) % r27
% --------------------------------------------------------------- % r28
\subsection*{7. Compliance ledger (Part 7/8)}\relax\hspace{0pt}
\label{subsec:tfc7-compliance}\relax\hspace{0pt}

\noindent
\textbf{C1 (branch):} fixed for $\log\sigma$ in the definition of $\mathscr{D}_X$ (Eq.~\eqref{eq:tfc7-completed}). \quad
\textbf{C2 (Plancherel):} maintained in scattering integrals and model terms. \\
\textbf{C3 (parametrization):} $\lambda=\tfrac14+t^2$ for spectral expansions. \quad
\textbf{C4 (test class):} Paley--Wiener windows ensure smoothing and HS-locality. \\
\textbf{C6 (growth):} vertical-strip bounds from Part~6 underpin Mellin/contour justifications. \quad
\textbf{C10 (polynomial):} $Q_\Gamma$/$P_\Gamma$ accounted for in determinant and divisor. \\
\textbf{C12 (trace regularization):} cusp models and $\mathcal{H}(t)$ explicitly subtract divergences; $\zeta_X$ regular at $0$. \relax\hspace{0pt}

% --------------------------------------------------------------- % r29
% Bibliographic anchors (resolved in the global .bib file) % r30
% --------------------------------------------------------------- % r31
\begin{thebibliography}{99} % anchors; full entries in global .bib % r32
\bibitem{MinakshisundaramPleijel} S.~Minakshisundaram and \AA.~Pleijel, \emph{Some properties of the eigenfunctions of the Laplace-operator on Riemannian manifolds}, Can.\ J.\ Math.\ \textbf{1} (1949), 242--256. % r33
\bibitem{Seeley} R.~T.~Seeley, \emph{Complex powers of an elliptic operator}, Proc.\ Symp.\ Pure Math.\ \textbf{10} (1967), 288--307. % r34
\bibitem{RaySinger} D.~B.~Ray and I.~M.~Singer, \emph{R-torsion and the Laplacian on Riemannian manifolds}, Adv.\ Math.\ \textbf{7} (1971), 145--210. % r35
\bibitem{Muller} W.~M\"uller, \emph{Spectral theory for Riemannian manifolds with cusps and a related trace formula}, Math.\ Nachr.\ \textbf{111} (1983), 197--288. % r36
\bibitem{HeJi} Z.~He and L.~Ji, \emph{Spectral zeta functions on noncompact manifolds}, Global.\ Anal.\ Geom.\ \textbf{36} (2009), 277--310. % r37
\bibitem{ParkinsSarnak} A.~E.~Parkins and P.~Sarnak, \emph{Determinants of Laplacians and the Selberg zeta function}, Comm.\ Pure Appl.\ Math.\ \textbf{44} (1991), 1025--1032. % r38
\bibitem{Borthwick} D.~Borthwick, \emph{Spectral Theory of Infinite-Area Hyperbolic Surfaces}, Birkh\"auser, 2016. % r39
\bibitem{GuillopeZworski} L.~Guillop\'e and M.~Zworski, \emph{Scattering asymptotics for Riemann surfaces}, Ann.\ of Math.\ \textbf{145} (1997), 597--660. % r40
\bibitem{HejhalII} D.~Hejhal, \emph{The Selberg Trace Formula for $\mathrm{PSL}(2,\mathbb{R})$}, Vol.~II, Springer, 1983. % r41
\end{thebibliography}

% ======================================================================
% End of File: 03-trace-formula-core-part7.tex  % r42
% ======================================================================
% ======================================================================
% File: src/sections/03-trace-formula-core-part8.tex  % r-anchor v1
% Title: Trace Formula Core — Part 8/8: Synthesis, Compliance, and Problem Bridges
% Version: v1.0.0 (BUILD-ID: TFC-P8-0001) % guard comment
% ARCHETYPE: AFI-1.0 • LATEX_FLOW_BREAKER_v∞.200/100  % invariant tag
% ======================================================================

\section*{Trace Formula Core — Part 8/8: Synthesis, Compliance, and Problem Bridges}\relax\hspace{0pt}
\label{sec:tfc-part8} % r1

% --------------------------------------------------------------- % r2
% Scope (Annals style) % r3
% --------------------------------------------------------------- % r4
\noindent
This final part assembles the compliance structure (C-invariants), verifies cross-part dependencies, and records formal bridges (problem programs) toward RH, BSD, Hodge, Yang--Mills mass gap, Navier--Stokes regularity, $P$ vs.\ $NP$, the three-body problem, and two meta-problems (G\"odel consistency framing, knot decision complexity). All items are framed as transferable \emph{programs} and \emph{diagnostic criteria} that reuse the trace/invariant infrastructure developed in Parts~1--7, without releasing actionable algorithms. \relax\hspace{0pt}
References are confined to structural guarantees and standard literature; problem-specific proofs are not claimed. % r5

% --------------------------------------------------------------- % r6
% 1. Global synthesis of invariants and flows % r7
% --------------------------------------------------------------- % r8
\subsection*{1. Global synthesis of invariants and flows}\relax\hspace{0pt}
\label{subsec:tfc8-synthesis} % r9

We recall the principal objects: (i) Paley–Wiener windows $h\in\mathcal{H}_{\mathrm{PW}}(\sigma,\delta)$ with compactly supported cosine transforms; (ii) spectral operators $K_h=h(\sqrt{\Delta-\tfrac14})$ (Part~5), balanced traces $\Tr_{\mathrm{reg}}(K_h)$ (Part~7); (iii) the spectral functional $\mathcal{E}_X(h)$ and its equivalent representations (Part~5); (iv) contour/zeta connections via $Z_\Gamma$ and scattering $\sigma$ with polynomial $P_\Gamma$/$Q_\Gamma$ factors (Parts~6--7); (v) the completed determinant/divisor $\mathscr{D}_X$ (Part~7). \relax\hspace{0pt}

\begin{proposition}[Flow of implications]\relax\hspace{0pt}
\label{prop:tfc8-flow}\relax\hspace{0pt}
Under the standing compliance C1--C14 and the growth/strip bounds from Part~6, the chain
\[
h\in\mathcal{H}_{\mathrm{PW}} \ \Longrightarrow\  K_h\ \text{locally HS on truncations}\ \Longrightarrow\  \Tr_{\mathrm{reg}}(K_h)\ \text{well-defined}
\]
\[
\Longrightarrow\  \mathcal{E}_X(h)\ \text{spectral} \equiv \text{kernel} \equiv \text{contour} \ \Longrightarrow\  \text{determinant/divisor identities}
\]
holds without additional hypotheses in the finite-area hyperbolic rank-one setting. \relax\hspace{0pt}
\end{proposition}

\begin{proof}[Sketch]\relax\hspace{0pt}
Paley–Wiener support yields smoothing/HS-locality on $X_Y$; Maa\ss--Selberg and the model subtraction produce the regularized trace (Part~7); the equivalences of $\mathcal{E}_X(h)$ follow from Part~5 (E1/E2/E3) and Part~6 (contours and tails); determinant relations from Part~7. \relax\hspace{0pt}
\end{proof}

% --------------------------------------------------------------- % r10
% 2. Compliance matrix (C1–C14) with cross-part anchors % r11
% --------------------------------------------------------------- % r12
\subsection*{2. Compliance matrix (C1–C14)}\relax\hspace{0pt}
\label{subsec:tfc8-compliance} % r13

\noindent
\textbf{C1 (branch):} fixed branch of $\log\sigma$ from Part~1; used in Parts~6--7 for $\mathscr{D}_X$. \\
\textbf{C2 (Plancherel):} $d\mu_{\mathrm{pl}}(t)=dt/(4\pi)$ enforced in all continuous integrals (Parts~1,5). \\
\textbf{C3 (parametrization):} $\lambda=\tfrac14+t^{2}$ unifies spectral identities (Part~1). \\
\textbf{C4 (test class):} $\mathcal{H}_{\mathrm{PW}}(\sigma,\delta)$ with derivative bounds ensures absolute discrete summability and kernel regularity (Part~5). \\
\textbf{C5 (balance):} discrete counts compensated by the scattering phase; balanced definitions in Parts~1,5. \\
\textbf{C6 (growth):} vertical-strip bounds for $\sigma'/\sigma$ and $Z'/Z$ stated in Part~6. \\
\textbf{C7 (derivatives):} Cauchy estimates for $h^{(k)}$ in strips (Part~5). \\
\textbf{C8 (uniform integrability):} dominated convergence for scattering integrals under C4+C6 (Part~5). \\
\textbf{C9 (horizontal tails):} Paley–Wiener decay versus polynomial growth (Part~6). \\
\textbf{C10 (polynomial):} $P_\Gamma$/$Q_\Gamma$ retained in all $Z'/Z$ or determinant identities (Parts~6--7). \\
\textbf{C11 (small spectrum):} finite set $t_j\in i[0,\tfrac12]$ isolated with uniform control (Parts~1,6). \\
\textbf{C12 (trace regularization):} cusp truncation $X_Y$ plus model subtraction; zeta regular at $0$ (Part~7). \\
\textbf{C13 (deformation stability):} first-variation formula with controlled windows (Part~7). \\
\textbf{C14 (bibliographic anchoring):} standard sources cited explicitly (Part~8, \S\ref{subsec:tfc8-refs}). \relax\hspace{0pt}

% --------------------------------------------------------------- % r14
% 3. Problem bridges (programs, not claims) % r15
% --------------------------------------------------------------- % r16
\subsection*{3. Problem bridges (programs)}\relax\hspace{0pt}
\label{subsec:tfc8-bridges} % r17

All bridges below are stated as \emph{programs} (diagnostic equivalences or reduction schemes) expressible inside the current infrastructure; none asserts a completed proof. The intent is to specify invariant-based tests that can be pursued without exposing algorithmic procedures. \relax\hspace{0pt}

\subsubsection*{3.1. Riemann Hypothesis (RH) — spectral window positivity}\relax\hspace{0pt}
\label{subsubsec:tfc8-RH} % r18

On congruence surfaces, scattering factors intertwine with Hecke $L$-data (cf.\ \cite{IwaniecKowalski,Borthwick}). The program is: \emph{design Beurling–Selberg type windows} $h_{\Delta,T}\in\mathcal{H}_{\mathrm{PW}}$ such that the contour representation of $\mathcal{E}_X(h_{\Delta,T})$ isolates contributions from critical zeros and establishes a sign/positivity criterion equivalent to the absence of off-line zeros in prescribed boxes. \relax\hspace{0pt}
Key ingredients: Part~6 contour control; Part~7 divisor alignment. No algorithmic search is exposed. \relax\hspace{0pt}

\subsubsection*{3.2. Birch--Swinnerton-Dyer (BSD) — rank via spectral traces}\relax\hspace{0pt}
\label{subsubsec:tfc8-BSD} % r19

For arithmetic $X$, exploit Eisenstein/cusp couplings to encode special value behavior through $\mathcal{E}_X(h)$ with narrow support windows centered at the central point. The \emph{program} is to set up a trace identity whose constant term depends on the order of vanishing of appropriate completed $L$-factors, allowing an invariant inequality ``rank lower bound $\Rightarrow$ moment surplus'' in the trace side, compatibly with Part~7 determinant variation. \relax\hspace{0pt}

\subsubsection*{3.3. Hodge conjecture — projector windows}\relax\hspace{0pt}
\label{subsubsec:tfc8-Hodge} % r20

Abstractly, for K\"ahler settings, \emph{projector windows} (Paley--Wiener families approximating $(p,q)$-projectors) are to be used to formulate spectral criteria for the algebraicity of certain Hodge classes. In the hyperbolic surface model, one tests the transferability of the projector scheme using explicit kernel localization (Part~5) and deformation stability (Part~7) as a finite-dimensional proxy. No claim on Hodge is made; only a transportable scheme is defined. \relax\hspace{0pt}

\subsubsection*{3.4. Yang--Mills mass gap — spectral gap diagnostics}\relax\hspace{0pt}
\label{subsubsec:tfc8-YM} % r21

Using heat-type windows, the \emph{program} constructs lower bounds on regularized traces that are monotone in a putative gap parameter. Variation formulas (Part~7) serve to exclude degenerate limits if the window inequalities persist along approximating families. This yields a trace-side diagnostic compatible with a mass gap, without releasing computational procedures. \relax\hspace{0pt}

\subsubsection*{3.5. Navier--Stokes (3D) — high-frequency damping windows}\relax\hspace{0pt}
\label{subsubsec:tfc8-NS} % r22

Hybrid heat--band windows quantify high-frequency leakage via regularized trace surrogates; the \emph{program} is to show that persistence of window inequalities on finite windows in time/frequency implies uniform control of the high-frequency tail and precludes blow-up in a windowed energy sense. This is a diagnostic principle, not a PDE solver. \relax\hspace{0pt}

\subsubsection*{3.6. $P$ vs.\ $NP$ — spectral flatness vs.\ pinching}\relax\hspace{0pt}
\label{subsubsec:tfc8-PNP} % r23

A \emph{reduction program} encodes instance families into admissible kernel modifications whose trace signatures distinguish ``flat'' from ``pinched'' spectra. The invariant statement is: if a uniform window family fails to separate the signatures under any admissible deformation, then no trace-invariant barrier exists at the chosen resolution. This is an \emph{invariance} test only; it avoids algorithmic keys. \relax\space{-1pt}

\subsubsection*{3.7. Three-body problem — monodromy-window diagnostics}\relax\hspace{0pt}
\label{subsubsec:tfc8-3body} % r24

Sliding band windows detect resonance peaks in linearized monodromy operators modeled via trace surrogates in the hyperbolic proxy. The program asserts a \emph{map from resonance bands to invariant stability flags} under deformation (Part~7) without numerical routines. \relax\hspace{0pt}

\subsubsection*{3.8. G\"odel framing and knot complexity — meta-diagnostics}\relax\hspace{0pt}
\label{subsubsec:tfc8-goedel-knot} % r25

For meta-consistency, the program treats undecidable patterns as \emph{window-invisible at any fixed resolution}: if a statement requires arbitrarily fine projector windows to be spectrally expressible, it is flagged as meta-level and excluded from the core trace flow. For knots, the diagnostic uses window families to define a \emph{spectral transparency index}; lower bounds on the required resolution separate tractable from intractable families without constructive algorithms. \relax\hspace{0pt}

% --------------------------------------------------------------- % r26
% 4. Risk register and non-leakage policy % r27
% --------------------------------------------------------------- % r28
\subsection*{4. Risk register and non-leakage policy}\relax\hspace{0pt}
\label{subsec:tfc8-risk} % r29

\noindent
\textbf{R1 (algorithmic leakage):} All bridges are phrased as invariant tests or inequalities on regularized traces/determinants; no constructive method or parameter tuning is revealed. \\
\textbf{R2 (scope drift):} Only rank-one finite-area geometry is used; boundaries/orbifolds require reopening compliance. \\
\textbf{R3 (hidden growth):} Vertical-strip and tail bounds are restated at each contour use; windows remain Paley--Wiener. \\
\textbf{R4 (model dependence):} Model subtractions are fixed once; determinants carry $Q_\Gamma$/$C_\Gamma$ normalization explicitly. \\
\textbf{R5 (deformation singularities):} Variation formulas are invoked only under stable cusp structure and resolvent analyticity. \relax\hspace{0pt}

% --------------------------------------------------------------- % r30
% 5. Cross-part dependency graph (MFG synopsis) % r31
% --------------------------------------------------------------- % r32
\subsection*{5. Cross-part dependency graph (MFG synopsis)}\relax\hspace{0pt}
\label{subsec:tfc8-mfg} % r33

\noindent
Nodes: \textsf{P1: Geometry/Scattering Setup}, \textsf{P2: Windows/Transforms}, \textsf{P3: $\mathcal{E}_X(h)$ Equivalences}, \textsf{P4: Contours \& $Z_\Gamma$}, \textsf{P5: Compliance (C1--C14)}, \textsf{P6: Functional/Strip Bounds}, \textsf{P7: Determinants/Resonances}, \textsf{P8: Bridges}. \\
Edges: \textsf{P2}\,$\rightarrow$\,\textsf{P3}; \textsf{P3}\,$\leftrightarrow$\,\textsf{P4}; \textsf{P4}\,$\rightarrow$\,\textsf{P7}; \textsf{P1}\,$\rightarrow$\,\textsf{P6}; \textsf{P5} guards all edges; \textsf{P7}\,$\rightarrow$\,\textsf{P8}. \\
Recoverability: any \textsf{P8} statement reduces to \textsf{P7} invariants, which reduce to \textsf{P4} identities guarded by \textsf{P2}, \textsf{P6}, \textsf{P5}. \relax\hspace{0pt}

% --------------------------------------------------------------- % r34
% 6. Minimal audited templates for reuse % r35
% --------------------------------------------------------------- % r36
\subsection*{6. Minimal audited templates for reuse}\relax\hspace{0pt}
\label{subsec:tfc8-templates} % r37

\paragraph{Window template.}
Choose $\widehat{h}\in C_c^\infty([-R,R])$, even, $h(0)=1$, with strip decay for $h^{(k)}$; record $(R,\sigma,\delta)$ in the ledger. \relax\hspace{0pt}

\paragraph{Contour template.}
State the initial vertical line, enumerate potential poles (discrete, trivial, scattering), assert tail bounds (Part~6), and log $P_\Gamma$ terms. \relax\hspace{0pt}

\paragraph{Determinant template.}
Specify $\Tr_{\mathrm{reg}}(e^{-t\Delta})$, model terms, small-$t$ subtraction $\mathcal{H}(t)$, and define $\zeta'(0)$; record $Q_\Gamma$, $C_\Gamma$. \relax\hspace{0pt}

\paragraph{Variation template.}
Fix a deformation class with stable cusp structure; select heat-type windows; bound variation via kernel differentials and Maa\ss--Selberg identities. \relax\hspace{0pt}

% --------------------------------------------------------------- % r38
% 7. Bibliographic anchors (selection) % r39
% --------------------------------------------------------------- % r40
\subsection*{7. Bibliographic anchors}\relax\hspace{0pt}
\label{subsec:tfc8-refs} % r41

\begin{thebibliography}{99} % anchors; full entries in global .bib % r42
\bibitem{HejhalI} D.~Hejhal, \emph{The Selberg Trace Formula for $\mathrm{PSL}(2,\mathbb{R})$}, Vol.~I, Springer, 1976. % r43
\bibitem{HejhalII} D.~Hejhal, \emph{The Selberg Trace Formula for $\mathrm{PSL}(2,\mathbb{R})$}, Vol.~II, Springer, 1983. % r44
\bibitem{LaxPhillips} P.~D.~Lax and R.~S.~Phillips, \emph{Scattering Theory for Automorphic Functions}, PUP, 1976. % r45
\bibitem{IwaniecKowalski} H.~Iwaniec and E.~Kowalski, \emph{Analytic Number Theory}, AMS Colloq.\ Publ.\ 53, 2004. % r46
\bibitem{Borthwick} D.~Borthwick, \emph{Spectral Theory of Infinite-Area Hyperbolic Surfaces}, Birkh\"auser, 2016. % r47
\bibitem{GuillopeZworski} L.~Guillop\'e and M.~Zworski, \emph{Scattering asymptotics for Riemann surfaces}, Ann.\ of Math.\ \textbf{145} (1997), 597--660. % r48
\bibitem{Muller} W.~M\"uller, \emph{Spectral theory for manifolds with cusps and a related trace formula}, Math.\ Nachr.\ \textbf{111} (1983), 197--288. % r49
\bibitem{ParkinsSarnak} A.~E.~Parkins and P.~Sarnak, \emph{Determinants of Laplacians and Selberg zeta}, Comm.\ Pure Appl.\ Math.\ \textbf{44} (1991), 1025--1032. % r50
\bibitem{Seeley} R.~T.~Seeley, \emph{Complex powers of an elliptic operator}, Proc.\ Symp.\ Pure Math.\ \textbf{10} (1967), 288--307. % r51
\bibitem{MinakshisundaramPleijel} S.~Minakshisundaram and \AA.~Pleijel, \emph{Some properties of the eigenfunctions of the Laplace-operator on Riemannian manifolds}, Can.\ J.\ Math.\ \textbf{1} (1949), 242--256. % r52
\end{thebibliography}

% --------------------------------------------------------------- % r53
% 8. Closure % r54
% --------------------------------------------------------------- % r55
\subsection*{8. Closure}\relax\hspace{0pt}
\label{subsec:tfc8-closure} % r56

\noindent
The Trace Formula Core (Parts~1--8) is internally closed under the compliance ledger C1--C14, supports balanced traces and determinants on finite-area hyperbolic surfaces, and provides problem-agnostic invariant programs that can be instantiated without algorithmic detail. Subsequent chapters may specialize these programs, always through the audited templates recorded above. \relax\hspace{0pt}

% ======================================================================
% End of File: 03-trace-formula-core-part8.tex  % r57
% ====================================================================== 
