% ======================================================================
% File: src/sections/02-preliminaries.tex
% Chapter 2 — Preliminaries and Notational Framework
% Part 1/5 — Geometric and Spectral Setting
% Diamond++ 10/20 Standard — Absolute Version (Polished to 20/10, MEA-Core-SS • sealed)
% ======================================================================

\chapter{Preliminaries and Notational Framework}
\label{chap:preliminaries}

\section{Geometric and Spectral Setting}
\label{sec:geom-spectral-setting}

% ------------------ DIAMOND++ SCOPE BOX (MEA-Core-SS) ------------------
% (Requires: \usepackage{tcolorbox})
\begin{tcolorbox}[colback=gray!5,colframe=gray!35,title=Scope \& Assumptions (MEA-Core-SS • enforced)]
\begin{itemize}
  \item \textbf{Completeness.} All Riemannian manifolds $(M,g)$ under consideration are \emph{complete}.
  \item \textbf{Core classes.} We treat:
        (i) compact manifolds without boundary;
        (ii) finite-area hyperbolic surfaces $X=\Gamma\backslash\mathbb{H}$ with cusps for cofinite Fuchsian groups $\Gamma\subset \mathrm{PSL}_2(\mathbb{R})$.
        Infinite-volume geometries and boundary value problems (Dirichlet/Neumann) are excluded unless explicitly stated.
  \item \textbf{Spectral split (noncompact).} On finite-area hyperbolic surfaces, the spectral measure consists of a discrete $L^2$-spectrum and a continuous component generated by Eisenstein series. Scattering data are encoded by the scattering matrix $\mathbf{S}(s)$ with determinant $\sigma(s)$.
  \item \textbf{Counting convention.} $N(\Lambda)$ denotes the \emph{discrete} counting function unless explicitly balanced by the scattering phase $\Xi(\lambda)$ (Krein's spectral shift).
  \item \textbf{Normalization discipline.} Eisenstein series, scattering coefficients, and Plancherel density follow the conventions stated in \S\ref{subsec:spectral-decomposition}; all constants and normalizations are audit-linked to Appendix~J (MEA-Core-SS ledger).
\end{itemize}
\end{tcolorbox}
% -----------------------------------------------------------------------

\subsection*{A. Classes of Manifolds $(M,g)$}
\label{subsec:classes}

We fix smooth complete Riemannian manifolds $(M,g)$ of dimension $d\ge1$ and distinguish:
\begin{enumerate}[label=(\roman*)]

  \item \textbf{Compact, no boundary.}
        The spectrum of $\Delta_g$ is purely discrete, nonnegative, and accumulates only at $+\infty$.
        Eigenfunctions $\{u_j\}_{j\ge0}$ form a complete orthonormal basis of $L^2(M,g)$.

  \item \textbf{Finite-area hyperbolic surfaces with cusps.}
        Prototype noncompact model: $X=\Gamma\backslash\mathbb{H}$ with $\Gamma$ cofinite in $\mathrm{PSL}_2(\mathbb{R})$.
        The spectral measure splits into a discrete $L^2$-spectrum $\{\lambda_j\}_{j\ge0}$ and a continuous spectrum generated by Eisenstein series
        $E_{\mathfrak a}(z,\tfrac12+it)$ attached to cusps $\mathfrak a$, with scattering matrix $\mathbf{S}(s)$ (size $\kappa\times \kappa$, $\kappa$ the number of cusps) and determinant $\sigma(s)=\det \mathbf{S}(s)$.

  \item \textbf{Excluded from the core.}
        Infinite-volume geometries and manifolds with boundary (requiring boundary conditions and modified trace/Plancherel formalisms) are excluded from the core development; whenever needed, they are addressed with explicit hypotheses.

\end{enumerate}

\subsection*{B. Laplace--Beltrami Operator and Spectrum}
\label{subsec:laplacian}

The Laplace--Beltrami operator acts on $C^\infty(M)$ by
\[
   \Delta_g f \;=\; -\mathrm{div}_g(\nabla_g f).
\]
On complete $(M,g)$, the Friedrichs extension realizes $\Delta_g$ as a self-adjoint,
nonnegative operator on $L^2(M,g)$ with quadratic-form domain $H^1(M)$.

\begin{itemize}

  \item \textbf{Compact case.}
        The spectrum is discrete:
        \[
          0=\lambda_0 < \lambda_1 \le \lambda_2 \le \cdots,\qquad \lambda_j\to+\infty.
        \]

  \item \textbf{Finite-area hyperbolic surfaces ($d=2$).}
        The spectral decomposition is
        \[
           \spec(\Delta_g)
           \;=\;
           \{\lambda_j\}_{j=0}^\infty
           \;\cup\;
           \big\{\tfrac14 + t^2 : t\in\mathbb{R}\big\},
        \]
        so the continuous branch starts at $\lambda_c=\tfrac14$. For the \emph{discrete} eigenvalues we use the spectral parametrization
        \[
           \lambda_j=
           \begin{cases}
              \tfrac14 + t_j^2, & t_j\in\mathbb{R}  \quad (\lambda_j\ge\tfrac14),\\[4pt]
              \tfrac14 - r_j^2, & t_j=i r_j,\; r_j\in(0,\tfrac12] \quad (0\le\lambda_j<\tfrac14).
           \end{cases}
        \]

\end{itemize}

\begin{remark}[Essential self-adjointness and the threshold \texorpdfstring{$\lambda_c=\tfrac14$}{lambda\_c=1/4}]
On complete $(M,g)$, $\Delta_g$ is essentially self-adjoint on $C_c^\infty(M)$, and the spectral theorem applies.
For $X=\Gamma\backslash\mathbb{H}$, the identification of the continuous spectrum with $\{\tfrac14+t^2: t\in\mathbb{R}\}$
arises from the principal-series representations of $\mathrm{SL}_2(\mathbb{R})$; hence the bottom of the continuum is $\lambda_c=\tfrac14$.
\end{remark}

\subsection*{C. Spectral Counting: Weyl and Selberg Asymptotics}
\label{subsec:weyl}

\paragraph{Compact case (Hörmander).}
For
\[
   N(\Lambda) \;=\; \#\{\lambda_j \le \Lambda\},
\]
one has
\[
   N(\Lambda) \;\sim\; \frac{\omega_d}{(2\pi)^d}\,\mathrm{vol}_g(M)\,\Lambda^{d/2},
   \qquad \Lambda\to\infty,
\]
where $\omega_d$ is the Euclidean volume of the unit ball in $\mathbb{R}^d$ \cite{Hormander1968}.

\paragraph{Finite-area hyperbolic surfaces: discrete counting.}
Let $N_{\mathrm{disc}}(\lambda)=\#\{\lambda_j\le\lambda\}$ count only the $L^2$-discrete spectrum.
Then, for $d=2$,
\[
  N_{\mathrm{disc}}(\lambda)
  \;=\; \frac{\mathrm{vol}(X)}{4\pi}\,\lambda \;+\; O\!\big(\sqrt{\lambda}\,\log\lambda\big),
  \qquad \lambda\to\infty,
\]
so the principal coefficient is $\frac{\mathrm{vol}(X)}{4\pi}$ \cite{Selberg1956,Hejhal1983,Hejhal1983II}.

\begin{definition}[Balanced counting]
By \emph{balanced} (or \emph{scattering-corrected}) counting we mean the quantity
\(
  N_{\mathrm{disc}}(\lambda)-\Xi(\lambda),
\)
i.e. the discrete $L^2$-count adjusted by the scattering phase so that the main term matches the Plancherel density of the model space.
\end{definition}

\paragraph{Balanced counting via the scattering phase (Hejhal, Lax--Phillips).}
Let $\mathbf{S}(s)$ be the $\kappa\times\kappa$ scattering matrix and $\sigma(s)=\det\mathbf{S}(s)$.
Define the (Krein) spectral shift
\[
  \Xi(\lambda) \;=\; \frac{1}{2\pi i}\,\log \sigma\!\Big(\tfrac12 + i\sqrt{\lambda - \tfrac14}\Big).
\]
\paragraph{Branch normalization.}
We fix the branch of $\log\sigma(\tfrac12+it)$ by analytic continuation from $t=+\infty$, so that
$\Xi(\lambda)\to 0$ as $\lambda\to\infty$. With this choice $\Xi(\lambda)$ is real-valued and defines the Krein spectral shift for the pair (geometric Laplacian, free model) in the Selberg/Lax--Phillips framework.
Then (see, e.g., \cite{Hejhal1983,Hejhal1983II,LaxPhillips1976})
\[
  N_{\mathrm{disc}}(\lambda) \;-\; \Xi(\lambda)
  \;=\; \frac{\mathrm{vol}(X)}{4\pi}\,\lambda \;+\; O\!\big(\sqrt{\lambda}\,\log\lambda\big),
\]
so scattering contributes at most logarithmic-order corrections without altering the leading coefficient.

\begin{remark}[Small eigenvalues]
Finite-area hyperbolic surfaces may possess finitely many ``small'' eigenvalues $\lambda_j<\tfrac14$.
These contribute only a bounded correction to $N_{\mathrm{disc}}(\lambda)$ and do not affect the leading term
$\frac{\vol(X)}{4\pi}\lambda$ nor the remainder $O(\sqrt{\lambda}\log\lambda)$ in the balanced identity.
\end{remark}

\begin{remark}[Dimension and generality]
For $d\ne 2$ or variable curvature, the compact main term scales as $\Lambda^{d/2}$;
noncompact balanced statements invoke the appropriate Plancherel density and low-energy thresholds.
We keep the $d=2$ hyperbolic setting as the canonical noncompact model throughout the core development.
\end{remark}

\subsection*{D. Spectral Decomposition and Plancherel Framework}
\label{subsec:spectral-decomposition}

\subsubsection*{D.1. General theory (Plancherel, Eisenstein normalization, functional calculus)}
Let $\{u_j\}$ be an orthonormal basis of eigenfunctions for the \emph{discrete} $L^2$-spectrum.
On finite-area hyperbolic surfaces, the continuous component is generated by Eisenstein series
$E_{\mathfrak{a}}(z,\tfrac12+it)$ attached to cusps $\mathfrak{a}$. We fix the following normalization.

\begin{remark}[Eisenstein normalization and scattering matrix]
For each cusp $\mathfrak a$ choose a scaling matrix $\sigma_{\mathfrak a}\in \mathrm{PSL}_2(\mathbb{R})$ mapping $\mathfrak a$ to $\infty$.
The Eisenstein series $E_{\mathfrak a}(z,s)$ is normalized so that its Fourier expansion at the cusp $\mathfrak a'$ reads
\[
  E_{\mathfrak a}\!\big(\sigma_{\mathfrak a'}z,s\big)
  \;=\;
  \delta_{\mathfrak a\mathfrak a'}\,y^s
  \;+\;
  \phi_{\mathfrak a\mathfrak a'}(s)\,y^{1-s}
  \;+\;
  \sum_{n\neq 0} c_{\mathfrak a\mathfrak a'}(n,s)\,\sqrt{y}\,K_{s-\frac12}(2\pi|n|y)\,e^{2\pi i n x},
\]
where $z=x+iy$. The \emph{scattering matrix} is
$\mathbf S(s)=\big(\phi_{\mathfrak a\mathfrak a'}(s)\big)_{\mathfrak a,\mathfrak a'}\in\mathbb C^{\kappa\times\kappa}$,
meromorphic with functional equation $\mathbf S(s)\mathbf S(1-s)=\mathbf I$; its determinant is the scattering determinant $\sigma(s)$ \cite{Hejhal1983II,LaxPhillips1976}.
Eisenstein series are not in $L^2(X)$; they furnish generalized eigenfunctions realizing the continuous spectrum in the spectral theorem.
\end{remark}

\begin{definition}[Spectral split]
On finite-area noncompact $X$ the spectral measure splits canonically into the $L^2$-discrete part and the continuous
part realized by Eisenstein series; we refer to this dichotomy as the \emph{spectral split}.
\end{definition}

\paragraph{Functional calculus (explicit projection measures).}
For $\Psi\in C_0^\infty(\mathbb{R})$ (and, by extension, bounded Borel $\Psi$ via the spectral theorem), the operator $\Psi(\Delta_g)$ acts through the spectral resolution
\[
  \Psi(\Delta_g) \;=\; \sum_{j} \Psi(\lambda_j)\,\langle \cdot, u_j\rangle u_j
  \;+\; \frac{1}{4\pi}\sum_{\mathfrak{a}}
      \int_{\mathbb{R}} \Psi\!\big(\tfrac14+t^2\big)\,
      \langle \cdot, E_{\mathfrak a}(\cdot,\tfrac12+it)\rangle\,
      E_{\mathfrak a}(z,\tfrac12+it)\,dt,
\]
where the factor $\frac{1}{4\pi}$ reflects the standard $dt$–normalization of the continuous spectral measure on hyperbolic surfaces.
All continuous-spectrum identities are read in the Plancherel/distributional sense.

\begin{remark}[Convergence and operator class]
All spectral series/integrals above are understood in the \emph{strong operator topology} on $L^2(X)$.
On compact $X$, $\Psi\in C_0^\infty(\mathbb{R})$ implies that $\Psi(\Delta_g)$ is smoothing and hence trace class.
On finite-area noncompact $X$, $\Psi(\Delta_g)$ is smoothing with a kernel smooth on $X\times X$; it is bounded on $L^2$
and Hilbert–Schmidt on compacta. Global trace class may fail without additional decay in the cuspidal region; in this monograph
we use $\Psi(\Delta_g)$ under $L^2$ pairings and within trace identities where continuous-spectrum contributions are handled
via the Plancherel measure and the scattering theory.
\end{remark}

\begin{remark}[Discrete vs continuous bookkeeping]
Whenever a formula sums over $\{\lambda_j\}$ or uses $N(\cdot)$, it concerns the \emph{discrete} part unless explicitly balanced by $\Xi(\lambda)$ or replaced by a Plancherel integral.
This convention is enforced by the MEA-Core-SS audit checks across the monograph.
\end{remark}

\subsubsection*{D.2. Example: The Modular Surface}
For $X=\mathrm{PSL}_2(\mathbb{Z})\backslash\mathbb{H}$ one has $\kappa=1$ cusp, and the continuous spectrum is $[\tfrac14,\infty)$.
The Eisenstein series $E(z,s)$ is normalized so that its constant term at $\infty$ is $y^s+\phi(s)y^{1-s}$; here $\mathbf S(s)=(\phi(s))$ is $1\times 1$ and $\sigma(s)=\phi(s)$.
The balanced counting identity specializes to
\[
  N_{\mathrm{disc}}(\lambda) - \Xi(\lambda)
  \;=\; \frac{\mathrm{vol}(X)}{4\pi}\,\lambda
  \;+\; O\!\big(\sqrt{\lambda}\,\log\lambda\big).
\]

\subsection*{E. Notation and Invariants (Audit Ledger Entry)}
\label{subsec:notation-invariants}

We record invariants used recurrently (all symbols are mirrored in the Glossary and Appendix~J, with provenance):
\begin{itemize}
  \item Dimension $d=\dim X$, volume $\mathrm{vol}(X)$, injectivity radius $\mathrm{inj}(X)$.
  \item Cuspidal data: number of cusps $\kappa$ and cusp widths $\{w_i\}$.
  \item Spectral gap $\beta_\Gamma$ (when applicable); bottom of continuum $\lambda_c$ ($=\tfrac14$ for $d=2$ hyperbolic surfaces).
  \item Discrete counting $N_{\mathrm{disc}}(\lambda)$; scattering determinant $\sigma(s)$; spectral shift $\Xi(\lambda)$.
\end{itemize}

\subsection*{F. Audit • Forward/Backward Links}
\label{subsec:audit-links}

\begin{itemize}
  \item \textbf{Audit outcome (sealed).}
        Completeness and finite-area scope fixed; discrete vs continuous spectrum distinguished and normalized; Weyl/Selberg asymptotics stated with balanced correction $\Xi(\lambda)$; Plancherel framework and explicit functional calculus recorded with normalizations and matrix sizes; convergence/branch conventions fixed.
  \item \textbf{Backward links.}
        Notation and invariants are mirrored in the Notation Glossary and Appendix~J (audit of constants, normalizations, and sources).
  \item \textbf{Forward links.}
        To \Cref{sec:spectral-decomposition} (Part 2/5) for test functions and functional-calculus refinements;
        to \Cref{sec:def-invariant} (Part 3/5) for the definition of the eono–fractal invariant;
        to \Cref{chap:kernel} for kernel truncations; to \Cref{chap:projector} for spectral projectors.
\end{itemize}

% ------------------ SOURCES (to be included in .bib) -------------------
% Selberg’s trace/counting:
%   @incollection{Selberg1956, author={Atle Selberg}, title={Harmonic analysis and discontinuous groups...}, booktitle={Proc. Sympos. Pure Math.}, year={1956}}
% Hejhal’s monographs:
%   @book{Hejhal1983, author={Dennis A. Hejhal}, title={The Selberg Trace Formula for PSL(2,R) I}, series={Lecture Notes in Math. 548}, publisher={Springer}, year={1983}}
%   @book{Hejhal1983II, author={Dennis A. Hejhal}, title={The Selberg Trace Formula for PSL(2,R) II}, series={Lecture Notes in Math. 1001}, publisher={Springer}, year={1983}}
% Lax–Phillips scattering:
%   @book{LaxPhillips1976, author={Peter D. Lax and Ralph S. Phillips}, title={Scattering Theory for Automorphic Functions}, Princeton UP, 1976}
% Hörmander’s Weyl law:
%   @article{Hormander1968, author={Lars Hörmander}, title={The spectral function of an elliptic operator}, journal={Acta Math.}, year={1968}}
% -----------------------------------------------------------------------

% ======================================================================
% End of Part 1/5 — Geometric and Spectral Setting (Polished, sealed)
% ======================================================================

