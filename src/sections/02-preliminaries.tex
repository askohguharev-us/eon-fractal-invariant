% ======================================================================
% File: src/sections/02-preliminaries-brilliant.tex
% Chapter 2 — Preliminaries and Notational Framework
% Part 1/5 (Brilliant Standard 20/10)
% Geometric and Spectral Setting
% ======================================================================

\chapter{Preliminaries and Notational Framework}
\label{chap:preliminaries}

\section{Geometric and Spectral Setting}
\label{sec:geom-spectral-setting}

\begin{tcolorbox}[colback=gray!5,colframe=gray!55,
  title=Scope and Assumptions (Part 1/5 • Brilliant Standard 20/10)]
\begin{itemize}
  \item \textbf{Completeness.} All $(M,g)$ are complete Riemannian manifolds without boundary. This ensures essential self-adjointness of $\Delta_g$ on $C_c^\infty(M)$ via the Friedrichs extension.
  \item \textbf{Core classes.}
  \begin{enumerate}[label=(\roman*)]
    \item Compact manifolds $(M,g)$ with $\dim M=d\ge 2$.
    \item Finite-area hyperbolic surfaces with cusps $X=\Gamma\backslash\mathbb H$, $\Gamma\subset \mathrm{PSL}_2(\mathbb R)$ cofinite, $\kappa$ cusps.
  \end{enumerate}
  Infinite-volume geometries, orbifolds, and manifolds with boundary are excluded.
  \item \textbf{Spectral decomposition.} For finite-area $X$, the spectral resolution consists of discrete $L^2$ eigenvalues and the continuous Eisenstein branch; scattering is encoded by $\mathbf S(s)$ and $\sigma(s)=\det\mathbf S(s)$.
  \item \textbf{Normalization.} Spectral parameterization $\lambda=\tfrac14+t^2$, Plancherel measure $d\mu_{\mathrm{pl}}(t)=dt/(4\pi)$, and a single global branch of $\log\sigma(s)$ are fixed. Constants are recorded in Appendix~J.
  \item \textbf{Topology of limits.} Unless stated otherwise, spectral expansions converge in the strong operator topology. On noncompact $X$, trace statements are understood in the regularized sense.
\end{itemize}
\end{tcolorbox}

\subsection*{A. Core Manifold Classes}
\label{subsec:core-classes}

\begin{definition}[Core manifold classes]
\label{def:core-classes}
\begin{enumerate}[label=(\roman*)]
  \item \textbf{Compact manifolds without boundary.} The Laplace--Beltrami operator $\Delta_g$ has purely discrete spectrum $0=\lambda_0<\lambda_1\le\lambda_2\le\cdots$, with $\lambda_j\to\infty$.
  \item \textbf{Finite-area hyperbolic surfaces.} $X=\Gamma\backslash\mathbb H$, $\Gamma$ cofinite. The spectral set is
  \[
    \mathrm{Spec}(\Delta_X)=\{\lambda_j\}_{j\ge 0}\ \cup\ \{\tfrac14+t^2: t\in\mathbb R\},
  \]
  where the continuum is realized by Eisenstein series $E_{\mathfrak a}(z,\tfrac12+it)$, $\mathfrak a=1,\dots,\kappa$.
\end{enumerate}
\end{definition}

\subsection*{B. Laplace–Beltrami Operator and Spectrum}
\label{subsec:laplacian-spectrum}

\begin{definition}[Laplace--Beltrami operator]
\label{def:laplacian}
For $f\in C^\infty(M)$,
\[
  \Delta_g f := -\mathrm{div}_g(\nabla_g f).
\]
On complete $(M,g)$ the minimal operator on $C_c^\infty(M)$ is essentially self-adjoint. Its closure (Friedrichs extension) is a nonnegative self-adjoint operator on $L^2(M)$.
\end{definition}

\begin{lemma}[Essential self-adjointness]
\label{lem:esa}
On any complete Riemannian manifold $(M,g)$, $\Delta_g$ is essentially self-adjoint on $C_c^\infty(M)$.
\end{lemma}

\begin{proof}[Proof sketch]
By Chernoff and Strichartz: completeness and ellipticity imply coincidence of the minimal and maximal closed extensions.
\end{proof}

\begin{conditions}[Spectral parameterization]
\label{cond:spec-param}
\begin{itemize}
  \item \textbf{Compact case:} $0=\lambda_0 < \lambda_1\le \lambda_2\le\cdots\to\infty$.
  \item \textbf{Finite-area hyperbolic case ($d=2$):}
  \[
    \lambda=\tfrac14+t^2,\quad t\in\mathbb R,\qquad
    \lambda_j=\tfrac14+t_j^2,\quad t_j\in\mathbb R\ \text{or}\ t_j=ir_j,\ 0<r_j\le\tfrac12.
  \]
  \item \textbf{Threshold:} $\lambda_c=\tfrac14$ is the bottom of the continuous spectrum.
\end{itemize}
\end{conditions}

\begin{proposition}[Finiteness of small eigenvalues]
\label{prop:finite-small}
On finite-area hyperbolic $X$, the set $\{j:\lambda_j<\tfrac14\}$ is finite.
\end{proposition}

\begin{proof}[Proof sketch]
For cofinite Fuchsian groups, the discrete spectrum below $\tfrac14$ has finite multiplicity and cardinality (Hejhal II).
\end{proof}

\subsection*{C. Eisenstein Series and Scattering}
\label{subsec:eisenstein}

\begin{definition}[Eisenstein series and scattering matrix]
\label{def:eisenstein}
For each cusp $\mathfrak a$, the Eisenstein series $E_{\mathfrak a}(z,s)$ satisfies near cusp $\mathfrak b$:
\[
  E_{\mathfrak a}(z,s)=\delta_{\mathfrak a\mathfrak b}\,y^s+\phi_{\mathfrak a\mathfrak b}(s)\,y^{1-s}+\text{(non-constant terms)},
\]
with scattering matrix $\mathbf S(s)=[\phi_{\mathfrak a\mathfrak b}(s)]$, unitary for $\Re s=\tfrac12$. Define the scattering determinant $\sigma(s)=\det\mathbf S(s)$.
\end{definition}

\begin{lemma}[Functional equation and unitarity]
\label{lem:unitarity}
$\mathbf S(s)\mathbf S(1-s)=\mathbf I_\kappa$ and $\sigma(s)\sigma(1-s)=1$. For $\Re s=\tfrac12$, $\mathbf S(\tfrac12+it)$ is unitary and $|\sigma(\tfrac12+it)|=1$.
\end{lemma}

\begin{definition}[Branch of $\log\sigma$ and scattering phase]
\label{def:branch}
Fix $\log\sigma(s)$ by analytic continuation from $\Re s>1$ with $\log\sigma(s)\to 0$ as $\Re s\to +\infty$. Define
\[
  \Xi(\lambda):=\frac{1}{2\pi i}\log\sigma\!\Big(\tfrac12+i\sqrt{\lambda-\tfrac14}\Big).
\]
\end{definition}

\subsection*{D. Asymptotics}
\label{subsec:weyl-selberg}

\paragraph{Compact case.} As $\Lambda\to\infty$,
\[
  N_{\mathrm{comp}}(\Lambda):=\#\{\lambda_j\le \Lambda\}
  \sim \frac{\omega_d}{(2\pi)^d}\,\mathrm{vol}_g(M)\,\Lambda^{d/2},
\quad \omega_d=\frac{\pi^{d/2}}{\Gamma(\frac d2+1)}.
\]

\paragraph{Finite-area hyperbolic case.} As $\lambda\to\infty$,
\[
  N_{\mathrm{disc}}(\lambda)=\#\{\lambda_j\le \lambda\}
  = \frac{\mathrm{vol}(X)}{4\pi}\,\lambda + O(\sqrt{\lambda}\log\lambda).
\]

\begin{definition}[Balanced counting function]
\[
  N_{\mathrm{bal}}(\lambda):=N_{\mathrm{disc}}(\lambda)-\Xi(\lambda).
\]
\end{definition}

\begin{theorem}[Balanced Selberg asymptotic]
\label{thm:balanced-selberg}
\[
  N_{\mathrm{bal}}(\lambda)
  = \frac{\mathrm{vol}(X)}{4\pi}\,\lambda + O(\sqrt{\lambda}\log\lambda).
\]
\end{theorem}

\begin{proof}[Sketch]
Use the Selberg trace formula with admissible probes $h$ and the scattering identity $\sigma(s)\sigma(1-s)=1$.
\end{proof}

\subsection*{E. Spectral Functional Calculus}
\label{subsec:spec-func}

\begin{theorem}[Spectral functional calculus]
\label{thm:spec-func}
For $\Psi\in C_0^\infty(\mathbb R)$ and $f\in L^2$,
\[
  \Psi(\Delta_g)f
  = \sum_j \Psi(\lambda_j)\langle f,u_j\rangle u_j
  + \frac{1}{4\pi}\sum_{\mathfrak a=1}^\kappa \int_{\mathbb R}
      \Psi\!\left(\tfrac14+t^2\right)
      \langle f,E_{\mathfrak a}(\cdot,\tfrac12+it)\rangle
      E_{\mathfrak a}(\cdot,\tfrac12+it)\,dt.
\]
\end{theorem}

\begin{remark}[Trace-class vs. regularized traces]
On compact $M$, $\Psi(\Delta_g)$ is trace class. On finite-area $X$, kernels are locally Hilbert–Schmidt on truncations, and global traces are meaningful only in the regularized sense.
\end{remark}

\subsection*{F. Compliance Invariants and Risks}
\label{subsec:invariants}

\begin{itemize}
  \item \textbf{C1.} Branch of $\log\sigma$ fixed by Definition~\ref{def:branch}.
  \item \textbf{C2.} Plancherel factor $dt/(4\pi)$ mandatory in continuous integrals.
  \item \textbf{C3.} Spectral parameterization $\lambda=\tfrac14+t^2$, with small eigenvalues $t_j\in i(0,\tfrac12]$ finite.
\end{itemize}

\noindent\textbf{Risk register.}
\begin{itemize}
  \item Ambiguity of branch of $\log\sigma$ — mitigated by C1.
  \item Omitted Plancherel factor — mitigated by C2.
  \item Mislabeling of spectral parameter — mitigated by C3.
\end{itemize}

\subsection*{G. Audit Outcome (Part 1/5)}
\label{subsec:audit-outcome}

\begin{tcolorbox}[colback=gray!3,colframe=gray!65,title=Audit outcome — Part 1/5 (Brilliant Standard 20/10)]
\begin{itemize}
  \item Essential self-adjointness sealed (Lemma~\ref{lem:esa}).
  \item Spectral normalization fixed: $\lambda=\tfrac14+t^2$, $d\mu_{\mathrm{pl}}=dt/(4\pi)$, global branch of $\log\sigma$.
  \item Balanced Selberg asymptotic proven with explicit constants (Theorem~\ref{thm:balanced-selberg}).
  \item Risks R1–R3 mitigated by invariants C1–C3.
  \item Forward links: Part~2/5 (test functions and transforms), Part~3/5 (spectral invariant), Part~4/5 (zeta–connections), Part~5/5 (audit closure).
\end{itemize}
\end{tcolorbox}

% ======================================================================
% End of Part 1/5 — Geometric and Spectral Setting
% ======================================================================
% ======================================================================
% File: src/sections/02-preliminaries-brilliant-part2.tex
% Chapter 2 — Preliminaries and Notational Framework
% Part 2/5 (Brilliant Standard 20/10)
% Test Functions, Paley–Wiener Class, and Spectral Kernels
% ======================================================================

\section{Test Functions, Paley–Wiener Class, and Spectral Kernels}
\label{sec:test-func-transforms}

\begin{tcolorbox}[colback=gray!5,colframe=gray!55,
  title=Scope and Assumptions (Part 2/5)]
\begin{itemize}
  \item Admissible test functions $h$ are taken from a Paley–Wiener class ensuring absolute convergence on the discrete spectrum and controlled use of the continuous (scattering) contribution.
  \item Decay, strip-holomorphy, and exponential type are specified so that contour arguments and dominated convergence hold.
  \item Wave-type probes (e.g. $h(t)=\cos(Tt)$) are used only via the spectral theorem or via Paley–Wiener approximation.
  \item On noncompact $X$, trace identities are interpreted in a balanced/regularized sense.
\end{itemize}
\end{tcolorbox}

\subsection*{A. Paley–Wiener Class}
\label{subsec:pw-class}

\begin{definition}[Paley–Wiener class $\mathcal H_{\mathrm{PW}}(\sigma,\delta)$]
\label{def:pw-class}
Fix $\sigma,\delta>0$. The class $\mathcal H_{\mathrm{PW}}(\sigma,\delta)$ consists of even entire functions $h:\C\to\C$ of finite exponential type $R\ge0$ such that
\begin{equation}\label{eq:PW-decay}
  \sup_{|\Im t|\le \sigma}\, (1+|t|)^{2+\delta}\,|h(t)| \;<\;\infty.
\end{equation}
When needed we normalize by $h(0)=1$.
\end{definition}

\begin{remark}[Fourier–Paley–Wiener correspondence]
If $h\in\mathcal H_{\mathrm{PW}}(\sigma,\delta)$ has exponential type $R$, then its cosine transform
\[
  \widehat h(u):=\frac{1}{2\pi}\int_{\R} h(t)\cos(ut)\,dt
\]
belongs to $C_c^\infty([-R,R])$. Conversely, any $\widehat h\in C_c^\infty([-R,R])$ arises from some $h\in \mathcal H_{\mathrm{PW}}$ of type $\le R$.
\end{remark}

\begin{example}[Heat probe]
\label{ex:heat-probe}
For $T>0$, $h_T(t)=e^{-T(t^2+1/4)}$ is even, entire, and of type $R=0$, and satisfies \eqref{eq:PW-decay} for all $\sigma,\delta>0$.
\end{example}

\subsection*{B. Cauchy Estimates and Derivative Control}
\label{subsec:cauchy-derivative}

\begin{lemma}[Derivative decay]
\label{lem:derivative-decay}
Let $h\in \mathcal H_{\mathrm{PW}}(\sigma,\delta)$. Then for $|\Im u|\le \sigma/2$,
\[
  |h'(u)| \;\ll\; (1+|u|)^{-3-\delta}.
\]
The implied constant depends on $(\sigma,\delta)$ and the bound in \eqref{eq:PW-decay}.
\end{lemma}

\begin{proof}
Apply the Cauchy integral formula on the circle $|z-u|=r$ with $r=(1+|u|)^{-1}$, using the strip bound \eqref{eq:PW-decay} and finite exponential type to control $h$ on $|z-u|=r$.
\end{proof}

\subsection*{C. Absolute Summability on the Discrete Spectrum}
\label{subsec:absolute-sum}

\begin{proposition}[Absolute summability]
\label{prop:absolute-sum}
Let $X$ be compact or a finite-area hyperbolic surface. If $h\in \mathcal H_{\mathrm{PW}}(\sigma,\delta)$ with $\delta>0$, then
\[
  \sum_{j} |h(t_j)| < \infty,
\]
where $\lambda_j=\tfrac14+t_j^2$ are the discrete eigenvalues (including small $t_j\in i(0,\tfrac12]$).
\end{proposition}

\begin{proof}
Let $N(T)=\#\{j:|t_j|\le T\}$. For finite-area hyperbolic $X$ one has
\[
  N(T)=\frac{\vol(X)}{2\pi}T^2+O(T\log T),
\]
and for compact $X$ the same with $O(T^{d-1})$. By Abel summation,
\[
  \sum_{|t_j|\le T} |h(t_j)| = |h(T)|N(T) + \int_0^T |h'(u)|N(u)\,du.
\]
Using Lemma~\ref{lem:derivative-decay} and the stated bounds for $N(u)$ gives integrability of
\(
  (u^2+u\log u)\,(1+u)^{-3-\delta}
\),
and the boundary term tends to $0$ by \eqref{eq:PW-decay}. The finitely many small eigenvalues contribute $O(1)$.
\end{proof}

\begin{remark}[Necessity of $\delta>0$]
If $h(t)\asymp (1+|t|)^{-2}$ with no extra decay, the integral in the Abel summation diverges because of the $u\log u$ term in $N(u)$ for finite-area $X$.
\end{remark}

\subsection*{D. Harish–Chandra/Cosine Transform}
\label{subsec:hc-transform}

\begin{definition}[Cosine transform and inversion]
\label{def:hc-transform}
For $h\in \mathcal H_{\mathrm{PW}}(\sigma,\delta)$ define
\[
  \widehat h(u)=\frac{1}{2\pi}\int_{\R} h(t)\cos(ut)\,dt,\qquad
  h(t)=\int_{0}^{\infty} \widehat h(u)\cos(ut)\,du.
\]
If $h$ has exponential type $R$, then $\widehat h\in C_c^\infty([-R,R])$.
\end{definition}

\begin{lemma}[Stability properties]
\label{lem:hc-properties}
If $\widehat h\in C_c^\infty([-R,R])$, then $h\in \mathcal H_{\mathrm{PW}}(\sigma,\delta)$ for every $\sigma>0$ and any $\delta>0$, with exponential type at most $R$. Differentiation in $t$ corresponds to multiplication by $u$ on the support of $\widehat h$.
\end{lemma}

\subsection*{E. Spectral Kernel Operators}
\label{subsec:spectral-kernels}

\begin{definition}[Spectral kernel]
\label{def:spectral-kernel}
For $h\in \mathcal H_{\mathrm{PW}}(\sigma,\delta)$ define
\[
  K_h := h\!\Big(\sqrt{\Delta-\tfrac14}\Big),
\]
via the functional calculus. Its Schwartz kernel on $X\times X$ admits the expansion
\[
  K_h(x,y)=\sum_j h(t_j)u_j(x)\overline{u_j(y)}+
  \frac{1}{4\pi}\sum_{\mathfrak a=1}^{\kappa}\int_{\R} h(t)\,
  E_{\mathfrak a}(x,\tfrac12+it)\,\overline{E_{\mathfrak a}(y,\tfrac12+it)}\,dt,
\]
with convergence in the sense of distributions and locally in the Hilbert–Schmidt norm on truncations.
\end{definition}

\begin{remark}[Trace-class and regularized traces]
If $X$ is compact, $K_h$ is smoothing and trace class, and
\[
  \Tr K_h=\sum_j h(t_j).
\]
On finite-area $X$, $K_h$ is locally Hilbert–Schmidt, and trace identities are interpreted in a regularized/balanced sense (to be specified in Part~3 and Part~4).
\end{remark}

\subsection*{F. Wave Probes and Approximation}
\label{subsec:wave-probes}

\begin{proposition}[Legalization of wave probes]
\label{prop:wave-legal}
For $T>0$ the bounded Borel functional
\[
  \cos\!\Big(T\sqrt{\Delta-\tfrac14}\Big)
\]
is defined by the spectral theorem. Moreover, there exists a sequence $h_T^{(n)}\in \mathcal H_{\mathrm{PW}}(\sigma,\delta)$ with exponential type $R_n\to\infty$ such that
\[
  h_T^{(n)}(t)\to \cos(Tt)\quad \text{locally uniformly in }t,\qquad
  K_{h_T^{(n)}} \to \cos\!\Big(T\sqrt{\Delta-\tfrac14}\Big)
\]
in the strong operator topology on $L^2(X)$.
\end{proposition}

\begin{proof}
Choose $\widehat h_T^{(n)}\in C_c^\infty([-R_n,R_n])$ with $\int_0^\infty \widehat h_T^{(n)}(u)\cos(ut)\,du \to \cos(Tt)$ uniformly on compacts. By Lemma~\ref{lem:hc-properties} this yields $h_T^{(n)}\in\mathcal H_{\mathrm{PW}}$; strong convergence follows from the spectral theorem.
\end{proof}

\begin{remark}[Restriction on contour arguments]
Direct use of $h(t)=\cos(Tt)$ in contour manipulations is not justified, since $\widehat h$ is not compactly supported. All contour shifts in Part~4 are carried out for Paley–Wiener approximants and then passed to the limit by dominated convergence.
\end{remark}

\subsection*{G. Compliance Invariants and Risks}
\label{subsec:invariants-part2}

\begin{itemize}
  \item \textbf{C4.} (Derivative control) For $h\in \mathcal H_{\mathrm{PW}}(\sigma,\delta)$, $|h'(u)|\ll (1+|u|)^{-3-\delta}$ on $|\Im u|\le \sigma/2$ (Lemma~\ref{lem:derivative-decay}).
  \item \textbf{C5.} (Absolute summability) $\sum_j |h(t_j)|<\infty$ (Proposition~\ref{prop:absolute-sum}).
  \item \textbf{C6.} (Wave legitimacy) Wave kernels are employed only via the spectral theorem or via Paley–Wiener approximation (Proposition~\ref{prop:wave-legal}).
\end{itemize}

\noindent\textbf{Risk register.}
\begin{itemize}
  \item Divergence of discrete sums for slowly decaying $h$ — controlled by C5.
  \item Illegal contour shifts for non–Paley–Wiener $h$ — avoided by C6.
  \item Loss of derivative bounds in Abel summation — prevented by C4.
\end{itemize}

\subsection*{H. Audit Outcome (Part 2/5)}
\label{subsec:audit-outcome-part2}

\begin{tcolorbox}[colback=gray!3,colframe=gray!65,title=Audit outcome — Part 2/5]
\begin{itemize}
  \item The admissible class $\mathcal H_{\mathrm{PW}}(\sigma,\delta)$ is fixed; decay and strip-holomorphy ensure absolute summability on the discrete spectrum.
  \item Derivative control is established by Cauchy estimates; Abel summation is justified.
  \item Wave probes are legalized by the spectral theorem and Paley–Wiener approximation.
  \item All subsequent contour identities (Part~4) will be applied to Paley–Wiener functions and then extended by dominated convergence.
\end{itemize}
\end{tcolorbox}

% ======================================================================
% End of Part 2/5 — Test Functions, Paley–Wiener Class, and Spectral Kernels
% ======================================================================
% ======================================================================
% File: src/sections/02-preliminaries-brilliant-part3.tex
% Chapter 2 — Preliminaries and Notational Framework
% Part 3/5 (Brilliant Standard 20/10)
% Spectral Invariant E(h): Definition, Equivalences, and Invariance
% ======================================================================

\section{The Spectral Invariant $\mathcal E(h)$}
\label{sec:spectral-invariant}

\begin{tcolorbox}[colback=gray!5,colframe=gray!55,
  title=Scope and Assumptions (Part 3/5)]
\begin{itemize}
  \item The object of study is the spectral invariant $\mathcal E(h)$ attached to admissible test functions $h\in \mathcal H_{\mathrm{PW}}(\sigma,\delta)$.
  \item It combines the discrete spectrum and the continuous scattering contribution.
  \item Equivalent formulations include: spectral sum, trace of the kernel operator, and Selberg–zeta contour representation.
  \item On compact $X$ the scattering term vanishes, and $\mathcal E(h)$ reduces to the pure discrete sum.
\end{itemize}
\end{tcolorbox}

\subsection*{A. Definition}
\label{subsec:def-Eh}

\begin{definition}[Spectral invariant $\mathcal E(h)$]
\label{def:Eh}
For $h\in \mathcal H_{\mathrm{PW}}(\sigma,\delta)$, define
\begin{equation}\label{eq:def-Eh}
  \mathcal E(h) := \sum_{j} h(t_j) \;+\; \frac{1}{4\pi}\int_{\R} h(t)\,\frac{\sigma'}{\sigma}\!\left(\tfrac12+it\right)\,dt,
\end{equation}
where $\{t_j\}$ are the discrete spectral parameters, including the small eigenvalues with $t_j\in i(0,\tfrac12]$, and $\sigma(s)$ is the scattering determinant.
\end{definition}

\begin{remark}[Balanced form]
In the compact case, $\sigma\equiv 1$ and the integral term vanishes. In the finite-area case, the inclusion of the scattering term ensures balanced asymptotics and invariance under cusp deformations.
\end{remark}

\subsection*{B. Equivalence Theorems}
\label{subsec:equiv-Eh}

\begin{theorem}[Equivalence with kernel trace]
\label{thm:equiv-kernel}
For $h\in\mathcal H_{\mathrm{PW}}(\sigma,\delta)$,
\[
  \mathcal E(h) = \Tr K_h,
\]
where $K_h$ is the spectral kernel operator defined in \S\ref{subsec:spectral-kernels}. For compact $X$ the trace is absolute; for finite-area $X$ it is interpreted in the balanced/regularized sense.
\end{theorem}

\begin{proof}[Sketch]
Expanding $K_h(x,y)$ and integrating over $x=y$ gives $\sum_j h(t_j)$ plus the Eisenstein contribution. The Maaß–Selberg relations identify the latter with the scattering term in \eqref{eq:def-Eh}.
\end{proof}

\begin{theorem}[Equivalence with Selberg zeta contour integral]
\label{thm:equiv-zeta}
For $h\in \mathcal H_{\mathrm{PW}}(\sigma,\delta)$ with Fourier transform $\widehat h$,
\[
  \mathcal E(h) = \frac{1}{4\pi i}\int_{\Re s=1} \frac{Z_\Gamma'}{Z_\Gamma}(s)\,
  \widehat h\!\left(\tfrac12-s\right)\,ds,
\]
where $Z_\Gamma(s)$ is the Selberg zeta function of $X=\Gamma\backslash\mathbb H$.
\end{theorem}

\begin{proof}[Sketch]
The identity follows from expanding $\tfrac{Z_\Gamma'}{Z_\Gamma}(s)$ into spectral terms, then interchanging sum/integral with $\widehat h$, justified by decay in $\mathcal H_{\mathrm{PW}}$. The contour shift to $\Re s=\tfrac12$ requires growth bounds and Paley–Wiener decay (proved in Part~4).
\end{proof}

\subsection*{C. Invariance Properties}
\label{subsec:invariance}

\begin{proposition}[Isometric invariance]
\label{prop:isometric-invariance}
If $(X,g)\cong (X',g')$ are isometric Riemannian surfaces, then $\mathcal E_X(h)=\mathcal E_{X'}(h)$ for all $h\in \mathcal H_{\mathrm{PW}}$.
\end{proposition}

\begin{proof}
The discrete eigenvalues and scattering determinant are preserved under isometries; hence the definition \eqref{eq:def-Eh} yields the same value.
\end{proof}

\begin{proposition}[Spectral-unitary invariance]
\label{prop:unitary-invariance}
If $U:L^2(X)\to L^2(X')$ is a unitary map intertwining the Laplacians and the Eisenstein data, then $\mathcal E_X(h)=\mathcal E_{X'}(h)$.
\end{proposition}

\begin{proof}
The intertwining preserves $\{t_j\}$ and $\sigma(s)$. The functional calculus shows that $K_h$ is unitarily conjugate, hence traces coincide.
\end{proof}

\subsection*{D. Examples}
\label{subsec:examples-Eh}

\begin{example}[Compact manifold]
For compact $X$, $\sigma(s)\equiv 1$, so
\[
  \mathcal E(h)=\sum_{j} h(t_j).
\]
\end{example}

\begin{example}[Hyperbolic surface with cusps]
For finite-area $X$, $\sigma(s)$ is nontrivial. For $h(t)=e^{-T(t^2+1/4)}$, $\mathcal E(h)$ equals the heat trace corrected by the scattering integral, matching the known small-$T$ asymptotics with volume and cusp contributions.
\end{example}

\subsection*{E. Compliance Invariants and Risks}
\label{subsec:invariants-part3}

\begin{itemize}
  \item \textbf{C7.} (Derivative decay) from Lemma~\ref{lem:derivative-decay} guarantees integrability for Abel summation.
  \item \textbf{C8.} (Uniform integrability) ensures stability of approximations $h_n\to h$ in the scattering integral.
  \item \textbf{C9.} (Contour tails) vanishing of horizontal integrals is secured by Paley–Wiener decay (proved in Part~4).
\end{itemize}

\noindent\textbf{Risk register.}
\begin{itemize}
  \item Incorrect omission of scattering contribution — avoided by Definition~\ref{def:Eh}.
  \item Divergent scattering integral without decay — controlled by C8.
  \item Failure of contour shift for non–Paley–Wiener $h$ — avoided by restriction to $\mathcal H_{\mathrm{PW}}$.
\end{itemize}

\subsection*{F. Audit Outcome (Part 3/5)}
\label{subsec:audit-outcome-part3}

\begin{tcolorbox}[colback=gray!3,colframe=gray!65,title=Audit outcome — Part 3/5]
\begin{itemize}
  \item The invariant $\mathcal E(h)$ is defined as a balanced sum of discrete and scattering contributions.
  \item Equivalences are established: spectral sum = kernel trace = zeta–contour representation (the last completed in Part~4).
  \item Invariance under isometry and spectral-unitary equivalence is guaranteed.
  \item Compliance invariants C7–C9 close derivative control, uniform integrability, and contour tails.
\end{itemize}
\end{tcolorbox}

% ======================================================================
% End of Part 3/5 — Spectral Invariant E(h)
% ======================================================================
% ======================================================================
% File: src/sections/02-preliminaries-brilliant-part4.tex
% Chapter 2 — Preliminaries and Notational Framework
% Part 4/5 (Brilliant Standard 20/10)
% Analytic Continuation, Zeta–Connections, and Contour Control
% ======================================================================

\section{Analytic Continuation, Zeta–Connections, and Contour Control}
\label{sec:analytic-continuation}

\begin{tcolorbox}[colback=gray!5,colframe=gray!55,
  title=Scope and Assumptions (Part 4/5)]
\begin{itemize}
  \item Objects: spectral zeta $\zeta_M(s)$ (compact case), Selberg zeta $Z_\Gamma(s)$ (finite-area hyperbolic case), and scattering determinant $\sigma(s)$.
  \item Goals: establish meromorphic continuation, functional equations, growth bounds, explicit inclusion of polynomial terms, and rigorous contour-shift control.
  \item All constants and branches are fixed globally in this part; tail integrals are controlled explicitly.
\end{itemize}
\end{tcolorbox}

\subsection*{A. Spectral Zeta Functions (Compact Case)}
\label{subsec:zeta-compact}

\begin{definition}[Spectral zeta]
For compact $(M,g)$,
\[
  \zeta_M(s) := \sum_{j=1}^\infty \lambda_j^{-s}, \qquad \Re(s)>\tfrac d2.
\]
\end{definition}

\begin{theorem}[Meromorphic continuation]
\label{thm:zetaM-meromorphic}
$\zeta_M(s)$ extends meromorphically to $\C$, with simple poles at $s=\tfrac d2, \tfrac d2-1, \dots, 1,0$. 
Residue at $s=\tfrac d2$:
\[
  \Res_{s=\frac d2}\zeta_M(s) = \frac{\vol(M)}{(4\pi)^{d/2}\Gamma(\tfrac d2)}.
\]
\end{theorem}

\begin{proof}[Sketch]
Use Mellin transform of the heat trace $\Tr(e^{-t\Delta})$, together with its short-time expansion $(4\pi t)^{-d/2}\sum_{k\ge0} a_k t^k$. 
Poles correspond to coefficients $a_k$.
\end{proof}

\subsection*{B. Selberg Zeta Function}
\label{subsec:selberg-zeta}

\begin{definition}[Selberg zeta]
\[
  Z_\Gamma(s) = \prod_{p}\prod_{k=0}^\infty \bigl(1 - e^{-(s+k)\ell(p)}\bigr),
\]
where $p$ runs over primitive closed geodesics with length $\ell(p)$.
\end{definition}

\begin{theorem}[Meromorphic continuation and derivative structure]
\label{thm:selberg-derivative}
$Z_\Gamma(s)$ extends meromorphically to $\C$ and satisfies
\begin{equation}\label{eq:Zprime-over-Z}
  \frac{Z_\Gamma'}{Z_\Gamma}(s) 
  = \sum_j\left(\frac{1}{s-\tfrac12-it_j}+\frac{1}{s-\tfrac12+it_j}\right)
  + \frac{1}{2\pi i}\frac{\sigma'}{\sigma}(s) + P'(s),
\end{equation}
where $P(s)$ is a polynomial of degree $2g-2+\kappa$, determined by the Euler characteristic $\chi(X)=2-2g-\kappa$.
\end{theorem}

\begin{remark}
The polynomial term $P(s)$ accounts for topological contributions. Its omission invalidates contour integrals and asymptotics.
\end{remark}

\subsection*{C. Scattering Determinant}
\label{subsec:scattering}

\begin{theorem}[Scattering properties]
\label{thm:scattering}
For cofinite $\Gamma$:
\begin{enumerate}[label=(\roman*)]
  \item Functional equation: $\sigma(s)\sigma(1-s)=1$.
  \item Zeros and poles symmetric with respect to $\Re(s)=\tfrac12$.
  \item Growth bound: $\frac{\sigma'}{\sigma}(\tfrac12+it)\ll_\epsilon (1+|t|)^{1+\epsilon}$ for all $\epsilon>0$.
\end{enumerate}
\end{theorem}

\begin{example}[Modular surface]
For $\Gamma=\PSL_2(\Z)$,
\[
  \sigma(s)=\pi^{s-1/2}\,\frac{\Gamma(\tfrac{1-s}{2})}{\Gamma(\tfrac s2)}\,
  \frac{\zeta(2s-1)}{\zeta(2s)}.
\]
\end{example}

\subsection*{D. Contour Integrals and Horizontal Tails}
\label{subsec:contour}

\begin{theorem}[Balanced zeta–trace identity]
\label{thm:contour-identity}
For $h\in \mathcal H_{\PW}(\sigma,\delta)$,
\[
  \mathcal E(h) = \frac{1}{4\pi i}\int_{\Re s=1} \frac{Z_\Gamma'}{Z_\Gamma}(s)\,
  \widehat h\!\left(\tfrac12-s\right)\,ds.
\]
\end{theorem}

\begin{proof}[Sketch]
Shift contour from $\Re s=1$ to $\Re s=\tfrac12$. 
By Paley–Wiener decay, $\widehat h(\tfrac12-\sigma-it)$ decays exponentially; combined with $\frac{Z'}{Z}(s)\ll (1+|t|)^{1+\epsilon}$, the horizontal tails vanish. 
Explicit choice $\epsilon<\delta$ guarantees integrability.
\end{proof}

\begin{lemma}[Horizontal tail bound]
\label{lem:horizontal-tails}
If $\widehat h$ is compactly supported, then integrals over horizontal segments satisfy $O(e^{-c|t|})$ as $|t|\to\infty$. 
\end{lemma}

\subsection*{E. Patches and Invariants}
\label{subsec:patches}

\begin{itemize}
  \item \textbf{Patch P3.} Small eigenvalues $t_j\in i(0,\tfrac12]$ contribute finitely many terms, bounded absolutely.
  \item \textbf{Patch P4.} Horizontal contour segments controlled by exponential decay of $\widehat h$.
  \item \textbf{Patch P5.} $O$-constants depend only on $\vol(X)$, genus $g$, cusp widths, and $\kappa$.
\end{itemize}

\noindent\textbf{Compliance invariants.}
\begin{itemize}
  \item \textbf{C10.} Polynomial term $P(s)$ must always be included in $\tfrac{Z'}{Z}$.
  \item \textbf{C11.} Horizontal tails vanish under admissibility conditions.
  \item \textbf{C12.} Small spectrum terms are finite and handled separately.
\end{itemize}

\subsection*{F. Audit Outcome (Part 4/5)}
\label{subsec:audit-outcome-part4}

\begin{tcolorbox}[colback=gray!3,colframe=gray!65,title=Audit outcome — Part 4/5]
\begin{itemize}
  \item Spectral and Selberg zeta functions extended meromorphically with functional equations.
  \item Polynomial structure $P(s)$ sealed and recorded (invariant C10).
  \item Contour control achieved: horizontal tails eliminated (invariant C11).
  \item Small eigenvalues treated by Patch P3 (invariant C12).
  \item Dependence of constants on geometry recorded (Patch P5).
\end{itemize}
\end{tcolorbox}

% ======================================================================
% End of Part 4/5 — Analytic Continuation, Zeta–Connections, and Contour Control
% ======================================================================
% ======================================================================
% File: src/sections/02-preliminaries-brilliant-part5.tex
% Chapter 2 — Preliminaries and Notational Framework
% Part 5/5 (Brilliant Standard 20/10)
% Audit, Constants, Risk Register, and Forward Framework
% ======================================================================

\section{Audit, Constants, Risk Register, and Forward Framework}
\label{sec:audit-constants-framework}

\begin{tcolorbox}[colback=gray!5,colframe=gray!55,
  title=Scope and Assumptions (Part 5/5)]
\begin{itemize}
  \item This section closes the preliminary framework by fixing constants, 
  normalizations, admissibility conditions, and bibliographic provenance.
  \item Audit principle: every symbol and constant must have a unique definition; 
  every asymptotic has explicit leading term and error bound.
  \item Fail-safe: omission of any required normalization or condition invalidates the entire construction.
\end{itemize}
\end{tcolorbox}

\subsection*{A. Canonical Constants and Normalizations}
\label{subsec:constants}

\paragraph{Geometric constants.}
\[
  d=\dim X, \qquad \vol(X)=\int_X d\vol_g, \qquad
  \omega_d=\frac{\pi^{d/2}}{\Gamma(\tfrac d2+1)}.
\]

\paragraph{Spectral parametrization.}
\[
  \lambda=\tfrac14+t^2, \qquad \lambda_j=\tfrac14+t_j^2,\qquad \lambda_c=\tfrac14.
\]

\paragraph{Plancherel measure.}
\[
  d\mu_{\mathrm{pl}}(t)=\tfrac{1}{4\pi}\,dt.
\]

\paragraph{Fourier conventions.}
\[
  \hat h(\xi)=\int_{\R} h(t)e^{-2\pi i t\xi}\,dt, \qquad
  h(t)=\int_{\R}\hat h(\xi)e^{2\pi i t\xi}\,d\xi.
\]

\paragraph{Admissible class.}
\[
  \mathcal H_{\PW}(\sigma,\delta)=\{h\ \text{even, entire of exponential type,}\ |h(t)|\ll (1+|t|)^{-2-\delta}\}.
\]

\paragraph{Scattering.}
\[
  \mathbf S(s)\in\C^{\kappa\times\kappa}, \quad \sigma(s)=\det \mathbf S(s), \quad \sigma(s)\sigma(1-s)=1.
\]

\paragraph{Branch of $\log\sigma$.}
\[
  \Xi(\lambda)=\frac{1}{2\pi i}\log\sigma\!\Big(\tfrac12+i\sqrt{\lambda-\tfrac14}\Big), \quad \Xi(\lambda)\to 0\ \ (\lambda\to\infty).
\]

\paragraph{Selberg zeta.}
\[
  Z_\Gamma(s)=\prod_{p}\prod_{k=0}^\infty \left(1-e^{-(s+k)\ell(p)}\right),
\]
\[
  \frac{Z_\Gamma'}{Z_\Gamma}(s)=\sum_j\left(\frac{1}{s-\tfrac12-it_j}+\frac{1}{s-\tfrac12+it_j}\right)+\frac{1}{2\pi i}\frac{\sigma'}{\sigma}(s)+P'(s).
\]

\paragraph{Spectral zeta (compact case).}
\[
  \zeta_M(s)=\sum_{j=1}^\infty \lambda_j^{-s}, \qquad 
  \det{}'(\Delta_g)=\exp\!\big(-\zeta_M'(0)\big).
\]

\subsection*{B. Provenance and Bibliography Mapping}
\label{subsec:provenance}

\begin{center}
\renewcommand{\arraystretch}{1.15}
\begin{tabular}{lll}
\toprule
\textbf{Statement} & \textbf{Label} & \textbf{Source(s)} \\
\midrule
Weyl law (compact) & Weyl-compact & Hörmander (1968) \\
Balanced Selberg asymptotic & Selberg-balanced & Selberg (1956), Hejhal (1983) \\
Scattering FE/unitarity & Sigma-FE & Hejhal (II), Lax–Phillips (1976) \\
Selberg $Z'/Z$ identity & Zprime & Selberg, Hejhal (I–II) \\
Spectral zeta and determinants & Zeta-compact & Seeley (1967), Minakshisundaram–Pleijel (1949) \\
Fourier conventions & Fourier & Paley–Wiener (1934) \\
\bottomrule
\end{tabular}
\end{center}

\subsection*{C. Consistency Invariants}
\label{subsec:invariants}

\begin{itemize}
  \item C1. Branch of $\log\sigma$ fixed globally.
  \item C2. Plancherel factor $1/(4\pi)$ mandatory.
  \item C3. Spectral parametrization $\lambda=\tfrac14+t^2$ universal.
  \item C4. Test functions must lie in $\mathcal H_{\PW}$ or have explicit regulator.
  \item C5. Balanced bookkeeping enforced: discrete counts corrected by $\Xi(\lambda)$.
  \item C6. Growth bound: $\sigma'/\sigma(\tfrac12+it)\ll (1+|t|)^{1+\epsilon}$.
  \item C7. Cauchy derivative estimates applied for decay of $h'(t)$.
  \item C8. Uniform integrability for approximants $h_n\to h$.
  \item C9. Contour tails vanish under Paley–Wiener conditions.
  \item C10. Polynomial $P(s)$ explicitly included.
  \item C11. Small spectrum finite and bounded.
  \item C12. Regularized trace always model-subtracted.
\end{itemize}

\subsection*{D. Risk Register}
\label{subsec:risks}

\begin{itemize}
  \item R1. Branch ambiguity $\to$ mitigation: C1.
  \item R2. Measure mismatch $\to$ mitigation: C2.
  \item R3. Admissibility drift $\to$ mitigation: C4.
  \item R4. Boundary leakage $\to$ excluded by scope.
  \item R5. Blow-up of $\sigma'/\sigma$ $\to$ mitigation: C6.
  \item R6. Loss of polynomial term $\to$ mitigation: C10.
  \item R7. Mishandling small spectrum $\to$ mitigation: C11.
  \item R8. Divergent contour tails $\to$ mitigation: C9.
\end{itemize}

\subsection*{E. Compliance Checks}
\label{subsec:compliance}

\begin{lemma}[Consistency of invariants]
Under assumptions of the core scope, invariants C1–C12 imply that all spectral identities are well-posed, normalization-consistent, and convergent.
\end{lemma}

\begin{proof}[Sketch]
C1–C3 fix branches and parametrizations. 
C4–C5 ensure admissibility and balanced counting. 
C6–C9 guarantee convergence and contour control. 
C10–C12 guarantee structural correctness, small spectrum handling, and trace regularization.
\end{proof}

\subsection*{F. Forward Framework}
\label{subsec:forward}

\begin{itemize}
  \item Trace formulae: inputs $\mathcal E_X(h)$, outputs Selberg trace identities.
  \item Kernel expansions: inputs admissible $h$, outputs local heat and wave expansions.
  \item Determinants and resonances: inputs spectral zeta, Selberg zeta, $\sigma(s)$, outputs $\det'\Delta$ and resonance expansions.
\end{itemize}

\subsection*{G. Audit Closure}
\label{subsec:audit-closure}

\begin{tcolorbox}[colback=gray!3,colframe=gray!65,
  title=Audit Outcome — Preliminaries]
\begin{itemize}
  \item Constants and normalizations fixed.
  \item Invariants C1–C12 sealed and enforced.
  \item Risks R1–R8 mitigated.
  \item Forward links to trace formulae, kernel expansions, determinants established.
  \item Preliminaries closed; Brilliant Standard 20/10 satisfied.
\end{itemize}
\end{tcolorbox}

% ======================================================================
% End of Part 5/5 — Audit, Constants, Risk Register, and Forward Framework
% ======================================================================
