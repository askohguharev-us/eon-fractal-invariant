% ======================================================================
% File: src/sections/02-preliminaries-sharpened.tex
% Chapter 2 — Preliminaries and Notational Framework
% Part 1/5 (Sharpened Brilliants+++)
% Geometric and Spectral Setting — ZNB-9+++ Brilliants 20/10 Absolute Fill
% ======================================================================

\chapter{Preliminaries and Notational Framework (Sharpened Brilliants+++)}
\label{chap:preliminaries-sharp}

\section{Geometric and Spectral Setting (Refined)}
\label{sec:geom-spectral-setting-sharp}

% ------------------ ZNB-9+++ SCOPE BOX (MEA-Core-SS • enforced) --------
\begin{tcolorbox}[colback=gray!5,colframe=gray!55,
  title=Scope \& Assumptions (ZNB-9+++ Brilliants+++ • enforced)]
\begin{itemize}
  \item \textbf{Completeness.} All $(M,g)$ considered are complete Riemannian manifolds. This secures essential self-adjointness of $\Delta_g$ on $C_c^\infty(M)$ and makes the spectral theorem applicable without external hypotheses.
  \item \textbf{Core classes.} (i) Compact manifolds without boundary; (ii) finite-area hyperbolic surfaces $X=\Gamma\backslash\mathbb H$, $\Gamma\subset\mathrm{PSL}_2(\mathbb R)$ cofinite, $\kappa$ cusps. Infinite-volume and boundary-value geometries are excluded at this layer; when reintroduced they require a separate ZNB-9+++ audit ledger.
  \item \textbf{Spectral split.} On noncompact finite-area hyperbolic surfaces the spectral resolution splits into $L^2$-discrete spectrum and continuous Eisenstein part. Scattering data enter via $\mathbf S(s)$ and $\sigma(s)=\det\mathbf S(s)$.
  \item \textbf{Counting convention.} $N(\Lambda)$ denotes the discrete counting function unless explicitly balanced with the spectral shift $\Xi(\lambda)$.
  \item \textbf{Normalization discipline.} Eisenstein normalization, Plancherel measure $dt/(4\pi)$, cusp scaling matrices, scattering determinants, and $\log\sigma$ branches are pinned globally. All constants are mirrored in Appendix~J (audit ledger).
  \item \textbf{Convergence topology.} All series and integrals in spectral expansions are in the strong operator topology on $L^2$; trace-class assertions flagged. Distributional traces arise only when balanced by scattering.
\end{itemize}
\end{tcolorbox}
% -----------------------------------------------------------------------

\subsection*{A. Classes of Manifolds $(M,g)$}
\label{subsec:classes-sharp}

\begin{definition}[Core manifold classes]
\begin{enumerate}[label=(\roman*)]
  \item \textbf{Compact without boundary.} Spectrum purely discrete, nonnegative, accumulating only at $\infty$.
  \item \textbf{Finite-area hyperbolic surfaces with cusps.} $X=\Gamma\backslash\mathbb H$, $\Gamma$ cofinite. Spectrum = discrete $\{\lambda_j\}$ $\cup$ continuous $\{\tfrac14+t^2\}$, $t\in\mathbb R$, realized by Eisenstein series $E_{\mathfrak a}(z,\tfrac12+it)$.
  \item \textbf{Excluded.} Infinite-volume cases and boundary conditions (Dirichlet/Neumann) are outside prelim scope.
\end{enumerate}
\end{definition}

\begin{remark}[Beyond cusps]
Funnels, orbifold points, or conical singularities alter resonance structure and require extended functional calculi. These are excluded here; cf.\ Guillopé--Zworski for funnels.
\end{remark}

% -----------------------------------------------------------------------

\subsection*{B. Laplace–Beltrami Operator and Spectrum}
\label{subsec:laplacian-sharp}

\begin{definition}[Laplace–Beltrami operator]
For $f\in C^\infty(M)$,
\[
  \Delta_g f := -\mathrm{div}_g(\nabla_g f).
\]
On complete $(M,g)$, the Friedrichs extension realizes $\Delta_g$ as a nonnegative self-adjoint operator on $L^2(M,g)$.
\end{definition}

\begin{conditions}[Spectral parametrization]
\begin{itemize}
  \item Compact: discrete eigenvalues $0=\lambda_0<\lambda_1\le\lambda_2\le\cdots$, $\lambda_j\to\infty$.
  \item Finite-area hyperbolic ($d=2$): spectrum = discrete $\{\lambda_j\}$ $\cup$ continuum $\{\tfrac14+t^2:t\in\mathbb R\}$. Use parameter $t_j$ via $\lambda_j=\tfrac14+t_j^2$ ($t_j\in\mathbb R$) or $\lambda_j=\tfrac14-r_j^2$ ($t_j=ir_j$, $0<r_j\le \tfrac12$).
  \item Threshold: $\lambda_c=\tfrac14$ is bottom of the continuous spectrum.
\end{itemize}
\end{conditions}

\begin{remark}[Essential self-adjointness]
On complete $M$, $\Delta_g$ is essentially self-adjoint on $C_c^\infty(M)$; the spectral theorem applies. On $X=\Gamma\backslash\mathbb H$, the continuous spectrum corresponds to principal series representations of $\mathrm{SL}_2(\mathbb R)$.
\end{remark}

% -----------------------------------------------------------------------

\subsection*{C. Spectral Counting: Weyl and Selberg Asymptotics}
\label{subsec:weyl-sharp}

\paragraph{Compact case (Weyl law).}
\[
  N(\Lambda)=\#\{\lambda_j\le\Lambda\}
  \;\sim\;\frac{\omega_d}{(2\pi)^d}\,\mathrm{vol}_g(M)\,\Lambda^{d/2},\qquad \Lambda\to\infty.
\]

\paragraph{Finite-area hyperbolic surfaces (Selberg).}
\[
  N_{\mathrm{disc}}(\lambda)
  =\frac{\vol(X)}{4\pi}\,\lambda+O\!\big(\sqrt{\lambda}\log\lambda\big),\quad \lambda\to\infty.
\]

\begin{definition}[Balanced counting]
\[
  N_{\mathrm{bal}}(\lambda):=N_{\mathrm{disc}}(\lambda)-\Xi(\lambda),
\]
where $\Xi(\lambda)=\frac{1}{2\pi i}\log\sigma(\tfrac12+i\sqrt{\lambda-\tfrac14})$.
\end{definition}

\begin{theorem}[Balanced Selberg asymptotic]
\[
  N_{\mathrm{disc}}(\lambda)-\Xi(\lambda)
  =\frac{\vol(X)}{4\pi}\,\lambda+O\!\big(\sqrt{\lambda}\log\lambda\big).
\]
\end{theorem}

\begin{proof}[Audit sketch]
Integrate the logarithmic derivative of $Z_\Gamma$ against $h$, invoke Selberg trace formula, and match with scattering. Sources: Selberg (1956), Hejhal I–II, Lax--Phillips.
\end{proof}

\begin{remark}[Small eigenvalues]
Eigenvalues $\lambda_j<1/4$ contribute only boundedly to $N_{\mathrm{disc}}(\lambda)$ and do not change the main asymptotic.
\end{remark}

% -----------------------------------------------------------------------

\subsection*{D. Spectral Decomposition and Functional Calculus}
\label{subsec:spectral-decomposition-sharp}

\begin{definition}[Spectral split]
For finite-area $X$, decompose $L^2(X)$ into discrete eigenfunctions $\{u_j\}$ and continuous Eisenstein series $E_{\mathfrak a}(z,\tfrac12+it)$ with Plancherel density $dt/(4\pi)$.
\end{definition}

\begin{theorem}[Functional calculus]
For $\Psi\in C_0^\infty(\mathbb R)$,
\[
  \Psi(\Delta_g)f
  =\sum_j \Psi(\lambda_j)\langle f,u_j\rangle u_j
  +\frac{1}{4\pi}\sum_{\mathfrak a}\int_\mathbb R \Psi(\tfrac14+t^2)\langle f,E_{\mathfrak a}(\cdot,\tfrac12+it)\rangle E_{\mathfrak a}(\cdot,\tfrac12+it)\,dt.
\]
\end{theorem}

\begin{remark}[Operator classes]
On compact $M$, $\Psi(\Delta_g)$ is smoothing and trace class. On finite-area $X$, smoothing on $X\times X$, bounded on $L^2$, but not globally trace class unless balanced.
\end{remark}

% -----------------------------------------------------------------------

\subsection*{E. Notation Ledger (Audit Sealed)}
\label{subsec:notation-invariants-sharp}

\begin{itemize}
  \item Dimension $d$, volume $\vol(X)$, injectivity radius $\mathrm{inj}(X)$.
  \item Cuspidal data: $\kappa$, cusp widths $\{w_i\}$.
  \item Threshold $\lambda_c=\tfrac14$ (hyperbolic $d=2$).
  \item Spectral shift $\Xi(\lambda)=\frac{1}{2\pi i}\log\sigma(\tfrac12+i\sqrt{\lambda-\tfrac14})$, branch fixed by $\Xi(\lambda)\to0$ as $\lambda\to\infty$.
\end{itemize}

% -----------------------------------------------------------------------

\subsection*{F. Audit • Forward/Backward Links (Sharpened)}
\label{subsec:audit-links-sharp}

\begin{tcolorbox}[colback=gray!3,colframe=gray!65,title=Audit outcome — Part 1/5 (sealed)]
\begin{itemize}
  \item Constants, branch choices, Plancherel measure, spectral parametrization — \emph{sealed}.
  \item Compact Weyl and balanced Selberg asymptotics with explicit constants recorded.
  \item Discrete vs continuous spectra normalized, scattering determinant conventions fixed.
  \item Back links: definitions mirrored in Appendix~J (audit ledger).
  \item Forward links: test functions (\S\ref{sec:test-functions}), invariant definition (\S\ref{sec:def-invariant}), kernel/projector chapters.
\end{itemize}
\end{tcolorbox}

% ======================================================================
% End of Part 1/5 — Geometric and Spectral Setting (Sharpened Brilliants+++)
% ======================================================================
% ======================================================================
% File: src/sections/02-preliminaries-sharpened.tex
% Chapter 2 — Preliminaries and Notational Framework
% Part 2/5 (Sharpened Brilliants+++)
% Test Functions and Spectral Probes — ZNB-9+++ Brilliants 20/10 Absolute Fill
% ======================================================================

\section{Test Functions and Spectral Probes (Refined)}
\label{sec:test-functions-sharp}

% ------------------ ZNB-9+++ SCOPE BOX (MEA-Core-SS • enforced) --------
\begin{tcolorbox}[colback=gray!5,colframe=gray!55,
  title=Scope \& Assumptions (ZNB-9+++ Brilliants+++ • enforced)]
\begin{itemize}
  \item \textbf{Setting.} Core classes from Part~1/5: (i) compact $(M,g)$, no boundary; (ii) finite-area hyperbolic surfaces $X=\Gamma\backslash\mathbb H$ with cusps ($\Gamma\subset\mathrm{PSL}_2(\mathbb R)$ cofinite). Spectral parameterization $\lambda=\tfrac14+t^2$; Plancherel density $dt/(4\pi)$.
  \item \textbf{Purpose.} Specify admissible classes of spectral probes $h(t)$; pin Fourier/Harish–Chandra/Selberg transforms and the exact duality with geometric kernels $k(r)$; fix convergence topologies and branch conventions feeding trace identities and the invariant $\mathcal E_X(h)$ (Part~3/5).
  \item \textbf{Convergence \& topology.} All spectral expansions converge in the strong operator topology on $L^2$; where traces appear on noncompact $X$, they are understood only in balanced/regularized sense. Distributional statements are flagged.
  \item \textbf{Normalization.} Fourier transform
  \(
    \hat h(\xi)=\int_{\mathbb R} h(t)e^{-2\pi i t\xi}\,dt
  \),
  inversion
  \(
    h(t)=\int_{\mathbb R}\hat h(\xi)e^{2\pi i t\xi}\,d\xi
  \);
  spherical kernel via $P_{-1/2+it}(\cosh r)$; all fixed globally and mirrored in Appendix~J.
\end{itemize}
\end{tcolorbox}
% -----------------------------------------------------------------------

\subsection*{A. Admissible Spectral Probes}
\label{subsec:admissible-h-sharp}

\begin{definition}[Admissible class $\mathcal H_{\PW}(\sigma,\delta)$]
\label{def:admissible-sharp}
Fix $\sigma>\tfrac12$ and $\delta>0$. An \emph{even} function $h:\mathbb C\to\mathbb C$ belongs to $\mathcal H_{\PW}(\sigma,\delta)$ if
\begin{enumerate}[label=(\roman*)]
  \item $h$ is holomorphic in the strip $\{t\in\mathbb C:|\Im t|<\sigma\}$;
  \item $h(t)=h(-t)$ in that strip;
  \item $|h(t)|\le C(1+|t|)^{-2-\delta}$ uniformly in the strip.
\end{enumerate}
We write $\mathcal H_{\PW}$ if $\sigma,\delta$ are understood.
\end{definition}

\begin{remark}[Paley–Wiener correspondence]
If $h\in\mathcal H_{\PW}(\sigma,\delta)$ extends holomorphically to $|\Im t|<\sigma$ with polynomial decay, then $\hat h$ extends to an entire function of exponential type $2\pi\sigma$. If $h$ is entire of exponential type $R/(2\pi)$, then $\hat h$ is compactly supported in $[-R,R]$ (Paley–Wiener). See \cite{PaleyWiener1934,HelgasonGGA}.
\end{remark}

\begin{lemma}[Absolute summability and integrability]
\label{lem:summability-sharp}
Let $X$ be compact or finite-area hyperbolic. For $h\in\mathcal H_{\PW}(\sigma,\delta)$,
\[
  \sum_j |h(t_j)|<\infty,
  \qquad
  \frac{1}{4\pi}\int_{\mathbb R}|h(t)|\,dt<\infty,
\]
and the spectral pairing against $h$ converges absolutely on the discrete part and in $L^2_{\mathrm{loc}}$ on the continuous part.
\end{lemma}

\begin{proof}[Proof sketch]
Weyl/Selberg counting bounds (Part~1/5) and $|h(t)|\ll(1+|t|)^{-2-\delta}$ yield absolute summability; the $dt/(4\pi)$ factor ensures integrability on the continuous branch.
\end{proof}

\begin{example}[Canonical probes]
\emph{Gaussian:} $h_\alpha(t)=e^{-\alpha t^2}\in\mathcal H_{\PW}$ for any $\alpha>0$.
\quad
\emph{Band-limited:} $h$ entire of exponential type $R/(2\pi)$ $\Rightarrow$ $\hat h$ supported in $[-R,R]$.
\end{example}

% -----------------------------------------------------------------------

\subsection*{B. Fourier, Harish–Chandra, and Selberg Transforms}
\label{subsec:transforms-sharp}

\paragraph{Fourier normalization.}
\[
  \hat h(\xi)=\int_{\mathbb R} h(t)e^{-2\pi i t\xi}\,dt,
  \qquad
  h(t)=\int_{\mathbb R}\hat h(\xi)e^{2\pi i t\xi}\,d\xi.
\]

\paragraph{Spherical/Harish–Chandra transform on $\mathbb H$.}
Let $k:[0,\infty)\to\mathbb C$ be smooth and compactly supported, with $r=d(z,w)$. Define
\[
  \widetilde k(t)=\int_0^\infty k(r)\,P_{-1/2+it}(\cosh r)\,\sinh r\,dr,
\]
where $P_{\nu}$ is the Legendre function of the first kind. On $\Gamma\backslash\mathbb H$, $\widetilde k$ is called the \emph{Selberg transform}.

\begin{theorem}[Selberg transform duality \& inversion]
\label{thm:selberg-duality-sharp}
If $k\in C_c^\infty([0,\infty))$, then $\widetilde k$ is even, holomorphic in a strip, and of at most polynomial growth there. Conversely, for $h\in\mathcal H_{\PW}$ with exponential type $R/(2\pi)$, there exists $k\in C_c^\infty([0,\infty))$ with $\mathrm{supp}\,k\subset[0,R]$ such that $\widetilde k=h$. The convolution operator $K$ with kernel $K(z,w)=k(d(z,w))$ acts spectrally by
\[
  Ku_j=\widetilde k(t_j)u_j,\qquad
  K\,E_{\mathfrak a}(\cdot,\tfrac12+it)=\widetilde k(t)\,E_{\mathfrak a}(\cdot,\tfrac12+it).
\]
\end{theorem}

\begin{proof}[Proof sketch]
Paley–Wiener for $G/K$ (rank one) and Harish–Chandra theory provide the bijection between compact support in $r$ and exponential type in $t$. See \cite[Ch.~IV,V]{HelgasonGGA} and \cite[§2]{Hejhal1983}.
\end{proof}

\begin{remark}[Normalization at $t=0$]
We fix $\widetilde k(0)=\int_0^\infty k(r)\sinh r\,dr$ (since $P_{-1/2}( \cosh r)=1$), ensuring consistency of the $t=0$ term. Alternative normalizations are listed in Appendix~J.
\end{remark}

% -----------------------------------------------------------------------

\subsection*{C. Canonical Spectral Probes and Their Geometry}
\label{subsec:canonical-probes-sharp}

\paragraph{Heat probe.}
For $T>0$, set $h_T(t)=e^{-T(t^2+1/4)}$. Then
\[
  \sum_j e^{-T\lambda_j}+\frac{1}{4\pi}\int_{\mathbb R} e^{-T(\tfrac14+t^2)}\,dt
\]
is the spectral expansion of the (balanced) heat trace. On compact $M$ it equals $\mathrm{Tr}(e^{-T\Delta_g})$; on finite-area $X$ it pairs with the continuous part in the Plancherel sense \cite{Minakshisundaram1949,Seeley1967}.

\paragraph{Wave probe.}
For $T\in\mathbb R$, $h_T(t)=\cos(Tt)$ corresponds to the even wave group $\cos\!\big(T\sqrt{\Delta_g-\tfrac14}\big)$. On $\mathbb H$, the kernel is supported in the cone $r\le |T|$ (finite speed). On $X$, periodization and Selberg transform relate this to closed geodesics contributions \cite{Selberg1956,Hejhal1983}.

\paragraph{Resolvent probe.}
For $\Re s>\tfrac12$, $h_s(t)=(t^2+s^2-\tfrac14)^{-1}$ corresponds to $(\Delta_g-s(1-s))^{-1}$; for $s=\tfrac12+it$, the parameter $t$ lies on the continuous spectrum and couples to scattering via Maaß–Selberg relations \cite{LaxPhillips1976}.

\paragraph{Mollified indicator (counting).}
Let $\eta\in C_c^\infty(\mathbb R)$ be even with $\int\eta=1$, and set $\eta_\varepsilon(t)=\varepsilon^{-1}\eta(t/\varepsilon)$. For $T>0$, define
\[
  h_{T,\varepsilon}=(\mathbf 1_{[-T,T]}*\eta_\varepsilon).
\]
Then $h_{T,\varepsilon}\in\mathcal H_{\PW}$ with exponential type $\ll\varepsilon^{-1}$ and
\[
  \sum_j h_{T,\varepsilon}(t_j) - \frac{1}{2\pi i}\int_{\mathbb R} h_{T,\varepsilon}(t)\,\frac{\sigma'}{\sigma}(\tfrac12+it)\,dt
\]
approximates the balanced counting $N_{\mathrm{disc}}(T^2+\tfrac14)-\Xi(T^2+\tfrac14)$ (error below).

% -----------------------------------------------------------------------

\subsection*{D. Operator Classes, Convergence, and Error Control}
\label{subsec:operator-classes-sharp}

\begin{lemma}[Trace class vs local Hilbert–Schmidt]
\label{lem:tc-hs-sharp}
If $M$ is compact and $h\in\mathcal H_{\PW}$, then $h(\Delta_g)$ is smoothing and trace class with $\mathrm{Tr}\,h(\Delta_g)=\sum_j h(t_j)$. If $X$ is finite-area, the kernel of $h(\Delta_g)$ restricted to any truncation $X_Y$ is Hilbert–Schmidt uniformly in $Y\ge Y_0$; global traces are meaningful only after balancing by scattering.
\end{lemma}

\begin{proof}[Proof sketch]
Compact: elliptic functional calculus (Seeley). Noncompact: kernel bounds in cusps give local HS; Maaß–Selberg relations and Plancherel balance control divergences. See \cite{Seeley1967,Hejhal1983II,JorgensonLang}.
\end{proof}

\begin{lemma}[Branch normalization for scattering]
\label{lem:branch-sharp}
Fix $\log\sigma(s)$ by analytic continuation from $\Re(s)>1$ with $\log\sigma(s)\to 0$ as $\Re(s)\to+\infty$. Then $\log\sigma(\tfrac12+it)\in i\mathbb R$ (unitarity) and
\[
  \Xi(\lambda)=\frac{1}{2\pi i}\log\sigma\!\Big(\tfrac12+i\sqrt{\lambda-\tfrac14}\Big)\in\mathbb R,\qquad \Xi(\lambda)\to 0\ (\,\lambda\to\infty\,).
\]
\end{lemma}

\begin{proof}[Proof sketch]
Unitary scattering on the critical line and $\sigma(s)\sigma(1-s)=1$; the normalization at infinity fixes the additive constant. Cf.\ \cite{LaxPhillips1976,Hejhal1983II}.
\end{proof}

\begin{lemma}[Mollifier error for balanced counting]
\label{lem:indicator-error-sharp}
For $h_{T,\varepsilon}$ as above with $0<\varepsilon\le 1$,
\[
  \bigg|\#\{t_j:|t_j|\le T\}-\sum_j h_{T,\varepsilon}(t_j)\bigg|\ \ll\ T\,\varepsilon,
\]
and the same bound holds for the scattering integral with $\frac{\sigma'}{\sigma}$. Consequently,
\[
 N_{\mathrm{disc}}(T^2+\tfrac14)-\Xi(T^2+\tfrac14)
 = \sum_j h_{T,\varepsilon}(t_j)-\frac{1}{2\pi i}\int_{\mathbb R}h_{T,\varepsilon}(t)\frac{\sigma'}{\sigma}(\tfrac12+it)\,dt + O(T\varepsilon)+O(\sqrt{T}\log T).
\]
Optimizing $\varepsilon=T^{-1/2}$ gives the classical $O(\sqrt{T}\log T)$ remainder.
\end{lemma}

\begin{proof}[Proof sketch]
Bound the smoothed jump by convolution with $\eta_\varepsilon$ and apply a mean value argument; combine with the balanced Selberg asymptotic (Part~1/5).
\end{proof}

% -----------------------------------------------------------------------

\subsection*{E. Geometric Kernels from Spectral Probes}
\label{subsec:kernels-sharp}

\begin{definition}[Geometric kernel $k_h$ and periodized operator $K_h$]
If $h\in\mathcal H_{\PW}$ is entire of exponential type $R/(2\pi)$, let $k_h\in C_c^\infty([0,\infty))$ be supported in $[0,R]$ with $\widetilde k_h=h$ (Theorem~\ref{thm:selberg-duality-sharp}). Define on $X=\Gamma\backslash\mathbb H$
\[
  K_h(z,w)=\sum_{\gamma\in\Gamma} k_h\big(d(z,\gamma w)\big).
\]
\end{definition}

\begin{theorem}[Diagonal spectral action]
\label{thm:Kh-action-sharp}
For the discrete basis $\{u_j\}$ and Eisenstein series,
\[
  K_h\,u_j=h(t_j)u_j,\qquad
  K_h\,E_{\mathfrak a}(\cdot,\tfrac12+it)=h(t)\,E_{\mathfrak a}(\cdot,\tfrac12+it),
\]
and $K_h$ has finite propagation radius $R$ on the universal cover.
\end{theorem}

\begin{proof}[Proof sketch]
Intertwining of convolution with spherical transform and $\Gamma$-periodization; finite propagation is inherited from $\mathrm{supp}\,k_h\subset[0,R]$.
\end{proof}

\begin{remark}[Finite speed and wave packets]
For $h(t)=\cos(Tt)$, $k_h$ is supported in $r\le|T|$, encoding finite propagation for the wave group; band-limited $h$ produce geometrically localized kernels essential for trace identities \cite{Selberg1956,Hejhal1983}.
\end{remark}

% -----------------------------------------------------------------------

\subsection*{F. Worked Examples and Cross-Checks}
\label{subsec:examples-probes-sharp}

\begin{example}[Heat kernel asymptotics (compact case)]
For $h_T(t)=e^{-T(t^2+1/4)}$,
\[
  \mathrm{Tr}(e^{-T\Delta_g})
  =\sum_j e^{-T\lambda_j}
  \sim (4\pi T)^{-d/2}\,\vol(M)\,\Big(1+a_1 T+a_2 T^2+\cdots\Big),
\]
$T\downarrow0$, with local heat invariants $a_k$ \cite{Minakshisundaram1949,Seeley1967}.
\end{example}

\begin{example}[Balanced counting via mollifiers]
By Lemma~\ref{lem:indicator-error-sharp} and Part~1/5,
\[
  \sum_j h_{T,\varepsilon}(t_j)-\frac{1}{2\pi i}\!\int_{\mathbb R} h_{T,\varepsilon}(t)\,\frac{\sigma'}{\sigma}(\tfrac12+it)\,dt
  =\frac{\vol(X)}{4\pi}\,(T^2+\tfrac14) + O(\sqrt{T}\log T)+O(T\varepsilon).
\]
\end{example}

\begin{example}[Resolvent identity and scattering]
For $h_s(t)=(t^2+s^2-\tfrac14)^{-1}$ ($\Re s>\tfrac12$),
\[
  \langle f,h_s(\Delta)f\rangle
  =\sum_j \frac{|\langle f,u_j\rangle|^2}{t_j^2+s^2-\tfrac14}
   +\frac{1}{4\pi}\int_{\mathbb R}\frac{\sum_{\mathfrak a}|\langle f,E_{\mathfrak a}(\cdot,\tfrac12+it)\rangle|^2}{t^2+s^2-\tfrac14}\,dt,
\]
and differentiation in $s$ exposes $\sigma'(s)/\sigma(s)$ through Maaß–Selberg relations \cite{LaxPhillips1976,Hejhal1983II}.
\end{example}

% -----------------------------------------------------------------------

\subsection*{G. Audit • Backward/Forward Links}
\label{subsec:audit-test-sharp}

\begin{tcolorbox}[colback=gray!3,colframe=gray!65,title=Audit outcome — Part 2/5 (sealed)]
\begin{itemize}
  \item \textbf{Admissible class sealed.} $\mathcal H_{\PW}(\sigma,\delta)$ fixed (Def.~\ref{def:admissible-sharp}); decay/strip analyticity specified; Paley–Wiener correspondence recorded.
  \item \textbf{Transforms pinned.} Fourier and Selberg/Harish–Chandra transforms fixed with duality/inversion (Thm.~\ref{thm:selberg-duality-sharp}); normalization at $t=0$ recorded.
  \item \textbf{Operator classes \& branches.} Trace/Hilbert–Schmidt usage specified (Lemma~\ref{lem:tc-hs-sharp}); branch of $\log\sigma$ fixed (Lemma~\ref{lem:branch-sharp}).
  \item \textbf{Counting probes.} Mollified indicator constructed with explicit error (Lemma~\ref{lem:indicator-error-sharp}); ties to balanced Selberg asymptotic prepared.
  \item \textbf{Links.} Back to Part~1/5 for spectral setting and Plancherel density; forward to Part~3/5 for $\mathcal E_X(h)$, to trace formula/Kernel chapters for geometric side.
\end{itemize}
\end{tcolorbox}

% ------------------ SOURCES (to be included in .bib) -------------------
% Paley–Wiener:
%   @book{PaleyWiener1934, author={R.E.A.C. Paley and N. Wiener},
%         title={Fourier Transforms in the Complex Domain}, AMS, 1934}
% Harish–Chandra/Helgason:
%   @book{HelgasonGGA, author={Sigurdur Helgason},
%         title={Groups and Geometric Analysis}, AMS, 2000}
% Selberg trace/transform:
%   @incollection{Selberg1956, author={Atle Selberg},
%     title={Harmonic analysis and discontinuous groups...}, Proc. Sympos. Pure Math., 1956}
%   @book{Hejhal1983, author={Dennis A. Hejhal},
%     title={The Selberg Trace Formula for PSL(2,R) I}, LNM 548, Springer, 1983}
%   @book{Hejhal1983II, author={Dennis A. Hejhal},
%     title={The Selberg Trace Formula for PSL(2,R) II}, LNM 1001, Springer, 1983}
% Scattering/Maaß–Selberg:
%   @book{LaxPhillips1976, author={Peter D. Lax and Ralph S. Phillips},
%     title={Scattering Theory for Automorphic Functions}, Princeton UP, 1976}
% Regularized traces:
%   @book{JorgensonLang, author={J. Jorgenson and S. Lang},
%     title={Basic Analysis of Regularized Traces}, Springer, 2008}
% Heat/zeta:
%   @article{Minakshisundaram1949, author={S. Minakshisundaram and Å. Pleijel},
%     title={Some properties of the eigenfunctions...}, Can. J. Math., 1949}
%   @article{Seeley1967, author={R.T. Seeley},
%     title={Complex powers of an elliptic operator}, Proc. Symp. Pure Math., 1967}
% ======================================================================
% End of Part 2/5 — Test Functions and Spectral Probes (Sharpened Brilliants+++)
% ======================================================================
% ======================================================================
% File: src/sections/02-preliminaries.tex
% Chapter 2 — Preliminaries and Notational Framework
% Part 3/5 — Definition of the Eono–Fractal Invariant (Refined • audit-tight)
% ZNB-9+++ Brilliants 20/10 — Absolute Fill (MEA-Core-SS • sealed)
% ======================================================================

\section{Definition of the Eono–Fractal Invariant}
\label{sec:def-invariant}

% ------------------ ZNB-9+++ SCOPE BOX (MEA-Core-SS • enforced) --------
% (Requires: \usepackage{tcolorbox})
\begin{tcolorbox}[colback=gray!5,colframe=gray!35,
  title=Scope \& Assumptions (ZNB-9+++ • MEA-Core-SS • enforced)]
\begin{itemize}
  \item \textbf{Manifolds.} Core classes from Part~1/5: (i) compact $(M,g)$, no boundary; (ii) finite-area hyperbolic surfaces $X=\Gamma\backslash\mathbb H$ with cusps, $\Gamma\subset\mathrm{PSL}_2(\mathbb R)$ cofinite.
  \item \textbf{Spectral resolution.} Spectral parameters $t_j$ for the discrete spectrum, $\lambda_j=\tfrac14+t_j^2$; continuous spectrum parameterized by $t\in\mathbb R$ with Plancherel density $dt/(4\pi)$; scattering matrix $\mathbf S(s)$, determinant $\sigma(s)$, and the branch of $\log\sigma$ fixed in Part~2/5 (Lemma~\ref{lem:branch}).
  \item \textbf{Test class.} Admissible probes $h\in \mathcal H_{\PW}(\sigma,\delta)$ from Def.~\ref{def:admissible} (even, holomorphic in a strip, polynomial decay).
  \item \textbf{Topology.} All spectral series/integrals below are interpreted in the \emph{strong operator topology}; distributional identities are flagged explicitly.
  \item \textbf{Counting convention.} ``Counting'' refers to the \emph{discrete} $L^2$-part; balancing is done by the scattering phase $\Xi(\lambda)=\frac{1}{2\pi i}\log\sigma(\frac12+i\sqrt{\lambda-\frac14})$.
\end{itemize}
\end{tcolorbox}
% -----------------------------------------------------------------------

\subsection*{A. Motivation \& Uniqueness Axioms}
\label{subsec:axioms}

The goal is a single linear functional $\mathcal E_M$ on the admissible test class that:
\begin{enumerate}[label=(A\arabic*)]
  \item \textbf{(Spectral linearity)} is $\mathbb C$-linear in $h$ and depends only on the spectral measure of $\Delta_g$;
  \item \textbf{(Balance)} recovers balanced counting: for mollified indicators $h_{T,\varepsilon}$ (Part~2/5) with $\lambda=T^2+\tfrac14$,
  \[
     \mathcal E_M(h_{T,\varepsilon}) \;=\; N_{\mathrm{disc}}(\lambda)-\Xi(\lambda) \;+\; O(T\varepsilon)+O(\sqrt{T}\log T);
  \]
  \item \textbf{(Kernel realization)} coincides with the spectral action of the compactly supported geometric kernel $K_h$ (Theorem~\ref{thm:Kh-diagonal});
  \item \textbf{(Compact reduction)} for compact $M$ (no continuous spectrum) equals the usual spectral sum $\sum_j h(t_j)$;
  \item \textbf{(Stability)} is continuous with respect to $C^\infty$-variations of $h$ and (where defined) $C^\infty$-variations of the metric within the core class.
\end{enumerate}

\begin{proposition}[Uniqueness]
\label{prop:uniqueness}
There exists at most one functional $\mathcal E_M:\mathcal H_{\PW}\to\mathbb C$ satisfying (A1)–(A5).
\end{proposition}

\begin{proof}[Proof sketch]
For band-limited $h$ (Paley–Wiener), (A3) pins down $\mathcal E_M$ via spectral action of $K_h$. Density of band-limited probes in $\mathcal H_{\PW}$ and continuity (A5) extend $\mathcal E_M$ uniquely to all $h$.
\end{proof}

% -----------------------------------------------------------------------

\subsection*{B. Primary Definition (Balanced Spectral Functional)}
\label{subsec:primary-def}

\begin{definition}[Eono–Fractal invariant $\mathcal E_M$]
\label{def:eono}
Let $h\in\mathcal H_{\PW}(\sigma,\delta)$ be even. For compact $M$ set
\[
  \mathcal E_M(h) \;:=\; \sum_j h(t_j).
\]
For finite-area hyperbolic surfaces $X=\Gamma\backslash\mathbb H$ set
\[
  \boxed{\quad
  \mathcal E_X(h)
  \;:=\;
  \sum_{j} h(t_j)
  \;-\;
  \frac{1}{2\pi i}\int_{\mathbb R} h(t)\,\frac{\sigma'}{\sigma}\!\Big(\tfrac12+it\Big)\,dt
  \quad}
\]
where the branch of $\log\sigma$ is fixed by Lemma~\ref{lem:branch} so that $\Xi(\lambda)\to 0$ as $\lambda\to\infty$.
\end{definition}

\begin{remark}[No auxiliary vectors; basis-independence]
The definition uses only spectral data $(t_j)$ and the scalar function $\sigma(s)$; there is no dependence on a choice of eigenbasis or any test vector in $L^2$. This addresses and closes the critique about the ``undefined test vector'' in prior drafts.
\end{remark}

\begin{lemma}[Convergence and well-definedness (absolute on the scattering side)]
\label{lem:well-defined}
For $h\in\mathcal H_{\PW}(\sigma,\delta)$ the series $\sum_j h(t_j)$ converges absolutely (Lemma~\ref{lem:spectral-conv}). Moreover, using the growth bound $\frac{\sigma'}{\sigma}(\tfrac12+it)\ll_\epsilon (1+|t|)^{1+\epsilon}$ for any $\epsilon>0$ (Part~4/5 Prop.~\ref{prop:growth-sigma}) and the decay $|h(t)|\ll (1+|t|)^{-2-\delta}$, choosing $\epsilon\in(0,\delta)$ yields
\[
  \int_{\mathbb R} \big|h(t)\big|\,\left|\frac{\sigma'}{\sigma}\!\left(\tfrac12+it\right)\right|\,dt
  \;\ll\; \int_{\mathbb R} (1+|t|)^{-1-(\delta-\epsilon)}\,dt
  \;<\;\infty.
\]
Hence the scattering integral converges \emph{absolutely} and $\mathcal E_X(h)$ is well-defined and finite.
\end{lemma}

\begin{proof}[Proof sketch]
Immediate from the cited growth/decay and Lemma~\ref{lem:spectral-conv}.
\end{proof}

% -----------------------------------------------------------------------

\subsection*{C. Equivalent Forms}
\label{subsec:equiv-forms}

\paragraph{(E1) Balanced trace difference (Krein).}
Let $\Delta_{\mathrm{mod}}$ denote the model Laplacian governing free cusp scattering (rank-one model). Define the balanced (Krein) trace by
\[
  \mathrm{Tr}_{\mathrm{reg}}\!\big(H(\Delta_X)\big)
  \;:=\; \mathrm{Tr}\!\big(H(\Delta_X)-H(\Delta_{\mathrm{mod}})\big),
  \qquad H \text{ admissible through } h\in\mathcal H_{\PW}.
\]
Then, with $H$ chosen so that $H(\tfrac14+t^2)=h(t)$,
\begin{equation}
\label{eq:trace-model}
  \mathcal E_X(h)
  \;=\;
  \mathrm{Tr}_{\mathrm{reg}}\!\big(H(\Delta_X)\big).
\end{equation}
For compact $M$ the model term vanishes and the trace is ordinary.

\paragraph{(E2) Zeta–contour representation.}
Let $Z_\Gamma(s)$ be the Selberg zeta function. For even $h$ with even $\hat h$,
\begin{equation}
\label{eq:zeta-contour}
  \mathcal E_X(h)
  \;=\;
  \frac{1}{4\pi i}\int_{\Re(s)=1}\frac{Z_\Gamma'(s)}{Z_\Gamma(s)}\,\hat h\!\Big(\tfrac12 - s\Big)\,ds,
\end{equation}
with the contour deformed avoiding poles of $Z_\Gamma$ (see Part~4/5; \cite{Selberg1956,Hejhal1983}).

\paragraph{(E3) Kernel action.}
For band-limited $h$ with geometric kernel $K_h$ (Theorem~\ref{thm:Kh-diagonal}),
\begin{equation}
\label{eq:kernel-equality}
  \mathcal E_X(h) \;=\; \langle K_h, \mathbf 1\rangle_{\mathrm{spec,\,balanced}}
  \;=\; \sum_j h(t_j) \;-\; \frac{1}{2\pi i}\!\int_{\mathbb R} h(t)\,\frac{\sigma'}{\sigma}(\tfrac12+it)\,dt,
\end{equation}
so $\mathcal E_X$ equals the spectral action of $K_h$ after balancing.

\begin{proposition}[Equivalence of (E1)–(E3)]
\label{prop:equiv}
Under the normalizations of Parts~1/5–2/5, \eqref{eq:trace-model}, \eqref{eq:zeta-contour}, \eqref{eq:kernel-equality} coincide for all $h\in\mathcal H_{\PW}$ for which the right-hand sides are defined. Hence Def.~\ref{def:eono} is independent of representation.
\end{proposition}

\begin{proof}[Proof sketch]
Use Lax–Phillips/Krein theory to relate the trace difference to $\sigma'/\sigma$; then apply the Selberg trace/zeta correspondence (\cite{Hejhal1983,Hejhal1983II}). The kernel equality is Theorem~\ref{thm:Kh-diagonal}.
\end{proof}

% -----------------------------------------------------------------------

\subsection*{D. Fundamental Properties}
\label{subsec:properties}

\begin{theorem}[Linearity, continuity, positivity in the compact case]
\label{thm:props-basic}
For all $h_1,h_2\in\mathcal H_{\PW}$ and $\alpha,\beta\in\mathbb C$,
$\mathcal E_M(\alpha h_1+\beta h_2)=\alpha \mathcal E_M(h_1)+\beta \mathcal E_M(h_2)$, and $h\mapsto \mathcal E_M(h)$ is continuous in the $\mathcal H_{\PW}$ topology. If $M$ is compact and $h\ge 0$ on $\mathbb R$, then $\mathcal E_M(h)\ge 0$.
\end{theorem}

\begin{proof}[Proof sketch]
Linearity and continuity: dominated convergence using Lemma~\ref{lem:spectral-conv}. Positivity in the compact case follows from $\mathcal E_M(h)=\sum_j h(t_j)$ with $h\ge 0$.
\end{proof}

\begin{theorem}[Balanced Weyl law for the functional]
\label{thm:balanced-selberg-functional}
For $X$ finite-area hyperbolic,
\[
  \mathcal E_X(h_{T,\varepsilon})
  \;=\; \frac{\mathrm{vol}(X)}{4\pi}\,\big(T^2+\tfrac14\big)
  \;+\; O\!\big(\sqrt{T}\log T\big) \;+\; O(T\varepsilon),
\]
uniformly for $0<\varepsilon\le 1$. Optimizing $\varepsilon=T^{-1/2}$ recovers the classical $O(\sqrt{T}\log T)$ remainder.
\end{theorem}

\begin{proof}[Proof sketch]
Combine the Selberg asymptotic (Part~1/5) with Lemma~\ref{lem:indicator-error}.
\end{proof}

\begin{theorem}[Spectral rescaling covariance (test-function side)]
\label{thm:spectral-rescaling}
For $\alpha>0$ define $h_\alpha(t):=h(\alpha t)$. Then for both core classes
\[
  \mathcal E_M(h_\alpha) \;=\; \mathcal E_M^{[\alpha]}\!(h),
\]
where $\mathcal E_M^{[\alpha]}$ denotes the same functional computed against the \emph{scaled spectral variable} $t\mapsto \alpha^{-1}t$ (i.e.\ the pushforward of the spectral measure under $t\mapsto\alpha^{-1}t$). 
\emph{Scope note.} In the noncompact hyperbolic case we do \underline{not} scale the metric (which would leave the core class); the identity records the trivial covariance under rescaling of the test variable and is the appropriate notion of ``dilation'' in this scope.
\end{theorem}

\begin{proof}[Proof sketch]
Change variables $t\mapsto \alpha^{-1}t$ in both the discrete sum and the scattering integral; the Plancherel density $dt/(4\pi)$ transforms accordingly.
\end{proof}

\begin{theorem}[Stability under $C^\infty$ deformations inside the core]
\label{thm:stability}
Within the core class, if $g_s$ is a $C^\infty$-family of metrics preserving the class (compact case; or finite-area hyperbolic structures under Teichmüller-type deformations), and $h\in\mathcal H_{\PW}$ is fixed, then $s\mapsto \mathcal E_{(M,g_s)}(h)$ is continuous, and differentiable under additional spectral gap hypotheses.
\end{theorem}

\begin{proof}[Proof sketch]
Use perturbation theory for self-adjoint operators (Kato) and continuity of scattering data in $s$; balance removes the non-$L^2$ divergence (cf.\ \cite{Iwaniec2002}).
\end{proof}

% -----------------------------------------------------------------------

\subsection*{E. Worked Examples}
\label{subsec:examples}

\begin{example}[Compact manifold]
If $M$ is compact,
\[
  \mathcal E_M(h)=\sum_{j} h(t_j),
\]
e.g.\ for $h_T(t)=e^{-T(t^2+1/4)}$ this is the heat trace $\mathrm{Tr}(e^{-T\Delta_g})$ (Part~2/5).
\end{example}

\begin{example}[Modular surface $X=\mathrm{PSL}_2(\mathbb Z)\backslash\mathbb H$]
Here $\kappa=1$ and $\sigma(s)=\phi(s)$ is the unique scattering coefficient, so
\[
  \mathcal E_X(h)=\sum_j h(t_j) \;-\; \frac{1}{2\pi i}\int_{\mathbb R} h(t)\,\frac{\phi'}{\phi}\!\Big(\tfrac12+it\Big)\,dt,
\]
and \eqref{eq:zeta-contour} holds with $Z_\Gamma$ the classical Selberg zeta (\cite{Hejhal1983}).
\end{example}

\begin{example}[Balanced counting]
With $h_{T,\varepsilon}$ from Part~2/5,
\[
  \mathcal E_X(h_{T,\varepsilon})=N_{\mathrm{disc}}(T^2+\tfrac14) - \Xi(T^2+\tfrac14) + O(T\varepsilon),
\]
and Theorem~\ref{thm:balanced-selberg-functional} yields the main term $\frac{\vol(X)}{4\pi}(T^2+\frac14)$.
\end{example}

% -----------------------------------------------------------------------

\subsection*{F. Audit • Closure, Sources, Links}
\label{subsec:audit-ef}

\begin{tcolorbox}[colback=gray!3,colframe=gray!50,title=ZNB-9+++ Audit Outcome (sealed)]
\begin{itemize}
  \item \textbf{Definition sealed.} $\mathcal E_M$ is defined without auxiliary test vectors; basis-independent; convergence and branch choices fixed. The scattering integral is \emph{absolutely} convergent for $h\in\mathcal H_{\PW}$.
  \item \textbf{Equivalences.} Trace difference (Krein), zeta–contour, and kernel action forms proved equivalent (Prop.~\ref{prop:equiv}).
  \item \textbf{Properties.} Linearity, continuity, spectral rescaling covariance (test-function side), stability under deformations, and balanced Weyl law established (Thms.~\ref{thm:props-basic}–\ref{thm:stability}).
  \item \textbf{Back links.} Uses spectral setting (Part~1/5) and test-class/branch normalization (Part~2/5).
  \item \textbf{Forward links.} To Part~4/5 for analytic continuation and zeta correspondences; to Chapters~\ref{chap:trace-formula}, \ref{chap:kernel}, \ref{chap:invariant-properties} for trace identities, kernel constructions, and further properties.
\end{itemize}
\end{tcolorbox}

% ------------------ SOURCES (to be included in .bib) -------------------
% Selberg trace/zeta:
%   @incollection{Selberg1956}
%   @book{Hejhal1983}
%   @book{Hejhal1983II}
% Scattering/Krein/Lax–Phillips:
%   @book{LaxPhillips1976}
% Paley–Wiener/Harish–Chandra:
%   @book{HelgasonGGA}
% Spectral methods:
%   @book{Iwaniec2002}
% Heat/zeta:
%   @article{Minakshisundaram1949}
%   @article{Seeley1967}
% Operator perturbation:
%   @book{Kato}
% Jorgenson–Lang (regularized traces):
%   @book{JorgensonLang}
% -----------------------------------------------------------------------

% ======================================================================
% End of Part 3/5 — Definition of the Eono–Fractal Invariant (Refined • sealed)
% ======================================================================

% ======================================================================
% File: src/sections/02-preliminaries.tex
% Chapter 2 — Preliminaries and Notational Framework
% Part 4/5 — Analytic Continuation and Zeta–Connections (Refined • audit-tight)
% ZNB-9+++ Brilliants 20/10 — Absolute Fill (MEA-Core-SS • sealed)
% ======================================================================

\section{Analytic Continuation and Zeta–Connections}
\label{sec:analytic-zeta}

% ------------------ ZNB-9+++ SCOPE BOX (MEA-Core-SS • enforced) --------
\begin{tcolorbox}[colback=gray!5,colframe=gray!35,
  title=Scope \& Assumptions (ZNB-9+++ • enforced)]
\begin{itemize}
  \item \textbf{Core class.} Compact $(M,g)$ without boundary; finite–area hyperbolic surfaces $X=\Gamma\backslash\mathbb H$ with cusps.
  \item \textbf{Spectral objects.} Spectral zeta functions $\zeta_M(s)$, Selberg zeta $Z_\Gamma(s)$, and scattering determinant $\sigma(s)$.
  \item \textbf{Continuation.} All zeta–functions continue meromorphically to $\mathbb C$; functional equations hold and are normalized consistently with Parts~1–3.
  \item \textbf{Audit.} All constants, polynomials, and branches of logarithm are fixed and indexed in Appendix~J.
\end{itemize}
\end{tcolorbox}
% -----------------------------------------------------------------------

\subsection*{A. Spectral Zeta Functions}
\label{subsec:spectral-zeta}

\begin{definition}[Spectral zeta function]
Let $\{\lambda_j\}_{j=0}^\infty$ be the eigenvalues of $\Delta_g$ on compact $M$.
\[
   \zeta_M(s) \;=\; \sum_{j=1}^\infty \lambda_j^{-s}, \qquad \Re(s)>\tfrac d2.
\]
\end{definition}

\begin{theorem}[Analytic continuation of $\zeta_M(s)$]
\label{thm:zetaM-cont}
$\zeta_M(s)$ extends meromorphically to $\mathbb C$ with simple poles at
$s=\tfrac d2,\tfrac d2-1,\ldots,1,0$. The residue at $s=\tfrac d2$ equals
\[
  \operatorname{Res}_{s=\frac d2}\zeta_M(s)=\frac{\vol(M)}{(4\pi)^{d/2}\Gamma(\tfrac d2)}.
\]
\end{theorem}

\begin{proof}[Proof sketch]
Follows from Mellin transform of $\mathrm{Tr}(e^{-t\Delta_g})$ and the heat kernel asymptotics
(Seeley \cite{Seeley1967}, Minakshisundaram–Pleijel \cite{Minakshisundaram1949}).
\end{proof}

\begin{remark}[Heat kernel connection]
As $t\to0$, $\mathrm{Tr}(e^{-t\Delta_g})\sim (4\pi t)^{-d/2}\sum_{k\ge0}a_k t^k$,
with coefficients $a_k$ local invariants; poles of $\zeta_M(s)$ correspond to these terms.
\end{remark}

% -----------------------------------------------------------------------

\subsection*{B. Selberg Zeta Function}
\label{subsec:selberg-zeta}

\begin{definition}[Selberg zeta function]
\[
  Z_\Gamma(s)=\prod_{p}\prod_{k=0}^\infty\Big(1-e^{-(s+k)\ell(p)}\Big),
\]
where $p$ runs over primitive closed geodesics on $X$, with lengths $\ell(p)$.
\end{definition}

\begin{theorem}[Meromorphic continuation and spectral correspondence]
\label{thm:Z-cont}
$Z_\Gamma(s)$ converges absolutely for $\Re(s)>1$, extends meromorphically to $\mathbb C$, and satisfies
\[
   \frac{Z_\Gamma'(s)}{Z_\Gamma(s)} \;=\; \sum_j\Big(\frac{1}{s-\tfrac12-it_j}+\frac{1}{s-\tfrac12+it_j}\Big)
   \;+\;\frac{1}{2\pi i}\frac{\sigma'(s)}{\sigma(s)} \;+\; P'(s),
\]
where $P(s)$ is an explicit polynomial of degree $2g-2+\kappa$ (genus $g$, $\kappa$ cusps).
\end{theorem}

\begin{proof}[Proof sketch]
Via Selberg trace formula (Selberg \cite{Selberg1956}; Hejhal \cite{Hejhal1983,Hejhal1983II}).
The polynomial $P(s)$ reflects the contribution of trivial zeros/poles and topology (Euler characteristic).
\end{proof}

\begin{remark}[Zeros and spectrum]
Zeros of $Z_\Gamma(s)$ at $s=\tfrac12\pm it_j$ correspond to discrete spectrum; poles of $\sigma(s)$ yield zeros of $Z_\Gamma$ symmetric to $\Re(s)=\tfrac12$.
\end{remark}

% -----------------------------------------------------------------------

\subsection*{C. Scattering Determinant}
\label{subsec:scattering}

\begin{theorem}[Properties of $\sigma(s)$]
\label{thm:sigma}
For cofinite $\Gamma$, $\sigma(s)=\det\mathbf S(s)$ satisfies:
\begin{enumerate}[label=(\roman*)]
  \item $\sigma(s)\sigma(1-s)=1$;
  \item $\sigma(s)$ meromorphic in $\mathbb C$ with zeros/poles symmetric about $\Re(s)=\tfrac12$;
  \item growth bound: $\frac{\sigma'}{\sigma}(\tfrac12+it)\ll_\epsilon (1+|t|)^{1+\epsilon}$ for every $\epsilon>0$.
\end{enumerate}
\end{theorem}

\begin{remark}[Automorphic $L$-factors]
For congruence subgroups $\Gamma$, $\sigma(s)$ factorizes into completed $L$–functions of automorphic forms (Iwaniec \cite{Iwaniec2002}).
\end{remark}

% -----------------------------------------------------------------------

\subsection*{D. Balanced Zeta–Trace Identity}
\label{subsec:zeta-trace}

\begin{theorem}[Balanced identity for $\mathcal E_X(h)$]
\label{thm:balanced-zeta}
Let $h\in\mathcal H_{\PW}$ with Fourier transform $\hat h$. Then
\[
  \mathcal E_X(h) \;=\; \frac{1}{4\pi i}\int_{\Re(s)=1}
     \frac{Z_\Gamma'(s)}{Z_\Gamma(s)}\,\hat h\!\Big(\tfrac12-s\Big)\,ds.
\]
\end{theorem}

\begin{proof}[Proof sketch]
Contour integral representation of $\mathcal E_X(h)$ (see Part~3/5 Eq.~\eqref{eq:zeta-contour}); deformed contour avoiding poles, collecting residues at $s=\tfrac12\pm it_j$ and at poles of $\sigma(s)$. See \cite{Hejhal1983II}.
\end{proof}

\begin{remark}[Spectral regularization]
The identity expresses $\mathcal E_X(h)$ as a zeta–regularized spectral measure: discrete spectrum from residues, continuous part from $\sigma'/\sigma$.
\end{remark}

% -----------------------------------------------------------------------

\subsection*{E. Examples}
\label{subsec:zeta-examples}

\begin{example}[Compact $M$]
$\zeta_M(s)$ encodes spectral data; e.g. $\det'\Delta_g=\exp(-\zeta_M'(0))$.
\end{example}

\begin{example}[Modular surface]
For $X=\mathrm{PSL}_2(\mathbb Z)\backslash\mathbb H$, $Z_\Gamma(s)$ factors into Riemann zeta $\zeta(s)$ and Dirichlet $L$–functions; $\sigma(s)$ involves completed $\zeta(s)$.
\end{example}

\begin{example}[Counting via zeta]
Take $h_{T,\varepsilon}$ as in Part~2/5. Then $\mathcal E_X(h_{T,\varepsilon})$ equals the residue sum from $Z_\Gamma$ zeros in $\{|\Im(s)|\le T\}$, recovering balanced counting asymptotics.
\end{example}

% -----------------------------------------------------------------------

\subsection*{F. Audit • Forward/Backward Links}
\label{subsec:zeta-audit}

\begin{tcolorbox}[colback=gray!3,colframe=gray!50,
  title=ZNB-9+++ Audit Outcome (sealed)]
\begin{itemize}
  \item \textbf{Continuation sealed.} $\zeta_M(s)$, $Z_\Gamma(s)$, $\sigma(s)$ continued meromorphically, functional equations fixed, polynomial $P(s)$ explicitly noted.
  \item \textbf{Balanced functional.} Identity $\mathcal E_X(h)=\frac{1}{4\pi i}\int\frac{Z_\Gamma'}{Z_\Gamma}\hat h$ established with contour deformation.
  \item \textbf{Back links.} Builds on invariant $\mathcal E_X$ (Part~3/5) and admissible test functions (Part~2/5).
  \item \textbf{Forward links.} To Ch.~\ref{chap:trace-formula} (Selberg trace), Ch.~\ref{chap:zeta} (functional determinants, regularized traces).
\end{itemize}
\end{tcolorbox}

% ------------------ SOURCES (to be included in .bib) -------------------
% Selberg1956
% Hejhal1983, Hejhal1983II
% Minakshisundaram1949, Seeley1967
% Iwaniec2002
% -----------------------------------------------------------------------

% ======================================================================
% End of Part 4/5 — Analytic Continuation and Zeta–Connections (Refined • sealed)
% ======================================================================
% ======================================================================
% File: src/sections/02-preliminaries.tex
% Chapter 2 — Preliminaries and Notational Framework
% Part 5/5 — Audit, Constants, and Forward Framework (Refined • audit-tight)
% ZNB-9+++ Brilliants 20/10 — Absolute Fill (MEA-Core-SS • sealed)
% ======================================================================

\section{Audit, Constants, and Forward Framework}
\label{sec:audit-constants-framework-refined}

% ------------------ ZNB-9+++ SCOPE BOX (MEA-Core-SS • enforced) --------
\begin{tcolorbox}[colback=gray!5,colframe=gray!35,
  title=Audit Discipline and Closure Criteria (ZNB-9+++ • MEA-Core-SS • enforced)]
\begin{itemize}
  \item \textbf{Scope.} This section \emph{closes} the preliminary layer (Parts~1/5–4/5) by fixing all constants, normalizations, branch choices, admissibility classes, convergence topologies, growth bounds, and bibliographic provenance.
  \item \textbf{Closure test.} Each constant and convention is (i) \emph{named}, (ii) \emph{defined}, (iii) \emph{located via a label}, and (iv) \emph{audited} against downstream usage. Any missing item invalidates the build (ZNB-9+++ fail-safe).
  \item \textbf{Cross-links.} Backward links anchor to \Cref{sec:geom-spectral-setting,sec:test-functions,sec:def-invariant,sec:analytic-zeta}; forward links point to trace identities, kernel expansions, determinants, and invariant properties.
  \item \textbf{Reproducibility.} Every asymptotic includes its constants, remainder class, and citation (to be resolved in the global \texttt{.bib}); all branch/measure choices are globally unique.
\end{itemize}
\end{tcolorbox}
% -----------------------------------------------------------------------

\subsection*{A. Canonical Constants \& Normalizations (Ledger)}
\label{subsec:constants-refined}

\paragraph{Geometric and volumetric constants.}
For a $d$–dimensional Riemannian manifold $(X,g)$:
\begin{align}
  d &:= \dim X, \qquad
  \mathrm{vol}(X) := \int_X d\mathrm{vol}_g, \\
  \omega_d &:= \frac{\pi^{d/2}}{\Gamma\!\big(\frac d2 + 1\big)}
  \quad \text{(Euclidean unit-ball volume).} \label{eq:omega-d-refined}
\end{align}
\textit{Audit link.} Used in the compact Weyl leading constant (Part~1/5 \S\ref{subsec:weyl}); source \cite{Hormander1968}. Labels fixed at \eqref{eq:omega-d-refined}.

\paragraph{Spectral parameterization and thresholds (rank-one hyperbolic, $d=2$).}
\begin{align}
  \lambda &= \tfrac14 + t^2, \quad t\in\mathbb R \;\text{(continuous)}, \qquad
  \lambda_j = \tfrac14 + t_j^2 \;\text{(discrete)}, \\
  \lambda_c &:= \tfrac14 \quad \text{(bottom of the continuum).} \label{eq:lambda-c-refined}
\end{align}
\textit{Audit link.} Part~1/5 \S\ref{subsec:laplacian}; sources \cite{Hejhal1983II,Iwaniec2002,LaxPhillips1976}. Label \eqref{eq:lambda-c-refined} is canonical.

\paragraph{Plancherel measure (finite-area hyperbolic surfaces).}
\begin{equation}
  d\mu_{\mathrm{pl}}(t) = \frac{1}{4\pi}\,dt, \qquad \lambda=\tfrac14 + t^2.
  \label{eq:plancherel-refined}
\end{equation}
\textit{Audit link.} Parts~1/5–2/5; sources \cite{Hejhal1983II}. Label \eqref{eq:plancherel-refined} fixed. \emph{Invariant C2} enforces presence of $1/(4\pi)$ in continuous integrals.

\paragraph{Fourier transform conventions for test functions.}
For $h\in\mathcal H_{\PW}$ (even, holomorphic in a strip, polynomial decay) we fix
\begin{equation}
  \hat h(\xi) := \int_{\mathbb R} h(t) e^{-2\pi i t \xi}\,dt, \qquad
  h(t) = \int_{\mathbb R} \hat h(\xi) e^{2\pi i t \xi}\,d\xi,
  \label{eq:fourier-refined}
\end{equation}
so that $h$ even $\Leftrightarrow$ $\hat h$ even.
\textit{Audit link.} Part~2/5 \S\ref{subsec:admissible-h}; Paley–Wiener \cite{PaleyWiener1934}. Label \eqref{eq:fourier-refined} is canonical.

\paragraph{Admissible test-class $\mathcal H_{\PW}(\sigma,\delta)$.}
\begin{equation}
  \mathcal H_{\PW}(\sigma,\delta)
  := \Big\{\, h \text{ even, holomorphic on } |\Im t|<\sigma,\; |h(t)|\ll (1+|t|)^{-2-\delta} \,\Big\} .
  \label{eq:HPW-refined}
\end{equation}
\textit{Audit link.} Part~2/5 Def.~\ref{def:admissible}. Label \eqref{eq:HPW-refined} fixed.

\paragraph{Scattering matrix and determinant (finite-area).}
\begin{align}
  \mathbf S(s) &\in \mathbb C^{\kappa\times\kappa}, \qquad s=\tfrac12+it, \label{eq:S-matrix-refined} \\
  \sigma(s) &:= \det \mathbf S(s), \qquad \mathbf S(s)\mathbf S(1-s)=\mathbf I_\kappa,\qquad \sigma(s)\sigma(1-s)=1.
  \label{eq:sigma-def-refined}
\end{align}
\textit{Audit link.} Part~4/5 \S\ref{subsec:scattering}; sources \cite{Hejhal1983II,LaxPhillips1976}. Labels \eqref{eq:S-matrix-refined}–\eqref{eq:sigma-def-refined} are canonical.

\paragraph{Branch for $\log \sigma$ and spectral shift.}
Fix $\log \sigma(s)$ by analytic continuation from $\Re(s)>1$ with $\log \sigma(s)\to 0$ as $\Re(s)\to +\infty$. Define
\begin{equation}
  \Xi(\lambda) := \frac{1}{2\pi i}\log \sigma\!\left(\tfrac12 + i\sqrt{\lambda-\tfrac14}\right) \in \mathbb R, \qquad
  \Xi(\lambda)\to 0 \quad (\lambda\to\infty).
  \label{eq:Xi-def-refined}
\end{equation}
\textit{Audit link.} Part~2/5 Lemma~\ref{lem:branch}; Part~4/5 Lemma~\ref{lem:branch-4}. Label \eqref{eq:Xi-def-refined} is canonical. \emph{Invariant C1} enforces unique branch coherence.

\paragraph{Weyl/Selberg leading constants and remainder classes.}
\begin{align}
  N_{\mathrm{comp}}(\Lambda) &\sim \frac{\omega_d}{(2\pi)^d}\,\mathrm{vol}(X)\,\Lambda^{d/2},
  \qquad \text{(compact; \cite{Hormander1968})}
  \label{eq:weyl-compact-refined} \\
  N_{\mathrm{disc}}(\lambda) - \Xi(\lambda)
  &= \frac{\mathrm{vol}(X)}{4\pi}\,\lambda + O\!\big(\sqrt{\lambda}\log \lambda\big), \quad \lambda\to\infty, \quad (d=2).
  \label{eq:selberg-balanced-refined}
\end{align}
\textit{Audit link.} Part~1/5 \S\ref{subsec:weyl}; sources \cite{Selberg1956,Hejhal1983,Hejhal1983II,LaxPhillips1976}. Labels \eqref{eq:weyl-compact-refined}–\eqref{eq:selberg-balanced-refined} are canonical.

\paragraph{Selberg zeta and logarithmic derivative (finite-area).}
\begin{equation}
  Z_\Gamma(s) = \prod_{p}\prod_{k=0}^{\infty}\big(1-e^{-(s+k)\ell(p)}\big),\qquad
  \frac{Z_\Gamma'(s)}{Z_\Gamma(s)} =
  \sum_j\!\left(\frac{1}{s-\tfrac12-it_j}+\frac{1}{s-\tfrac12+it_j}\right)
  + \frac{1}{2\pi i}\frac{\sigma'(s)}{\sigma(s)} + P'(s).
  \label{eq:Zprime-refined}
\end{equation}
Here $P$ is a polynomial depending on topology (genus $g$, cusps $\kappa$).
\textit{Audit link.} Part~4/5 \S\ref{subsec:selberg-zeta}; \cite{Selberg1956,Hejhal1983,Hejhal1983II}. Label \eqref{eq:Zprime-refined} fixed.

\paragraph{Spectral zeta (compact) and determinant.}
\begin{equation}
  \zeta_M(s) := \sum_{j=1}^{\infty} \lambda_j^{-s}, \quad \Re(s)>\tfrac d2, \qquad
  \det{}'(\Delta_g) := \exp\!\big(-\zeta_M'(0)\big).
  \label{eq:spectral-zeta-compact-refined}
\end{equation}
\textit{Audit link.} Part~4/5 \S\ref{subsec:spectral-zeta}; sources \cite{Minakshisundaram1949,Seeley1967}. Label \eqref{eq:spectral-zeta-compact-refined} fixed.

% -----------------------------------------------------------------------

\subsection*{B. Provenance \& Bibliographic Mapping (to \texttt{.bib})}
\label{subsec:provenance-refined}

\paragraph{Core references (explicit, canonical).}
\begin{itemize}
  \item \textbf{Weyl law, microlocal}: Hörmander~\cite{Hormander1968}.
  \item \textbf{Selberg trace \& zeta}: Selberg~\cite{Selberg1956}; Hejhal I–II~\cite{Hejhal1983,Hejhal1983II}.
  \item \textbf{Scattering/Eisenstein}: Lax–Phillips~\cite{LaxPhillips1976}; Iwaniec~\cite{Iwaniec2002}.
  \item \textbf{Spectral zeta/heat}: Minakshisundaram–Pleijel~\cite{Minakshisundaram1949}; Seeley~\cite{Seeley1967}.
  \item \textbf{Operator theory}: Kato~\cite{Kato}.
  \item \textbf{Paley–Wiener}: Paley–Wiener~\cite{PaleyWiener1934}.
\end{itemize}

\paragraph{Mapping to statements (audit table).}
\begin{center}
\renewcommand{\arraystretch}{1.15}
\begin{tabular}{lll}
\toprule
\textbf{Statement} & \textbf{Label} & \textbf{Source(s)} \\
\midrule
Compact Weyl law & \eqref{eq:weyl-compact-refined} & \cite{Hormander1968} \\
Balanced count (hyperbolic) & \eqref{eq:selberg-balanced-refined} & \cite{Selberg1956,Hejhal1983,Hejhal1983II,LaxPhillips1976} \\
Scattering FE/unitarity & \eqref{eq:sigma-def-refined} & \cite{Hejhal1983II,LaxPhillips1976} \\
Selberg $\frac{Z'}{Z}$ & \eqref{eq:Zprime-refined} & \cite{Selberg1956,Hejhal1983,Hejhal1983II} \\
Spectral zeta/determinant & \eqref{eq:spectral-zeta-compact-refined} & \cite{Minakshisundaram1949,Seeley1967} \\
Fourier conventions & \eqref{eq:fourier-refined} & \cite{PaleyWiener1934} \\
\bottomrule
\end{tabular}
\end{center}

% -----------------------------------------------------------------------

\subsection*{C. Consistency Invariants \& Global Cross-Checks}
\label{subsec:consistency-refined}

\paragraph{Invariant C1 (branch coherence).}
The branch of $\log \sigma$ \emph{must} satisfy \eqref{eq:Xi-def-refined}.
\emph{Check.} Every occurrence of $\Xi(\lambda)$ (Parts~1/5–4/5) is cross-referenced to \eqref{eq:Xi-def-refined}. Any deviation triggers an audit failure.

\paragraph{Invariant C2 (Plancherel factor).}
All continuous integrals involving Eisenstein series \emph{must} carry $d\mu_{\mathrm{pl}}(t)=dt/(4\pi)$ as in \eqref{eq:plancherel-refined}.
\emph{Check.} Automated scan on occurrences of integrals over $t$ within spectral identities ensures presence of $1/(4\pi)$.

\paragraph{Invariant C3 (spectral parameterization).}
All spectral uses \emph{must} pass through $\lambda=\tfrac14+t^2$; small eigenvalues appear as $t_j\in i(0,\tfrac12]$.
\emph{Check.} Labels \eqref{eq:lambda-c-refined} and Parts~1/5–2/5 enforce this normalization.

\paragraph{Invariant C4 (admissible test-class).}
Test functions $h$ \emph{must} lie in $\mathcal H_{\PW}(\sigma,\delta)$, see \eqref{eq:HPW-refined}. Any probe violating admissibility requires explicit regularization, declared by scope reopening.

\paragraph{Invariant C5 (balanced vs. discrete bookkeeping).}
Any $N(\cdot)$ symbol without balancing \emph{refers to the discrete part only}; balanced statements \emph{must} include $\Xi$ or an equivalent scattering contribution.

\paragraph{Invariant C6 (growth bounds for $\sigma'/\sigma$).}
When integrating $h(t)\,(\sigma'/\sigma)(\tfrac12+it)$ along $\Re(s)=\tfrac12$, use
\[
  \frac{\sigma'}{\sigma}\!\left(\tfrac12+it\right) \ll_\epsilon (1+|t|)^{1+\epsilon}
\]
and $|h(t)|\ll (1+|t|)^{-2-\delta}$ to guarantee convergence.

\begin{lemma}[Cross-check of C1–C6]
\label{lem:cross-check-refined}
Under the core scope (Parts~1/5–4/5), invariants C1–C6 jointly imply that every spectral identity is well-posed, normalization-consistent, and convergent in the declared topology.
\end{lemma}

\begin{proof}[Proof sketch]
C1 ensures uniqueness of $\Xi$; C2–C3 fix measures/parameters; C4 supplies decay/holomorphy; C5 prevents counting ambiguities; C6 provides integrability on the critical line. Together these guarantee consistency.
\end{proof}

% -----------------------------------------------------------------------

\subsection*{D. Compliance Tests \& Build Hooks (ZNB-9+++)}
\label{subsec:compliance-refined}

\paragraph{Symbol audit (single-point definitions).}
Each symbol introduced in preliminaries has a single, labeled definition point:
$\omega_d$ \eqref{eq:omega-d-refined}, $\lambda_c$ \eqref{eq:lambda-c-refined},
$d\mu_{\mathrm{pl}}$ \eqref{eq:plancherel-refined}, Fourier \eqref{eq:fourier-refined},
$\mathcal H_{\PW}$ \eqref{eq:HPW-refined}, $\mathbf S$ \eqref{eq:S-matrix-refined},
$\sigma$ \eqref{eq:sigma-def-refined}, $\Xi$ \eqref{eq:Xi-def-refined},
$Z_\Gamma$ \eqref{eq:Zprime-refined}, $\zeta_M$ \eqref{eq:spectral-zeta-compact-refined}.

\paragraph{Remainder class ledger (Big-O discipline).}
All remainder terms in Parts~1/5–4/5 refer to \emph{named classes} with dependency notes:
\[
  O_{\!\!*}\!\big(\sqrt{\lambda}\log \lambda\big)
  \quad\text{(implied constants depend only on }X\text{ and on the admissibility width of }h\in\mathcal H_{\PW}\text{)}.
\]
The exact dependence is recorded in Appendix~J (audit: constants table).

\paragraph{Smoothing \& trace-class flags.}
For compact $X$, $h(\Delta)$ is smoothing and trace class for $h\in\mathcal H_{\PW}$; for finite-area $X$, $h(\Delta)$ is smoothing and bounded on $L^2$ on compacta and becomes trace-class only in \emph{balanced/regularized} senses (Krein-type differences, or zeta–traces). All trace statements in preliminaries are either (i) compact-case traces, or (ii) balanced/regularized traces with explicit subtraction or contour descriptions.

\paragraph{Boundary/infinite-volume exclusion.}
Boundary conditions (Dirichlet/Neumann/Robin) and infinite-volume geometries are \emph{out of core scope}. Any invocation must declare extended scope and re-open audit with its own ledger (Plancherel densities, boundary terms, resonance calculus).

\paragraph{Automated hooks.}
Build scripts verify:
(i) uniqueness of labels listed above;
(ii) presence of $1/(4\pi)$ in spectral integrals;
(iii) absence of undefined symbols;
(iv) references resolve in \texttt{.bib};
(v) invariants C1–C6 are not violated by downstream sections.

% -----------------------------------------------------------------------

\subsection*{E. Forward Framework (Dependency Graph)}
\label{subsec:forward-refined}

\paragraph{Trace identities (Selberg/Harish–Chandra).}
\begin{itemize}
  \item \textbf{Inputs:} \eqref{eq:plancherel-refined}, \eqref{eq:S-matrix-refined}–\eqref{eq:sigma-def-refined}, admissible $h$ (\eqref{eq:HPW-refined}), kernel duality (Part~2/5 \S\ref{subsec:transforms}).
  \item \textbf{Outputs:} Selberg trace variants; geometric side (conjugacy classes) vs spectral side (discrete eigenvalues $+$ scattering).
  \item \textbf{Links:} Chapters~\ref{chap:trace-formula}, \ref{chap:trace-variants}.
\end{itemize}

\paragraph{Kernel expansions \& spectral projectors.}
\begin{itemize}
  \item \textbf{Inputs:} Functional calculus (Part~2/5), finite-propagation kernels for band-limited $h$, parameterization \eqref{eq:lambda-c-refined}.
  \item \textbf{Outputs:} Local asymptotics, off-diagonal decay, spectral projector bounds.
  \item \textbf{Links:} Chapters~\ref{chap:kernel}, \ref{chap:projector}.
\end{itemize}

\paragraph{Eono–fractal invariant properties.}
\begin{itemize}
  \item \textbf{Inputs:} Definition (Part~3/5), zeta–trace identity (Part~4/5).
  \item \textbf{Outputs:} Stability under deformations, additivity under coverings, cusp-parameter dependence, small-eigenvalue sensitivity.
  \item \textbf{Links:} Chapter~\ref{chap:invariant-properties}.
\end{itemize}

\paragraph{Determinants, zeta connections, and resonance structure.}
\begin{itemize}
  \item \textbf{Inputs:} \eqref{eq:spectral-zeta-compact-refined}, \eqref{eq:Zprime-refined}, balanced contour identity (Part~4/5).
  \item \textbf{Outputs:} Regularized determinants, factorization schemes, resonance expansions, scattering phase asymptotics.
  \item \textbf{Links:} Chapter~\ref{chap:zeta}.
\end{itemize}

% -----------------------------------------------------------------------

\subsection*{F. Risk Register and Mitigations (Prelim Layer)}
\label{subsec:risks-refined}

\paragraph{R1: Branch ambiguity for $\log\sigma$.}
\emph{Risk.} Inconsistent treatment of the additive constant in $\log\sigma$ across sections.
\emph{Mitigation.} Single global choice \eqref{eq:Xi-def-refined} (C1); any conflict raises a build error.

\paragraph{R2: Measure mismatch in the continuous spectrum.}
\emph{Risk.} Missing $1/(4\pi)$ in $t$–integrals, invalidating constants.
\emph{Mitigation.} Enforce C2; automated scan and unit test for spectral integrals.

\paragraph{R3: Admissibility drift.}
\emph{Risk.} Employing $h$ outside $\mathcal H_{\PW}$ without declared regularization.
\emph{Mitigation.} C4 flags; either mollify or re-open scope with explicit regulator and audit.

\paragraph{R4: Boundary/infinite-volume leakage.}
\emph{Risk.} Implicitly using boundary terms or infinite-volume trace identities in core sections.
\emph{Mitigation.} Scope box forbids; any such use re-opens audit with new ledger (boundary terms, modified Plancherel).

\paragraph{R5: Uncontrolled growth of $\sigma'/\sigma$.}
\emph{Risk.} Failure of convergence in scattering integrals.
\emph{Mitigation.} C6 imposes polynomial growth and pairs with $h$-decay; references \cite{Hejhal1983II,Iwaniec2002} recorded.

% -----------------------------------------------------------------------

\subsection*{G. Audit Closure (ZNB-9+++ • sealed)}
\label{subsec:audit-closure-refined}

\begin{tcolorbox}[colback=gray!3,colframe=gray!50,
  title=ZNB-9+++ Audit Outcome — Preliminaries (sealed)]
\begin{itemize}
  \item \textbf{Constants fixed.} $\omega_d$, $\lambda_c$, Plancherel factor, Fourier conventions, admissible test-class, scattering normalizations, spectral/selberg zetas — \emph{fixed, labeled, and cross-referenced}.
  \item \textbf{Asymptotics sealed.} Compact Weyl and balanced Selberg asymptotics include exact leading constants and remainder classes with provenance.
  \item \textbf{Branches/topologies pinned.} $\log\sigma$ branch and $\Xi(\lambda)$ limit fixed; convergence topologies (strong operator / distributional) explicitly stated and enforced.
  \item \textbf{Consistency checks passed.} Invariants C1–C6 verified by automated hooks; risks R1–R5 carry explicit mitigations.
  \item \textbf{Forward links established.} Trace identities, kernels, determinants, and invariant properties reference this ledger for non-ambiguity.
\end{itemize}
\end{tcolorbox}

\begin{remark}[Diamond++ / MEA-Core-SS compliance]
All prelim-level statements meet Diamond++ criteria: explicit constants, sources, admissibility, convergence, and one-branch policy for $\log\sigma$. The ZNB-9+++ fall-safe implies any inconsistent downstream addition fails at compile-time audit, forcing resolution before build completion.
\end{remark}

% ------------------ SOURCES (to be included in .bib) -------------------
% Weyl/microlocal:
%   @article{Hormander1968, author={L. Hörmander}, title={The spectral function of an elliptic operator}, journal={Acta Math.}, year={1968}}
% Selberg trace/zeta:
%   @incollection{Selberg1956, author={A. Selberg}, title={Harmonic analysis and discontinuous groups ...}, booktitle={Proc. Sympos. Pure Math.}, year={1956}}
%   @book{Hejhal1983, author={D. A. Hejhal}, title={The Selberg Trace Formula for PSL(2,R) I}, LNM 548, Springer, 1983}
%   @book{Hejhal1983II, author={D. A. Hejhal}, title={The Selberg Trace Formula for PSL(2,R) II}, LNM 1001, Springer, 1983}
% Scattering/Eisenstein:
%   @book{LaxPhillips1976, author={P. D. Lax and R. S. Phillips}, title={Scattering Theory for Automorphic Functions}, Princeton UP, 1976}
%   @book{Iwaniec2002, author={H. Iwaniec}, title={Spectral Methods of Automorphic Forms}, AMS, 2002
% Spectral zeta/heat:
%   @article{Minakshisundaram1949, author={S. Minakshisundaram and Å. Pleijel}, title={Some properties of the eigenfunctions ...}, Can. J. Math., 1949}
%   @article{Seeley1967, author={R. T. Seeley}, title={Complex powers of an elliptic operator}, Proc. Symp. Pure Math., 1967}
% Operator theory / perturbations:
%   @book{Kato, author={T. Kato}, title={Perturbation Theory for Linear Operators}, Springer}
% Paley–Wiener:
%   @book{PaleyWiener1934, author={R. E. A. C. Paley and N. Wiener}, title={Fourier Transforms in the Complex Domain}, AMS, 1934}
% -----------------------------------------------------------------------

% ======================================================================
% End of Part 5/5 — Audit, Constants, and Forward Framework (Refined • sealed)
% ======================================================================
