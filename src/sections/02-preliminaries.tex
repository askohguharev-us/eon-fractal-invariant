% ======================================================================
% File: src/sections/02-preliminaries-sharpened.tex
% Chapter 2 — Preliminaries and Notational Framework
% Part 1/5 (Sharpened Brilliants+++)
% Geometric and Spectral Setting — ZNB-9+++ Brilliants 20/10 Absolute Fill
% ======================================================================

\chapter{Preliminaries and Notational Framework (Sharpened Brilliants+++)}
\label{chap:preliminaries-sharp}

\section{Geometric and Spectral Setting (Refined)}
\label{sec:geom-spectral-setting-sharp}

% ------------------ ZNB-9+++ SCOPE BOX (MEA-Core-SS • enforced) --------
\begin{tcolorbox}[colback=gray!5,colframe=gray!55,
  title=Scope \& Assumptions (ZNB-9+++ Brilliants+++ • enforced)]
\begin{itemize}
  \item \textbf{Completeness.} All $(M,g)$ considered are complete Riemannian manifolds. This secures essential self-adjointness of $\Delta_g$ on $C_c^\infty(M)$ and makes the spectral theorem applicable without external hypotheses.
  \item \textbf{Core classes.} (i) Compact manifolds without boundary; (ii) finite-area hyperbolic surfaces $X=\Gamma\backslash\mathbb H$, $\Gamma\subset\mathrm{PSL}_2(\mathbb R)$ cofinite, $\kappa$ cusps. Infinite-volume and boundary-value geometries are excluded at this layer; when reintroduced they require a separate ZNB-9+++ audit ledger.
  \item \textbf{Spectral split.} On noncompact finite-area hyperbolic surfaces the spectral resolution splits into $L^2$-discrete spectrum and continuous Eisenstein part. Scattering data enter via $\mathbf S(s)$ and $\sigma(s)=\det\mathbf S(s)$.
  \item \textbf{Counting convention.} $N(\Lambda)$ denotes the discrete counting function unless explicitly balanced with the spectral shift $\Xi(\lambda)$.
  \item \textbf{Normalization discipline.} Eisenstein normalization, Plancherel measure $dt/(4\pi)$, cusp scaling matrices, scattering determinants, and $\log\sigma$ branches are pinned globally. All constants are mirrored in Appendix~J (audit ledger).
  \item \textbf{Convergence topology.} All series and integrals in spectral expansions are in the strong operator topology on $L^2$; trace-class assertions flagged. Distributional traces arise only when balanced by scattering.
\end{itemize}
\end{tcolorbox}
% -----------------------------------------------------------------------

\subsection*{A. Classes of Manifolds $(M,g)$}
\label{subsec:classes-sharp}

\begin{definition}[Core manifold classes]
\begin{enumerate}[label=(\roman*)]
  \item \textbf{Compact without boundary.} Spectrum purely discrete, nonnegative, accumulating only at $\infty$.
  \item \textbf{Finite-area hyperbolic surfaces with cusps.} $X=\Gamma\backslash\mathbb H$, $\Gamma$ cofinite. Spectrum = discrete $\{\lambda_j\}$ $\cup$ continuous $\{\tfrac14+t^2\}$, $t\in\mathbb R$, realized by Eisenstein series $E_{\mathfrak a}(z,\tfrac12+it)$.
  \item \textbf{Excluded.} Infinite-volume cases and boundary conditions (Dirichlet/Neumann) are outside prelim scope.
\end{enumerate}
\end{definition}

\begin{remark}[Beyond cusps]
Funnels, orbifold points, or conical singularities alter resonance structure and require extended functional calculi. These are excluded here; cf.\ Guillopé--Zworski for funnels.
\end{remark}

% -----------------------------------------------------------------------

\subsection*{B. Laplace–Beltrami Operator and Spectrum}
\label{subsec:laplacian-sharp}

\begin{definition}[Laplace–Beltrami operator]
For $f\in C^\infty(M)$,
\[
  \Delta_g f := -\mathrm{div}_g(\nabla_g f).
\]
On complete $(M,g)$, the Friedrichs extension realizes $\Delta_g$ as a nonnegative self-adjoint operator on $L^2(M,g)$.
\end{definition}

\begin{conditions}[Spectral parametrization]
\begin{itemize}
  \item Compact: discrete eigenvalues $0=\lambda_0<\lambda_1\le\lambda_2\le\cdots$, $\lambda_j\to\infty$.
  \item Finite-area hyperbolic ($d=2$): spectrum = discrete $\{\lambda_j\}$ $\cup$ continuum $\{\tfrac14+t^2:t\in\mathbb R\}$. Use parameter $t_j$ via $\lambda_j=\tfrac14+t_j^2$ ($t_j\in\mathbb R$) or $\lambda_j=\tfrac14-r_j^2$ ($t_j=ir_j$, $0<r_j\le \tfrac12$).
  \item Threshold: $\lambda_c=\tfrac14$ is bottom of the continuous spectrum.
\end{itemize}
\end{conditions}

\begin{remark}[Essential self-adjointness]
On complete $M$, $\Delta_g$ is essentially self-adjoint on $C_c^\infty(M)$; the spectral theorem applies. On $X=\Gamma\backslash\mathbb H$, the continuous spectrum corresponds to principal series representations of $\mathrm{SL}_2(\mathbb R)$.
\end{remark}

% -----------------------------------------------------------------------

\subsection*{C. Spectral Counting: Weyl and Selberg Asymptotics}
\label{subsec:weyl-sharp}

\paragraph{Compact case (Weyl law).}
\[
  N(\Lambda)=\#\{\lambda_j\le\Lambda\}
  \;\sim\;\frac{\omega_d}{(2\pi)^d}\,\mathrm{vol}_g(M)\,\Lambda^{d/2},\qquad \Lambda\to\infty.
\]

\paragraph{Finite-area hyperbolic surfaces (Selberg).}
\[
  N_{\mathrm{disc}}(\lambda)
  =\frac{\vol(X)}{4\pi}\,\lambda+O\!\big(\sqrt{\lambda}\log\lambda\big),\quad \lambda\to\infty.
\]

\begin{definition}[Balanced counting]
\[
  N_{\mathrm{bal}}(\lambda):=N_{\mathrm{disc}}(\lambda)-\Xi(\lambda),
\]
where $\Xi(\lambda)=\frac{1}{2\pi i}\log\sigma(\tfrac12+i\sqrt{\lambda-\tfrac14})$.
\end{definition}

\begin{theorem}[Balanced Selberg asymptotic]
\[
  N_{\mathrm{disc}}(\lambda)-\Xi(\lambda)
  =\frac{\vol(X)}{4\pi}\,\lambda+O\!\big(\sqrt{\lambda}\log\lambda\big).
\]
\end{theorem}

\begin{proof}[Audit sketch]
Integrate the logarithmic derivative of $Z_\Gamma$ against $h$, invoke Selberg trace formula, and match with scattering. Sources: Selberg (1956), Hejhal I–II, Lax--Phillips.
\end{proof}

\begin{remark}[Small eigenvalues]
Eigenvalues $\lambda_j<1/4$ contribute only boundedly to $N_{\mathrm{disc}}(\lambda)$ and do not change the main asymptotic.
\end{remark}

% -----------------------------------------------------------------------

\subsection*{D. Spectral Decomposition and Functional Calculus}
\label{subsec:spectral-decomposition-sharp}

\begin{definition}[Spectral split]
For finite-area $X$, decompose $L^2(X)$ into discrete eigenfunctions $\{u_j\}$ and continuous Eisenstein series $E_{\mathfrak a}(z,\tfrac12+it)$ with Plancherel density $dt/(4\pi)$.
\end{definition}

\begin{theorem}[Functional calculus]
For $\Psi\in C_0^\infty(\mathbb R)$,
\[
  \Psi(\Delta_g)f
  =\sum_j \Psi(\lambda_j)\langle f,u_j\rangle u_j
  +\frac{1}{4\pi}\sum_{\mathfrak a}\int_\mathbb R \Psi(\tfrac14+t^2)\langle f,E_{\mathfrak a}(\cdot,\tfrac12+it)\rangle E_{\mathfrak a}(\cdot,\tfrac12+it)\,dt.
\]
\end{theorem}

\begin{remark}[Operator classes]
On compact $M$, $\Psi(\Delta_g)$ is smoothing and trace class. On finite-area $X$, smoothing on $X\times X$, bounded on $L^2$, but not globally trace class unless balanced.
\end{remark}

% -----------------------------------------------------------------------

\subsection*{E. Notation Ledger (Audit Sealed)}
\label{subsec:notation-invariants-sharp}

\begin{itemize}
  \item Dimension $d$, volume $\vol(X)$, injectivity radius $\mathrm{inj}(X)$.
  \item Cuspidal data: $\kappa$, cusp widths $\{w_i\}$.
  \item Threshold $\lambda_c=\tfrac14$ (hyperbolic $d=2$).
  \item Spectral shift $\Xi(\lambda)=\frac{1}{2\pi i}\log\sigma(\tfrac12+i\sqrt{\lambda-\tfrac14})$, branch fixed by $\Xi(\lambda)\to0$ as $\lambda\to\infty$.
\end{itemize}

% -----------------------------------------------------------------------

\subsection*{F. Audit • Forward/Backward Links (Sharpened)}
\label{subsec:audit-links-sharp}

\begin{tcolorbox}[colback=gray!3,colframe=gray!65,title=Audit outcome — Part 1/5 (sealed)]
\begin{itemize}
  \item Constants, branch choices, Plancherel measure, spectral parametrization — \emph{sealed}.
  \item Compact Weyl and balanced Selberg asymptotics with explicit constants recorded.
  \item Discrete vs continuous spectra normalized, scattering determinant conventions fixed.
  \item Back links: definitions mirrored in Appendix~J (audit ledger).
  \item Forward links: test functions (\S\ref{sec:test-functions}), invariant definition (\S\ref{sec:def-invariant}), kernel/projector chapters.
\end{itemize}
\end{tcolorbox}

% ======================================================================
% End of Part 1/5 — Geometric and Spectral Setting (Sharpened Brilliants+++)
% ======================================================================
% ======================================================================
% File: src/sections/02-preliminaries-sharpened.tex
% Chapter 2 — Preliminaries and Notational Framework
% Part 2/5 (Sharpened Brilliants+++)
% Test Functions and Spectral Probes — ZNB-9+++ Brilliants 20/10 Absolute Fill
% ======================================================================

\section{Test Functions and Spectral Probes (Refined)}
\label{sec:test-functions-sharp}

% ------------------ ZNB-9+++ SCOPE BOX (MEA-Core-SS • enforced) --------
\begin{tcolorbox}[colback=gray!5,colframe=gray!55,
  title=Scope \& Assumptions (ZNB-9+++ Brilliants+++ • enforced)]
\begin{itemize}
  \item \textbf{Setting.} Core classes from Part~1/5: (i) compact $(M,g)$, no boundary; (ii) finite-area hyperbolic surfaces $X=\Gamma\backslash\mathbb H$ with cusps ($\Gamma\subset\mathrm{PSL}_2(\mathbb R)$ cofinite). Spectral parameterization $\lambda=\tfrac14+t^2$; Plancherel density $dt/(4\pi)$.
  \item \textbf{Purpose.} Specify admissible classes of spectral probes $h(t)$; pin Fourier/Harish–Chandra/Selberg transforms and the exact duality with geometric kernels $k(r)$; fix convergence topologies and branch conventions feeding trace identities and the invariant $\mathcal E_X(h)$ (Part~3/5).
  \item \textbf{Convergence \& topology.} All spectral expansions converge in the strong operator topology on $L^2$; where traces appear on noncompact $X$, they are understood only in balanced/regularized sense. Distributional statements are flagged.
  \item \textbf{Normalization.} Fourier transform
  \(
    \hat h(\xi)=\int_{\mathbb R} h(t)e^{-2\pi i t\xi}\,dt
  \),
  inversion
  \(
    h(t)=\int_{\mathbb R}\hat h(\xi)e^{2\pi i t\xi}\,d\xi
  \);
  spherical kernel via $P_{-1/2+it}(\cosh r)$; all fixed globally and mirrored in Appendix~J.
\end{itemize}
\end{tcolorbox}
% -----------------------------------------------------------------------

\subsection*{A. Admissible Spectral Probes}
\label{subsec:admissible-h-sharp}

\begin{definition}[Admissible class $\mathcal H_{\PW}(\sigma,\delta)$]
\label{def:admissible-sharp}
Fix $\sigma>\tfrac12$ and $\delta>0$. An \emph{even} function $h:\mathbb C\to\mathbb C$ belongs to $\mathcal H_{\PW}(\sigma,\delta)$ if
\begin{enumerate}[label=(\roman*)]
  \item $h$ is holomorphic in the strip $\{t\in\mathbb C:|\Im t|<\sigma\}$;
  \item $h(t)=h(-t)$ in that strip;
  \item $|h(t)|\le C(1+|t|)^{-2-\delta}$ uniformly in the strip.
\end{enumerate}
We write $\mathcal H_{\PW}$ if $\sigma,\delta$ are understood.
\end{definition}

\begin{remark}[Paley–Wiener correspondence]
If $h\in\mathcal H_{\PW}(\sigma,\delta)$ extends holomorphically to $|\Im t|<\sigma$ with polynomial decay, then $\hat h$ extends to an entire function of exponential type $2\pi\sigma$. If $h$ is entire of exponential type $R/(2\pi)$, then $\hat h$ is compactly supported in $[-R,R]$ (Paley–Wiener). See \cite{PaleyWiener1934,HelgasonGGA}.
\end{remark}

\begin{lemma}[Absolute summability and integrability]
\label{lem:summability-sharp}
Let $X$ be compact or finite-area hyperbolic. For $h\in\mathcal H_{\PW}(\sigma,\delta)$,
\[
  \sum_j |h(t_j)|<\infty,
  \qquad
  \frac{1}{4\pi}\int_{\mathbb R}|h(t)|\,dt<\infty,
\]
and the spectral pairing against $h$ converges absolutely on the discrete part and in $L^2_{\mathrm{loc}}$ on the continuous part.
\end{lemma}

\begin{proof}[Proof sketch]
Weyl/Selberg counting bounds (Part~1/5) and $|h(t)|\ll(1+|t|)^{-2-\delta}$ yield absolute summability; the $dt/(4\pi)$ factor ensures integrability on the continuous branch.
\end{proof}

\begin{example}[Canonical probes]
\emph{Gaussian:} $h_\alpha(t)=e^{-\alpha t^2}\in\mathcal H_{\PW}$ for any $\alpha>0$.
\quad
\emph{Band-limited:} $h$ entire of exponential type $R/(2\pi)$ $\Rightarrow$ $\hat h$ supported in $[-R,R]$.
\end{example}

% -----------------------------------------------------------------------

\subsection*{B. Fourier, Harish–Chandra, and Selberg Transforms}
\label{subsec:transforms-sharp}

\paragraph{Fourier normalization.}
\[
  \hat h(\xi)=\int_{\mathbb R} h(t)e^{-2\pi i t\xi}\,dt,
  \qquad
  h(t)=\int_{\mathbb R}\hat h(\xi)e^{2\pi i t\xi}\,d\xi.
\]

\paragraph{Spherical/Harish–Chandra transform on $\mathbb H$.}
Let $k:[0,\infty)\to\mathbb C$ be smooth and compactly supported, with $r=d(z,w)$. Define
\[
  \widetilde k(t)=\int_0^\infty k(r)\,P_{-1/2+it}(\cosh r)\,\sinh r\,dr,
\]
where $P_{\nu}$ is the Legendre function of the first kind. On $\Gamma\backslash\mathbb H$, $\widetilde k$ is called the \emph{Selberg transform}.

\begin{theorem}[Selberg transform duality \& inversion]
\label{thm:selberg-duality-sharp}
If $k\in C_c^\infty([0,\infty))$, then $\widetilde k$ is even, holomorphic in a strip, and of at most polynomial growth there. Conversely, for $h\in\mathcal H_{\PW}$ with exponential type $R/(2\pi)$, there exists $k\in C_c^\infty([0,\infty))$ with $\mathrm{supp}\,k\subset[0,R]$ such that $\widetilde k=h$. The convolution operator $K$ with kernel $K(z,w)=k(d(z,w))$ acts spectrally by
\[
  Ku_j=\widetilde k(t_j)u_j,\qquad
  K\,E_{\mathfrak a}(\cdot,\tfrac12+it)=\widetilde k(t)\,E_{\mathfrak a}(\cdot,\tfrac12+it).
\]
\end{theorem}

\begin{proof}[Proof sketch]
Paley–Wiener for $G/K$ (rank one) and Harish–Chandra theory provide the bijection between compact support in $r$ and exponential type in $t$. See \cite[Ch.~IV,V]{HelgasonGGA} and \cite[§2]{Hejhal1983}.
\end{proof}

\begin{remark}[Normalization at $t=0$]
We fix $\widetilde k(0)=\int_0^\infty k(r)\sinh r\,dr$ (since $P_{-1/2}( \cosh r)=1$), ensuring consistency of the $t=0$ term. Alternative normalizations are listed in Appendix~J.
\end{remark}

% -----------------------------------------------------------------------

\subsection*{C. Canonical Spectral Probes and Their Geometry}
\label{subsec:canonical-probes-sharp}

\paragraph{Heat probe.}
For $T>0$, set $h_T(t)=e^{-T(t^2+1/4)}$. Then
\[
  \sum_j e^{-T\lambda_j}+\frac{1}{4\pi}\int_{\mathbb R} e^{-T(\tfrac14+t^2)}\,dt
\]
is the spectral expansion of the (balanced) heat trace. On compact $M$ it equals $\mathrm{Tr}(e^{-T\Delta_g})$; on finite-area $X$ it pairs with the continuous part in the Plancherel sense \cite{Minakshisundaram1949,Seeley1967}.

\paragraph{Wave probe.}
For $T\in\mathbb R$, $h_T(t)=\cos(Tt)$ corresponds to the even wave group $\cos\!\big(T\sqrt{\Delta_g-\tfrac14}\big)$. On $\mathbb H$, the kernel is supported in the cone $r\le |T|$ (finite speed). On $X$, periodization and Selberg transform relate this to closed geodesics contributions \cite{Selberg1956,Hejhal1983}.

\paragraph{Resolvent probe.}
For $\Re s>\tfrac12$, $h_s(t)=(t^2+s^2-\tfrac14)^{-1}$ corresponds to $(\Delta_g-s(1-s))^{-1}$; for $s=\tfrac12+it$, the parameter $t$ lies on the continuous spectrum and couples to scattering via Maaß–Selberg relations \cite{LaxPhillips1976}.

\paragraph{Mollified indicator (counting).}
Let $\eta\in C_c^\infty(\mathbb R)$ be even with $\int\eta=1$, and set $\eta_\varepsilon(t)=\varepsilon^{-1}\eta(t/\varepsilon)$. For $T>0$, define
\[
  h_{T,\varepsilon}=(\mathbf 1_{[-T,T]}*\eta_\varepsilon).
\]
Then $h_{T,\varepsilon}\in\mathcal H_{\PW}$ with exponential type $\ll\varepsilon^{-1}$ and
\[
  \sum_j h_{T,\varepsilon}(t_j) - \frac{1}{2\pi i}\int_{\mathbb R} h_{T,\varepsilon}(t)\,\frac{\sigma'}{\sigma}(\tfrac12+it)\,dt
\]
approximates the balanced counting $N_{\mathrm{disc}}(T^2+\tfrac14)-\Xi(T^2+\tfrac14)$ (error below).

% -----------------------------------------------------------------------

\subsection*{D. Operator Classes, Convergence, and Error Control}
\label{subsec:operator-classes-sharp}

\begin{lemma}[Trace class vs local Hilbert–Schmidt]
\label{lem:tc-hs-sharp}
If $M$ is compact and $h\in\mathcal H_{\PW}$, then $h(\Delta_g)$ is smoothing and trace class with $\mathrm{Tr}\,h(\Delta_g)=\sum_j h(t_j)$. If $X$ is finite-area, the kernel of $h(\Delta_g)$ restricted to any truncation $X_Y$ is Hilbert–Schmidt uniformly in $Y\ge Y_0$; global traces are meaningful only after balancing by scattering.
\end{lemma}

\begin{proof}[Proof sketch]
Compact: elliptic functional calculus (Seeley). Noncompact: kernel bounds in cusps give local HS; Maaß–Selberg relations and Plancherel balance control divergences. See \cite{Seeley1967,Hejhal1983II,JorgensonLang}.
\end{proof}

\begin{lemma}[Branch normalization for scattering]
\label{lem:branch-sharp}
Fix $\log\sigma(s)$ by analytic continuation from $\Re(s)>1$ with $\log\sigma(s)\to 0$ as $\Re(s)\to+\infty$. Then $\log\sigma(\tfrac12+it)\in i\mathbb R$ (unitarity) and
\[
  \Xi(\lambda)=\frac{1}{2\pi i}\log\sigma\!\Big(\tfrac12+i\sqrt{\lambda-\tfrac14}\Big)\in\mathbb R,\qquad \Xi(\lambda)\to 0\ (\,\lambda\to\infty\,).
\]
\end{lemma}

\begin{proof}[Proof sketch]
Unitary scattering on the critical line and $\sigma(s)\sigma(1-s)=1$; the normalization at infinity fixes the additive constant. Cf.\ \cite{LaxPhillips1976,Hejhal1983II}.
\end{proof}

\begin{lemma}[Mollifier error for balanced counting]
\label{lem:indicator-error-sharp}
For $h_{T,\varepsilon}$ as above with $0<\varepsilon\le 1$,
\[
  \bigg|\#\{t_j:|t_j|\le T\}-\sum_j h_{T,\varepsilon}(t_j)\bigg|\ \ll\ T\,\varepsilon,
\]
and the same bound holds for the scattering integral with $\frac{\sigma'}{\sigma}$. Consequently,
\[
 N_{\mathrm{disc}}(T^2+\tfrac14)-\Xi(T^2+\tfrac14)
 = \sum_j h_{T,\varepsilon}(t_j)-\frac{1}{2\pi i}\int_{\mathbb R}h_{T,\varepsilon}(t)\frac{\sigma'}{\sigma}(\tfrac12+it)\,dt + O(T\varepsilon)+O(\sqrt{T}\log T).
\]
Optimizing $\varepsilon=T^{-1/2}$ gives the classical $O(\sqrt{T}\log T)$ remainder.
\end{lemma}

\begin{proof}[Proof sketch]
Bound the smoothed jump by convolution with $\eta_\varepsilon$ and apply a mean value argument; combine with the balanced Selberg asymptotic (Part~1/5).
\end{proof}

% -----------------------------------------------------------------------

\subsection*{E. Geometric Kernels from Spectral Probes}
\label{subsec:kernels-sharp}

\begin{definition}[Geometric kernel $k_h$ and periodized operator $K_h$]
If $h\in\mathcal H_{\PW}$ is entire of exponential type $R/(2\pi)$, let $k_h\in C_c^\infty([0,\infty))$ be supported in $[0,R]$ with $\widetilde k_h=h$ (Theorem~\ref{thm:selberg-duality-sharp}). Define on $X=\Gamma\backslash\mathbb H$
\[
  K_h(z,w)=\sum_{\gamma\in\Gamma} k_h\big(d(z,\gamma w)\big).
\]
\end{definition}

\begin{theorem}[Diagonal spectral action]
\label{thm:Kh-action-sharp}
For the discrete basis $\{u_j\}$ and Eisenstein series,
\[
  K_h\,u_j=h(t_j)u_j,\qquad
  K_h\,E_{\mathfrak a}(\cdot,\tfrac12+it)=h(t)\,E_{\mathfrak a}(\cdot,\tfrac12+it),
\]
and $K_h$ has finite propagation radius $R$ on the universal cover.
\end{theorem}

\begin{proof}[Proof sketch]
Intertwining of convolution with spherical transform and $\Gamma$-periodization; finite propagation is inherited from $\mathrm{supp}\,k_h\subset[0,R]$.
\end{proof}

\begin{remark}[Finite speed and wave packets]
For $h(t)=\cos(Tt)$, $k_h$ is supported in $r\le|T|$, encoding finite propagation for the wave group; band-limited $h$ produce geometrically localized kernels essential for trace identities \cite{Selberg1956,Hejhal1983}.
\end{remark}

% -----------------------------------------------------------------------

\subsection*{F. Worked Examples and Cross-Checks}
\label{subsec:examples-probes-sharp}

\begin{example}[Heat kernel asymptotics (compact case)]
For $h_T(t)=e^{-T(t^2+1/4)}$,
\[
  \mathrm{Tr}(e^{-T\Delta_g})
  =\sum_j e^{-T\lambda_j}
  \sim (4\pi T)^{-d/2}\,\vol(M)\,\Big(1+a_1 T+a_2 T^2+\cdots\Big),
\]
$T\downarrow0$, with local heat invariants $a_k$ \cite{Minakshisundaram1949,Seeley1967}.
\end{example}

\begin{example}[Balanced counting via mollifiers]
By Lemma~\ref{lem:indicator-error-sharp} and Part~1/5,
\[
  \sum_j h_{T,\varepsilon}(t_j)-\frac{1}{2\pi i}\!\int_{\mathbb R} h_{T,\varepsilon}(t)\,\frac{\sigma'}{\sigma}(\tfrac12+it)\,dt
  =\frac{\vol(X)}{4\pi}\,(T^2+\tfrac14) + O(\sqrt{T}\log T)+O(T\varepsilon).
\]
\end{example}

\begin{example}[Resolvent identity and scattering]
For $h_s(t)=(t^2+s^2-\tfrac14)^{-1}$ ($\Re s>\tfrac12$),
\[
  \langle f,h_s(\Delta)f\rangle
  =\sum_j \frac{|\langle f,u_j\rangle|^2}{t_j^2+s^2-\tfrac14}
   +\frac{1}{4\pi}\int_{\mathbb R}\frac{\sum_{\mathfrak a}|\langle f,E_{\mathfrak a}(\cdot,\tfrac12+it)\rangle|^2}{t^2+s^2-\tfrac14}\,dt,
\]
and differentiation in $s$ exposes $\sigma'(s)/\sigma(s)$ through Maaß–Selberg relations \cite{LaxPhillips1976,Hejhal1983II}.
\end{example}

% -----------------------------------------------------------------------

\subsection*{G. Audit • Backward/Forward Links}
\label{subsec:audit-test-sharp}

\begin{tcolorbox}[colback=gray!3,colframe=gray!65,title=Audit outcome — Part 2/5 (sealed)]
\begin{itemize}
  \item \textbf{Admissible class sealed.} $\mathcal H_{\PW}(\sigma,\delta)$ fixed (Def.~\ref{def:admissible-sharp}); decay/strip analyticity specified; Paley–Wiener correspondence recorded.
  \item \textbf{Transforms pinned.} Fourier and Selberg/Harish–Chandra transforms fixed with duality/inversion (Thm.~\ref{thm:selberg-duality-sharp}); normalization at $t=0$ recorded.
  \item \textbf{Operator classes \& branches.} Trace/Hilbert–Schmidt usage specified (Lemma~\ref{lem:tc-hs-sharp}); branch of $\log\sigma$ fixed (Lemma~\ref{lem:branch-sharp}).
  \item \textbf{Counting probes.} Mollified indicator constructed with explicit error (Lemma~\ref{lem:indicator-error-sharp}); ties to balanced Selberg asymptotic prepared.
  \item \textbf{Links.} Back to Part~1/5 for spectral setting and Plancherel density; forward to Part~3/5 for $\mathcal E_X(h)$, to trace formula/Kernel chapters for geometric side.
\end{itemize}
\end{tcolorbox}

% ------------------ SOURCES (to be included in .bib) -------------------
% Paley–Wiener:
%   @book{PaleyWiener1934, author={R.E.A.C. Paley and N. Wiener},
%         title={Fourier Transforms in the Complex Domain}, AMS, 1934}
% Harish–Chandra/Helgason:
%   @book{HelgasonGGA, author={Sigurdur Helgason},
%         title={Groups and Geometric Analysis}, AMS, 2000}
% Selberg trace/transform:
%   @incollection{Selberg1956, author={Atle Selberg},
%     title={Harmonic analysis and discontinuous groups...}, Proc. Sympos. Pure Math., 1956}
%   @book{Hejhal1983, author={Dennis A. Hejhal},
%     title={The Selberg Trace Formula for PSL(2,R) I}, LNM 548, Springer, 1983}
%   @book{Hejhal1983II, author={Dennis A. Hejhal},
%     title={The Selberg Trace Formula for PSL(2,R) II}, LNM 1001, Springer, 1983}
% Scattering/Maaß–Selberg:
%   @book{LaxPhillips1976, author={Peter D. Lax and Ralph S. Phillips},
%     title={Scattering Theory for Automorphic Functions}, Princeton UP, 1976}
% Regularized traces:
%   @book{JorgensonLang, author={J. Jorgenson and S. Lang},
%     title={Basic Analysis of Regularized Traces}, Springer, 2008}
% Heat/zeta:
%   @article{Minakshisundaram1949, author={S. Minakshisundaram and Å. Pleijel},
%     title={Some properties of the eigenfunctions...}, Can. J. Math., 1949}
%   @article{Seeley1967, author={R.T. Seeley},
%     title={Complex powers of an elliptic operator}, Proc. Symp. Pure Math., 1967}
% ======================================================================
% End of Part 2/5 — Test Functions and Spectral Probes (Sharpened Brilliants+++)
% ======================================================================
ммирт
