% ─────────────────────────────────────────────
% Tome II : Absolutum invariant — Preface (v10.1)
% Path    : src/sections-2/00-preface-abs_invariant.tex
% Mode    : DIAMOND-STRICT | Academic-Annals | BRILLIANT-LUMEN v10.1
% ─────────────────────────────────────────────

\chapter*{Preface}
\addcontentsline{toc}{chapter}{Preface}

\begin{center}
\textbf{The Absolutum invariant: analytic–spectral foundations without new axioms}
\end{center}

\vspace{0.75em}

\noindent\textbf{Foundational framework.}
All results of Tome~II are formulated within the classical set theory \textbf{ZFC}.
Whenever class-level arguments are convenient, we use the conservative notational extension \textbf{GBC} 
(with global choice, but all results are interpretable in \textbf{ZFC} via standard coding).  
Constructive variants (CZF/IZF) are indicated explicitly when relevant; otherwise, classical logic is assumed.  
No new axioms beyond this foundation are introduced in the present tome.  
The phrase “without axioms’’ denotes “without additional axioms beyond the chosen base system.’’

\vspace{0.5em}
\noindent\textbf{Position.}
Tome~I (\emph{Eon–fractal invariant}) established the categorical and local–global scaffolding.
Tome~II (\emph{Absolutum invariant}) develops the complete analytic–spectral layer upon that structure.
Tome~III will employ the invariant exclusively within pure mathematics, addressing limit and classical millennium problems, yet never introducing applied interpretations or computational keys.

\vspace{0.5em}
\noindent\textbf{Objective.}
We construct a universal analytic–spectral object, the \emph{Absolutum invariant} \(\mathfrak A\),
which unifies sheaf–theoretic descent, operator–theoretic spectra, and cohomological obstructions within a single measurable framework.
The invariant is developed through provable closure, stability, and functoriality statements sufficient for applications in higher theoretical analysis.

\vspace{0.5em}
\noindent\textbf{Methodology and editorial policy.}
The exposition follows the schema
\[
\mathrm{Definition} \;\longrightarrow\; \mathrm{Lemma} \;\longrightarrow\; \mathrm{Theorem} \;\longrightarrow\; \mathrm{Correctness\ Test}.
\]
Each notion appears by explicit definition, and each assertion is proved or precisely referenced.
\textbf{Correctness Tests (CT)} are \emph{mathematical theorems} of the form
\[
\forall X\,(\varphi(X)\;\rightarrow\;\psi(X)),
\]
where \(\varphi,\psi\) belong to the formal signature of the tome.
They are not assumed to be algorithmically decidable unless stated.  
A dual structure is maintained: the formal body (proofs) and concise orientation notes (nonformal context), which never enter logical derivations.

\vspace{0.5em}
\noindent\textbf{Minimal semantic vocabulary.}
Let \(\Gamma\) denote a small site with finite Čech coverings.
Over \(\Gamma\), we consider separable Hilbert fields, unital \(C^\ast\)-algebra sheaves with faithful semifinite traces, 
and measurable families of self-adjoint Dirac/Laplace-type operators with compact resolvent on each object \(C\in\Gamma\).
Local data are glued by descent, and obstructions are encoded cohomologically.

\vspace{0.5em}
\noindent\textbf{Meta-operators as formal devices.}
We employ three rigorously defined meta-operators, each serving a structural role within the formal system:

\begin{itemize}
  \item \textbf{\(\Omega\)} (\emph{domain guard}). A predicate \(\Omega(C)\) on contexts \(C\in\mathrm{Ob}(\Gamma)\) specifying admissibility 
  (measurability, separability, trace–compatibility, ellipticity).  
  All quantifiers are explicitly restricted to objects \(C\) satisfying \(\Omega(C)\);  
  statements for \(C\not\models\Omega(C)\) are vacuous.  
  Hence “boundary of applicability’’ means “all claims are formed under \(\Omega\).’’
  
  \item \textbf{\(\Sigma\)} (\emph{selection functional}). A choice operator
  \(\Sigma:\mathrm{Sect}(C)/\!\sim\,\to \mathrm{Sect}(C)\)
  selecting a representative in each nonempty gauge–equivalence class once existence is established.  
  Proofs are invariant under the choice of representative.
  If \(\Sigma\) requires the Axiom of Choice (AC), this dependency is stated explicitly.

  \item \textbf{\(\Psi\)} (\emph{reflexive endofunctor}).  
  An endofunctor \(\Psi:\mathsf{Inv}\to\mathsf{Inv}\) on the category of invariants with natural transformations 
  \(\eta:\mathrm{Id}\Rightarrow\Psi\) and \(\mu:\Psi^2\Rightarrow\Psi\) forming an idempotent monad.  
  \(\Psi\) is conservative: if a statement \(P\) in the base language holds for an invariant \(I\), then \(P\) holds for \(\Psi(I)\) and conversely, provided \(P\) is preserved by the structure maps.  
  The existence of \(\Psi\) is established in Theorem~2.3.7.
\end{itemize}

\vspace{0.5em}
\noindent\textbf{Scope of CT blocks.}
CT statements establish:  
(i) descent and globalization under curvature and compatibility bounds;  
(ii) stability under refinement and gauge transformations;  
(iii) scale control (\(\varepsilon\)-regularity and \(\Gamma\)-limits);  
(iv) spectral continuity and meromorphic control for zeta–functions.  
Each chapter ends with an explicit CT block.

\vspace{0.5em}
\noindent\textbf{Relations among tomes.}
Tome~I provides the local–global grammar;  
Tome~II develops analytic–spectral theorems;  
Tome~III employs only their corollaries, adding no new principles.  
See Chapter~\ref{chap:spectral-descent} for the main construction 
and Theorem~\ref{thm:absolutum-universality} for the universal property.

\vspace{0.5em}
\noindent\textbf{Reading protocol.}
Each chapter opens with a mini–table of contents and a dependency map.  
Forward and backward references are explicit.  
Orientation notes are typographically separated and excluded from proofs.  
All quantifiers, domains, and operators are defined before use.

\vspace{1.25em}
\begin{flushright}
\textit{Alexander S. Kozhukharev}\\
Independent Researcher, Moscow\\
Tome II: \emph{Absolutum invariant}\\
\textit{Version BRILLIANT–LUMEN v10.1 (arXiv/Annals ready)}
\end{flushright}

% ─────────────────────────────────────────────
% End of Preface
% ─────────────────────────────────────────────
